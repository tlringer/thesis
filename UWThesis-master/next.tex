\section*{The Next Era: Proof Engineering for All}

% 2030? could do decades for each of this lol. I guess only if energy

Proof engineering has come a long way, but it is still accessible mostly to experts, and perhaps the occasional practitioner.
Proof repair has made proof engineering easier for experts, and a bit easier for practitioners.
But there is a lot more that we as a community can do to bring proof engineering to all: not just experts and practitioners,
but also software engineers and potential users from in other domains.

I thus conclude with a discussion of 12 future project summaries building up to the next era.
If any of these ideas inspire you, please work with me to bring them to life.

\subsection*{Proof Engineering for Experts}

Unifying theme: lateral reach. Some examples:

\paragraph{More Proof Assistants} Thoughts from PUMPKIN Pi on Isabelle/HOL, future work from PUMPKIN PATCH.

\paragraph{More Changes} Version updates, isolating large changes (PUMPKIN PATCH), relations more general than equivalences (PUMPKIN Pi).

\paragraph{More Styles} ML for decompiler (PUMPKIN Pi, REPLICA) : more for diverse proof styles (PUMPKIN PATCH). Note that this is a WIP, but sketch out project, challenges, future ideas, expectations, evaluation a bit.

\subsection*{Proof Engineering for Practitioners}

Unifying theme: usability. Some examples:

\paragraph{Automation} More search procedures for automatic configuration, e-graphs from PUMPKIN Pi, custom unification heuristics.

\paragraph{Integration} IDE \& CI integration, HCI for repair.

\paragraph{Evaluation} repair challenge, user studies ideas (PUMPKIN PATCH, REPLICA, panel w/ Benjamin Pierce, QED at large). (maybe look for more ideas, this can be merged with integration if need be).

\subsection*{Proof Engineering for Software Engineers}

Unifying theme: mixed methods verification, or the 2030 vision from Twitter thread. Some examples:

\paragraph{Gradual Verification} A continuum from testing to verification, tools to help with that.

\paragraph{Tool-Assisted Proof Development} Tool-assisted development to follow good design principles for verificattion (James Wilcox conversation, final REPLICA takeaway).

\paragraph{Specification Inference} Analysis to infer specs (TA1).

\subsection*{Proof Engineering for New Domains}

Unifying theme: collaboration, new abstractions for new domains). Some examples:

\paragraph{Machine Learning} Fairification \& other ML correctness properties. Some stuff here but more.

\paragraph{Cryptography} Lots of stuff here but not thinking broadly enough. What about cryptographic proof systems? ZK and beyond. Recall email thread.

\paragraph{Something Else} Look for more in survey paper, email, DARPA TAs, Twitter. Healthcare perhaps? % https://twitter.com/TaliaRinger/status/1373747841944883201
