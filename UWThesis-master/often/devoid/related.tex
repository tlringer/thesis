\section{Related Work}
\label{sec:related}

\subparagraph*{Ornaments.}
\toolnameb automates discovery of and lifting across algebraic ornaments in a higher-order dependently typed language.
In the decade since the discovery of ornaments~\cite{mcbride}, there have been a number
of formalizations and embedded implementations of ornaments~\cite{Dagand:2013:CTO:2591370.2591396, ko2013relational, dagand2014transporting, ko2016programming, dagand2017essence}.
\toolnameb is the first tool for ornamentation to operate over a non-embedded dependently typed language.
It essentially moves the automation-heavy approach of Ornamentation in ML~\cite{Williams2017},
which operates on non-embedded ML code, into the type theory that forms the basis of theorem provers like Coq. 
In doing so, it takes advantage of the properties of algebraic ornaments~\cite{mcbride}.
It also introduces the first search algorithm to identify ornaments, which in the past 
was identified as a ``gap'' in the literature~\cite{ko2016programming}.

\subparagraph*{Lifting Proofs.}
\toolnameb identifies and lifts proofs along a specific equivalence 
similar to that from existing ornaments work~\cite{ko2016programming}.
The need to automatically lift functions and proofs
across equivalences and other relations is a long-standing challenge for proof 
engineers~\cite{magaud2000changing, barthe2001type, magaud2003changing, huffman2013lifting, zimmermann2015automatic, cohen:hal-01414881}.
The univalence axiom from Homotopy Type Theory~\cite{univalent2013homotopy} enables transparent transport of proofs;
cubical type theory~\cite{cohen2016cubical} gives univalence a constructive interpretation. 

Our work is closely related to \textit{Equivalences for Free!}~\cite{tabareau2017equivalences}, which brings this full circle, 
using mathematical properties of univalence to enable lifting across equivalences in a substantial subset of CIC$_\omega$ without relying
on the univalence axiom. In doing so, it introduces and formalizes the relation that our specification depends upon,
and implements a framework for lifting in Coq. This framework is more general than \toolnameb: It lifts along any equivalence, not
just ornamental promotions, and can handle opaque terms,
with the caveat that users must prove each equivalence themselves; \toolnameb requires non-opaque terms and lifts along the class of equivalences that correspond
to ornamental promotions, taking advantage of the mathematical properties of ornaments to eliminate the need for explicit applications of section
and retraction, and to discover and prove certain equivalences automatically. These mathematical properties
allow us to automatically
lift the induction principle and eliminate references to old terms,
which is beneficial for performance.

Similarly, our work is related to CoqEAL~\cite{cohen:hal-01414881}, which transfers functions along arbitrary relations
between types. As these relations do not necessarily need to be equivalences, this framework is more general
than our work. Similar tradeoffs between automation and generality apply: CoqEAL produces functions that refer to the old type,
and does not yet support automatic inference of relations. In addition, CoqEAL currently only supports automatic transfer of functions,
and does not yet handle proofs.

These tools may provide an alternative backend for \toolnameb. Furthermore,
our search algorithm may help discover relations that make these tools easier to use,
and our lifting algorithm may help improve automation and efficiency for 
certain relations in these tools.

\subparagraph*{Program and Proof Reuse.}
The problem that we solve is fundamentally about proof reuse,
which applies software reuse principles to ITPs. 
There is a wealth of work in proof reuse, from tactic languages~\cite{felty1994generalization} and logical frameworks~\cite{caplan1995logical},
to tools for proof abstraction and generalization~\cite{pons2000generalization, johnsen2004theorem},
to domain-specific methodologies~\cite{Delaware:2011:PLT:2048066.2048113} and frameworks~\cite{Delaware:2013:MLC:2429069.2429094}.

\toolnameb focuses on the specific problem of reuse
when adding fully-determined indices to types.
Other approaches to this problem include combinators which definitionally reduce to desirable terms~\cite{DBLP:journals/corr/abs-1803-08150} in the language Cedille,
and automatic generation of conversion functions in Ghostbuster~\cite{McDonell:2016:GTS:2951913.2951914} for GADTs in Haskell.
Our work focuses on a type theory different from both of these, in which the properties that allow for such combinators in Cedille are not present, and in which dependent types introduce challenges not present in Haskell.

\toolnameb is not the first tool to combine search with reuse. 
Optician~\cite{miltner2017synthesizing} synthesizes bidirectional string transformations;
a similar approach may help extend tooling to handle transformations for low-level data.
\textsc{Pumpkin Patch}~\cite{ringer2018adapting} 
searches the difference in proofs for patches that can be used to repair proofs broken by changes;
\toolnameb uses a similar approach to identify functions
that form an equivalence. The resulting tools are complementary: \toolnameb supports the addition
of indices and hypotheses, which \textsc{Pumpkin Patch} does not support; \textsc{Pumpkin Patch} supports changes
in values, which \toolnameb does not support. 

