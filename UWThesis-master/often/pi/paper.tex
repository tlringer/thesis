%% For double-blind review submission, w/o CCS and ACM Reference (max submission space)
\documentclass[sigplan,screen]{acmart}
%% For double-blind review submission, w/ CCS and ACM Reference
%\documentclass[acmsmall,review,anonymous]{acmart}\settopmatter{printfolios=true}
%% For single-blind review submission, w/o CCS and ACM Reference (max submission space)
%\documentclass[acmsmall,review]{acmart}\settopmatter{printfolios=true,printccs=false,printacmref=false}
%% For single-blind review submission, w/ CCS and ACM Reference
%\documentclass[acmsmall,review]{acmart}\settopmatter{printfolios=true}
%% For final camera-ready submission, w/ required CCS and ACM Reference
%\documentclass[acmsmall]{acmart}\settopmatter{}

%%% The following is specific to PLDI '21 and the paper
%%% 'Proof Repair across Type Equivalences'
%%% by Talia Ringer, RanDair Porter, Nathaniel Yazdani, John Leo, and Dan Grossman.
%%%
\setcopyright{acmlicensed}
\acmPrice{15.00}
\acmDOI{10.1145/3453483.3454033}
\acmYear{2021}
\copyrightyear{2021}
\acmSubmissionID{pldi21main-p43-p}
\acmISBN{978-1-4503-8391-2/21/06}
\acmConference[PLDI '21]{Proceedings of the 42nd ACM SIGPLAN International Conference on Programming Language Design and Implementation}{June 20--25, 2021}{Virtual, UK}
\acmBooktitle{Proceedings of the 42nd ACM SIGPLAN International Conference on Programming Language Design and Implementation (PLDI '21), June 20--25, 2021, Virtual, UK}

%% Bibliography style
\bibliographystyle{ACM-Reference-Format}
%% Citation style
%% Note: author/year citations are required for papers published as an
%% issue of PACMPL.
%\citestyle{acmauthoryear}   %% For author/year citations


%%%%%%%%%%%%%%%%%%%%%%%%%%%%%%%%%%%%%%%%%%%%%%%%%%%%%%%%%%%%%%%%%%%%%%
%% Note: Authors migrating a paper from PACMPL format to traditional
%% SIGPLAN proceedings format must update the '\documentclass' and
%% topmatter commands above; see 'acmart-sigplanproc-template.tex'.
%%%%%%%%%%%%%%%%%%%%%%%%%%%%%%%%%%%%%%%%%%%%%%%%%%%%%%%%%%%%%%%%%%%%%%


%% Some recommended packages.
\usepackage{booktabs}   %% For formal tables:
                        %% http://ctan.org/pkg/booktabs
\usepackage{subcaption} %% For complex figures with subfigures/subcaptions
                        %% http://ctan.org/pkg/subcaption

\usepackage{enumerate} % for lists
\usepackage{listings} % for code
\usepackage{xspace} % so we don't need to figure out spacing after \toolname every time
\usepackage{mathpartir} % for inference rules
\usepackage{bbm} % to render N for the natural numbers
\usepackage{syntax} % for code highlighting
\usepackage{xcolor} % for code highlighting
\usepackage{colortbl} % to black out table cells
\usepackage{multirow} % for tables
\usepackage{tikz} % for fancy circles and smileys

% Nice rendering of Coq code
\lstdefinelanguage{coq}{
    keywords={Repair, module, Theorem, Proof, Record, Lemma, Definition, Abort, Qed, forall, Inductive, Type, Prop, Set, fun, fix, forall, Require, Import, Fixpoint, match, end, with, as, return, struct, Qed, Defined, let},
    basicstyle=\linespread{0.95}\small\ttfamily,
    keywordstyle=\color{blue},
    commentstyle=\itshape\rmfamily,
    showstringspaces=false,
    columns=flexible,
    breaklines=true,
    texcl=true,
    mathescape=true,
    tabsize=4,
    stringstyle=\color{brown},
    escapeinside={(@}{@)},
}

\lstset{language=coq} % default

\newcommand\toolname{\textsc{Pumpkin P}i\xspace} % tool name
\newcommand\company{Galois\xspace} % company name
\newcommand\A{$A$\xspace} % recurring type A
\newcommand\B{$B$\xspace} % recurring type B
\newcommand{\reducedstrut}{\vrule width 0pt height .9\ht\strutbox depth .9\dp\strutbox\relax} % for \codediff and \codeauto
\newcommand{\codediff}[1]{%
  \begingroup
  \setlength{\fboxsep}{0pt}%
  \colorbox{orange!25}{\reducedstrut#1\/}%
  \endgroup
} % to highlight the difference between two code blocks
\newcommand{\codesima}[1]{%
  \begingroup
  \setlength{\fboxsep}{0pt}%
  \colorbox{orange!25}{\reducedstrut#1\raisebox{0.8ex}{\scalebox{0.66}{2}}\/}%
  \endgroup
} % to highlight the similarities between two code blocks
\renewcommand{\textsuperscript}[1]{}
\newcommand{\codesimb}[1]{%
  \begingroup
  \setlength{\fboxsep}{0pt}%
  \colorbox{red!25}{\reducedstrut#1\raisebox{0.8ex}{\scalebox{0.66}{1}}\/}%
  \endgroup
} % to highlight the similarities between two code blocks
\newcommand{\codesimc}[1]{%
  \begingroup
  \setlength{\fboxsep}{0pt}%
  \colorbox{yellow!25}{\reducedstrut#1\raisebox{0.8ex}{\scalebox{0.66}{3}}\/}%
  \endgroup
} % to highlight the similarities between two code blocks
\newcommand{\codesimd}[1]{%
  \begingroup
  \setlength{\fboxsep}{0pt}%
  \colorbox{green!25}{\reducedstrut#1\raisebox{0.8ex}{\scalebox{0.66}{4}}\/}%
  \endgroup
} % to highlight the similarities between two code blocks
\newcommand{\codesime}[1]{%
  \begingroup
  \setlength{\fboxsep}{0pt}%
  \colorbox{blue!25}{\reducedstrut#1\raisebox{0.8ex}{\scalebox{0.66}{5}}\/}%
  \endgroup
} % to highlight the similarities between two code blocks
\newcommand{\codeauto}[1]{%
  \begingroup
  \setlength{\fboxsep}{0pt}%
  \colorbox{cyan!30}{\reducedstrut#1\/}%
  \endgroup
} % to highlight automatically-generated terms
\newcommand{\mysubsubsec}[1]{\vspace{0.40em} \noindent {{\textbf{#1.}}}}

% https://tex.stackexchange.com/questions/7032/good-way-to-make-textcircled-numbers
\newcommand*\circled[1]{\tikz[baseline=(char.base)]{
            \node[shape=circle,draw,inner sep=0.5pt] (char) {#1};}}

% https://tex.stackexchange.com/questions/58901/something-between-frownie-and-smiley/
\DeclareRobustCommand\good{\tikz[baseline=(char.base)]{
    \draw circle (1.6mm);
\node[fill,circle,inner sep=0.5pt] (left eye) at (135:0.8mm) {};
\node[fill,circle,inner sep=0.5pt] (right eye) at (45:0.8mm) {};
\draw (-145:0.8mm) arc (-145:-35:0.8mm);
}}

\DeclareRobustCommand\ok{\tikz[baseline=(char.base)]{
    \draw circle (1.6mm);
\node[fill,circle,inner sep=0.5pt] (left eye) at (135:0.8mm) {};
\node[fill,circle,inner sep=0.5pt] (right eye) at (45:0.8mm) {};
\draw (-135:0.9mm) -- (-45:0.9mm);
}}

\DeclareRobustCommand\bad{\tikz[baseline=(char.base)]{
    \draw circle (1.6mm);
\node[fill,circle,inner sep=0.5pt] (left eye) at (135:0.8mm) {};
\node[fill,circle,inner sep=0.5pt] (right eye) at (45:0.8mm) {};
\draw (-135:0.9mm) arc (145:35:0.8mm);
}}

\begin{document}

%% Title information
\title[]{Proof Repair Across Type Equivalences}         %% [Short Title] is optional;
                                        %% when present, will be used in
                                        %% header instead of Full Title.
%\titlenote{}             %% \titlenote is optional;
                                        %% can be repeated if necessary;
                                        %% contents suppressed with 'anonymous'
%\subtitle{Subtitle}                     %% \subtitle is optional
%\subtitlenote{with subtitle note}       %% \subtitlenote is optional;
                                        %% can be repeated if necessary;
                                        %% contents suppressed with 'anonymous'


%% Author information
%% Contents and number of authors suppressed with 'anonymous'.
%% Each author should be introduced by \author, followed by
%% \authornote (optional), \orcid (optional), \affiliation, and
%% \email.
%% An author may have multiple affiliations and/or emails; repeat the
%% appropriate command.
%% Many elements are not rendered, but should be provided for metadata
%% extraction tools.

%% Author with single affiliation.
\author{Talia Ringer}

\affiliation{
  \institution{University of Washington}  
  \country{USA}                    %% \country is recommended
}
\email{tringer@cs.washington.edu}          %% \email is recommended

%% Author with two affiliations and emails.
\author{RanDair Porter}

\affiliation{
  \institution{University of Washington}  
  \country{USA}                    %% \country is recommended
}
\email{randair@uw.edu}         %% \email is recommended

\author{Nathaniel Yazdani}

\affiliation{
  \institution{Northeastern University}  
  \country{USA}                    %% \country is recommended
}
\email{yazdani.n@husky.neu.edu}         %% \email is recommended

\author{John Leo}

\affiliation{
  \institution{Halfaya Research}  
  \country{USA}                    %% \country is recommended
}
\email{leo@halfaya.org}         %% \email is recommended

\author{Dan Grossman}

\affiliation{
  \institution{University of Washington}  
  \country{USA}                    %% \country is recommended
}
\email{djg@cs.washington.edu}         %% \email is recommended

%% Abstract
%% Note: \begin{abstract}...\end{abstract} environment must come
%% before \maketitle command
\begin{abstract}
We describe a new approach to automatically repairing broken proofs in the Coq proof assistant in response to changes in types.
Our approach combines a configurable proof term transformation with a decompiler from proof terms to suggested tactic scripts.
The proof term transformation implements transport across equivalences in a way that removes references to the old version of the changed type and does not rely on axioms beyond those Coq assumes.

We have implemented this approach in \toolname, an extension to the \textsc{Pumpkin Patch} Coq plugin suite for proof repair.
We demonstrate \toolname's flexibility on eight case studies,
including supporting a benchmark from a user study,
easing development with dependent types,
porting functions and proofs between unary and binary numbers,
and supporting an industrial proof engineer to interoperate between Coq and other verification tools more easily.
\end{abstract}

%% 2012 ACM Computing Classification System (CSS) concepts
%% Generate at 'http://dl.acm.org/ccs/ccs.cfm'.
\begin{CCSXML}
<ccs2012>
<concept>
<concept_id>10011007.10011006.10011008</concept_id>
<concept_desc>Software and its engineering~General programming languages</concept_desc>
<concept_significance>500</concept_significance>
</concept>
<concept>
<concept_id>10003456.10003457.10003521.10003525</concept_id>
<concept_desc>Social and professional topics~History of programming languages</concept_desc>
<concept_significance>300</concept_significance>
</concept>
</ccs2012>
\end{CCSXML}

\ccsdesc[500]{Software and its engineering~General programming languages}
\ccsdesc[300]{Social and professional topics~History of programming languages}
%% End of generated code


%% Keywords
%% comma separated list
\keywords{proof engineering, proof repair, proof reuse}  %% \keywords are mandatory in final camera-ready submission


%% \maketitle
%% Note: \maketitle command must come after title commands, author
%% commands, abstract environment, Computing Classification System
%% environment and commands, and keywords command.
\maketitle

%% Body
\chapter{Introduction}

% Current intro sources, mixed together:
% - some new content
% - research statement
% - QED at Large (intro and why proof engineering matters)
% - PUMPKIN Pi
% TODO somewhere, clarify scope of verification here

% TODO can say ``see survey paper''

What would it take to empower programmers of all skill levels across all domains to formally prove
the absence of costly or dangerous bugs in software systems---that is, to formally \textit{verify} them?

Verification has already come a long way toward this since its inception,
especially when it comes to the scale of systems that can be verified.
The seL4~\cite{Klein2009} verified operating system (OS) microkernel, for example,
is the effort of a team of proof engineers spanning more than
a million lines of proof, costing over 20 person-years.
Given a famous 1977 critique of verification~\cite{DeMillo1977} (emphasis mine):

\begin{quote}
\textit{A sufficiently fanatical researcher}
might be willing to devote \textit{two or 
three years} to verifying a significant 
piece of software if he could be 
assured that the software would remain stable.
\end{quote}
I could argue that, over 40 years, either verification has become easier,
or researchers have become more fanatical. Unfortunately, not all has changed (emphasis still mine):

\begin{quote}
But real-life programs need to 
be maintained and modified. 
There is \textit{no reason to believe} that verifying a modified program is any 
easier than verifying the original the 
first time around.
\end{quote}
As we will soon see, this remains so difficult that sometimes, even experts give up in the face of change. % TODO cite section

This thesis aims to change that by taking advantage of a missed opportunity: tools for developing verified systems
have no understanding of how these systems evolve over time, so they miss out on crucial information.
This thesis introduces a new class of verification tools called \textit{proof repair} tools.
Proof repair tools understand how software systems evolve, and use the crucial information that evolution carries
to automatically evolve proofs about those systems.
This gives us reason to believe.

\section{Thesis}

% TODO segue sentence
Proof repair falls under the umbrella of \textit{proof engineering}: the technologies that make it easier
to develop and maintain verified systems.
Much like software engineering scales programming to large systems, so proof engineering scales verification to large systems. 
In recent years, proof engineers have verified OS microkernels~\cite{Klein2009, Klein2014micro}, machine learning systems~\cite{TODO}, distributed systems~\cite{TODO}, constraint solvers~\cite{TODO}, web browser kernels~\cite{TODO}, compilers~\cite{Leroy:POPL06, Leroy2009}, file systems~\cite{TODO}, and even a quantum optimizer~\cite{TODO}.
As we will soon see, practitioners have found these verified systems to be more robust and secure in deployment. % TODO cite section
% TODO maybe include the Why Proof Engineering Matters from QED at Large in Chapter 2 somewhere. Maybe move this too.

Proof engineering focuses in particular on verified systems that have been
developed using special tools called \textit{proof assistants} or interactive theorem provers (ITPs).
Examples of proof assistants include Coq~\cite{coq}, Isabelle/HOL~\cite{isabelle}, 
HOL Light~\cite{hollight}, and Lean~\cite{lean}.
The proof assistant that I focus on in this thesis will be the Coq proof assistant.
A discussion of how this work carries over to other proof assistants is in Section~\ref{sec:related}.

To develop a verified system using a proof assistant like Coq, the proof engineer does three things:

\begin{enumerate}
\item implements a program using a functional programming language,
\item specifies what it means for the program to be correct, and
\item proves that the program satisfies the specification.
\end{enumerate}
This proof assistant then automatically checks this proof with a small trusted part of its system~\cite{Barendregt2002,Barendregt2351}.
If the proof is correct, then the program satisfies its specification---it is \textit{verified}.

Proof repair automatically fixes broken proofs in response to changes in programs and specifications.
For example, a proof engineer who optimizes an algorithm may change the program, but not the specification; a proof engineer who adapts an OS to new hardware may change both. Even a small change to a program or specification can break many proofs, especially in large systems.
Changing a verified library, for example, can break proofs about programs that depend on that library---and those breaking changes
can be outside of the proof engineers' control.

Proof repair views these broken proofs as bugs that a tool can patch.
In doing so, it shows that there \textit{is} reason to believe that verifying a modified system should often, in practical use cases, be easier than verifying the original the first time around,
even when proof engineers do not follow good development processes,
or when change occurs outside of proof engineers' control.
More formally:

\begin{quote}
\textbf{Thesis}: Changes in programs, specifications, and proofs carry information that a tool can extract, generalize, and apply to fix other proofs broken by the same change. A tool that automates this can save work for proof engineers relative to reference manual repairs in practical use cases.
\end{quote}

\section{Approach}

\begin{figure}
\caption{TODO}
\label{fig:workflow}
\end{figure}

% TODO here, say: we assume Coq in here, see survey paper for more, tactics versus no tactics, de bruijn, where Coq is,
% many ideas generalize but eh, doesn't adapt exactly as-is.
% then expand on that in related work or conclusion

% TODO the amount of background: in reading guide, give resources for learning Coq in more detail, but don't go into too much Coq detail here

The way that proof engineers typically write proofs can obfuscate the information that changes in programs, specifications, and proofs carry.
The typical proof engineering workflow in Coq is interactive: The proof engineer passes Coq high-level
search procedures called \textit{tactics} (like \lstinline{induction}), and Coq responds to each tactic
by refining the current goal to some subgoal (like the proof obligation for the base case). This loop of tactics and goals 
continues until no goals remain, at which point the proof engineer has constructed a high-level sequence of tactics called a \textit{proof script}.
To check the proof, the proof assistant compiles it down to a low-level representation called a \textit{proof term},
then checks that the proof term has the expected type.
Figure~\ref{fig:workflow} illustrates this workflow.

The high-level language of tactics can abstract away important details that a proof repair tool needs,
but the low-level language of proof terms can be brittle and challenging to work with.
Crucially, though, the low-level language comes with lots of structure and strong guarantees.
My approach to proof repair works in the low-level language to take advantage of that.
It then builds back up to the high-level language in the end.

By working at the low-level language, it is able to systematically and with strong guarantees extract and generalize the 
information that breaking changes carry,
then apply those changes to fix other proofs broken by the same change.
But by later building up to the high-level language,
it can in the end produce proofs that integrate more naturally with proof engineering workflows.

This works using a combination of semantic differencing and program transformations in this low-level language.
In particular, it uses a semantic differencing algorithm to extract information from a breaking change in a program, specification, or proof.
It then combines that with program transformations to generalize and, in some cases, apply that information to fix other 
proofs broken by the same change.
In the end, it uses a prototype decompiler to get from the low-level language back up to the high-level language,
so that proof engineers can continue to work in that language going forward. % TODO lol this sucks but whatever it's a first draft

\section{Results}

The technical results of this thesis are threefold:

\begin{enumerate}
\item the \textbf{design} of differencing algorithms \& program transformations for proof repair, 
\item an \textbf{implementation} of a proof repair tool suite, and
\item \textbf{case studies} to evaluate the tool suite on real proof repair scenarios.
\end{enumerate}
Viewing the thesis statement as a theorem, the proof is as follows: % TODO link to particular sections here

\begin{quote}
\textbf{Thesis Proof}: Changes in programs, specifications, and proofs carry information that a tool can extract, generalize, and apply to fix other proofs broken by the same change (by \textbf{design} and \textbf{implementation}). A tool that automates this can save work for proof engineers relative to reference manual repairs in practical use cases (by \textbf{case studies}).
\end{quote}

\paragraph{Design}
The design describes semantic differencing algorithms to extract information from breaking changes in verified systems,
along with proof term transformations to generalize and, in some cases, apply the information to fix proofs broken by the change. % TODO redundant but shitty first draft text so whatever
The semantic differencing algorithms compare the old and new versions of a changed term or type,
and from that find a diff that describes that information corresponding to that change;
the transformations then use that diff to transform some term to a more general fix.
The details vary by the class of change supported.
These design is guided heavily by foundational developments in depenedent type theory;
the theory is sprinkle throughout as appropriate. % TODO may want to list or link to sections here
More details including limitations are in the corresponding chapters. % TODO link to those chapters

% TODO move these to correpsonding sections
\iffalse
I summarize these below; more details including limitations are in the corresponding chapters.

\paragraph{Proof Repair by Example} % TODO names of classifiers below suck, but whatever fix later
Chapter~\ref{ch:example} describes semantic differencing algorithms and proof term transformations for example-based proof repair.  % TODO more specific section here would be good, and onward
The corresponding differencing algorithms run in response to a breaking change in the content of a program or specification. % TODO clarify ``certain''?
They look at the difference between the old and new versions of a single example patched proof in response to that change.
When they succeed, the diff that they find is a list of \textit{patch candidates}: functions that map back and forth between the old and new versions
of the example patch proof in a particular context, but do not yet generalize to other contexts.
The corresponding proof term transformations then try to transform each of those candidate patch functions to a more general \textit{reusable proof patch}:
a function that can be applied with traditional proof automation to fix other proofs broken by the same change.
In summary, proof repair by example has this profile:

\begin{itemize}
\item \textbf{class}: changes in the content of a program or specification 
\item \textbf{diff}: a list of patch candidates
\item \textbf{source}: a single example patched proof
\item \textbf{term}: a single patch candidate
\item \textbf{repair}: a reusable proof patch
\end{itemize}
Note that proof repair by example does not take care of the application step---this is handled using traditional proof automation in Coq.
\fi

\iffalse
\paragraph{Proof Repair Across Type Equivalences}
Chapter~\ref{chapt:pi} describes semantic differencing algorithms and proof term transformations for proof repair in response
to changes that can be described by a type equivalence. % TODO hand-wavy; cite
The corresponding differencing algorithms run in response to a breaking change in the structure of a datatype.
They look at the difference between the old and new versions of that datatype.
When they succeed, the diff that they find is a \textit{type equivalence}: a pair of functions that map back and forth between the
two versions of the datatype (possibly with some additional information) and are mutual inverses.
The corresponding proof term transformation then transforms a proof term defined over the old version of the dataype
directly to a proof term defined over the new version of the datatype.
In summary, proof repair across type equivalence has this profile:

\begin{itemize}
\item \textbf{class}: changes in the structure of a datatype
\item \textbf{diff}: a type equivalence
\item \textbf{source}: the changed datatype
\item \textbf{term}: a proof term defined over the old version of the datatype
\item \textbf{repair}: a proof term defined over the new version of the datatype
\end{itemize}
\fi

\paragraph{Implementation}
The implementation shows that in fact \textit{a tool} can extract and generalize the information that changes carry,
and then apply that information to fix other proofs broken by the same change.
This implementation comes in the form of a proof repair tool suite for Coq called \sysnamelong (Proof Updater Mechanically Passing Knowledge Into New Proofs, Assisting the Coq Hacker).
\sysnamelong implements two kinds of proof repair: proof repair by example (Chapter~\ref{ch:example}) % TODO implementation chapters instead
and proof repair across type equivalences (Chapter~\ref{chapt:pi}). % 8.9.1 
Notably, since all repairs that \sysnamelong produces are checked Coq in the end, \sysnamelong does not extend \textit{Trusted Computing Base} (TCB):
the set of unverified components that the correctnes of the proof development depends on~\cite{TODO}. % TODO not just cite by also make this way less shitty of a definition
In total, \sysnamelong is about 15000 lines of code implemented in OCaml.
These 15000 lines of code consist of three plugins and a library,
which together bridge the gap between the theory supported by design and the practical proof repair needed for the case studies.
Toward that end, five notable features include:

\begin{enumerate}
\item a preprocessing tool to support features in the implementation language missing from the theory,
\item a prototype decompiler from proof terms to proof scripts for better workflow integration,
\item optimizations for efficiency,
\item meaningful error messages for usability, and
\item additional automation for applying patches. 
\end{enumerate}
More details and other features are in the corresponding chapters. % TODO link to those chapters

\paragraph{Case Studies}
The case studies show that \sysnamelong can save work for proof engineers relative to reference manual repairs in practical use cases. % TODO link to corresponding chapters instead
(This paragraph is not done yet, but not really needed to understand flow.)

% Dan: It might be good either in 1 or in both 3 and 4 to lay out the key /results/ -- you have _design_ of the transformation, 
% the highly non-trivial _implementation_ in Coq in a way that doesn't extend the TCB, and you have substantial case studies to _evaluate_.
% You have a bit of formalism and metatheory, but that is more for exposition than for specific results (am I right?) -- you are 
% guided by the theory to build tools that actually work in this unforgiving domain -- "hacks won't work".

\section{Reading Guide}
\label{sec:guide}

% Dan: Instead, I strongly advocate an explicit section in Chapter 1 that lays this all out for the reader: what prior publications are being leaned on, where does text from those reappear, and most importantly, where -- in explicit sections of forthcoming chapters -- are expanded explanations and additional data.  Voila, your thesis is now both a self-contained coherent document and a useful 'appendix' for people who have already read large parts of what is to follow.

This thesis assumes some background in \kl{proof engineering}, type theory, and (to a lesser extent) the \kl{Coq} \kl{proof assistant}. 
I strongly encourage readers of all backgrounds who would like more context to better understand this thesis
look to my survey paper on proof engineering~\cite{PGL-045}, which includes a detailed list of resources
and is available for free on my website: \url{https://dependenttyp.es}.

I recommend that readers with less background on proof engineering, dependent type theory, or the Coq proof assistant
take time to digest Chapter~\ref{chapt:mot} before moving on---though I recommend that even Coq experts read Chapter~\ref{chapt:mot}!
Chapters~\ref{ch:example} and~\ref{chapt:pi} get rather technical, so it is normal not to understand every detail,
though you may always contact me with questions.

\subsection*{Previously Published Material}

While this thesis is self-contained, it centers material from two previously published papers:

\begin{itemize}
\item \textbf{Talia Ringer}, Nathaniel Yazdani, John Leo, and Dan Grossman. \intro{Adapting Proof Automation to Adapt Proofs}~\cite{ringer2018adapting}. CPP 2018.
\item \textbf{Talia Ringer}, RanDair Porter, Nathaniel Yazdani, John Leo, and Dan Grossman. \intro{Proof Repair Across Type Equivalences}~\cite{Ringer2021}. PLDI 2021.
\end{itemize}
It also includes material from three other papers:

\begin{itemize}
\item \textbf{Talia Ringer}, Nathaniel Yazdani, John Leo, and Dan Grossman. \intro{Ornaments for Proof Reuse in Coq}~\cite{Ringer2019}. ITP 2019.
\item \textbf{Talia Ringer}, Alex Sanchez-Stern, Dan Grossman, and Sorin Lerner. \intro{\textsc{REPLica}: REPL Instrumentation for Coq Analysis}~\cite{replica}. CPP 2020.
\item \textbf{Talia Ringer}, Karl Palmskog, Ilya Sergey, Milos Gligoric, and Zachary Tatlock. \intro{QED at Large: A Survey of Engineering of Formally Verified Software}~\cite{PGL-045}. Foundations and Trends® in Programming Languages: Vol. 5: No. 2-3, pp 102-281. 2019. 
\end{itemize}
Below is a map from each of these papers to corresponding sections,
along with an explanation of what is new in this thesis and what is omitted.
All of these papers can be found for free on my website.

\paragraph{Adapting Proof Automation to Adapt Proofs}
The bulk of Chapter~\ref{ch:example} comes from this paper,
though the content is significantly reorganized and reframed.
The introducton and conclusion of Chapter~\ref{ch:example} are fresh content.
Sections~\ref{sec:pumpkin-approach}, \ref{sec:pumpkin-diff}, \ref{sec:pumpkin-trans}, and~\ref{sec:pumpkin-impl}
all include additions and elaborations not found in the original paper.
Chapter~\ref{sec:related} includes some related work from this paper,
and Chapter~\ref{chapt:conclusions} includes some future work from this paper.

\paragraph{Proof Repair Across Type Equivalences}
Parts of the introduction and Section~\ref{sec:mot-theory} come from this paper.
The bulk of Chapter~\ref{chapt:pi} comes from this paper,
though the content is likewise reorganized and reframed.
The conclusion of Chapter~\ref{chapt:pi} is fresh content.
Sections~\ref{sec:pi-approach}, \ref{sec:pi-diff}, \ref{sec:pi-trans}, and~\ref{sec:pi-implementation}
all include additions and elaborations not found in the original paper.
Chapter~\ref{sec:related} includes some related work from this paper,
and Chapter~\ref{chapt:conclusions} includes some future work from this paper.

\paragraph{Ornaments for Proof Reuse in Coq}
The example from Section~\ref{sec:mot-dev} comes from this paper, though most of the text is new.
Parts of Section~\ref{sec:mot-theory} also come from this paper.
Section~\ref{sec:pi-diff} uses a simplified version of the search algorithm from this paper as an example.
Section~\ref{sec:eval} includes the evaluation from this paper with additional context.
Chapter~\ref{sec:related} includes some related work from this paper.
This thesis retires the name of the tool from this paper (\textsc{Devoid})
and uses the name of the generalized version of the tool from \kl{Proof Repair Across Type Equivalences} (\toolnamec) in its place.

\paragraph{\textsc{REPLica}}
Section~\ref{sec:irl} includes a few samples of this paper, as does the abstract.

\paragraph{QED at Large}
Chapter~\ref{chapt:mot} includes a few samples of this paper.
Chapter~\ref{sec:related} includes a large amount of related work from this paper.

\subsubsection*{Authorship Statements}

The material in this thesis draws on work that I did with
four student and postdoctoral coauthors: \kl{Nathaniel Yazdani}, \kl{RanDair Porter}, \kl{Alex Sanchez-Stern},
and \intro{Karl Palmskog}.
Below is a summary of the contributions of each of those coauthors,
indexed for later reference.
The contributions of my faculty and professional coauthors---\intro{John Leo}, \kl{Dan Grossman}, \kl{Zach Tatlock},
\intro{Ilya Sergey}, \intro{Milos Gligoric}, and \intro{Sorin Lerner}---were of course also extremely valuable:

\paragraph{Nathaniel Yazdani}
I worked with \kl{Nate} starting from when he was an undergraduate student.
Nate contributed conceptually to all three proof repair papers his name appears on,
helped with a number of the case studies,
implemented important features on the critical path to success,
and did some of the writing about his contributions.
Nate's contributions include:

\begin{enumerate}
\item a tool for preprocessing proof developments into a format suitable for repair,
\item higher-order transformations for applying proof term transformations over entire libraries, and
\item a key early insight about equality.
\end{enumerate}
All three of these were necessary to scale proof repair to help real proof engineers in practical scenarios.

\paragraph{RanDair Porter}
\kl{RanDair} joined the project as an undergraduate.
RanDair implemented a prototype decompiler from proof terms to proof scripts,
and wrote a description of the behavior of the decompiler that I built on in the corresponding paper.
This decompiler was necessary for integrating proof repair tools with real proof engineering workflows,
and it continues to inspire exciting work to this day.

\paragraph{Alex Sanchez-Stern}
\kl{Alex} worked with me as a PhD student on a user study of proof engineers during my visit \kl{UCSD}.
Alex designed, implemented, deployed, and evaluated one of the two analyses in the user study paper.
He also helped substantially in building the infrastructure necessary to deploy the user study,
and wrote large sections of the paper.
The user study and paper would not have happened without Alex.

\paragraph{Karl Palmskog}
\kl{Karl} was a postdoctoral researcher when he joined me on the survey paper.
Karl wrote entire chapters of the survey paper.
I could not have written that paper without Karl.

\subsubsection*{Pronouns}

In this thesis, I use ``I'' to refer to work that I did as part of my thesis work,
even though of course no work happens in a vacuum.
I use the names of my coauthors like ``\kl{Nate}'' or ``\kl{RanDair}'' when referring to work that my coauthors did,
when I was operating primarily in an advisory role.
When I collaborated with my coauthors, I name them and myself, like ``\kl{Nate} and I,''
and then (when not ambiguous) I use ``we'' thereafter.
Throughout, I also use mathematical ``we'' to mean both myself and the reader.

When I discuss a rhetorical proof engineer who does not actually exist,
like ``the proof engineer,'' I always use ``she''---this is a small attempt
to seed the world with data that counteracts stereotypes. 
When preserving anonymity of a particular person, I always use singular ``they.''
Otherwise, I use the pronouns that the person prefers.





\section{A Simple Motivating Example}
\label{sec:overview}

Consider a simple example of using \toolnamec: repairing proofs after swapping the list constructors (Figure~\ref{fig:listswap}).
This is inspired by a similar change from a user study of proof engineers (Section~\ref{sec:search}).
Even such a simple change can cause trouble, as in this proof from the Coq standard library:\footnote{We use induction instead of pattern matching.}

\begin{lstlisting}
Lemma rev_app_distr {A} :(@\vspace{-0.04cm}@)
  $\forall$ (x y : list A), rev (x ++ y) = rev y ++ rev x.(@\vspace{-0.04cm}@)
Proof.(@\vspace{-0.04cm}@)
  induction x as [| a l IHl].(@\vspace{-0.04cm}@)
  induction y as [| a l IHl].(@\vspace{-0.04cm}@)
  simpl. auto.(@\vspace{-0.04cm}@)
  simpl. rewrite app_nil_r; auto.(@\vspace{-0.04cm}@)
  intro y. simpl.(@\vspace{-0.04cm}@)
  rewrite (IHl y). rewrite app_assoc; trivial.(@\vspace{-0.04cm}@)
Qed.(@\vspace{-0.05cm}@)
\end{lstlisting}
This theorem says that appending (\lstinline{++}) two lists and reversing (\lstinline{rev}) the result behaves the same as appending
the reverse of the second list onto the reverse of the first list.
When we change the \lstinline{list} type, the proof no longer works.
%This theorem statement \lstinline{rev_app_distr} defined over the old version of \lstinline{list} is our \textit{old specification}.
%When we change the \lstinline{list} type, we get the \textit{new specification}.
%But the \textit{old proof} or tactic script no longer works with this new specification.
To repair this proof with \toolnamec, we run this command:

\begin{figure}
\begin{minipage}{0.46\textwidth}
   \lstinputlisting[firstline=1, lastline=3]{often/pi/listswap.tex}
\end{minipage}
\hfill
\begin{minipage}{0.46\textwidth}
   \lstinputlisting[firstline=5, lastline=7]{often/pi/listswap.tex}
\end{minipage}
\vspace{-0.4cm}
\caption{The updated \lstinline{list} (bottom) is the old \lstinline{list} (top) with its two constructors swapped (\codediff{orange}).}
\label{fig:listswap}
\end{figure}

\iffalse
\begin{figure*}
\begin{minipage}{0.46\textwidth}
   \lstinputlisting[firstline=1, lastline=3]{often/pi/listswap.tex}
\end{minipage}
\hfill
\begin{minipage}{0.46\textwidth}
   \lstinputlisting[firstline=5, lastline=7]{often/pi/listswap.tex}
\end{minipage}
\vspace{-0.4cm}
\caption{The updated \lstinline{list} (right) is the old \lstinline{list} (left) with its two constructors swapped (\codediff{orange}).}
\label{fig:listswap}
\end{figure*}
\fi

\begin{figure*}
\codeauto{%
\begin{minipage}{0.48\textwidth}
\lstinputlisting[firstline=1, lastline=13]{often/pi/equivproof.tex}
\end{minipage}
\hfill
\begin{minipage}{0.48\textwidth}
\lstinputlisting[firstline=15, lastline=28]{often/pi/equivproof.tex}
\end{minipage}}
\vspace{-0.3cm}
\caption{Two functions between \lstinline{Old.list} and \lstinline{New.list} (top) that form an equivalence (bottom).}
\label{fig:equivalence}
\end{figure*}

\begin{lstlisting}
Repair Old.list New.list in rev_app_distr.(@\vspace{-0.05cm}@)
\end{lstlisting}
assuming the old and new list types from Figure~\ref{fig:listswap} are in modules \lstinline{Old} and \lstinline{New}.
This suggests a proof script that succeeds (in $\codeauto{light blue}$ to denote \toolnamec produces it automatically):

\begin{lstlisting}[backgroundcolor=\color{cyan!30}]
Proof.(@\vspace{-0.04cm}@)
  intros x. induction x as [a l IHl| ]; intro y0.(@\vspace{-0.04cm}@)
  - simpl. rewrite IHl. simpl.(@\vspace{-0.04cm}@)
    rewrite app_assoc. auto.(@\vspace{-0.04cm}@)
  - induction y0 as [a l H| ].(@\vspace{-0.04cm}@)
    + simpl. rewrite app_nil_r. auto.(@\vspace{-0.04cm}@)
    + auto.(@\vspace{-0.04cm}@)
Qed.(@\vspace{-0.05cm}@)
\end{lstlisting}
where the dependencies (\lstinline{rev}, \lstinline{++}, \lstinline{app_assoc}, and \lstinline{app_nil_r}) have
also been updated automatically~\href{https://github.com/uwplse/pumpkin-pi/blob/silent/plugin/coq/Swap.v}{\circled{1}}. % Swap.v
If we would like, we can manually modify this to something that more closely matches the style of the original proof:

\begin{lstlisting}
Proof.(@\vspace{-0.04cm}@)
  induction x as [a l IHl|].(@\vspace{-0.04cm}@)
  intro y. simpl.(@\vspace{-0.04cm}@)
  rewrite (IHl y). rewrite app_assoc; trivial.(@\vspace{-0.04cm}@)
  induction y as [a l IHl|].(@\vspace{-0.04cm}@)
  simpl. rewrite app_nil_r; auto.(@\vspace{-0.04cm}@)
  simpl. auto.(@\vspace{-0.04cm}@)
Qed.(@\vspace{-0.04cm}@)
\end{lstlisting}
We can even repair the entire list module from the Coq standard library all at once by running the \lstinline{Repair module}
command~\href{https://github.com/uwplse/pumpkin-pi/blob/silent/plugin/coq/Swap.v}{\circled{1}}. % Swap.v
When we are done, we can get rid of \lstinline{Old.list}. % entirely.

The key to success is taking advantage of Coq's structured proof term language:
Coq compiles every proof script to a proof term in a language called Gallina that is based on the calculus of inductive 
constructions---\toolnamec repairs that term.
\toolnamec then decompiles the repaired proof term (with optional hints from the original proof script) back 
to a proof script that the proof engineer can maintain.
%Here, \toolnamec transforms the proof term Coq compiles \lstinline{rev_app_distr} to,
%and then decompiles that transformed proof term to the proof script in light blue.

In contrast, updating the poorly structured proof script directly would not be straightforward.
Even for the simple proof script above, grouping tactics by line, there are $6! = 720$ permutations of this proof script.
It is not clear which lines to swap since these tactics do not have a semantics beyond the searches their evaluation performs.
Furthermore, just swapping lines is not enough: even for such a simple change, we must also swap
arguments, so \lstinline{induction x as [| a l IHl]} becomes \lstinline{induction x as [a l IHl|]}.
%Handling even swapping constructors this way would require a search procedure that would not generalize to other changes.
Valentin Robert's thesis~\cite{robert2018} describes the challenges of repairing tactics in detail.
\toolnamec's approach circumvents this challenge.

%\toolnamec's approach circumvents this challenge. % by transforming proof terms, then decompiling the transformed proof term to a tactic script.
%By decompiling the transformed proof term, \toolnamec is able to suggest a tactic script in the end.
%As later sections show, this approach is much more general than just permuting constructors.




\section{Problem Definition}
\label{sec:key1}

\toolname can do much more than permute constructors. % approach is more general than just permuting constructors:
Given an equivalence between types \A and \B,
\toolname repairs functions and proofs defined over \A to instead refer to \B (Section~\ref{sec:scope}).
It does this in a way that allows for removing references to \A, which is essential for proof repair,
since \A may be an old version of an updated type (Section~\ref{sec:repair}).

 %We can view proof repair as a form of 
%\textit{proof reuse}~\cite{Ringer2019, felty1994generalization, caplan1995logical, pons2000generalization, johnsen2004theorem}, % TODO consider citation list
%or reusing proofs about one specification (say, from another library, or from within the same proof development)
%to derive proofs about another specification.
%The difference is that in standard proof reuse, both of these specifications continue to exist.
%In contrast, proof repair is the process of reusing proofs across \textit{two versions of a single specification},
%only one of which---the new version---must continue to exist.
%That is, the old version of the specification may be removed after updating proofs to use the new version (Section~\ref{sec:repair}).
%The key to supporting proof repair is to build a proof reuse tool that can handle that additional challenge (Section~\ref{sec:time}).

%\begin{quote}
%\textbf{Insight 1}:
%Proof repair is a form of proof reuse---reusing proofs about one specification to derive proofs about another specification---with 
%the additional challenge that one of the specifications may cease to exist (Section~\ref{sec:repair}).
%The key to supporting proof repair is to build a proof reuse
%tool that can handle that additional challenge (Section~\ref{sec:time}).
%\end{quote}

\subsection{Scope: Type Equivalences}
\label{sec:scope}

\toolname repairs proofs in response to changes in types that correspond to \textit{type equivalences}~\cite{univalent2013homotopy},
or pairs of functions that map between two types and are mutual inverses.\footnote{The adjoint follows, and \toolname includes machinery to prove it~\href{https://github.com/uwplse/pumpkin-pi/blob/v2.0.0/plugin/src/automation/search/equivalence.ml}{\circled{10}}~\href{https://github.com/uwplse/pumpkin-pi/blob/v2.0.0/plugin/theories/Adjoint.v}{\circled{23}}.}
When a type equivalence between types \A and \B exists, those types are \textit{equivalent} (denoted \A $\simeq$ \B). % for example:
Figure~\ref{fig:equivalence} shows a type equivalence between the two versions of \lstinline{list}
from Figure~\ref{fig:listswap} that \toolname discovered and proved automatically~\href{https://github.com/uwplse/pumpkin-pi/blob/v2.0.0/plugin/coq/Swap.v}{\circled{1}}.

%
%\begin{lstlisting}
%Old.list $\simeq$ New.list
%\end{lstlisting}

To give some intuition for what kinds of changes can be described by equivalences, we preview two changes below.
See Table~\ref{fig:changes} on page~\pageref{fig:changes} for more examples.

\mysubsubsec{Factoring out Constructors}
Consider changing the type \lstinline{I} to the type \lstinline{J} 
in Figure~\ref{fig:equivalence2}.
\lstinline{J} can be viewed as \lstinline{I} with its two constructors \lstinline{A} and \lstinline{B} pulled out to a
new argument of type \lstinline{bool} for a single constructor.
With \toolname, the proof engineer can repair functions and proofs about \lstinline{I} to instead use \lstinline{J},
as long as she configures \toolname to describe which constructor 
of \lstinline{I} maps to \lstinline{true} and which maps to \lstinline{false}.
This information about constructor mappings induces an equivalence \lstinline{I }$\simeq$\lstinline{ J}
across which \toolname repairs functions and proofs.
File \href{https://github.com/uwplse/pumpkin-pi/blob/v2.0.0/plugin/coq/playground/constr_refactor.v}{\circled{2}} shows an example of this, mapping \lstinline{A} to \lstinline{true} and \lstinline{B} to false,
and repairing proofs of De Morgan's laws. % constr_refactor.v
%
%It uses \toolname to automatically repair functions and proofs over \lstinline{I}, like:

%\begin{lstlisting}
%Theorem demorgan_1 : $\forall$ (i1 i2 : I),(@\vspace{-0.04cm}@)
%  neg (and i1 i2) = or (neg i1) (neg i2).(@\vspace{-0.04cm}@)
%Proof.(@\vspace{-0.04cm}@)
%  intros i1 i2.(@\vspace{-0.04cm}@)
%  induction i1; auto.(@\vspace{-0.04cm}@)
%Qed.
%\end{lstlisting}
%to corresponding functions and proofs over \lstinline{J}, like:
%
%\begin{lstlisting}[backgroundcolor=\color{cyan!30}]
%Theorem demorgan_1 : $\forall$ (j1 j2 : J),(@\vspace{-0.04cm}@)
%  neg (and j1 j2) = or (neg j1) (neg j2).(@\vspace{-0.04cm}@)
%Proof.(@\vspace{-0.04cm}@)
%  intros j1 j2.(@\vspace{-0.04cm}@)
%  induction j1 (@\codediff{as [b]. induction b as [ | ]}@); auto.(@\vspace{-0.04cm}@)
%Qed.
%\end{lstlisting}
%These repaired functions and proofs refer to \lstinline{J} in place of \lstinline{I}.
%Otherwise, they behave the same way as the functions and proofs over \lstinline{I} up to the equivalence between
%\lstinline{I} and \lstinline{J}---Section~\ref{sec:repair} explains this intuition more formally.

\mysubsubsec{Adding a Dependent Index}
At first glance, the word \textit{equivalence} may seem to imply that \toolname can support only changes in
which the proof engineer does not add or remove information.
But equivalences are more powerful than they may seem.
%The idea is, when possible, to separate out the new information
%into a projection of a $\Sigma$ type or a constructor of a sum type.
%roofs about this new information become the proof obligation for the proof engineer,
%and \toolname automates the rest.
Consider, for example, changing a list to a length-indexed vector (Figure~\ref{fig:listtovect}).
\toolname can repair functions and proofs about lists to functions and proofs about vectors of particular lengths~\href{https://github.com/uwplse/pumpkin-pi/blob/v2.0.0/plugin/coq/examples/Example.v}{\circled{3}}, % Example.v
since $\Sigma$\lstinline{(l:list T).length l = n }$\simeq$\lstinline{ vector T n}.
From the proof engineer's perspective, after updating specifications from \lstinline{list} to \lstinline{vector},
to fix her functions and proofs, she must additionally prove invariants about the lengths of her lists.
\toolname makes it easy to separate out that proof obligation, then automates the rest.

More generally, in homotopy type theory, with the help of quotient types, it is possible to form an equivalence
from a relation, even when the relation is not an equivalence~\cite{angiuli2020internalizing}.
While Coq lacks quotient types,
it is possible to achieve a similar outcome and use \toolname for changes that add or remove information
when those changes can be expressed as equivalences between $\Sigma$ types or sum types.
%
%(\href{https://github.com/uwplse/pumpkin-pi/blob/master/plugin/coq/examples/Example.v}{\lstinline{Example.v}}).%%
%
%With the proof reuse tool \textsc{Devoid}~\cite{Ringer2019},
%it is possible to repair proofs about lists to proofs about vectors of \textit{some} length, since:
%
%\begin{lstlisting}
%packed_vect T := $\Sigma$(n : nat).vector T n.
%list T $\simeq$ packed_vector T.
%\end{lstlisting}
%This is enough to automatically repair a lemma about lists:
%
%\begin{lstlisting}
%$\forall$ {A B} (l1 : list A) (l2 : list B),(@\vspace{-0.04cm}@)
%  zip_with pair l1 l2 = zip l1 l2.
%\end{lstlisting}
%to a lemma about vectors of some length:
%
%\begin{lstlisting}
%$\forall$ {A B} (l1 : (@\codediff{packed\_vect A}@)) (l2 : (@\codediff{packed\_vect B}@)),(@\vspace{-0.04cm}@)
%  zip_with pair l1 l2 = zip l1 l2.
%\end{lstlisting}
%recursively updating dependencies \lstinline{zip} and \lstinline{zip_with}.
%It is not enough, however, to help the proof engineer get from that to a proof about vectors \textit{of a particular length}:
%
%\begin{lstlisting}
%$\forall$ {A B} (@\codediff{n}@) (l1 : (@\codediff{vector A n}@)) (l2 : (@\codediff{vector B n}@)),(@\vspace{-0.04cm}@)
%  zip_with pair (@\codediff{n}@) l1 l2 = zip (@\codediff{n}@) l1 l2.
%\end{lstlisting}
%\textsc{Devoid} leaves this step to the proof engineer.
%\toolname, in contrast, can handle this step as well (\href{https://github.com/uwplse/pumpkin-pi/blob/master/plugin/coq/examples/Example.v}{\lstinline{Example.v}}).
%The key is to repair functions and proofs across this equivalence:

%Section~\ref{sec:search} shows this and other case studies using \toolname to repair real proofs
%informed by the needs of proof engineers.

\subsection{Goal: Transport with a Twist}
\label{sec:repair}

The goal of \toolname is to implement a kind of proof reuse known as \textit{transport}~\cite{univalent2013homotopy},
but in a way that is suitable for repair.
Informally, transport takes a term $t$ and produces a term $t'$ that is the same as $t$ modulo an equivalence $A \simeq B$.
If $t$ is a function, then $t'$ behaves the same way modulo the equivalence;
if $t$ is a proof, then $t'$ proves the same theorem the same way modulo the equivalence.

When transport across $A \simeq B$ takes $t$ to $t'$,
we say that $t$ and $t'$ are \textit{equal up to transport}
across that equivalence (denoted $t \equiv_{A \simeq B} t'$).\footnote{This notation should be interpreted in a metatheory with \textit{univalence}---a property that Coq lacks---or it should be approximated in Coq.
The details of transport with univalence are in \citet{univalent2013homotopy}, and an approximation in Coq is in \citet{tabareau2017equivalences}. For equivalent \A and \B, there can be many equivalences $A \simeq B$.
Equality up to transport is across a \textit{particular} equivalence, but we erase this in the 
notation.}
In Section~\ref{sec:overview}, the original append function \lstinline{++} over \lstinline{Old.list}
and the repaired append function \lstinline{++} over \lstinline{New.list} that \toolname produces are
equal up to transport across the equivalence from Figure~\ref{fig:equivalence}, since (by \lstinline{app_ok}~\href{https://github.com/uwplse/pumpkin-pi/blob/v2.0.0/plugin/coq/Swap.v}{\circled{1}}):

\begin{lstlisting}
$\forall$ T (l1 l2 : Old.list T),(@\vspace{-0.04cm}@)
  swap T (l1 ++ l2) = (swap T l1) ++ (swap T l2).(@\vspace{-0.05cm}@)
\end{lstlisting}
The original \lstinline{rev_app_distr} is equal to the repaired proof up to transport,
since both prove the same thing the same way up to the equivalence, and up to the changes in \lstinline{++}
and \lstinline{rev}.

Transport typically works by applying the functions that make up the equivalence to convert
inputs and outputs between types.
This approach would not be suitable for repair, since it does not make it possible to remove the old type \A.
\toolname implements transport in a way that allows for removing references to \A---by proof term transformation.

%The goal of a proof repair tool like \toolname is to define a transport method that
%can remove references to the old specification, %rather than converting back and forth like standard transport methods.
%%That way, the proof repair tool can produce proofs that no longer refer in any way to the old specification,
%since the old specification may no longer exist.

%Section~\ref{sec:overview} showed a simple case of this: \toolname
%reused the proof of \lstinline{rev_app_distr} defined over \lstinline{Old.list}
%to generate a new proof of \lstinline{rev_app_distr} defined over equivalent \lstinline{New.list}.
%Furthermore, it did so in a way that removed all references to \lstinline{Old.list} in the proof
%and in its dependencies.
%That way, after calling \lstinline{Repair}, \lstinline{Old.list} could be removed.

%\subsection{A Tool for Proof Repair Across Equivalences}
%\label{sec:time}

\begin{figure}
\begin{minipage}{0.48\columnwidth}
\lstinputlisting[firstline=1, lastline=3]{equiv2.tex}
\end{minipage}
\hfill
\begin{minipage}{0.48\columnwidth}
\lstinputlisting[firstline=5, lastline=7]{equiv2.tex}
\end{minipage}
\vspace{-0.4cm}
\caption{The old type \lstinline{I} (left) is either \lstinline{A} or \lstinline{B}. The new type \lstinline{J} (right) is \lstinline{I} with \lstinline{A} and \lstinline{B} factored out to \lstinline{bool} (\codediff{orange}).}
\label{fig:equivalence2}
\end{figure}

\begin{figure}
\begin{minipage}{0.48\textwidth}
   \lstinputlisting[firstline=1, lastline=3]{listtovect.tex}
\end{minipage}
\hfill
\begin{minipage}{0.58\textwidth}
   \lstinputlisting[firstline=5, lastline=7]{listtovect.tex}
\end{minipage}
\vspace{-0.4cm}
\caption{A vector (bottom) is a list (top) indexed by its length (\codediff{orange}). Vectors effectively make it possible to enforce length invariants about lists at compile time.}
\label{fig:listtovect}
\end{figure}

\iffalse
\begin{figure*}
\begin{minipage}{0.40\textwidth}
   \lstinputlisting[firstline=1, lastline=3]{listtovect.tex}
\end{minipage}
\hfill
\begin{minipage}{0.58\textwidth}
   \lstinputlisting[firstline=5, lastline=7]{listtovect.tex}
\end{minipage}
\vspace{-0.4cm}
\caption{A vector (right) is a list (left) indexed by its length.}
\label{fig:listtovect}
\end{figure*}
\fi

\begin{figure*}
\small
\begin{grammar}
<i> $\in \mathbbm{N}$, <v> $\in$ Vars, <s> $\in$ \{ Prop, Set, Type<i> \}

<t> ::= <v> \hspace{0.06cm} | \hspace{0.06cm} <s> \hspace{0.06cm} | \hspace{0.06cm} $\Pi$ (<v> : <t>) . <t> \hspace{0.06cm} | \hspace{0.06cm} $\lambda$ (<v> : <t>) . <t> \hspace{0.06cm} | \hspace{0.06cm} <t> <t> \hspace{0.06cm} | \hspace{0.06cm} Ind (<v> : <t>)\{<t>,\ldots,<t>\} \hspace{0.06cm} | \hspace{0.06cm} Constr (<i>, <t>) \hspace{0.06cm} | \hspace{0.06cm} Elim(<t>, <t>)\{<t>,\ldots,<t>\}
\end{grammar}
\vspace{-0.3cm}
\caption{Syntax for CIC$_\omega$ from \citet{Timany2015FirstST} with (from left to right) variables, sorts, dependent types, functions, application, inductive types, inductive constructors, and primitive eliminators.}
\label{fig:syntax}
\end{figure*}


\section{The Transformation}
\label{sec:key2}

\begin{figure*}
\begin{minipage}{0.48\textwidth}
\begin{lstlisting}
DepConstr(0, list T) : list T := Constr((@\codediff{0}@), list T).(@\vspace{-0.04cm}@)
DepConstr(1, list T) t l : list T :=(@\vspace{-0.04cm}@)
  Constr ((@\codediff{1}@), list T) t l.(@\vspace{-0.04cm}@)
(@\vspace{-0.14cm}@)
DepElim(l, P) { p$_{\mathtt{nil}}$, p$_{\mathtt{cons}}$ } : P l :=(@\vspace{-0.04cm}@)
  Elim(l, P) { (@\codediff{p$_{\mathtt{nil}}$}@), (@\codediff{p$_{\mathtt{cons}}$}@) }.(@\vspace{-0.04cm}@)
\end{lstlisting}
\end{minipage}
\hfill
\begin{minipage}{0.48\textwidth}
\begin{lstlisting}
DepConstr(0, list T) : list T := Constr((@\codediff{1}@), list T).(@\vspace{-0.04cm}@)
DepConstr(1, list T) t l : list T :=(@\vspace{-0.04cm}@)
  Constr((@\codediff{0}@), list T) t l.(@\vspace{-0.04cm}@)
(@\vspace{-0.14cm}@)
DepElim(l, P) { p$_{\mathtt{nil}}$, p$_{\mathtt{cons}}$ } : P l :=(@\vspace{-0.04cm}@)
  Elim(l, P) { (@\codediff{p$_{\mathtt{cons}}$}@), (@\codediff{p$_{\mathtt{nil}}$}@) }.(@\vspace{-0.04cm}@)
\end{lstlisting}
\end{minipage}
\vspace{-0.3cm}
\caption{The dependent constructors and eliminators for old (left) and new (right) \lstinline{list}, with the difference in \codediff{orange}.}
\vspace{-0.1cm}
\label{fig:listconfig}
\end{figure*}

At the heart of \toolname is a configurable proof term transformation for transporting
proofs across equivalences~\href{https://github.com/uwplse/pumpkin-pi/blob/v2.0.0/plugin/src/automation/lift/lift.ml}{\circled{4}}. % lift.ml
It is a generalization of the transformation from an earlier version of \toolname called
\textsc{Devoid}~\cite{Ringer2019}, which solved this problem a particular class of equivalences.
%\toolname moves the reasoning specific to that class of equivalences into the configuration. 

%\begin{quote}
%\textbf{Insight 2}:
%A configurable proof term transformation can be used to build such a proof repair tool,
%and the result can handle many different kinds of changes.
%\end{quote}

The transformation takes as input a deconstructed equivalence that we call a \textit{configuration}.
This section introduces the configuration (Section~\ref{sec:configurable}),
defines the transformation that builds on that (Section~\ref{sec:generic}),
then specifies correctness criteria for the configuration (Section~\ref{sec:art}).
Section~\ref{sec:implementation} describes the additional work needed to implement this transformation.

\mysubsubsec{Conventions}
All terms that we introduce in this section are in the Calculus of Inductive Constructions (CIC$_{\omega}$), the type theory
that Coq's proof term language Gallina implements.
CIC$_{\omega}$ is based on the Calculus of Constructions (CoC), a variant of the lambda calculus with polymorphism (types that dependent on types) and dependent types (types that depend on terms)~\cite{coquand:inria-00076024}. CIC$_{\omega}$ extends CoC with
inductive types~\cite{inductive}.
Inductive types are defined solely by their constructors (like \lstinline{nil} and \lstinline{cons} for \lstinline{list}) and eliminators (like the induction principle for \lstinline{list}); this section assumes that these eliminators are primitive.

The syntax for CIC$_{\omega}$ with primitive eliminators is in Figure~\ref{fig:syntax}.
The typing rules are standard.
We assume inductive types $\Sigma$ with constructor $\exists$ and projections $\pi_l$ and $\pi_r$,
and an equality type \lstinline{=} with constructor \lstinline{eq_refl}.
We use $\vec{t}$ and $\{t_1, \ldots, t_n\}$ to denote lists of terms.

\subsection{The Configuration}
\label{sec:configurable}

The configuration is the key to building a proof term transformation that implements transport in a way that is suitable for repair.
Each configuration corresponds to an equivalence \A $\simeq$ \B.
It deconstructs the equivalence into things that talk about \A, and things that talk about \B.
It does so in a way that hides details
specific to the equivalence, like the order or number of arguments to an induction principle or type.

At a high level, the configuration helps the transformation achieve two goals: preserve equality up to transport across the equivalence 
between \A and \B, and produce well-typed terms.
This configuration is a pair of pairs:

\begin{lstlisting}
((DepConstr, DepElim), (Eta, Iota))(@\vspace{-0.05cm}@)
\end{lstlisting}
each of which corresponds to one of the two goals:
\lstinline{DepConstr} and \lstinline{DepElim} define how to transform constructors and eliminators, thereby preserving the equivalence, and 
\lstinline{Eta} and \lstinline{Iota} define how to transform $\eta$-expansion and $\iota$-reduction of constructors and eliminators, thereby producing well-typed terms.
Each of these is defined in CIC$_{\omega}$ for each equivalence.
%\textbf{Configure} passes this configuration to \textbf{Transform}.

%Section~\ref{sec:art} describes how the four parts of this configuration must relate to one another in order for the proof
%term transformation to work correctly, and proves that every equivalence induces a configuration.

\mysubsubsec{Preserving the Equivalence}
To preserve the equivalence, the configuration ports terms over \A to terms over \B by viewing each
term of type \B as if it were an \A.
This way, the rest of the transformation can replace values of \A with values of \B, and
inductive proofs about \A with inductive proofs about \B, %, then recursively transform
%subterms 
all without changing the order or number of arguments.

The two configuration parts responsible for this are \lstinline{DepConstr}
and \lstinline{DepElim} (\textit{dependent constructors} and \textit{eliminators}).
These describe how to construct and eliminate \A and \B, wrapping the types with a common inductive structure.
The transformation requires the same number of dependent constructors and cases in dependent eliminators for \A and \B,
even if \A and \B are types with different numbers of constructors
(\A and \B need not even be inductive; see Sections~\ref{sec:art} and~\ref{sec:search}).

For the \lstinline{list} change from Section~\ref{sec:overview},
the configuration that \toolname discovers uses the dependent constructors
and eliminators in Figure~\ref{fig:listconfig}. The dependent constructors for \lstinline{Old.list}
are the normal constructors with the order unchanged,
while the dependent constructors for \lstinline{New.list} swap constructors
back to the original order.
Similarly, the dependent eliminator for \lstinline{Old.list} is the normal eliminator for \lstinline{Old.list},
while the dependent eliminator for \lstinline{New.list} swaps cases.

As the name hints, these constructors and eliminators can be dependent.
Consider the type of vectors of some length:

\begin{lstlisting}
packed_vect T := $\Sigma$(n : nat).vector T n.(@\vspace{-0.05cm}@)
\end{lstlisting}
\toolname can port proofs across the equivalence between this type and \lstinline{list T}~\href{https://github.com/uwplse/pumpkin-pi/blob/v2.0.0/plugin/coq/examples/Example.v}{\circled{3}}. % Example.v
The dependent constructors \toolname discovers pack the index into an existential, like:

\begin{lstlisting}
DepConstr(0, packed_vect) : packed_vect T :=(@\vspace{-0.04cm}@)
  $\exists$ (Constr(0, nat)) (Constr(0, vector T)).(@\vspace{-0.05cm}@)
\end{lstlisting}
and the eliminator it discovers eliminates the projections:

\begin{lstlisting}
DepElim(s, P) { f$_0$ f$_1$ } : P ($\exists$ ($\pi_l$ s) ($\pi_r$ s)) :=(@\vspace{-0.04cm}@)
  Elim($\pi_r$ s, $\lambda$(n : nat)(v : vector T n).P ($\exists$ n v)) {(@\vspace{-0.04cm}@)
    f$_0$,(@\vspace{-0.04cm}@)
    ($\lambda$(t : T)(n : nat)(v : vector T n).f$_1$ t ($\exists$ n v))(@\vspace{-0.04cm}@)
  }.(@\vspace{-0.05cm}@) 
\end{lstlisting}

In both these examples, the interesting work moves into the configuration:
the configuration for the first swaps constructors and cases,
and the configuration for the second maps constructors and cases over \lstinline{list} to constructors and cases over \lstinline{packed_vect}. %packs constructors and eliminates projections.
That way, the transformation need not add, drop, or reorder arguments.
%In essence, all of the difficult work moves into the configuration.
Furthermore, both examples use automatic configuration, so \toolname's \textbf{Configure} component 
discovers \lstinline{DepConstr} and \lstinline{DepElim} from just the types \A and \B, taking care of even the difficult work.

\mysubsubsec{Producing Well-Typed Terms}
The other configuration parts \lstinline{Eta} and \lstinline{Iota} deal with producing well-typed terms,
in particular by transporting equalities.
CIC$_{\omega}$ distinguishes between two important kinds of equality: those that hold by reduction (\textit{definitional} equality), and those that hold by proof (\textit{propositional} equality).
That is, two terms \lstinline{t} and \lstinline{t'} of type \lstinline{T} are definitionally equal if they reduce to the same normal form,
and propositionally equal if there is a proof that \lstinline{t = t'} using the inductive
equality type \lstinline{=} at type \lstinline{T}. Definitionally equal terms are necessarily propositionally equal, but 
the converse is not in general true.

When a datatype changes, sometimes, definitional equalities defined over the old version of that type must become propositional.
A naive proof term transformation may fail to generate well-typed terms if it does not account for this.
Otherwise, if the transformation transforms a term \lstinline{t : T} to some \lstinline{t' : T'}, it does not necessarily
transform \lstinline{T} to \lstinline{T'}~\cite{tabareau2019marriage}.

\lstinline{Eta} and \lstinline{Iota} describe how to transport equalities.
More formally, they define $\eta$-expansion and $\iota$-reduction of \A and \B,
which may be propositional rather than definitional,
and so must be explicit in the transformation.
$\eta$-expansion describes how to expand a term to apply a constructor to an eliminator in a way that preserves propositional equality,
and is important for defining dependent eliminators~\cite{nlab:eta-conversion}.
$\iota$-reduction ($\beta$-reduction for inductive types) describes how to reduce an elimination of a constructor~\cite{nlab:beta-reduction}.

The configuration for the change from \lstinline{list} to \lstinline{packed_vect} has propositional \lstinline{Eta}.
It uses $\eta$-expansion for $\Sigma$:

\begin{lstlisting}
Eta(packed_vect) := $\lambda$(s:packed_vect).$\exists$ ($\pi_l$ s) ($\pi_r$ s).(@\vspace{-0.05cm}@)
\end{lstlisting}
which is propositional and not definitional in Coq.
Thanks to this, we can forego the assumption that our language has primitive projections (definitional $\eta$ for $\Sigma$).

\begin{figure}
\begin{minipage}{0.44\columnwidth}
   \lstinputlisting[firstline=1, lastline=8]{nattobin.tex}
\end{minipage}
\hfill
\begin{minipage}{0.54\columnwidth}
   \lstinputlisting[firstline=10, lastline=17]{nattobin.tex}
\end{minipage}
\vspace{-0.2cm}
\caption{A unary natural number \lstinline{nat} (left) is either zero (\lstinline{0}) or the successor of some other natural number (\lstinline{S}).
A binary natural number \lstinline{N} (right) is either zero (\lstinline{N0}) or a positive binary number (\lstinline{Npos}), where a positive binary number is either 1 (\lstinline{xH}), or the result of shifting left and adding 1 (\lstinline{xI}) or
0 (\lstinline{xO}). Unary and binary natural numbers are equivalent, but have different inductive structures.
Consequentially, definitional equalities over \lstinline{nat} may become propositional over \lstinline{N}.}
\vspace{-0.2cm}
\label{fig:nattobin}
\end{figure}

Each \lstinline{Iota}---one per constructor---describes and proves the $\iota$-reduction behavior
of \lstinline{DepElim} on the corresponding case.
This is needed, for example, to port proofs about unary numbers \lstinline{nat} to
proofs about binary numbers \lstinline{N} (Figure~\ref{fig:nattobin}).
While we can define \lstinline{DepConstr} and \lstinline{DepElim} to induce an equivalence
between them~\href{https://github.com/uwplse/pumpkin-pi/blob/v2.0.0/plugin/coq/nonorn.v}{\circled{5}}, % nonorn.v
we run into trouble reasoning about applications of \lstinline{DepElim},
since proofs about \lstinline{nat} that hold by reflexivity do not necessarily hold by reflexivity over \lstinline{N}. 
For example, in Coq, while \lstinline{S (n + m)  = S n + m} holds by reflexivity over \lstinline{nat},
when we define \lstinline{+} with \lstinline{DepElim} over \lstinline{N},
the corresponding theorem over \lstinline{N} does not hold by reflexivity.

To transform proofs about \lstinline{nat} to proofs about \lstinline{N}, we must transform \textit{definitional} $\iota$-reduction over \lstinline{nat} to \textit{propositional} $\iota$-reduction over \lstinline{N}.
For our choice of \lstinline{DepConstr} and \lstinline{DepElim},
$\iota$-reduction is definitional over \lstinline{nat}, since a proof of:

\begin{lstlisting}
$\forall$ P p$_\texttt{0}$ p$_\texttt{S}$ n,(@\vspace{-0.04cm}@)
  DepElim((@\codediff{DepConstr(1, nat) n}@), P) { p$_\texttt{0}$, p$_\texttt{S}$ } =(@\vspace{-0.04cm}@)
  (@\codediff{p$_\texttt{S}$}@) n (DepElim(n, P) { p$_\texttt{0}$, p$_\texttt{S}$ }).(@\vspace{-0.05cm}@)
\end{lstlisting}
holds by reflexivity.
\lstinline{Iota} for \lstinline{nat} in the \lstinline{S} case is a rewrite by that proof by reflexivity~\href{https://github.com/uwplse/pumpkin-pi/blob/v2.0.0/plugin/coq/nonorn.v}{\circled{5}},
with type:

\begin{lstlisting}
$\forall$ P p$_\texttt{0}$ p$_\texttt{S}$ n (Q: P (DepConstr(1, nat) n) $\rightarrow$ s),(@\vspace{-0.04cm}@)
  Iota(1, nat, Q) :(@\vspace{-0.04cm}@)
    Q ((@\codediff{p$_\texttt{S}$}@) n (DepElim(n, P) { p$_\texttt{0}$, p$_\texttt{S}$ })) $\rightarrow$(@\vspace{-0.04cm}@)
    Q (DepElim((@\codediff{DepConstr(1, nat) n}@), P) { p$_\texttt{0}$, p$_\texttt{S}$ }).(@\vspace{-0.05cm}@)
\end{lstlisting}
In contrast, $\iota$ for \lstinline{N} is propositional, since the 
theorem: %over \lstinline{N}:

\begin{lstlisting}
$\forall$ P p$_\texttt{0}$ p$_\texttt{S}$ n,(@\vspace{-0.04cm}@)
  DepElim((@\codediff{DepConstr(1, N) n}@), P) { p$_\texttt{0}$, p$_\texttt{S}$ } =(@\vspace{-0.04cm}@)
  (@\codediff{p$_\texttt{S}$}@) n (DepElim(n, P) { p$_\texttt{0}$, p$_\texttt{S}$ }).(@\vspace{-0.05cm}@)
\end{lstlisting}
no longer holds by reflexivity.
\lstinline{Iota} for \lstinline{N} is a rewrite by the propositional equality that proves this theorem~\href{https://github.com/uwplse/pumpkin-pi/blob/v2.0.0/plugin/coq/nonorn.v}{\circled{5}},
with type:

\begin{lstlisting}
$\forall$ P p$_\texttt{0}$ p$_\texttt{S}$ n (Q: P (DepConstr(1, N) n) $\rightarrow$ s),(@\vspace{-0.04cm}@)
  Iota(1, N, Q) :(@\vspace{-0.04cm}@)
    Q ((@\codediff{p$_\texttt{S}$}@) n (DepElim(n, P) { p$_\texttt{0}$, p$_\texttt{S}$ })) $\rightarrow$(@\vspace{-0.04cm}@)
    Q (DepElim((@\codediff{DepConstr(1, N) n}@), P) { p$_\texttt{0}$, p$_\texttt{S}$ }).(@\vspace{-0.05cm}@)
\end{lstlisting}
By replacing \lstinline{Iota} over \lstinline{nat} with \lstinline{Iota} over \lstinline{N},
the transformation replaces rewrites by reflexivity over \lstinline{nat} to rewrites by propositional equalities over \lstinline{N}.
That way, \lstinline{DepElim} behaves the same over \lstinline{nat} and \lstinline{N}.

Taken together over both \A and \B, \lstinline{Iota} describes how the inductive structures of \A and \B differ.
The transformation requires that \lstinline{DepElim} over \A and over \B have the same structure
as each other, so if \A and \B \textit{themselves} have the same 
inductive structure (if they are \textit{ornaments}~\cite{mcbride}),
then if $\iota$ is definitional for \A, it will be possible to choose
\lstinline{DepElim} with definitional $\iota$ for \B.
Otherwise, if \A and \B (like \lstinline{nat} and \lstinline{N}) have different inductive structures,
then definitional $\iota$ over one would become propositional $\iota$ over the other.
%For the case of \lstinline{nat} and \lstinline{N},
%the need for propositional $\iota$ was noted as far back as \citet{magaud2000changing}.
%\lstinline{Iota} in the configuration encodes this more generally.


\subsection{The Proof Term Transformation}
\label{sec:generic}

\begin{figure*}
\begin{mathpar}
\mprset{flushleft}
\small
\hfill\fbox{$\Gamma$ $\vdash$ $t$ $\Uparrow$ $t'$}\vspace{-0.3cm}\\

\inferrule[Dep-Elim]
  { \Gamma \vdash a \Uparrow b \\ \Gamma \vdash p_{a} \Uparrow p_b \\ \Gamma \vdash \vec{f_{a}}\phantom{l} \Uparrow \vec{f_{b}} }
  { \Gamma \vdash \mathrm{DepElim}(a,\ p_{a}) \vec{f_{a}} \Uparrow \mathrm{DepElim}(b,\ p_b) \vec{f_{b}} }

\inferrule[Dep-Constr]
{ \Gamma \vdash \vec{t}_{a} \Uparrow \vec{t}_{b} } %\\ TODO must we explicitly lift A to B if we want to handle parameters/indices?
{ \Gamma \vdash \mathrm{DepConstr}(j,\ A)\ \vec{t}_{a} \Uparrow \mathrm{DepConstr}(j,\ B)\ \vec{t}_{b}  }

\inferrule[Eta]
  { \\ }
  { \Gamma \vdash \mathrm{Eta}(A) \Uparrow \mathrm{Eta}(B) }

\inferrule[Iota]
  { \Gamma \vdash q_A \Uparrow q_B \\ \Gamma \vdash \vec{t_A} \Uparrow \vec{t_B} }
  { \Gamma \vdash \mathrm{Iota}(j,\ A,\ q_A)\ \vec{t_A} \Uparrow \mathrm{Iota}(j,\ B,\ q_B)\ \vec{t_B} }

\inferrule[Equivalence]
  { \\ }
  { \Gamma \vdash A\ \Uparrow B }

\inferrule[Constr]
{ \Gamma \vdash T \Uparrow T' \\ \Gamma \vdash \vec{t} \Uparrow \vec{t'} }
{ \Gamma \vdash \mathrm{Constr}(j,\ T)\ \vec{t} \Uparrow \mathrm{Constr}(j,\ T')\ \vec{t'} }

\inferrule[Ind]
  { \Gamma \vdash T \Uparrow T' \\ \Gamma \vdash \vec{C} \Uparrow \vec{C'}  }
  { \Gamma \vdash \mathrm{Ind} (\mathit{Ty} : T) \vec{C} \Uparrow \mathrm{Ind} (\mathit{Ty} : T') \vec{C'} }

%% Application
\inferrule[App]
 { \Gamma \vdash f \Uparrow f' \\ \Gamma \vdash t \Uparrow t'}
 { \Gamma \vdash f t \Uparrow f' t' }

\inferrule[Elim] % TODO wait why do we have c here when it clearly refers to the term we eliminate over? um
  { \Gamma \vdash c \Uparrow c' \\ \Gamma \vdash Q \Uparrow Q' \\ \Gamma \vdash \vec{f} \Uparrow \vec{f'}}
  { \Gamma \vdash \mathrm{Elim}(c, Q) \vec{f} \Uparrow \mathrm{Elim}(c', Q') \vec{f'}  }

% Lamda
\inferrule[Lam]
  { \Gamma \vdash t \Uparrow t' \\ \Gamma \vdash T \Uparrow T' \\ \Gamma,\ t : T \vdash b \Uparrow b' }
  {\Gamma \vdash \lambda (t : T).b \Uparrow \lambda (t' : T').b'}

% Product
\inferrule[Prod]
  { \Gamma \vdash t \Uparrow t' \\ \Gamma \vdash T \Uparrow T' \\ \Gamma,\ t : T \vdash b \Uparrow b' }
  {\Gamma \vdash \Pi (t : T).b \Uparrow \Pi (t' : T').b'}

\inferrule[Var]
  { v \in \mathrm{Vars} }
  {\Gamma \vdash v \Uparrow v}

%\inferrule[Sort]
%  { \\ }
%  {\Gamma \vdash s \Uparrow s}
\end{mathpar}
\vspace{-0.3cm}
\caption{Transformation for transporting terms across $A \simeq B$ with configuration \lstinline{((DepConstr, DepElim), (Eta, Iota))}.}
\label{fig:final}
\end{figure*}

\begin{figure*}
\begin{minipage}{0.49\textwidth}
\begin{lstlisting}
(* 1: original term *)(@\vspace{-0.04cm}@)
$\lambda$ (T : Type) (l m : Old.list T) .(@\vspace{-0.04cm}@)
 Elim(l, $\lambda$(l: Old.list T).Old.list T $\rightarrow$ Old.list T)) {(@\vspace{-0.04cm}@)
   ($\lambda$ m . m),(@\vspace{-0.04cm}@)
   ($\lambda$ t _ IHl m . Constr(1, Old.list T) t (IHl m))(@\vspace{-0.04cm}@)
 } m.(@\vspace{-0.04cm}@)
(@\vspace{-0.10cm}@)
(* 2: after unifying (@\texttt{with}@) configuration *)(@\vspace{-0.04cm}@)
$\lambda$ (T : Type) (l m : (@\codediff{A}@)) .(@\vspace{-0.04cm}@)
 (@\codediff{DepElim}@)(l, $\lambda$(l: (@\codediff{A}@)).(@\codediff{A}@) $\rightarrow$ (@\codediff{A}@))) {(@\vspace{-0.04cm}@)
   ($\lambda$ m . m)(@\vspace{-0.04cm}@)
   ($\lambda$ t _ IHl m . (@\codediff{DepConstr}@)(1, (@\codediff{A}@)) t (IHl m))(@\vspace{-0.04cm}@)
 } m.(@\vspace{-0.04cm}@)
\end{lstlisting}
\end{minipage}
\hfill
\begin{minipage}{0.49\textwidth}
\begin{lstlisting}
(* 4: reduced to final term *)(@\vspace{-0.04cm}@)
$\lambda$ (T : Type) (l m : New.list T) .(@\vspace{-0.04cm}@)
 Elim(l, $\lambda$(l: New.list T).New.list T $\rightarrow$ New.list T)) {(@\vspace{-0.04cm}@)
   ($\lambda$ t _ IHl m . Constr(0, New.list T) t (IHl m)),(@\vspace{-0.04cm}@)
   ($\lambda$ m . m)(@\vspace{-0.04cm}@)
 } m.(@\vspace{-0.04cm}@)
(@\vspace{-0.10cm}@)
(* 3: after transforming *)(@\vspace{-0.04cm}@)
$\lambda$ (T : Type) (l m : (@\codediff{B}@)) .(@\vspace{-0.04cm}@)
 (@\codediff{DepElim}@)(l, $\lambda$(l: (@\codediff{B}@)).(@\codediff{B}@) $\rightarrow$ (@\codediff{B}@))) {(@\vspace{-0.04cm}@)
   ($\lambda$ m . m)(@\vspace{-0.04cm}@)
   ($\lambda$ t _ IHl m . (@\codediff{DepConstr}@)(1, (@\codediff{B}@)) t (IHl m))(@\vspace{-0.04cm}@)
 } m.(@\vspace{-0.04cm}@)
\end{lstlisting}
\end{minipage}
\vspace{-0.3cm}
\caption{Swapping cases of the append function, counterclockwise, the input term: 1) unmodified, 2) unified with the configuration, 3) ported to the updated type, and 4) reduced to the output.}
\label{fig:appswap1}
\end{figure*}

Figure~\ref{fig:final} shows the proof term transformation $\Gamma \vdash t \Uparrow t'$ that forms the core of \toolname.
%Like the transformation from \textsc{Devoid},
The transformation is parameterized over equivalent types \A and \B (\textsc{Equivalence})
as well as the configuration. %terms, which appear in the transformation explicitly.
It assumes $\eta$-expanded functions.
It implicitly constructs an updated context $\Gamma'$ in which to interpret $t'$, but this is not needed for computation.

The proof term transformation is (perhaps deceptively) simple by design:
it moves the bulk of the work into the configuration,
and represents the configuration explicitly.
Of course, typical proof terms in Coq do not apply these configuration
terms explicitly.
\toolname does some additional work using \textit{unification heuristics} to get real proof terms into this format before running the transformation.
It then runs the proof term transformation, which transports proofs across the equivalence that corresponds to the configuration.

\mysubsubsec{Unification Heuristics}
The transformation does not fully describe the search procedure for transforming terms that \toolname implements.
Before running the transformation, \toolname \textit{unifies} subterms with particular \A (fixing parameters and indices),
and with applications of configuration terms over \A. 
The transformation then transforms configuration terms over \A
to configuration terms over \B.
Reducing the result produces the output term defined over \B.

Figure~\ref{fig:appswap1} shows this with the list append function \lstinline{++} from Section~\ref{sec:overview}.
To update \lstinline{++} (top left), \toolname unifies \lstinline{Old.list T} with \A, and \lstinline{Constr} and \lstinline{Elim}
with \lstinline{DepConstr} and \lstinline{DepElim} (bottom left).
After unification, the transformation recursively substitutes \B
for \A, which moves \lstinline{DepConstr} and \lstinline{DepElim}
to construct and eliminate over the updated type (bottom right).
This reduces to a term with swapped constructors and cases over \lstinline{New.list T} (top right).

In this case, unification is straightforward. % since \lstinline{DepConstr} and \lstinline{DepElim} correspond to
%\lstinline{Constr} and \lstinline{Elim} directly.
This can be more challenging when configuration terms are dependent.
This is especially pronounced with definitional \lstinline{Eta} and \lstinline{Iota},
which typically are implicit (reduced) in real code.
%This problem is exactly why \citet{tabareau2019marriage} speculated that converting definitional to propositional equalities
%like we do with \lstinline{Iota} may, in general, be intractable.
To handle this, \toolname implements custom \textit{unification heuristics} for each search procedure
that unify subterms with applications of configuration terms, and that instantiate parameters and dependent indices in those subterms~\href{https://github.com/uwplse/pumpkin-pi/blob/v2.0.0/plugin/src/automation/lift/liftconfig.ml}{\circled{6}}. % liftconfig.ml
The transformation in turn assumes that all existing parameters and indices are determined and instantiated
by the time it runs.

\toolname falls back to Coq's unification for manual configuration and when these custom heuristics fail.
When even Coq's unification is not enough, \toolname relies on proof engineers to provide hints
in the form of annotations~\href{https://github.com/uwplse/pumpkin-pi/blob/v2.0.0/plugin/coq/nonorn.v}{\circled{5}}.

\begin{figure*}
\begin{minipage}{0.43\textwidth}
\begin{lstlisting}
section: $\forall$ (a : A), g (f a) = a.(@\vspace{-0.04cm}@)
retraction: $\forall$ (b : B), f (g b) = b.(@\vspace{-0.04cm}@)
(@\vspace{-0.14cm}@)
constr_ok: $\forall$ $j$ $\vec{x_A}$ $\vec{x_B}$, $\vec{x_A}$ $\equiv_{A \simeq B}$ $\vec{x_B}$ $\rightarrow$(@\vspace{-0.04cm}@)
  DepConstr($j$, A) $\vec{x_A}$ $\equiv_{A \simeq B}$ DepConstr(j, B) $\vec{x_B}$.(@\vspace{-0.04cm}@)
(@\vspace{-0.14cm}@)
elim_ok: $\forall$ a b P$_A$ P$_B$ $\vec{f_A}$ $\vec{f_B}$,(@\vspace{-0.04cm}@)
  a $\equiv_{A \simeq B}$ b $\rightarrow$(@\vspace{-0.04cm}@)
  P$_A$ $\equiv_{(A \rightarrow s) \simeq (B \rightarrow s)}$ P$_B$ $\rightarrow$(@\vspace{-0.04cm}@)
  $\forall$ $j$, $\vec{f_A}$[j] $\equiv_{\xi (A, P_A, j) \simeq \xi (B, P_B, j)}$ $\vec{f_B}$[j]$\rightarrow$(@\vspace{-0.04cm}@)
  DepElim(a, P$_A$) $\vec{f_A}$ $\equiv_{(P a) \simeq (P b)}$ DepElim(b, P$_B$) $\vec{f_A}$.(@\vspace{-0.04cm}@)
\end{lstlisting}
\end{minipage}
\hfill
\begin{minipage}{0.56\textwidth}
\begin{lstlisting}
elim_eta(A): $\forall$ a P $\vec{f}$, DepElim(a, P) $\vec{f}$ : P (Eta(A) a).(@\vspace{-0.04cm}@)
eta_ok(A): $\forall$ (a : A), Eta(A) a = a.(@\vspace{-0.04cm}@)
(@\vspace{-0.14cm}@)
(@\phantom{constr_ok: $\forall$ $j$ $\vec{x_A}$ $\vec{x_B}$,}@)(@\vspace{-0.04cm}@)
(@\phantom{  DepConstr($j$, A) $\vec{x_A}$ $\equiv_{A \simeq B}$ DepConstr(j, B) $\vec{x_B}$.}@)(@\vspace{-0.04cm}@)
(@\vspace{-0.14cm}@)
iota_ok(A): $\forall$ $j$ P $\vec{f}$ $\vec{x}$ (Q: P(Eta(A) (DepConstr($j$, A) $\vec{x}$)) $\rightarrow$ s),(@\vspace{-0.04cm}@)
  Iota(A, j, Q) : (@\vspace{-0.04cm}@)
    Q (DepElim(DepConstr(j, A) $\vec{x}$, P) $\vec{f}$) $\rightarrow$ (@\vspace{-0.04cm}@)
    Q (rew $\leftarrow$ eta_ok(A) (DepConstr(j, A) $\vec{x}$) in(@\vspace{-0.04cm}@)
      ($\vec{f}$[j]$\ldots$(DepElim(IH$_0$, P) $\vec{f}$)$\ldots$(DepElim(IH$_n$, P) $\vec{f}$)$\ldots$)).(@\vspace{-0.04cm}@)
\end{lstlisting}
\end{minipage}
% Q (eq_rect (dep_constr_A_0 b) (fun H : A => P H) (f0 b) (eta_A (dep_constr_A_0 b)) (eq_sym (eta_OK_A (dep_constr_A_0 b)))).
\iffalse
\begin{minipage}{0.44\textwidth}
\smallmath{$f := \lambda(a : A).\mathrm{DepElim}(a, \lambda(a : A).B) { \lambda \ldots \mathrm{DepConstr}(0, B) \ldots, \ldots }.$}
%[\
%g := \lambda(b : B).\mathrm{DepElim}(b, \lambda(b : B).A)\\
%  { \lambda \ldots \mathrm{DepConstr}(0, A) \ldots, \ldots }.\\
%section : \forall (a : A), g (f a) = a.\\
%retraction : \forall (b : B), f (g b) = b.\\
%\\
%constr_ok :\\
%  \forall j, DepConstr(j, A) \equiv_{A \simeq B} DepConstr(j, B).\\
%\\
%elim_ok : \forall a b (P : A \rightarrow s) (Q : B \rightarrow s),\\
%  a \equiv_{A \simeq B} b \rightarrow\\
%  P \equiv_{A \simeq B} Q \rightarrow\\
%  DepElim(a, P) \equiv_{A \simeq B} DepElim(b, Q).\\
%\]
\end{minipage}
\hfill
\begin{minipage}{0.55\textwidth}
\begin{lstlisting}
elim_eta(A) : $\forall$ (a : A) (P : A $\rightarrow$ Type) $\vec{f}$, DepElim(a, P) $\vec{f}$ : P (Eta(A) a).(@\vspace{-0.04cm}@)
eta_ok(A) : $\forall$ (a : A), Eta(A) a = a.(@\vspace{-0.04cm}@)
(@\vspace{-0.14cm}@)
iota_ok(A) : $\forall$ j (P : A $\rightarrow$ Type) $\vec{f}$ $\vec{x}$(@\vspace{-0.04cm}@)
    (Q : P (Eta(A) (DepConstr(j, A) $\vec{x}$)) $\rightarrow$ Type),(@\vspace{-0.04cm}@)
  Q (DepElim(DepConstr(j, A) $\vec{x}$, P) $\vec{f}$) $\rightarrow$ (@\vspace{-0.04cm}@)
  Q (rew $\leftarrow$ eta_ok(A) (DepConstr(j, A) $\vec{x}$) (@\vspace{-0.04cm}@)
     in ($\vec{f}$[j] $\ldots$ (DepElim(IH$_0$, P) $\vec{f}$) $\ldots$ (DepElim(IH$_n$, P) $\vec{f}$) $\ldots$))(@\vspace{-0.04cm}@)
:= Iota(A, j, Q).(@\vspace{-0.04cm}@)
\end{lstlisting}
% Q (eq_rect (dep_constr_A_0 b) (fun H : A => P H) (f0 b) (eta_A (dep_constr_A_0 b)) (eq_sym (eta_OK_A (dep_constr_A_0 b)))).
\end{minipage}
\fi
\iffalse
\begin{mathpar}
\mprset{flushleft}
\small

%f := $\lambda$(a : A).DepElim(a, $\lambda$(a : A).B)(@\vspace{-0.04cm}@)
%  { $\lambda$ $\ldots$ DepConstr(0, B) $\ldots$, $\ldots$ }.(@\vspace{-0.04cm}@)
%g := $\lambda$(b : B).DepElim(b, $\lambda$(b : B).A)(@\vspace{-0.04cm}@)
%  { $\lambda$ $\ldots$ DepConstr(0, A) $\ldots$, $\ldots$ }.(@\vspace{-0.04cm}@)

\inferrule[Is-Equivalence]
 {  \Gamma \vdash \mathrm{section} : \Pi (a : A) . \mathrm{g}\ (\mathrm{f}\ a)\ =\ a \\\\ \Gamma \vdash \mathrm{retraction} : \Pi (b : B) . \mathrm{f}\ (\mathrm{g}\ b)\ =\ b }
 { \Gamma \vdash \mathrm{is\_equivalence}(\mathrm{f}, \mathrm{g}, \mathrm{section}, \mathrm{retraction}) }

\inferrule[Constr-OK]
  { \Gamma \vdash \mathrm{DepConstr}(j, A)\ \vec{x_A} \equiv_{A \simeq B} \mathrm{DepConstr}(j, B)\ \vec{x_B} }
  { \Gamma \vdash \mathrm{constr\_ok}(j) }

\inferrule[Elim-OK]
  { \Gamma \vdash a \equiv_{A \simeq B} b \\\\ \Gamma \vdash P_A \equiv_{A \simeq B} P_B \\\\ \Gamma \vdash \vec{f_A} \equiv_{A \simeq B} \vec{f_B} \\\ \Gamma \vdash \mathrm{DepElim}(a,\ P_A)\ \vec{f_A} \equiv_{A \simeq B} \mathrm{DepElim}(b,\ P_B)\ \vec{f_B}  }
  { \Gamma \vdash \mathrm{elim\_ok}(j) }

\inferrule[Elim-Eta]
  { \Gamma \vdash a : A \\ \Gamma \vdash P : \Pi (a : A) . s }
  { \Gamma \vdash \mathrm{DepElim}(a, P)\ \vec{f} : P\ (\mathrm{Eta}(A)\ a) }

\inferrule[Eta-OK]
  { \Gamma \vdash T \in \{A, B\} }
  { \Gamma \vdash \mathrm{eta\_ok}(T) : \Pi (t : T) . \mathrm{Eta}(T)\ t\ =\ t }

%iota_ok(A) : $\forall$ j (P : A $\rightarrow$ Type) $\vec{f}$ $\vec{x}$(@\vspace{-0.04cm}@)
%    (Q : P (Eta(A) (DepConstr(j, A) $\vec{x}$)) $\rightarrow$ Type),(@\vspace{-0.04cm}@)
%  Q (DepElim(DepConstr(j, A) $\vec{x}$, P) $\vec{f}$) $\rightarrow$ (@\vspace{-0.04cm}@)
%  Q (rew $\leftarrow$ eta_ok(A) (DepConstr(j, A) $\vec{x}$) in ($\vec{f}$[j] $\ldots$ (DepElim(IH$_0$, P) $\vec{f}$) $\ldots$ (DepElim(IH$_n$, P) $\vec{f}$) $\ldots$))(@\vspace{-0.04cm}@)
%:= Iota(A, j, Q).(@\vspace{-0.04cm}@)
\end{mathpar}
\fi
\vspace{-0.2cm}
\caption{Correctness criteria for a configuration to ensure that the transformation
preserves equivalence (left) coherently with equality (right, shown for \A; \B is similar). \lstinline{f} and \lstinline{g} are defined in text. $s$, $\vec{f}$, $\vec{x}$, and $\vec{\mathtt{IH}}$ represent
sorts, eliminator cases, constructor arguments, and inductive hypotheses. $\xi$ $(A,$ $P,$ $j)$ is the type 
of \lstinline{DepElim(A, P)} at \lstinline{DepConstr(j, A)} (similarly for \B).} %, respectively.}
\label{fig:spec}
\end{figure*}
 % TODO sigs for Iota here are not quite correct---Q is not bound. Also need to tweak to deal w/ eta, and to use eta_OK, and to relate both eta
% TODO (!!) Define f and g with some schema like:
% ж(A, B) := λ(a : A).DepElim(a, $\lambda$(a : A).B){ $\lambda$ ... DepConstr(0, B) ..., ... }
% ж(A, B) := λ(b : B).DepElim(b, $\lambda$(b : B).A){ $\lambda$ ... DepConstr(0, A) ..., ... }

% $\mathrm{E}_{A_i}\ (p_A : \mathrm{P}_A)$ := $\xi(A,\ p_A,\ \mathrm{Constr}(i,\ A),\ C_{A_i})$

\mysubsubsec{Specifying a Correct Transformation}
The implementation of this transformation in \toolname produces a term that Coq type checks, and so does not
add to the trusted computing base.
As \toolname is an engineering tool, there is no need to formally prove the transformation correct, though doing so would be satisfying.
The goal of such a proof would be to show that % the transformation preserves equality up to transport along the equivalence $A \simeq B$,
%while no longer referring to the old specification.
%That is, we need that 
if $\Gamma \vdash t \Uparrow t'$,
then $t$ and $t'$ are equal up to transport, and $t'$ refers to \B in place of \A.
%This is the same as the correctness criterion for the program transformation from \textsc{Devoid} that this is based on,
%with the transformation generalized to handle other equivalences beyond the class that \textsc{Devoid} supports.
The key steps in this transformation that make this possible are porting terms along the configuration % corresponding
%to a particular equivalence 
(\textsc{Dep-Constr}, \textsc{Dep-Elim}, \textsc{Eta}, and \textsc{Iota}).
%The rest is straightforward.
For metatheoretical reasons, without additional axioms, a proof of this theorem in Coq can only be approximated~\cite{tabareau2017equivalences}.
It would be possible to generate per-transformation proofs of correctness, but this does not serve an engineering need.

\subsection{Specifying Correct Configurations}
\label{sec:art}

%Both when designing a search procedure for an automatic configuration and when
%configuring \toolname manually, choosing a configuration is important,
%and it is not always straightforward.
%This section specifies what it means for a configuration to be correct. % and gives some intuition as to why.
%Section~\ref{sec:search} shows some useful example configurations.
%The configuration instantiates the proof term transformation to a particular equivalence between \A and \B.

Choosing a configuration necessarily depends in some way on the proof engineer's intentions:
there can be infinitely many equivalences that correspond to a 
change, only some of which are useful (for example~\href{https://github.com/uwplse/pumpkin-pi/blob/v2.0.0/plugin/coq/playground/refine_unit.v}{\circled{7}}, any \A is equivalent to \lstinline{unit} refined by \A). % refine_unit.v
And there can be many configurations that correspond
to an equivalence, some of which will produce terms that are more useful or efficient than others
(consider \lstinline{DepElim} converting through several intermediate types).

While we cannot control for intentions, we \textit{can} specify what it means for a chosen configuration to be correct:
Fix a configuration. Let \lstinline{f} be the function that uses \lstinline{DepElim} to eliminate \A and \lstinline{DepConstr} to construct \B,
and let \lstinline{g} be similar. %Assume a univalent metatheory in which equality up to transport is defined.
Figure~\ref{fig:spec} specifies the correctness criteria for the configuration.
These criteria relate \lstinline{DepConstr}, \lstinline{DepElim}, \lstinline{Eta}, and \lstinline{Iota}
in a way that preserves equivalence coherently with equality.

\mysubsubsec{Equivalence}
To preserve the equivalence (Figure~\ref{fig:spec}, left), \lstinline{DepConstr} and \lstinline{DepElim} must form an equivalence
%between \A and \B.
(\lstinline{section} and \lstinline{retraction} must hold for \lstinline{f} and \lstinline{g}).
%uses  \lstinline{DepElim} to eliminate \B and \lstinline{DepConstr} to construct \A.
\lstinline{DepConstr} over \A and \B must be equal up to transport across that equivalence (\lstinline{constr_ok}), 
and similarly for \lstinline{DepElim} (\lstinline{elim_ok}).
%An example proves this on the change from \lstinline{list T} to \lstinline{packed_vect T} in the 
%univalent parametricity framework.\footnote{\url{https://github.com/CoqHott/univalent_parametricity/commit/7dc14e69942e6b3302fadaf5356f9a7e724b0f3c}}
Intuitively, \lstinline{constr_ok} and \lstinline{elim_ok} guarantee that the transformation
correctly transports dependent constructors and dependent eliminators,
as doing so will preserve equality up to transport for those subterms.
This makes it possible for the transformation
to avoid applying \lstinline{f} and \lstinline{g}, instead porting terms from \A directly to \B.

%Furthermore, since CIC$_{\omega}$ is constructive, the \textit{only} way to construct an \A (respectively \B) is to use its constructors,
%and the \textit{only} way to eliminate an \A (respectively \B) is to apply its eliminator.
%Finally, since these form an equivalence, all ways of constructing or eliminating \A and \B are covered by these dependent constructors and %eliminators.
%So, as long as we are able to unify subterms with applications of \lstinline{DepConstr} and \lstinline{DepElim},
%\textsc{Dep-Constr} and \textsc{Dep-Elim} should preserve correctness of the transformation and cover all values and eliminations of \A and \B.

\begin{figure*}
\small
\begin{grammar}
<v> $\in$ Vars, <t> $\in$ CIC$_{\omega}$

<p> ::= intro <v> \hspace{0.05cm} | \hspace{0.05cm} rewrite <t> <t> \hspace{0.05cm} | \hspace{0.05cm} symmetry \hspace{0.05cm} | \hspace{0.05cm} apply <t> \hspace{0.05cm} | \hspace{0.05cm} induction <t> <t> \{ <p>, \ldots, <p> \} \hspace{0.05cm} | \hspace{0.05cm} split \{ <p>, <p> \} \hspace{0.05cm} | \hspace{0.05cm} left \hspace{0.05cm} | \hspace{0.05cm} right \hspace{0.05cm} | \hspace{0.05cm} <p> . <p>
\end{grammar}
\vspace{-0.4cm}
\caption{Qtac syntax.}
\vspace{-0.4cm}
\label{fig:ltacsyntax1}
\end{figure*}

\begin{figure*}
\begin{mathpar}
\mprset{flushleft}
\small
\hfill\fbox{$\Gamma$ $\vdash$ $t$ $\Rightarrow$ $p$}\vspace{-0.5cm}\\

\inferrule[Intro]
  { \Gamma,\ n : T \vdash b \Rightarrow p }
  { \Gamma \vdash \lambda (n : T) . b \Rightarrow \mathrm{intro}\ n.\ p }

\inferrule[Symmetry]
  { \Gamma \vdash H \Rightarrow p }
  { \Gamma \vdash \mathtt{eq\_sym}\ H \Rightarrow \mathrm{symmetry}.\ p }

\inferrule[Split]
  { \Gamma \vdash l \Rightarrow p \\ \Gamma \vdash r \Rightarrow q }
  { \Gamma \vdash \mathrm{Constr}(0,\ \wedge)\ l\ r \Rightarrow \mathrm{split} \{ p, q \}.\ }\\

\inferrule[Left]
  { \Gamma \vdash H \Rightarrow p }
  { \Gamma \vdash \mathrm{Constr}(0,\ \vee)\ H \Rightarrow \mathrm{left}.\ p }

\inferrule[Right]
  { \Gamma \vdash H \Rightarrow p }
  { \Gamma \vdash \mathrm{Constr}(1,\ \vee)\ H \Rightarrow \mathrm{right}.\ p }

\inferrule[Rewrite]
  { \Gamma \vdash H_1 : x = y \\ \Gamma \vdash H_2 \Rightarrow p }
  { \Gamma \vdash \mathrm{Elim}(H_1,\ P) \{ x,\ H_2,\ y \} \Rightarrow \mathrm{symmetry}.\ \mathrm{rewrite}\ P\ H_1.\ p }\\

\inferrule[Induction]
  { \Gamma \vdash \vec{f} \Rightarrow \vec{p} }
  { \Gamma \vdash \mathrm{Elim}(t,\ P)\ \vec{f} \Rightarrow \mathrm{induction}\ P\ t\ \vec{p} }

\inferrule[Apply]
  { \Gamma \vdash t \Rightarrow p }
  { \Gamma \vdash f t \Rightarrow \mathrm{apply}\ f.\ p }

\inferrule[Base]
  { \\ }
  { \Gamma \vdash t \Rightarrow \mathrm{apply}\ t }
\end{mathpar}
\vspace{-0.4cm}
\caption{Qtac decompiler semantics.}
\label{fig:someantics}
\end{figure*}

\mysubsubsec{Equality}
To ensure coherence with equality (Figure~\ref{fig:spec}, right),
\lstinline{Eta} and \lstinline{Iota} must prove $\eta$ and $\iota$.
That is, \lstinline{Eta} must have the same definitional behavior as the dependent eliminator (\lstinline{elim_eta}),
and must behave like identity (\lstinline{eta_ok}).
Each \lstinline{Iota} must prove and rewrite along the simplification (\textit{refolding}~\cite{boutillier:tel-01054723}) behavior that corresponds to a case of the dependent eliminator (\lstinline{iota_ok}).
This makes it possible for the transformation to
avoid applying \lstinline{section} and \lstinline{retraction}.

\mysubsubsec{Correctness}
With these correctness criteria for a configuration, we get the completeness result (proven in Coq~\href{https://github.com/uwplse/pumpkin-pi/blob/v2.0.0/plugin/coq/playground/arbitrary.v}{\circled{8}}) that every equivalence induces a configuration. % arbitrary.v
We also obtain an algorithm for the soundness result that every configuration induces an equivalence.

The algorithm to prove \lstinline{section} is as follows (\lstinline{retraction} is similar):
replace \lstinline{a} with \lstinline{Eta(A) a} by \lstinline{eta_ok(A)}.
Then, induct using \lstinline{DepElim} over \A.
For each case $i$, the proof obligation is to show that \lstinline{g (f a)} is equal to \lstinline{a},
where \lstinline{a} is \lstinline{DepConstr(A, i)} applied to the non-inductive arguments (by \lstinline{elim_eta(A)}).
Expand the right-hand side using \lstinline{Iota(A, i)}, then expand it again using \lstinline{Iota(B, i)}
(destructing over each \lstinline{eta_ok} to apply the corresponding \lstinline{Iota}).
The result follows by definition of \lstinline{g} and \lstinline{f}, and by reflexivity.

\mysubsubsec{Automatic Configuration}
\toolname implements four search procedures for automatic configuration~\href{https://github.com/uwplse/pumpkin-pi/blob/v2.0.0/plugin/src/automation/lift/liftconfig.ml}{\circled{6}}.
Three of the four procedures are based on the search procedure from 
\textsc{Devoid}~\cite{Ringer2019},
while the remaining procedure instantiates the types \A and \B of a generic configuration that can be defined inside of Coq directly.
%Two use similar algorithms,
%due to space constraints, 
%we do not discuss these in detail.

The algorithm above is essentially what \textbf{Configure} uses to generate functions \lstinline{f} and \lstinline{g} for these configurations~\href{https://github.com/uwplse/pumpkin-pi/blob/v2.0.0/plugin/src/automation/search/search.ml}{\circled{9}}, % search.ml
and also generate proofs \lstinline{section} and \lstinline{retraction} that these functions form an equivalence~\href{https://github.com/uwplse/pumpkin-pi/blob/v2.0.0/plugin/src/automation/search/equivalence.ml}{\circled{10}}. % equivalence.ml
To minimize dependencies, \toolname does not produce proofs of \lstinline{constr_ok} and \lstinline{elim_ok} directly,
as stating these theorems cleanly would require either a special framework~\cite{tabareau2017equivalences}
or a univalent type theory~\cite{univalent2013homotopy}.
If the proof engineer wishes, it is possible to prove these in individual cases~\href{https://github.com/uwplse/pumpkin-pi/blob/v2.0.0/plugin/coq/playground/arbitrary.v}{\circled{8}}, % arbitray.v
but this is not necessary in order to use \toolname. %---they simply need to hold.

%First we need that \lstinline{DepElim} over $A$ into \lstinline{DepConstr} over $B$ and \lstinline{DepElim} over $B$ into
%\lstinline{DepConstr} over $A$ form an equivalence between $A$ and $B$. When that's true, I think it should hold that \lstinline{DepElim} over $A$
%and \lstinline{DepElim} over $B$ are in univalent relation with one another. If not, then that's an extra condition.
%Finally, we need the transformation to preserve definitional equalities. Not sure about the general case, but for vectors and lists,
%we need:

%\begin{lstlisting}
%  $\forall$ A l (f : $\forall$ (l : sigT (Vector.t A)), l = l),
%    vect_dep_elim A (fun l => l = l) (f nil) (fun t s _ => f (cons t s)) l = f (id_eta l).
%\end{lstlisting}
%and:

%\begin{lstlisting}
%Definition elim_id (A : Type) (s : {H : nat & t A H}) :=
%  vect_dep_elim
%    A
%    (fun _ => {H : nat & t A H})
%    nil
%    (fun (h : A) _ IH =>
%      cons h IH)
%    s.

% $\forall$ A h s,
%    exists (H : cons h (elim_id A s) = elim_id A (cons h s)),
%      H = eq_refl.
%\end{lstlisting}
%More generally, for each constructor index $j$, define:

%\begin{lstlisting}
%  eqc (j, B) (f : $\forall$ b : B, b = b) :=
%    fun ... (* TODO get the hypos from the type of the eliminator *) =>
%      f (DepConstr (j, B)) (* TODO args *)%%

  %elim_id := (* TODO *)
%\end{lstlisting}
%Then we need:

%\begin{enumerate}
%\item $\forall b f, \mathrm{DepElim}(b,\ p_{b}) \{\mathrm{eqc} (1, B) f, \ldots, \mathrm{eqc} (n, B) f\} = f (\mathrm{Eta}(A) a) $
%\item Something relating the constructors and \lstinline{elim_id} to reflexivity
%\end{enumerate}
%and similarly for $A$.

%Really the point of these conditions is that from them, with some restrictions on input terms, we can get
%that lifting terms gives us the same type that we'd get from lifting the type. But there are still
%some restrictions (see the few that fail).

%It's probably not always possible to define these three things for every equivalence.
%Could generalize by rewriting. But this lets us avoid the rewriting problem from Nicolas' paper.

% TODO how does this get us something like primitive projections? Just makes Eta definitionally equal to regular Id?

% TODO so we can probably just frame search in terms of DepConstr and DepElim and then generate proofs about this on an ad-hoc basis
% and get away with not including the specific details of our instantiations. We can give examples instead, give intuition, and say we generate
% the proofs in Coq

%For the second one we need not just an eliminator rule but also an identity rule.
%DEVOID assumed primitive projections which let them get away without thinking of this,
%but then had this weirdly ad-hoc ``repacking'' thing in their implementation.
%It turns out this is just a more general identity rule, which basically says what
%the identity function should lift to so that the transformation preserves definitional equalities.
%Actually deciding when to run this rule is one of the biggest challenges in practice,
%so we'll talk about that more in the implementation section.


\section{Decompiling Proof Terms to Tactics}
\label{sec:decompiler}

\textbf{Transform} produces a proof term,
while the proof engineer typically writes and maintains proof scripts made up of tactics.
We improve usability thanks to the realization that, since Coq's proof term language Gallina is very structured,
we can decompile these Gallina terms to suggested Ltac proof scripts for the proof engineer to maintain.

%\begin{quote}
%\textbf{Insight 3}: The transformed proof terms can then be translated back to tactic scripts.
%\end{quote}

\textbf{Decompile} implements a prototype of this translation~\href{https://github.com/uwplse/coq-plugin-lib/tree/9ef05815c261de9c99b604c6b581ba1c4fbc1a46/src/coq/decompiler/decompiler.ml}{\circled{11}}: % decompiler.ml
it translates a proof term to a suggested proof script that attempts to prove the same theorem the same way.
Note that this problem is not well defined: while there is always a proof script that 
works (applying the proof term with the \lstinline{apply} tactic), the result is often qualitatively unreadable.
This is the baseline behavior to which the decompiler defaults.
The goal of the decompiler is to improve on that baseline as much as possible,
or else suggest a proof script that is close enough to correct that the proof engineer can
manually massage it into something that works and is maintainable.

\textbf{Decompile} achieves this in two passes: The first pass decompiles proof terms to proof scripts that use a predefined set of tactics.
The second pass improves on suggested tactics by simplifying arguments, substituting tacticals, and using
hints like custom tactics and decision procedures.

\mysubsubsec{First Pass: Basic Proof Scripts}
The first pass takes Coq terms and produces tactics in Ltac, the proof script language for Coq.
Ltac can be confusing to reason about, since Ltac tactics can refer to Gallina terms, and the semantics of Ltac depends both on the
semantics of Gallina and on the implementation of proof search procedures written in OCaml.
To give a sense of how the first pass works without the clutter of these details, we start by defining a mini decompiler that 
implements a simplified version of the first pass.
Section~\ref{sec:second} explains how we scale this to the implementation.

The mini decompiler takes CIC$_{\omega}$ terms and produces tactics in a 
mini version of Ltac which we call Qtac.
The syntax for Qtac is in Figure~\ref{fig:ltacsyntax1}.
Qtac includes hypothesis introduction (\lstinline{intro}),
rewriting (\lstinline{rewrite}), symmetry of equality (\lstinline{symmetry}),
application of a term to prove the goal (\lstinline{apply}), induction (\lstinline{induction}),
case splitting of conjunctions (\lstinline{split}),
constructors of disjunctions (\lstinline{left} and \lstinline{right}), and
composition (\lstinline{.}).
Unlike in Ltac, \lstinline{induction} and \lstinline{rewrite} take a motive explicitly (rather than relying on unification),
and \lstinline{apply} creates a new subgoal for each function argument.
%The implementation reasons about Ltac and so does not make these assumptions.

The semantics for the mini decompiler $\Gamma \vdash t \Rightarrow p$ are in Figure~\ref{fig:someantics} (assuming $=$, \lstinline{eq_sym}, $\wedge$, and $\vee$ are defined as in Coq).
As with the real decompiler, the mini decompiler defaults to the proof script
that applies the entire proof term with \lstinline{apply} (\textsc{Base}).
Otherwise, it improves on that behavior by recursing over the proof term and constructing a proof script using a predefined set of tactics.

\iffalse
\begin{figure*}
\begin{minipage}{0.48\textwidth}
\begin{lstlisting}
fun (@\codesimb{(y0 : list A)}@) =>(@\vspace{-0.04cm}@)
  (@\codesima{list_rect}@) _ _  (fun (@\codesima{a l H}@) =>(@\vspace{-0.04cm}@)
    (@\codesimc{eq_ind_r}@) _ (@\codesimd{eq_refl}@) (@\codesimc{(app_nil_r (rev l) (a::[]))}@))(@\vspace{-0.04cm}@)
    (@\codesime{eq_refl}@)(@\vspace{-0.04cm}@)
    (@\codesima{y0}@)(@\vspace{-0.04cm}@)
\end{lstlisting}
\end{minipage}
\begin{minipage}{0.48\textwidth}
\begin{lstlisting}
(@\vspace{-0.14cm}@)
- (@\codesimb{intro y0.}@) (@\codesima{induction y0 as [a l H|].}@)(@\vspace{-0.04cm}@)
  + (@\codesimc{simpl. rewrite app_nil_r.}@) (@\codesimd{auto.}@)(@\vspace{-0.04cm}@)
  + (@\codesime{auto.}@)(@\vspace{-0.04cm}@)
(@\vspace{-0.14cm}@)
\end{lstlisting}
\end{minipage}
\vspace{-0.3cm}
\caption{Proof term (left) and decompiled proof script (right) for the base case of 
\lstinline{rev_app_distr} (Section~\ref{sec:overview}),  with corresponding terms and tactics 
grouped by color and number.}
\label{fig:rainbow}
\end{figure*}
\fi

For the mini decompiler, this is straightforward: Lambda terms become introduction (\textsc{Intro}).
Applications of \lstinline{eq_sym} become symmetry of equality (\textsc{Symmetry}).
Constructors of conjunction and disjunction map to the respective tactics (\textsc{Split}, \textsc{Left}, and \textsc{Right}).
Applications of equality eliminators compose symmetry (to orient the rewrite direction) with rewrites (\textsc{Rewrite}),
and all other applications of eliminators become induction (\textsc{Induction}).
The remaining applications become apply tactics (\textsc{Apply}).
In all cases, the decompiler recurses, breaking into cases, until only the \textsc{Base}
case holds. % at which point we are done.

While the mini decompiler is very simple, only a few small changes are needed
to move this to Coq.
%The result can already handle some of the example proofs \toolname has produced.
The generated proof term of \lstinline{rev_app_distr} from Section~\ref{sec:overview},
for example, consists only of induction, rewriting, simplification, and reflexivity (solved by \lstinline{auto}).
Figure~\ref{fig:rainbow} shows the proof term for the base case of \lstinline{rev_app_distr} 
alongside the proof script that \toolname suggests.
This script is fairly low-level and close to the proof term, but it is already something that the proof engineer
can step through to understand, modify, and maintain.
There are few differences from the mini decompiler needed to produce this,
for example handling of rewrites in both directions (\lstinline{eq_ind_r} as opposed to \lstinline{eq_ind}),
simplifying rewrites,
and turning applications of \lstinline{eq_refl} into \lstinline{reflexivity} or \lstinline{auto}.

\mysubsubsec{Second Pass: Better Proof Scripts}
The implementation of \textbf{Decompile} first runs something similar to the mini decompiler, then modifies the suggested tactics to produce a more natural proof script~\href{https://github.com/uwplse/coq-plugin-lib/tree/9ef05815c261de9c99b604c6b581ba1c4fbc1a46/src/coq/decompiler/decompiler.ml}{\circled{11}}. % decompiler.ml
For example, it cancels out sequences of \lstinline{intros} and \lstinline{revert},
inserts semicolons, and removes extra arguments to \lstinline{apply} and \lstinline{rewrite}. %, ensuring the result still holds. % TODO rewrite
It can also take tactics from the proof engineer (like part of the old proof script) as hints,
then iteratively replace tactics with those hints, checking for correctness.
This makes it possible for suggested scripts to include custom tactics and decision procedures.
%We omit the details due to space constraints. TODO add back for PLDI?
%Further improvements could come from preserving comments and indentation, or automatically using information from the old 
%version of the proof script rather than asking for it explicitly.

%In fact, since \toolname uses an existing command to translate pattern matching and fixpoints to eliminators,
%\textit{all} of the proof terms that \toolname produces will use induction and rewriting instead.
%Because we have control over output terms, even a mini decompiler gets us pretty far.

% TODO add any new things RanDair implements, like exists






\section{Implementation}
\label{sec:impl}

The transformation and mini decompiler abstract many of the challenges
of building a tool for proof engineers. % many details needed to build a tool that reaches proof engineers.
%and the mini decompiler abstracts a lot of the details that make Ltac so useful to proof engineers---and so painful to 
%reason about automatically.
This section describes how we solved some of these challenges.
%for both the transformation (Section~\ref{sec:implementation}) and the decompiler (Section~\ref{sec:second}).
%Section~\ref{sec:discussion} describes some remaining challenges and our plans to address them. % in the future.

\subsection{Implementing the Transformation}
\label{sec:implementation}

\mysubsubsec{Termination}
When a subterm unifies with a configuration term, this suggests that \toolname \textit{can}
transform the subterm, but it does not necessarily mean that it \textit{should}.
In some cases, doing so would result in nontermination.
For example, if \B is a refinement of \A, then we can always run \textsc{Equivalence}
over and over again, forever.
%\textsc{Devoid} ruled out this case by simply prohibiting the case where \B refers to \A, but we found it sometimes
%useful to support this case.
We thus include some simple termination checks in our code~\href{https://github.com/uwplse/pumpkin-pi/blob/v2.0.0/plugin/src/automation/lift/liftrules.ml}{\circled{12}}. % liftrules.ml

\mysubsubsec{Intent}
Even when termination is guaranteed, whether to transform a subterm depends on intent.
That is, \toolname automates the case of porting \textit{every} \A to \B,
but proof engineers sometimes wish to port only \textit{some} $A$s to $B$s.
\toolname has some support for this using an interactive workflow~\href{https://github.com/uwplse/pumpkin-pi/blob/v2.0.0/plugin/coq/minimal_records.v}{\circled{13}},
with plans for automatic support in the future. % minimal_records.v, but show this
%We helped the proof engineer do this by interacting with \toolname using a particular workflow.

\mysubsubsec{From CIC$_{\omega}$ to Coq}
The implementation~\href{https://github.com/uwplse/pumpkin-pi/blob/v2.0.0/plugin/src/automation/lift/lift.ml}{\circled{4}} % lift.ml
of the transformation handles language differences to scale from CIC$_{\omega}$ to Coq.
We use the existing \lstinline{Preprocess}~\cite{Ringer2019} command to turn pattern matching and fixpoints into 
eliminators.
We handle refolding of constants in constructors using \lstinline{DepConstr}.

\begin{figure}
\begin{lstlisting}
fun (@\codesimb{(y0 : list A)}@) =>(@\vspace{-0.04cm}@)
  (@\codesima{list_rect}@) _ _  (fun (@\codesima{a l H}@) =>(@\vspace{-0.04cm}@)
    (@\codesimc{eq_ind_r}@) _ (@\codesimd{eq_refl}@) (@\codesimc{(app_nil_r (rev l) (a::[]))}@))(@\vspace{-0.04cm}@)
    (@\codesime{eq_refl}@)(@\vspace{-0.04cm}@)
    (@\codesima{y0}@)(@\vspace{-0.04cm}@)
(@\vspace{-0.04cm}@)
- (@\codesimb{intro y0.}@) (@\codesima{induction y0 as [a l H|].}@)(@\vspace{-0.04cm}@)
  + (@\codesimc{simpl. rewrite app_nil_r.}@) (@\codesimd{auto.}@)(@\vspace{-0.04cm}@)
  + (@\codesime{auto.}@)(@\vspace{-0.04cm}@)
\end{lstlisting}
\vspace{-0.3cm}
\caption{Proof term (top) and decompiled proof script (bottom) for the base case of 
\lstinline{rev_app_distr} (Section~\ref{sec:overview}), with corresponding terms and tactics 
grouped by color \& number.}
\label{fig:rainbow}
\end{figure}


\mysubsubsec{Reaching Real Proof Engineers}
Many of our design decisions in implementing \toolname were informed by our partnership with
an industrial proof engineer (Section~\ref{sec:search}).
For example, the proof engineer rarely had the patience to wait more than ten seconds
for \toolname to port a term,
so we implemented optional aggressive caching, even caching intermediate subterms
encountered while running the transformation~\href{https://github.com/uwplse/pumpkin-pi/blob/v2.0.0/plugin/src/cache/caching.ml}{\circled{14}}. % TODO caching.ml
We also added a cache to tell \toolname not to $\delta$-reduce certain terms~\href{https://github.com/uwplse/pumpkin-pi/blob/v2.0.0/plugin/src/cache/caching.ml}{\circled{14}}.
With these caches, the proof engineer found \toolname efficient enough to use on a code base with tens of thousands of lines of code and proof.

 % caching.ml
%These caches are implemented in \href{https://github.com/uwplse/pumpkin-pi/blob/master/plugin/src/cache/caching.ml}{caching.ml}.
%or recurse into certain modules.
% set certain terms or modules as opaque to \toolname, to prevent unnecessary $\delta$-reduction.

The experiences of proof engineers also inspired new features.
For example, we implemented a search procedure to generate custom eliminators %(\href{https://github.com/uwplse/pumpkin-pi/blob/master/plugin/src/automation/search/smartelim.ml}{smartelim.ml})
to help reason about types like $\Sigma$\lstinline{(l : list T).length l = n}
by reasoning separately about the projections~\href{https://github.com/uwplse/pumpkin-pi/blob/v2.0.0/plugin/src/automation/search/smartelim.ml}{\circled{15}}. %smartelim.ml
We added informative error messages~\href{https://github.com/uwplse/pumpkin-pi/blob/v2.0.0/plugin/src/lib/ornerrors.ml}{\circled{22}} to help the proof engineer distinguish between user errors and bugs. % TODO link to errors
These features helped with workflow integration. % tactic decompiler helped with integration into proof engineering workflows.

\begin{table*}
\small
  \begin{tabular}{|l|l|l|c|l|l|}
    \hline
    \textbf{Class} & \textbf{Config.} & \textbf{Examples} & \textbf{Sav.} & \textbf{Repair Tools} & \textbf{Search Tools} \\
    \hline
    \multirow[t]{2}{*}{Algebraic Ornaments} & \multirow[t]{2}{*}{Auto} & List to Packed Vector, hs-to-coq \href{https://github.com/uwplse/pumpkin-pi/blob/v2.0.0/plugin/coq/examples/Example.v}{\circled{3}} % Example.v
    & \good & \toolname, \textsc{Devoid}, UP & \toolname, \textsc{Devoid} \\
    & & List to Packed Vector, Std. Library \href{https://github.com/uwplse/pumpkin-pi/blob/v2.0.0/plugin/coq/examples/ListToVect.v}{\circled{16}} % ListToVect.v
    & \good & \toolname, \textsc{Devoid}, UP & \toolname, \textsc{Devoid} \\
    \hline
    Unpack Sigma Types & Auto & Vector of Particular Length, hs-to-coq \href{https://github.com/uwplse/pumpkin-pi/blob/v2.0.0/plugin/coq/examples/Example.v}{\circled{3}} % Example.v
    & \good & \toolname, UP & \toolname \\
    \hline
    \multirow[t]{3}{*}{Tuples \& Records} & \multirow[t]{3}{*}{Auto} & Simple Records \href{https://github.com/uwplse/pumpkin-pi/blob/v2.0.0/plugin/coq/minimal_records.v}{\circled{13}} % minimal_records.v 
     & \good & \toolname, UP & \toolname \\
    & & Parameterized Records \href{https://github.com/uwplse/pumpkin-pi/blob/v2.0.0/plugin/coq/more_records.v}{\circled{17}} % more_records.v
    & \good & \toolname, UP & \toolname \\
    & & Industrial Use \href{https://github.com/Ptival/saw-core-coq/tree/dump-wip}{\circled{18}} %(\href{https://github.com/Ptival/saw-core-coq/tree/dump-wip}{saw-core-coq})
    & \good & \toolname, UP & \toolname \\
    \hline
    \multirow[t]{3}{*}{Permute Constructors} & \multirow[t]{3}{*}{Auto} & List, Standard Library \href{https://github.com/uwplse/pumpkin-pi/blob/v2.0.0/plugin/coq/Swap.v}{\circled{1}}
    & \good & \toolname, UP & \toolname \\
     & & Modifying a PL, \textsc{REPLica} Benchmark \href{https://github.com/uwplse/pumpkin-pi/blob/v2.0.0/plugin/coq/Swap.v}{\circled{1}} % Swap.v 
     & \ok & \toolname, UP  & \toolname \\
    & & Large Ambiguous Enum \href{https://github.com/uwplse/pumpkin-pi/blob/v2.0.0/plugin/coq/Swap.v}{\circled{1}} % Swap.v
    & \ok & \toolname, UP & \toolname \\
    \hline
    Add new Constructors & Mixed & PL Extension, \textsc{REPLica} Benchmark \href{https://github.com/uwplse/pumpkin-pi/blob/v2.0.0/plugin/coq/playground/add_constr.v}{\circled{19}} % (\href{https://github.com/uwplse/pumpkin-pi/blob/master/plugin/coq/playground/add_constr.v}{add_constr.v})
    & \bad & \toolname & \toolname (partial) \\
    \hline
    Factor out Constructors & Manual & External Example \href{https://github.com/uwplse/pumpkin-pi/blob/v2.0.0/plugin/coq/playground/constr_refactor.v}{\circled{2}} % (\href{https://github.com/uwplse/pumpkin-pi/blob/master/plugin/coq/playground/constr_refactor.v}{constr_refactor.v}) 
    & \good & \toolname, UP & None \\
    \hline
    Permute Hypotheses & Manual & External Example \href{https://github.com/uwplse/pumpkin-pi/blob/v2.0.0/plugin/coq/playground/flip.v}{\circled{20}} %(\href{https://github.com/uwplse/pumpkin-pi/blob/master/plugin/coq/playground/flip.v}{flip.v}) 
    & \bad & \toolname, UP & None \\
    \hline
    \multirow[t]{2}{*}{Change Ind. Structure} & \multirow[t]{2}{*}{Manual} & Unary to Binary, Classic Benchmark \href{https://github.com/uwplse/pumpkin-pi/blob/v2.0.0/plugin/coq/nonorn.v}{\circled{5}} %(\href{https://github.com/uwplse/pumpkin-pi/blob/master/plugin/coq/nonorn.v}{nonorn.v})
     & \ok & \toolname, Magaud & None \\
     & & Vector to Finite Set, External Example \href{https://github.com/uwplse/pumpkin-pi/blob/v2.0.0/plugin/coq/playground/fin.v}{\circled{21}} % (\href{https://github.com/uwplse/pumpkin-pi/blob/master/plugin/coq/playground/fin.v}{fin.v}) 
     & \good & \toolname & None \\
    \hline
  \end{tabular}
\vspace{0.05cm}
  \caption{Some changes using \toolname (left to right): class of changes, kind of configuration, examples, whether using \toolname saved development time relative to reference manual repairs (\good\xspace if yes, \ok\xspace if comparable, \bad\xspace if no), and Coq tools we know of that support repair along (Repair) or automatic proof of (Search) the equivalence corresponding to each example. Tools considered are \textsc{Devoid}~\cite{Ringer2019}, the Univalent Parametricity (UP) white-box transformation~\cite{tabareau2019marriage}, and the tool from \citet{magaud2000changing}. \toolname is the only one that suggests tactics.
More nuanced comparisons to these and more are in Section~\ref{sec:related}.}
\vspace{-0.4cm}
\label{fig:changes}
\end{table*}

\subsection{Implementing the Decompiler}
\label{sec:second}

\mysubsubsec{From Qtac to Ltac}
The mini decompiler assumes more predictable versions of \lstinline{rewrite} and \lstinline{induction}
than those in Coq. \textbf{Decompile} includes additional logic to reason about these tactics~\href{https://github.com/uwplse/coq-plugin-lib/blob/9ef05815c261de9c99b604c6b581ba1c4fbc1a46/src/coq/decompiler/decompiler.ml}{\circled{11}}. % decompiler.ml
For example, Qtac assumes that there is only one \lstinline{rewrite} direction. Ltac has two rewrite directions,
and so the decompiler infers the direction from the motive.

Qtac also assumes that both tactics take the inductive motive explicitly,
while in Coq, both tactics infer the motive automatically.
Consequentially, Coq sometimes fails to infer the correct motive.
% infers the wrong motive, % without manipulation of goals and hypotheses,
%or fails to infer a motive at all.
%This is especially common for \lstinline{rewrite}, which is purely syntactic.
To handle induction, the decompiler strategically uses \lstinline{revert} to manipulate the goal
so that Coq can better infer the motive.
To handle rewrites, it uses \lstinline{simpl} to refold the goal before rewriting.
Neither of these approaches is guaranteed to work, so the proof engineer may sometimes need to tweak the suggested proof script appropriately.
Even if we pass Coq's induction principle an explicit motive, Coq still sometimes fails due
to unrepresented assumptions.
Long term, using another tactic like \lstinline{change} or \lstinline{refine} before applying these tactics
may help with cases for which Coq cannot infer the correct motive.

\mysubsubsec{From CIC$_{\omega}$ to Coq}
Scaling the decompiler to Coq introduces let bindings, which are generated by 
tactics like \lstinline{rewrite in}, \lstinline{apply in}, and \lstinline{pose}.
\textbf{Decompile} implements~\href{https://github.com/uwplse/coq-plugin-lib/blob/9ef05815c261de9c99b604c6b581ba1c4fbc1a46/src/coq/decompiler/decompiler.ml}{\circled{11}} % decompiler.ml
support for \lstinline{rewrite in} and \lstinline{apply in} similarly to how it supports
\lstinline{rewrite} and \lstinline{apply}, except that it ensures that the unmanipulated hypothesis does not occur in the body of the let expression,
it swaps the direction of the rewrite, and it recurses into any generated subgoals.
In all other cases, it uses \lstinline{pose}, a catch-all for let bindings.

\mysubsubsec{Forfeiting Soundness}
While there is a way to always produce a correct proof script,
\textbf{Decompile} deliberately forfeits soundness to suggest more useful tactics.
For example, it may suggest the \lstinline{induction} tactic, but leave the step of motive inference to the proof engineer.
We have found these suggested tactics easier to work with (Section~\ref{sec:search}).
Note that in the case the suggested proof script is not quite correct,
it is still possible to use the generated proof term directly.

\mysubsubsec{Pretty Printing}
After decompiling proof terms, \textbf{Decompile} pretty prints the result~\href{https://github.com/uwplse/coq-plugin-lib/blob/9ef05815c261de9c99b604c6b581ba1c4fbc1a46/src/coq/decompiler/decompiler.ml}{\circled{11}}.
Like the mini decompiler, \textbf{Decompile} represents its output using a predefined grammar of Ltac tactics,
albeit one that is larger than Qtac, and that also includes tacticals.
It maintains the recursive proof structure for formatting. %, then uses that to print proofs of subgoals using bullet points.
%It displays the resulting proof script to the proof engineer, who can modify it as needed.
%It includes scripts that automate the process of printing all of these tactic proofs to a Coq file,
%in case the proof engineer does not want an interactive workflow.
\toolname keeps all output terms from \textbf{Transform} in the Coq environment in case the decompiler does not succeed.
Once the proof engineer has the new proof, she can remove the old one.



\section{Case Studies: \textsc{Pumpkin P}i Eight Ways}
\label{sec:search}

This section summarizes eight case studies using \toolname,
corresponding to the eight rows in Table~\ref{fig:changes}.
%For each case study, we explain the configuration used, walk through an example, and describe lessons learned.
These case studies highlight \toolname's flexibility in handling diverse scenarios,
%including in one case for an industrial user with unanticipated workflow.
the success of automatic configuration for better workflow integration, % for supported changes,
the preliminary success of the prototype decompiler,
and clear paths to better serving proof engineers.
Detailed walkthroughs are in the code.

\mysubsubsec{Algebraic Ornaments: Lists to Packed Vectors}
The transformation in \toolname is a generalization of the transformation from \textsc{Devoid}.
\textsc{Devoid} supported proof reuse across \textit{algebraic ornaments}, which describe relations
between two inductive types, where one type is the other indexed by a fold~\cite{mcbride}.
A standard example is the relation between a list and a
length-indexed vector (Figure~\ref{fig:listtovect}).

\toolname implements a search procedure for automatic configuration of algebraic ornaments.
%The search procedure moves the work specific to algebraic ornaments into the \toolname configuration.
The result is all functionality from \textsc{Devoid}, plus tactic suggestions.
%which allows it to implement the  functionality from \textsc{Devoid}.
In file~\href{https://github.com/uwplse/pumpkin-pi/blob/v2.0.0/plugin/coq/examples/Example.v}{\circled{3}}, we used this to port
functions and a proof from lists to vectors of \textit{some} length, since \lstinline{list T} $\simeq$ \lstinline{packed_vect T}.
The decompiler helped us write proofs in the order of hours that we had found
too hard to write by hand,
though the suggested tactics did need massaging.
% as the decompiler struggled with motive inference
%for induction with dependent types.
%Additional effort is needed to improve tactic suggestions for dependent types.

\begin{figure*}
\begin{minipage}{0.48\textwidth}
   \lstinputlisting[firstline=1, lastline=9]{replica.tex}
\end{minipage}
\hfill
\begin{minipage}{0.48\textwidth}
   \lstinputlisting[firstline=10, lastline=18]{replica.tex}
\end{minipage}
\vspace{-0.3cm}
\caption{A simple language (left) and the same language with two swapped constructors and an added constructor (right).}
\vspace{0.1cm}
\label{fig:replica}
\end{figure*}

\mysubsubsec{Unpack Sigma Types: Vectors of Particular Lengths}
%The previous case study showed how to get between lists and vectors of \textit{some} length.
In the same file~\href{https://github.com/uwplse/pumpkin-pi/blob/v2.0.0/plugin/coq/examples/Example.v}{\circled{3}}, we then ported functions and proofs to vectors of a \textit{particular} length, 
like \lstinline{vector T n}.
\textsc{Devoid} had left this step to the proof engineer.
We supported this in \toolname by chaining the previous change
with an automatic configuration for unpacking sigma types.
%To support this, we used one additional automatic configuration for unpacking sigma types.
%This configuration corresponds to the equivalence between sigma types at a particular projection
%and the same type escaping the sigma type. %, in our example $\Sigma$\lstinline{(s : packed_vect T).}$\pi_l$\lstinline{ s = n }$\simeq$\lstinline
%{ vector T n}.
By composition, this transported proofs across the equivalence from Section~\ref{sec:key1}.

Two tricks helped with workflow integration for this change:
1) have the search procedure view \lstinline{vector T n} as 
$\Sigma$\lstinline{(v : vector T m).n = m} for some \lstinline{m},
then let \toolname instantiate those equalities via unification heuristics, %before transforming,
and 2) generate a custom eliminator for combining
list terms with length invariants.
%This gave us a proof of this lemma (with \lstinline{zip_with} and \lstinline{zip} operating over lists at given lengths):
The resulting workflow works not just for lists and vectors, but for any algebraic ornament,
automating manual effort from \textsc{Devoid}.
The suggested tactics were helpful for writing proofs in the order of hours
that we had struggled with manually over the course of days, but only after massaging.
More effort is needed to improve tactic suggestions for dependent types.

\mysubsubsec{Tuples \& Records: Industrial Use}
An industrial proof engineer at the company \company has been using \toolname in proving
correct an implementation of the TLS handshake protocol.
%While this is ongoing work, thus far,
%\toolname has helped \company integrate Coq with their existing verification workflow.
\company had been using a custom solver-aided verification language to prove correct C programs,
but had found that at times, the constraint solvers got stuck.
%and they could not progress on proofs about those programs.
They had built a compiler that translates their language into Coq's specification language Gallina,
that way proof engineers could finish stuck proofs interactively using Coq.
However, due to language differences, they had found the generated Gallina programs and specifications difficult to work with.

The proof engineer used \toolname to port the automatically generated functions and specifications to more
human-readable functions and specifications, wrote Coq proofs about those functions and specifications, then
used \toolname to port those proofs back to
proofs about the original functions and specifications.
%This workflow has allowed for industrial integration with Coq and has helped the proof engineer write functions and proofs
%that would have otherwise been difficult.
So far, they have used at least three automatic configurations,
but they most often used an automatic configuration for porting compiler-produced anonymous tuples
to named records, as in file~\href{https://github.com/Ptival/saw-core-coq/tree/dump-wip}{\circled{18}}. % TODO aux material %\footnote{\url{https://github.com/Ptival/saw-core-coq/tree/dump-wip}} TODO note in guide it got a bit abandoned because proof engineer got stuck in france...
%The proof engineer was able to use \toolname to integrate Coq into an existing proof engineering
%workflow using solver-aided tools at \company.
The workflow was a bit nonstandard,
so there was little need for tactic suggestions.
The proof engineer reported an initial time investment learning how to use \toolname,
followed by later returns.
%In the initial days, we worked closely with the proof engineer;
%later, the proof engineer worked independently and reached out occasionally.
%The proof engineer was able to work independently, but found two challenges with workflow integration:
%1) they sometimes could not distinguish between user errors and bugs,
%and 2) they waited only about ten seconds at most for \toolname to return.
%Both informed improvements to \toolname, like better error messages, caching,
%and a way to tell \toolname not to $\delta$-reduce certain terms.

\mysubsubsec{Permute Constructors: Modifying a Language}
The swapping example from Section~\ref{sec:overview} was inspired by benchmarks 
from the \textsc{Replica} user study of proof engineers~\cite{replica}.
A change from one of the benchmarks is in Figure~\ref{fig:replica}.
The proof engineer had a simple language represented by an inductive type \lstinline{Term},
as well as some definitions and proofs about the language.
The proof engineer swapped two constructors in the language,
and added a new constructor \lstinline{Bool}.

This case study and the next case study break this change into two parts.
In the first part, we used \toolname with automatic configuration to repair functions and proofs about the language
after swapping the constructors~\href{https://github.com/uwplse/pumpkin-pi/blob/v2.0.0/plugin/coq/Swap.v}{\circled{1}}.
%We also succeeded at more difficult variants of this,
%like swapping constructors with the same type, renaming all of the constructors,
%permuting more than two constructors,
%or permuting and renaming constructors at the same time.
With a bit of human guidance to choose the permutation from a list of suggestions,
\toolname repaired everything,
though the original tactics would have also worked,
so there was not a difference in development time.
%This is a property of the particular proofs that we had access to;
%As Section~\ref{sec:overview}
%and more advanced variants of this benchmark show~\circled{1}, even for simple changes, this is not always true.
% TODO what happens with the tactic decompiler for these?
%The entire \lstinline{Swap.v} file, which includes swapping constructors of every function in the \lstinline{list} module and
%its dependencies, four variants of the \textsc{Replica} swapping change,
%and testing a large and ambiguous permutation of a 30 constructor \lstinline{Enum},
%took \toolname less than 90 seconds total. % TODO specs, reproduction
%Each variant of the \textsc{Replica} swapping change took \toolname less than 5 seconds, % TODO specs, reproduction
%and the change adding a constructor took \toolname less than 30 seconds. % TODO specs, reproduction

\mysubsubsec{Add new Constructors: Extending a Language}
We then used \toolname to repair functions 
after adding the new constructor in Figure~\ref{fig:replica}, separating out the proof obligations for the new constructor from the old terms~\href{https://github.com/uwplse/pumpkin-pi/blob/v2.0.0/plugin/coq/playground/add_constr.v}{\circled{19}}.
%The resulting functions were guaranteed to preserve the behavior of the old terms. % TODO do the proofs too
This change combined manual and automatic configuration.
We defined an inductive type \lstinline{Diff} and (using partial automation) a configuration to port the terms across the equivalence \lstinline{Old.Term + Diff} $\simeq$ \lstinline{New.Term}.
This resulted in case explosion, but was formulaic, and pointed to a clear path for automation of this class of changes.
The repaired functions guaranteed preservation of the behavior of the original functions. %over \lstinline{Old.Term} before the change.

Adding constructors was less simple than swapping.
For example, \toolname did not yet save us time over the proof engineer from the user study;
fully automating the configuration would have helped significantly.
In addition, the repaired terms were (unlike in the swap case) inefficient compared to human-written terms.
For now, they make good regression tests for the human-written terms---in the future,
we hope to automate the discovery of the more efficient terms,
or use the refinement framework CoqEAL~\cite{cohen:hal-01113453}
to get between proofs of the inefficient and efficient terms.

\vspace{0.105cm}
\mysubsubsec{Factor out Constructors: External Example}
The change from Figure~\ref{fig:equivalence2} came at the request of a non-author.
We supported this using a manual configuration that described which constructor to map to \lstinline{true}
and which constructor to map to \lstinline{false}~\href{https://github.com/uwplse/pumpkin-pi/blob/v2.0.0/plugin/coq/playground/constr_refactor.v}{\circled{2}}.
The configuration was very simple for us to write, and the repaired tactics were immediately useful.
The development time savings were on the order of minutes for a small proof development.
Since most of the modest development time went into writing the configuration,
we expect time savings would increase for a larger development.

\vspace{0.105cm}
\mysubsubsec{Permute Hypotheses: External Example}
The change in \href{https://github.com/uwplse/pumpkin-pi/blob/v2.0.0/plugin/coq/playground/flip.v}{\circled{20}} came at the request of a different non-author (a cubical type theory expert),
and shows how to  use \toolname to swap two hypotheses of a type, since \lstinline{T1} $\rightarrow$ \lstinline{T2} $\rightarrow$ \lstinline{T3} $\simeq$
\lstinline{T2} $\rightarrow$ \lstinline{T1} $\rightarrow$ \lstinline{T3}.
This configuration was manual.
Since neither type was inductive, this change used the generic construction for any equivalence.
%instantiated to this particular equivalence.
This worked well, but necessitated some manual annotation due to the lack of custom unification heuristics for 
manual configuration, and so did not yet save development time, and likely still would not have had the proof development been larger.
Supporting custom unification heuristics would improve this workflow.

\vspace{0.105cm}
\mysubsubsec{Change Inductive Structure: Unary to Binary}
In \href{https://github.com/uwplse/pumpkin-pi/blob/v2.0.0/plugin/coq/nonorn.v}{\circled{5}}, we used \toolname to support a classic example of changing inductive structure:
updating unary to binary numbers,
as in Figure~\ref{fig:nattobin}.
% TODO fin and vect? mention somewhere?
Binary numbers allow for a fast addition function, found in the Coq standard library.
In the style of \citet{magaud2000changing}, we used \toolname to derive a slow binary
addition function that does not refer to \lstinline{nat},
and to port proofs from unary to slow binary addition.
We then showed that the ported theorems hold over fast binary addition.

%The implementation of this is in \lstinline{nonorn.v}.
The configuration for \lstinline{N} used definitions from the Coq standard library
for \lstinline{DepConstr} and \lstinline{DepElim} that had the desired behavior with no changes.
\lstinline{Iota} over the successor case was a rewrite by a lemma
from the standard library that reduced the successor case of the eliminator that we used for \lstinline{DepElim}:

\begin{lstlisting}
N.peano_rect_succ : $\forall$ P pO pS n,(@\vspace{-0.04cm}@)
  N.peano_rect P pO pS (N.succ n) =(@\vspace{-0.04cm}@)
  pS n (N.peano_rect P pO pS n).(@\vspace{-0.05cm}@)
\end{lstlisting}
%
%\begin{lstlisting}
%Lemma iota_1 :(@\vspace{-0.04cm}@)
%  $\forall$ P pO pS n (Q : P (dep_constr_1 n) $\rightarrow$ Type),(@\vspace{-0.04cm}@)
%     Q (pS n (dep_elim P pO pS n)) $\rightarrow$(@\vspace{-0.04cm}@)
%     Q (dep_elim P pO pS (dep_constr_1 n)).(@\vspace{-0.04cm}@)
%Proof.(@\vspace{-0.04cm}@)
%  intros. unfold dep_elim, dep_constr_1. rewrite N.peano_rect_succ. auto.(@\vspace{-0.04cm}@)
%Defined.
%\end{lstlisting}
The need for nontrivial \lstinline{Iota} comes from the fact that \lstinline{N} and \lstinline{nat}
have different inductive structures.
By writing a manual configuration with this \lstinline{Iota}, it was possible for us to implement this transformation 
that had been its own tool.

While porting addition from \lstinline{nat} to \lstinline{N} was automatic after configuring \toolname,
porting proofs about addition took more work.
Due to the lack of unification heuristics for manual configuration,
we had to annotate the proof term to tell \toolname that implicit casts in the inductive cases of proofs were applications of \lstinline{Iota}
over \lstinline{nat}.
These annotations were formulaic, but tricky to write.
Unification heuristics would go a long way toward improving the workflow. % for this use case.

After annotating, we obtained automatically repaired proofs about slow binary addition,
which we found simple to port to fast binary addition.
We hope to automate this last step in the future using CoqEAL. %~\cite{cohen:hal-01113453}.
Repaired tactics were partially useful, but failed to understand custom eliminators like \lstinline{N.peano_rect}, and to generate useful
tactics for applications of \lstinline{Iota}; both of these are clear paths to more useful tactics.
The development time for this proof with \toolname was comparable to reference manual repairs by external proof engineers.
Custom unification heuristics would help bring returns on investment for experts in this use case.

%\end{enumerate}
%\item \toolname was flexible. Each search procedure for automatic configuration
%was simple to write, and handled an entire class of equivalences corresponding to real use cases.
%Manual configuration was possible for interesting use cases.
%\item \toolname had good enough workflow integration to support real use cases.
%The tactic decompiler showed promising early results with clear paths to improvements.
%Some workflows were unanticipated and informed design changes in \toolname.
%\end{enumerate}

\iffalse
\subsection{\textsc{Replica} Benchmark Variants}
\label{sec:replica}

\begin{figure}
\begin{minipage}{0.49\columnwidth}
   \lstinputlisting[firstline=1, lastline=9]{replica.tex}
\end{minipage}
\hfill
\begin{minipage}{0.49\columnwidth}
   \lstinputlisting[firstline=11, lastline=19]{replica.tex}
\end{minipage}
\vspace{-0.3cm}
\caption{A simple language (left) and the same language with two swapped constructors and an added constructor (right).}
\vspace{-0.1cm}
\label{fig:replica}
\end{figure}

The swapping example from Section~\ref{sec:overview} was inspired by benchmarks 
from the \textsc{Replica} user study of proof engineers~\cite{replica}.
A change corresponding to part of one of the benchmarks is shown in Figure~\ref{fig:replica}.
The proof engineer had a simple language represented by an inductive type \lstinline{Term},
as well as some definitions and proofs about the language.
The proof engineer swapped two constructors in the term language,
and added a new constructor \lstinline{Bool}.

We used \toolname to automatically configure the proof term transformation to swap these constructors,
then repair all of the functions and proofs.
We also succeeded at more difficult variants of this,
like swapping constructors with the same type, renaming all of the constructors,
permuting more than two constructors,
or permuting and renaming constructors at the same time.
In all cases, with a bit of human guidance, \toolname repaired the functions and proofs.

We then extended the inductive type to add the new constructor, and used \toolname to repair terms,
separating out the proof obligations for the new constructor from the old terms.
The resulting terms were guaranteed to preserve the behavior of the old terms.

\subsubsection{Configuration}

The swap change used an automatic configuration that handles permuting and renaming constructors of inductive types.
This is one of the simplest configurations.
The only nontrivial part is that \lstinline{DepConstr} over the updated type permutes back the constructors:

\begin{lstlisting}[backgroundcolor=\color{cyan!30}]
dep_constr_0 (H : Z) : Term := (@\codediff{Int}@) H.(@\vspace{-0.04cm}@)
dep_constr_1 (H H0 : Term) : Term := (@\codediff{Eq}@) H H0.
\end{lstlisting}
so that they align with the original constructors.
The eliminator similarly permutes cases:

\begin{lstlisting}[backgroundcolor=\color{cyan!30}]
dep_elim P f0 f1 f2 f3 f4 f5 f6 t : P t :=(@\vspace{-0.04cm}@)
  Old.Term_rect P f0 (@\codediff{f2}@) (@\codediff{f1}@) f3 f4 f5 f6 t.
\end{lstlisting}

Adding the constructor combined a manual configuration with an automatic configuration.
The details of this are in \href{https://github.com/uwplse/pumpkin-pi/blob/v2.0.0/plugin/coq/playground/add_constr.v}{add_constr.v}.
At a high level, we defined an inductive type \lstinline{Diff} and (using partial automation) a configuration to port the terms across this equivalence:

\begin{lstlisting}
Old.Term + (@\codediff{Diff}@) $\simeq$ New.Term
\end{lstlisting}
This made it simple to guarantee preservation of the behavior over \lstinline{Old.Term} before the change.

\subsubsection{Example}

We used \toolname to automatically update the functions and proofs in \href{https://github.com/uwplse/pumpkin-pi/blob/v2.0.0/plugin/coq/Swap.v}{\lstinline{Swap.v}} from the \textsc{Replica} benchmark.
This included functions about \lstinline{Term}, as well as a large record \lstinline{EpsilonLogic} that encoded the semantics of the language,
plus a proof about that record.
%\begin{lstlisting}
%Theorem eval_eq_true_or_false : $\forall$ (L : EpsilonLogic) env (t1 t2 : Term),(@\vspace{-0.04cm}@)
%  L.eval env (Eq t1 t2) = L.vTrue $\vee$ L.eval env (Eq t1 t2) = L.vFalse.
%\end{lstlisting}
\toolname automatically updated all of these. For the inductive proof, it produced
a new inductive proof that used the induction principle with the swapped constructors.
It also discovered all other 23 type-correct permutations of constructors, all available for selection to 
prove the appropriate equivalence and use to update functions and proofs.
It presented the desired transformation as the first option in the list, so that all we had to do
was pass the argument \lstinline{mapping 0} to \lstinline{Repair module} for it to handle this change.
It was also able to handle the more advanced variants of this change.

In \href{https://github.com/uwplse/pumpkin-pi/blob/v2.0.0/plugin/coq/playground/add_constr.v}{add_constr.v},
we then updated the functions to add the new constructor.
This resulted in some case explosion, but was very formulaic.
Since the functions \toolname produced were guaranteed to preserve the behavior of the old functions,
we used them as regression tests to check the behavior of the handwritten functions from the benchmark.

\subsubsection{Lessons}

Swapping constructors was simple---it used a search procedure that handles
any permutation or renaming of constructors.
Extending constructors was less simple.
Most notably, the repaired terms were (unlike in the swap case) inefficient compared to human-written terms.
For now, they make good regression tests for the human-written terms---in the future,
we hope to automate the discovery of the more efficient term as well.

The swap change alone was simple enough that \toolname would not have been necessary
for updating the proofs for the initial change---keeping the tactics the same would have also worked.
%This is a property of the particular proofs that we had access to;
As we saw in Sections~\ref{sec:overview} and~\ref{sec:key1}, even for simple changes, this is not always true.
More advanced variants of this benchmark that we tried did necessitate \toolname,
and \toolname worked well for those. % TODO what happens with the tactic decompiler for these?
The entire \lstinline{Swap.v} file, which includes swapping constructors of every function in the \lstinline{list} module and
its dependencies, four variants of the \textsc{Replica} swapping change,
and testing a large and ambiguous permutation of a 30 constructor \lstinline{Enum},
took \toolname less than 90 seconds total. % TODO specs, reproduction
Each variant of the \textsc{Replica} swapping change took \toolname less than 5 seconds, % TODO specs, reproduction
and the change adding a constructor took \toolname less than 30 seconds. % TODO specs, reproduction

\subsection{Vectors from Lists}
\label{sec:dep}

The proof term transformation in \toolname is based on the proof term transformation from
\textsc{Devoid}, an earlier version of \toolname.
\textsc{Devoid} supported proof reuse across \textit{algebraic ornaments}, which describe relations
between two inductive types, where one type is exactly the other type indexed by a fold~\cite{mcbride}.
The standard example of this is the relation between a list and a
length-indexed vector, like we saw in Figure~\ref{fig:listtovect} in Section~\ref{sec:key1}.

With \toolname, we were able to not only support algebraic ornaments as in \textsc{Devoid},
but also automate effort that had previously been manual.
Several proof engineers including the industrial proof engineer, a Reddit user,
and someone on the \lstinline{coq-club} message board contacted us expressing interest in using this functionality.
%, though we do not yet know if the latter two ended up using \toolname. % TODO honestly ask

\subsubsection{Configuration}

We used two automatic configurations to ease development with dependent types using algebraic ornaments.
The first instantiates the proof term transformation to the transformation from \textsc{Devoid}:
it ports functions and proofs from the input type to the ornamented type at \textit{some} index,
like \lstinline{packed_vect T} from Section~\ref{sec:key1}.
The second provides the missing link to get proofs at a \textit{particular} index, like \lstinline{vector T n};
\textsc{Devoid} had left this to the proof engineer.

For lists and vectors, the search procedure for the first configuration used the equivalence:

\begin{lstlisting}
list T $\simeq$ packed_vect T
\end{lstlisting}
The configuration it found for \lstinline{list} was simple: identity for \lstinline{Eta},
the list constructors for \lstinline{DepConstr}, the list eliminator for \lstinline{DepElim},
and definitional \lstinline{Iota}.
For \lstinline{vector}, it set \lstinline{Eta} to $\eta$-expand $\Sigma$ types:

\begin{lstlisting}[backgroundcolor=\color{cyan!30}]
eta (T : Type) (s : packed_vect T) : packed_vect T :=(@\vspace{-0.04cm}@)
  $\exists$ ($\pi_l$ s) ($\pi_r$ s).
\end{lstlisting}
it set \lstinline{DepConstr} to pack constructors: % TODO shrink this now that some of it is in the other section

\begin{lstlisting}[backgroundcolor=\color{cyan!30}]
dep_constr_0 T : $\exists$ 0 (Vector.nil A).(@\vspace{-0.04cm}@)
dep_constr_1 T t s :=(@\vspace{-0.04cm}@)
  $\exists$ (S ($\pi_l$ s)) (Vector.cons ($\pi_l$ s) t ($\pi_r$ s)).
\end{lstlisting}
it set \lstinline{DepElim} to eliminate its projections:

\begin{lstlisting}[backgroundcolor=\color{cyan!30}]
dep_elim T P pnil pcons : P (eta T s) :=(@\vspace{-0.04cm}@)
vector_rect T(@\vspace{-0.04cm}@)
  (fun n v => P ($\exists$ n v))(@\vspace{-0.04cm}@)
  pnil(@\vspace{-0.04cm}@)
  (fun t n v => pcons t ($\exists$ n v))(@\vspace{-0.04cm}@)
  ($\pi_l$ s) ($\pi_r$ s).
\end{lstlisting}
and it used definitional \lstinline{Iota}.
%This configuration was enough to capture all of the search and lifting functionality from \textsc{Devoid}.

To get from lists to vectors \textit{at a particular length}, we used one additional automatic configuration.
This configuration corresponds to the equivalence between sigma types at a particular projection
and the same type escaping the sigma type, in our example:

\begin{lstlisting}
$\Sigma$(s : packed_vect T).$\pi_l$ s = n $\simeq$ vector T n
\end{lstlisting}
By composition with the initial equivalence, this transports proofs
across the equivalence we want:

\begin{lstlisting}
$\Sigma$(l : list T).length l = n $\simeq$ vector T n
\end{lstlisting}
since these equivalences are equal up to transport along the first equivalence.

The second configuration carries equality proofs over the indices,
which are then instantiated for the transformation using unification heuristics.
For example, it views \lstinline{vector T n} as implicitly representing $\Sigma$\lstinline{(v : vector T m).n = m} for some \lstinline{m}.
This is seen in \lstinline{eta}, here: 
% TODO both should take the same number of arguments, even if eta_A takes fewer. also does this belong in iota? this is weird honestly

\begin{lstlisting}[backgroundcolor=\color{cyan!30}]
eta T n m (H : n = m) (v : vector T m): vector T n :=(@\vspace{-0.04cm}@)
  eq_rect m (vector T) v n H.
\end{lstlisting}
which is the identity function generalized over any equal index.
Since there are no changes in inductive types, \lstinline{DepElim} and \lstinline{DepConstr} are trivial,
and \lstinline{Iota} does not change.

%\begin{lstlisting}
%(@\codeauto{dep_elim}@) (T : Type) (P : $\forall$ (n : nat), vector T n $\rightarrow$ Type) (p : $\forall$ n v, P n (eta T n n v)) (v : vector T n) : %P (eta T n n v) :=
%  p.
%\end{lstlisting}

\subsubsection{Example}

The expanded example from the \textsc{Devoid} paper is in \href{https://github.com/uwplse/pumpkin-pi/blob/v2.0.0/plugin/coq/examples/Example.v}{\lstinline{Example.v}}.
The \textsc{Devoid} example ported a list \lstinline{zip} function,
a \lstinline{zip_with} function, and a proof \lstinline{zip_with_is_zip} relating the two
functions from lists to vectors of some length.
It then manually ported those proofs to proofs over vectors at a particular length.
The updated \toolname example automates this last step.

%The workflow for this was a bit different than it was with \textsc{Devoid}.
To handle this, we used a custom eliminator \toolname generated to combine the list functions
with length invariants, and to combine the list proofs with the proofs about those length invariants.
This gave us a proof of this lemma (with \lstinline{zip_with} and \lstinline{zip} operating over lists at given lengths):

\begin{lstlisting}
Lemma zip_with_is_zip : $\forall$ A B n(@\vspace{-0.04cm}@)
  (v1: $\Sigma$(l1: (@\codediff{list A}@)).(@\codediff{length}@) l1 = n)(@\vspace{-0.04cm}@)
  (v2: $\Sigma$(l2: (@\codediff{list B}@)).(@\codediff{length}@) l2 = n),(@\vspace{-0.04cm}@)
    zip_with pair n v1 v2 = zip n v1 v2.
\end{lstlisting}
We then ran \lstinline{Repair module} using the first
configuration, which proved this lemma:

\begin{lstlisting}[backgroundcolor=\color{cyan!30}]
Lemma zip_with_is_zip : $\forall$ A B n(@\vspace{-0.04cm}@)
  (v1: $\Sigma$(l1: (@\codediff{packed_vect A}@)).(@\codediff{$\pi_l$}@) l1 = n)(@\vspace{-0.04cm}@)
  (v2: $\Sigma$(l2: (@\codediff{packed_vect B}@)).(@\codediff{$\pi_l$}@) l2 = n),(@\vspace{-0.04cm}@)
    zip_with pair v1 v2 = zip v1 v2.
\end{lstlisting}
with functions \lstinline{zip_with} and \lstinline{zip} updated as well.
We composed this with \lstinline{Repair module} on the second configuration,
which proved the final lemma:

\begin{lstlisting}[backgroundcolor=\color{cyan!30}]
Lemma zip_with_is_zip : $\forall$ A B n(@\vspace{-0.04cm}@)
  (v1: (@\codediff{vector A n}@)) (v2: (@\codediff{vector B n}@)),(@\vspace{-0.04cm}@)
    zip_with pair v1 v2 = zip v1 v2.
\end{lstlisting}
with functions \lstinline{zip_with} and \lstinline{zip} repaired as well.

\subsubsection{Lessons}

%\paragraph{Configuration \& Flexibility}
%Implementing the search procedure for the first configuration was straightforward.
%Implementing the search procedure for the second configuration was less straightforward.
%It is still an open challenge to define complete unification heuristics to port any arbitrary proof
%along the second configuration in either direction,
%though this does not impact the case study.

%\paragraph{Workflow Integration}
The result was a simplified workflow from \textsc{Devoid} that automated steps 
that had previously been manual.
%\toolname implemented some additional automation to generate eliminators that help separate out this additional information
%for repair. 
The tactic decompiler suggested tactics that helped us write tactic proofs ourselves that had been
prohibitively difficult for us to write by hand.
However, the suggested tactics did need massaging, as the decompiler struggled with motive inference
for induction with dependent types.
Additional effort is needed to further improve tactic integration with dependent types.
% TODO how long did compiling the file take?

\subsection{Unary and Binary Numbers}
\label{sec:bin}

All of the case studies so far have dealt with pairs of types with the same inductive structure,
or with new information.
Some of the oldest problems in the transport literature deal with \textit{changing} inductive
structure without adding any new information.
We have realized a classic example~\cite{magaud2000changing} of this with \toolname using a manual configuration:
updating unary to binary natural numbers.
% TODO fin and vect? mention somewhere?

In Coq, a binary number \lstinline{N} (see Section~\ref{sec:key2}, Figure~\ref{fig:nattobin}) is either zero or a positive binary number. A positive binary number
is either 1 (\lstinline{xH}), or the result of shifting left and adding 1 (\lstinline{xI})
or nothing (\lstinline{xO}).
This allows for a fast addition function, found in the Coq standard library.
In the style of \citet{magaud2000changing}, we use \toolname to derive a slow binary
addition function that does not refer to \lstinline{nat}.
From that, we port proofs over unary addition to binary addition,
removing all references to \lstinline{nat}, and show that they hold over fast binary addition too.

\subsubsection{Configuration}
We configured this manually using the \lstinline{Configure} command,
which takes the configuration as an argument.
The result is in \href{https://github.com/uwplse/pumpkin-pi/blob/v2.0.0/plugin/coq/nonorn.v}{\lstinline{nonorn.v}}.
The configuration for \lstinline{nat} was straightforward.
For \lstinline{N}, we used functions from the Coq standard library that
behaved like the \lstinline{nat} constructors:

\begin{lstlisting}
dep_constr_0 : N := 0%N.(@\vspace{-0.04cm}@)
dep_constr_1 : N $\rightarrow$ N := N.succ.
\end{lstlisting}
and an eliminator from the Coq standard library that behaves like the \lstinline{nat} eliminator:

\begin{lstlisting}
dep_elim P (pO : P dep_constr_0) (pS : $\forall$n, P n $\rightarrow$ P (dep_constr_1 n)) (n : N) : P n :=(@\vspace{-0.04cm}@)
  N.peano_rect P pO pS n.
\end{lstlisting}
\lstinline{Iota} was almost written for us.
The standard library had a lemma that reduced the successor case:

\begin{lstlisting}
N.peano_rect_succ :(@\vspace{-0.04cm}@)
  $\forall$ P pO pS n, N.peano_rect P pO pS (N.succ n) = pS n (N.peano_rect P pO pS n).
\end{lstlisting}
\lstinline{Iota} over the successor case was a simple rewrite by this lemma:

\begin{lstlisting}
Lemma iota_1 :(@\vspace{-0.04cm}@)
  $\forall$ P pO pS n (Q : P (dep_constr_1 n) $\rightarrow$ Type),(@\vspace{-0.04cm}@)
     Q (pS n (dep_elim P pO pS n)) $\rightarrow$(@\vspace{-0.04cm}@)
     Q (dep_elim P pO pS (dep_constr_1 n)).(@\vspace{-0.04cm}@)
Proof.(@\vspace{-0.04cm}@)
  intros. unfold dep_elim, dep_constr_1. rewrite N.peano_rect_succ. auto.(@\vspace{-0.04cm}@)
Defined.
\end{lstlisting}

The need for a nontrivial \lstinline{Iota} comes from the fact that \lstinline{N} has a different
inductive structure from \lstinline{nat}, and is noted as far back as \citet{magaud2000changing}.
This corresponds to a broader pattern---it captures the essence of the change in inductive structure.
\toolname's configurable proof term transformation captures that intuition.

\subsubsection{Example}

We ported unary addition from \lstinline{nat} to \lstinline{N} fully automatically:

\begin{lstlisting}
Repair nat N in add as slow_add.
\end{lstlisting}
The result (tellingly named) has the same slow behavior as the \lstinline{add} function over \lstinline{nat}.
However, it no longer refers to \lstinline{nat} in any way.
Like \citet{magaud2000changing}, we found it easy to manually prove that
this has the same behavior as fast binary addition:

\begin{lstlisting}
Lemma add_fast_add: $\forall$ (n m : Bin.nat), slow_add n m = N.add n m.(@\vspace{-0.04cm}@)
Proof.(@\vspace{-0.04cm}@)
  induction n using N.peano_rect; intros m; auto. unfold slow_add.(@\vspace{-0.04cm}@)
  rewrite N.peano_rect_succ. (* $\leftarrow$ iota_1 *)(@\vspace{-0.04cm}@)
  unfold slow_add in IHn. rewrite IHn. rewrite N.add_succ_l. reflexivity.(@\vspace{-0.04cm}@)
Qed.
\end{lstlisting}

We then used \toolname again to transform a proof:
\begin{lstlisting}
add_n_Sm : $\forall$ (n m : nat), S (add n m) = add n (S m).
\end{lstlisting}
from \lstinline{add} to \lstinline{slow_add}:

\begin{lstlisting}[backgroundcolor=\color{cyan!30}]
slow_add_n_Sm : $\forall$ (n m : N), N.succ (slow_add n m) = slow_add n (N.succ m).
\end{lstlisting}
This was not quite as push-button.
It involved a manual expansion step, turning implicit casts in the inductive case
into explicit applications of \lstinline{Iota} over \A.
These applications were formulaic, but tricky to write.
Once we had that, though, we could run the same \lstinline{Repair} command
to get \lstinline{slow_add_n_Sm}.
Showing that the same theorem held over fast binary addition was then
straightforward:

\begin{lstlisting}
Lemma add_n_Sm : $\forall$ n m, Bin.succ (N.add n m) = N.add n (Bin.succ m).
Proof.
  intros. repeat rewrite $\leftarrow$ add_fast_add. apply slow_plus_n_Sm.
Qed.
\end{lstlisting}

\subsubsection{Lessons}

\lstinline{Iota} was the key to supporting this case.
It was enough to implement this transformation that had previously been its own tool
just by writing a configuration with \lstinline{Iota}. 

The file took under a second for us to compile using \toolname.
We did not need tactic suggestions for this workflow,
but we did note that supporting custom eliminators like \lstinline{N.peano_rect} would be a simple way
to improve the decompiler.
The most difficult part was manually expanding proofs about \lstinline{nat}
to apply \lstinline{Iota},
as there is not yet a way to supply custom unification heuristics for manual configuration.
We discuss ideas for this in Sections~\ref{sec:related} and~\ref{sec:discussion}.

\subsection{Industrial Use}
\label{sec:industry}

An industrial proof engineer at Galois has been using \toolname in proving
correct an implementation of the TLS handshake protocol.
While this is ongoing work, thus far,
\toolname has helped \company integrate Coq with their existing verification workflow.

\company had been using a custom solver-aided verification language to prove correct C programs.
They had found that at times, those constraint solvers got stuck, and they could not
progress on proofs about those programs.
They had built a compiler that translates their solver-aided language into Coq's specification language Gallina,
so proof engineers could finish stuck proofs interactively using Coq.
However, the generated Gallina programs and specifications were sometimes too difficult to work with.

A proof engineer at \company has used \toolname to work with those automatically generated programs and specifications
with the following workflow:

\begin{enumerate}
\item use \toolname to update the automatically generated functions and specifications into more
human-readable functions and specifications, then
\item write Coq proofs about the more human-readable functions and specifications, then finally
\item use \toolname again to update those proofs about human-readable functions and specifications back to
proofs about the original automatically generated functions and specifications.
\end{enumerate}
This workflow has allowed for industrial integration with Coq and has helped the proof engineer write functions and proofs
that would have otherwise been difficult.

% TODO will add better proofs here once Val sends them

\subsubsection{Configuration}

\begin{figure*}
\begin{minipage}{0.33\textwidth}
   \lstinputlisting[firstline=1, lastline=10]{records.tex}
\end{minipage}
\hfill
\begin{minipage}{0.66\textwidth}
   \lstinputlisting[firstline=12, lastline=21]{records.tex}
\end{minipage}
\vspace{-0.5cm}
\caption{Two unnamed tuples (left) and corresponding named records (right).}
\label{fig:records}
\end{figure*}

Some example proofs that the proof engineer used \toolname for
can be found in a branch of the proof engineer's repository.\footnote{\url{https://github.com/Ptival/saw-core-coq/tree/dump-wip}}
The proof engineer used \toolname to port anonymous tuples produced by \company' compiler
to named records, as in the example in Figure~\ref{fig:records}.

We implemented a search procedure for the proof engineer to automatically configure the proof term transformation to an equivalence
between nested tuples and named records.
The search procedure triggered automatically when the proof engineer called the \lstinline{Repair} command.
%It set \A to be the record type and \B to be the tuple.
The configuration it found for the record type was straightforward: identity for \lstinline{Eta},
the record constructor for \lstinline{DepConstr}, the record eliminator for \lstinline{DepElim}, and definitional \lstinline{Iota}.
For the tuple, it set \lstinline{Eta} to expand and project, for example for \lstinline{Handshake}:
\begin{lstlisting}[backgroundcolor=\color{cyan!30}]
eta (H : Handshake) : Handshake := (fst H, snd H).
\end{lstlisting}
it set \lstinline{DepConstr} to apply the pair constructor:

\begin{lstlisting}[backgroundcolor=\color{cyan!30}]
dep_constr_0 (handshakeType : seq 32 bool) (messageNumber : seq 32 bool) :=(@\vspace{-0.04cm}@)
  (handshakeType, messageNumber).
\end{lstlisting}
and it set \lstinline{DepElim} to eliminate over the pair:

\begin{lstlisting}[backgroundcolor=\color{cyan!30}]
dep_elim P (f: $\forall$ h m, P (dep_constr_0 h m)) (H : Handshake) : P (eta H) :=(@\vspace{-0.04cm}@)
  prod_rect(@\vspace{-0.04cm}@)
    (fun (p : seq 32 bool * seq 32 bool) => P (eta p))(@\vspace{-0.04cm}@)
    (fun (h : seq 32 bool) (m : seq 32 bool) => f h m)(@\vspace{-0.04cm}@)
    H.
\end{lstlisting}
For \lstinline{Connection}, it found similar \lstinline{Eta}, \lstinline{DepConstr}, and \lstinline{DepElim},
except that \lstinline{Eta} included all nine projections, \lstinline{DepConstr} \textit{recursively} applied the
pair constructor, and \lstinline{DepElim} \textit{recursively} eliminated the pair.
This induced an equivalence between the nested tuple and record,
which \toolname generated and proved automatically.

\subsubsection{Example}
Using this configuration, the proof engineer automatically ported this compiler-generated function:

\begin{lstlisting}
cork (c : Connection) : Connection :=(@\vspace{-0.04cm}@)
  (fst c, (bvAdd _ (fst (snd c)) (bvNat _ 1), snd (snd c))).
\end{lstlisting}
to the corresponding function over records:

\begin{lstlisting}[backgroundcolor=\color{cyan!30}]
cork (c : (@\texttt{Record}@).Connection) : (@\texttt{Record}@).Connection := {|(@\vspace{-0.04cm}@)
  clientAuthFlag := clientAuthFlag c; (@\hspace{0.43cm}@) corked := bvAdd _ (corked c) (bvNat _ 1);(@\vspace{-0.04cm}@)
  corkedIO := corkedIO c; (@\hspace{2.36cm}@) handshake := handshake c;(@\vspace{-0.04cm}@)
  isCachingEnabled := isCachingEnabled c; keyExchangeEPH := keyExchangeEPH c;(@\vspace{-0.04cm}@)
  mode := mode c; (@\hspace{3.64cm}@) resumeFromCache := resumeFromCache c;(@\vspace{-0.04cm}@)
  serverCanSendOCSP := serverCanSendOCSP c(@\vspace{-0.04cm}@)
|}.
\end{lstlisting}
The proof engineer then wrote proofs that record, for example:

\begin{lstlisting}
Lemma corkLemma :(@\vspace{-0.04cm}@)
  $\forall$ (c : (@\texttt{Record}@).Connection), corked c = bvNat 2 0 $\rightarrow$ corked (cork c) = bvNat 2 1.(@\vspace{-0.04cm}@)
Proof.(@\vspace{-0.04cm}@)
  intros []. simpl. intros H. subst. reflexivity.(@\vspace{-0.04cm}@)
Defined.
\end{lstlisting} % TODO better proof
and then used \toolname to automatically port these back to proofs about the original function:

\begin{lstlisting}[backgroundcolor=\color{cyan!30}]
Lemma corkLemma :(@\vspace{-0.04cm}@)
  $\forall$ (c : Connection), fst (snd c) = bvNat 2 0 $\rightarrow$ fst (snd (cork c)) = bvNat 2 1.
\end{lstlisting} % TODO decompiler

\subsubsection{Lessons}

%but also to help write dependently typed functions.
So far, the proof engineer has used at least three automatic configurations.
Two had existed already, while the one for tuples and records was
added in response to the proof engineer's needs.
This search procedure was easy for us to implement. %using techniques from
%the repair and reuse tools \textsc{Pumpkin Patch} and \textsc{Devoid}.
Flexibility could be further improved by exposing an interface to allow proof engineers to
write search procedures themselves.

The proof engineer was able to use \toolname to integrate Coq into an existing proof engineering
workflow using solver-aided tools at \company.
The workflow for using \toolname itself was a bit nonstandard,
and there was little need for tactic proofs about the compiler-generated functions and specifications.
In the initial days, we worked closely with the proof engineer;
later, the proof engineer worked independently and reached out occasionally by email or phone.
\toolname was usable enough for this to work, but we found two challenges with workflow integration:
the proof engineer sometimes could not distinguish between user errors and bugs in our code,
and the proof engineer typically waited only about ten seconds at most for \toolname to return.
Both observations informed improvements to \toolname, like better error messages, caching of transformed subterms,
and the ability to tell \toolname not to $\delta$-reduce certain terms.
% TODO how long did compiling the file take?
\fi

% TODO what does the tactic decompiler do for this? It's broken. Why?

%\subsubsection{Algebraic}

%It is straightforward to fit the search algorithm from DEVOID into this framework, and in fact
%we can loosen the restriction that the language has primitive projections.
%Let $A$ be $A$ from DEVOID, let $B_{ind}$ be $B$ from DEVOID, let $I_B$ be $I_B$ from DEVOID,
%and let \lstinline{index} be \lstinline{index} from DEVOID.
%Let $B$ wrap $B_{ind}$ packed into a sigma type:

%\begin{lstlisting}
%B := $\lambda$ ($\vec{t}$ : $\vec{T}$) . ($\Sigma$ (i : I$_B$ $\vec{t}$) . B$_{ind}$ (index i $\vec{t}$))
%\end{lstlisting}
%Let $\vec{T_{B_j}}$ be the arguments of constructor type $C_{B_j}$ (type of constructor of $B_{\mathrm{ind}}$).
%Define \lstinline{DepConstr(j, B)} recursively using the following derivation (based on and same fall-through convention as the DEVOID paper %for now,
%and I'd prefer to move this away from a derivation but not sure how to do so and maintain formality): % TODO check

%\begin{mathpar}
%\mprset{flushleft}
%\small
%\hfill\fbox{$\Gamma$ $\vdash$ $(T_A, T_B)$ $\Downarrow_{C}$ $t$}\\%

%\inferrule[Dep-Constr-Conclusion]
%  { \Gamma \vdash \vec{t_{B_j}} : \vec{T_{B_j}} \\ \Gamma \vdash Constr(j, B)\ \vec{t_{B_j}} : B_{\mathrm{ind}} \vec{i_B}  }
%  { \Gamma \vdash (A\ \vec{i_A},\ B_{\mathrm{ind}}\ \vec{i_B}) \Downarrow_{p_{c}} \exists\ (\vec{i_B}[\mathrm{off}\ A\ B]) (Constr(j, B)\ \vec{t_{B_j}}) }

%\inferrule[Dep-Constr-Index] % new hypothesis for index
%  { \mathrm{new}\ n_B\ b_B \\ \Gamma,\ n_B : t_B \vdash (\Pi (n_A : t_A) . b_A,\ b_B) \Downarrow_{i_{c}} t }
%  {  \Gamma \vdash (\Pi (n_A : t_A) . b_A,\ \Pi (n_B : t_B) . b_B) \Downarrow_{C} t}

%\inferrule[Dep-Constr-IH] % inductive hypothesis
%  { \Gamma,\ n_B : B\ \vec{i_B} \vdash (b_A [n_B / n_A], b_B [\pi_l\ n_B / \vec{i_B}[\mathrm{off}\ A\ B]]) \Downarrow_{C} t }
%  { \Gamma \vdash (\Pi (n_A : A\ \vec{i_A}) . b_A, \Pi (n_B : B\ \vec{i_B}) \Downarrow_{C} \lambda (n_B : B\ \vec{i_B}) . t }

%\inferrule[Dep-Constr-Prod] % otherwise, unchanged (when we get rid of the gross fall-through thing, needs not new, and needs to check t_A and t_B not IHs)
%  { \Gamma,\ n_B : t_B \vdash (b_A [n_B / n_A], b_B) \Downarrow_{C} t }
%  { \Gamma \vdash (\Pi (n_A : t_A) . b_A, \Pi (n_B, t_B) . b_B) \Downarrow_{C} \lambda (n_B : t_B) . t }\\

%\inferrule[Dep-Constr]
%{ \Gamma \vdash Constr(j, A) : C_{A_j} \\ \Gamma \vdash (C_{A_j}, C_{B_j}) \Downarrow_{C} t }
%{ \Gamma \vdash (Constr(j, A), Constr(j, B_{\mathrm{ind}}) \Downarrow_{C} t }
%\end{mathpar}
%and \lstinline{DepElim(b, p)} similarly:

%\begin{mathpar}
%TODO
%\end{mathpar}

%Then:

%\begin{lstlisting}
%DepConstr(j, A) : C$_{A_{j}}$ := Constr(j, A)
%DepConstr(j, B) : C$_{A_{j}}$[B / A] := DepConstr(j, B)

%DepElim(a, p){f$_{1}$, $\ldots$, f$_{n}$} : p a := Elim(a, p){f$_{1}$, $\ldots$, f$_{n}$}
%DepElim(b, p){f$_{1}$, $\ldots$, f$_{n}$} : p b := DepElim(b, p)

%Eta(A) := $\lambda$(a : A).a
%Eta(B) := $\lambda$(b : B).$\exists$ ($\pi_l$ b) ($\pi_r$ b)
%\end{lstlisting}

% TODO investigate below projection thing, and write in when you finish
%For now assume we have some \lstinline{pack} function to pack into an existential;
%this is just for convenience.
%The indexer is just the first projection of this lifted across the eliminator rule, AFAIK---note this isn't exactly $\Pi_{l}$ like we use
%in the tool, but is really an eliminated $\Pi_{l}$? I will need to check on this, it's the only weird part.
%Also assume some \lstinline{index_args} function to add the new index to the appropriate arguments---I'll
%elaborate on this later but it's also something search needs to find and it's determined in terms of the \lstinline{indexer} that search finds.
%Also now, we no longer assume primitive projections.

%\subsubsection{Unpack sigma}

%This one is kind of weird but it gets us user-friendly types. I'll explain later.

%\begin{lstlisting}
%DepConstr(j, A) := (* TODO pack into existential, deal with equality *)
%DepConstr(j, B) : C$_{B_{j}}$ := Constr(j, B)

%DepElim(a, p){f$_{1}$, $\ldots$, f$_{n}$} : p a := (* TODO *)
%DepElim(b, p){f$_{1}$, $\ldots$, f$_{n}$} : p b := Elim(b, p){f$_{1}$, $\ldots$, f$_{n}$}

%Eta(A) := $\lambda$(a : A).$\exists$ ($\exists$ ($\pi_l$ ($\pi_l$ a)) ($\pi_r$ ($\pi_l$ a))) ($\pi_r$ a)
%Eta(B) := $\lambda$(b : B).b
%\end{lstlisting}

%\subsubsection{Records and tuples}

%This one should be easier. We'll play a similar trick with $B$ and $B_{ind}$ like we do for algebraic,
%and give things similar names.
%Then:

%\begin{lstlisting}
%DepConstr(j, A) : C$_{A_{j}}$ := Constr(j, A)
%DepConstr(j, B) : C$_{A_{j}}$[B / A] := $\lambda$ ($\vec{t_{A_j}}$ : $\vec{T_{A_j}}$) . (* TODO recursively pack into pair *)

%DepElim(a, p){f$_{1}$, $\ldots$, f$_{n}$} : p a := Elim(a, p){f$_{1}$, $\ldots$, f$_{n}$}
%DepElim(b, p){f$_{1}$, $\ldots$, f$_{n}$} : p b := (* TODO recursively eliminate product *)

%Eta(A) := $\lambda$(a : A).a
%Eta(B) := (* TODO recursive eta *)
%\end{lstlisting}

\chapter{Related Work}

% TODO whatever else isn't here yet, and some of this might be factored out or partially factored out---all papers, including survey, plus generals

\section{Programs}

\subsection*{Program Refactoring} 

Refactoring~\cite{Mens:2004:SSR:972215.972286}.

\subsection*{Program Repair} 

% From PUMPKIN PATCH, unchanged

Adapting proofs to changes is essentially program repair
for dependently typed languages. 
Program repair tools for 
languages with non-dependent type 
systems~\cite{Pei:2014:APR:2731750.2731779, Long:2016:APG:2837614.2837617, Le:2017:SSS:3106237.3106309, Mechtaev:2016:ASM:2884781.2884807, Monperrus2015} 
may have applications in the context of a dependently typed language.
Similarly, our work may have applications within program repair in these languages:
Future applications of our approach may repurpose it to repair programs for functional languages.

\subsection*{Ornaments}

% From PUMPKIN PATCH, unchanged

Ornaments~\cite{Dagand17jfp, Williams:2014:OP:2633628.2633631}
separate the computational and logical components of a datatype, and may
make proofs more resilient to datatype changes.

\subsection*{Programming by Example}

% From PUMPKIN PATCH, unchanged

Our approach generalizes an example that the programmer provides.
This is similar to programming by example, a subfield of 
program synthesis~\cite{DBLP:journals/ftpl/GulwaniPS17}. 
This field addresses different challenges in different logics,
but may drive solutions to similar problems in a dependently typed language.

\subsection*{Differencing \& Incremental Computation}

% From PUMPKIN PATCH, unchanged

Existing work in differencing and incremental computation may help 
improve our semantic differencing component.
Type-directed diffing~\cite{Miraldo:2017:TDS:3122975.3122976}
finds differences in algebraic data types.
Semantics-based change impact analysis~\cite{Autexier:2010:SCI:1860559.1860580} models semantic differences
between documents.
Differential assertion checking~\cite{differential-assertion-checking-2} analyzes different
versions of a program for relative correctness with respect to a specification.
Incremental $\lambda$-calculus~\cite{Cai:2014:TCH:2594291.2594304} introduces a general model for program changes.
All of these may be useful for improving semantic differencing.

\section{Proofs}

\subsection*{Proof Reuse}

% From PUMPKIN PATCH, unchanged

Our approach reimagines the problem of proof reuse in the context of proof automation.
While we focus on changes that occur over time, traditional proof reuse techniques can help
improve our approach.
Existing work in proof reuse focuses on transferring proofs between isomorphisms,
either through extending the type system~\cite{Barthe:2001:TIP:646793.704711} or through an automatic method~\cite{Magaud2002}.
This is later generalized and implemented in Isabelle~\cite{Huffman2013} and Coq~\cite{ZimmermannH15, tabareau:hal-01559073};
later methods can also handle implications. 
%Transfer tactics apply these functions but do not infer them, while our approach
%infers these functions but does not apply them.
Integrating a transfer tactic with a proof patch finding tool will create an end-to-end
tool that can both find patches and apply them automatically.

Proof reuse for extended inductive types~\cite{Boite2004} adapts proof obligations
to structural changes in inductive types. Later work~\cite{Mulhern06proofweaving} proposes a method
to generate proofs for new constructors. These approaches may be useful when extending the differencing
component to handle structural changes. Existing work in theorem reuse and proof generalization~\cite{Felty1994, pons00, Johnsen2004} abstracts existing proofs for reusability, and may be useful
for improving the abstraction component.
Our work focuses on the components critical to searching for patches; these complementary approaches
can drive improvements to the components.

\subsection*{Proof Evolution}

% From PUMPKIN PATCH, unchanged

There is a small body of work on change and dependency management for verification,
both to evaluate impact of potential changes and maximize reuse~\cite{873647, Autexier:2010:CMH:1986659.1986663}
and to optimize build performance~\cite{Celik:2017:IRP:3155562.3155588}.
These approaches may help isolate changes, which is necessary to identify future benchmarks, integrate
with CI systems, and fully support version updates.

\subsection*{Proof Refactoring}

\subsection*{Proof Repair}

\subsection*{Proof Design}

% From PUMPKIN PATCH, unchanged:

Existing proof engineering work addresses brittleness
by planning for changes~\cite{proof-eng} and designing theorems and proofs that make maintenance less of an issue.
Design principles for specific domains (such as formal metatheory~\cite{Aydemir2008, Delaware2013POPL, Delaware2013ICFP})
can make verification more tractable. CertiKOS~\cite{certikos} introduces the idea of a deep specification to
ease verification of large systems.
These design principles and frameworks are complementary to our approach.
Even when programmers use informed design principles,
changes outside of the programmer's control can break proofs;
our approach addresses these changes.

\subsection*{Proof Automation}

% From PUMPKIN PATCH, unchanged:

We address a missed opportunity in proof automation for ITP: searching
for patches that can fix broken proofs.
This is complementary to existing automation techniques. Nonetheless, there is a wealth
of work in proof automation that makes proofs more resilient to change.
Powerful tactics like \lstinline{crush}~\cite{chlipala:cpdt} can make
proofs more resilient to changes. 
Hammers like Isabelle's sledgehammer~\cite{Blanchette2013} can make proofs agnostic to some low-level changes.
Recent work~\cite{coqhammer} paves the way for a hammer in Coq.
Even the most powerful tactics cannot address all changes;
our hope is to open more possibilities for automation.

Powerful project-specific tactics~\cite{chlipala:cpdt, Chlipala2013} can help prevent low-level maintenance tasks.
Writing these tactics requires good engineering~\cite{Gonthier2011} and domain-specific knowledge,
and these tactics still sometimes break in the face of change.
A future patching tool may be able to repair tactics; the debugging process
for adapting a tactic is not too dissimilar to providing an example to a tool.

Rippling~\cite{rippling} is a technique for automating inductive proofs that uses restricted rewrite rules to
guide the inductive hypothesis toward the conclusion; this may guide improvements to the
differencing, abstraction, and specialization components.
The abstraction and factoring components address specific classes of unification problems;
recent developments to higher-order unification~\cite{Miller:2012:PHL:2331097} may help
improve these components.
Lean~\cite{selsam:lean} introduces the first congruence closure algorithm for dependent type theory that
relies only on the Uniqueness of Identity Proofs (UIP) axiom. While UIP is not fundamental to Coq,
it is frequently assumed as an axiom; when it is, it may be tractable to use a similar algorithm to improve the tool.

GALILEO~\cite{bundyreasoning} repairs faulty physics theories
in the context of a classical higher-order logic (HOL); there is preliminary work extending this
style of repair to mathematical proofs. 
Knowledge-sharing methods~\cite{tgck-cicm14} can adapt some proofs across different representations of HOL.
These complementary approaches may guide extensions to support decidable domains and classical logics.

\subsection*{Transport}

\subsection*{Parametricity}

\subsection*{Refinement}


\chapter{Conclusions \& Future Work}
\label{chapt:conclusions}

Through a combination of differencing and proof term transformations,
my proof repair tool suite can extract, generalize, and apply the information that a change carries to fix proofs broken by the same change.
Proof repair can save and in fact already has saved work for proof engineers relative to reference manual repairs in practical use cases.
And so proof repair is reason to believe that verifying a modified system should often, in practical use cases, be easier than verifying the original the first time around,
even when proof engineers do not follow good development processes,
or when change occurs outside of proof engineers' control.

Do not just take my word for it.
Consider a recent article by a proof engineer saying just this (emphasis mine): % TODO link: https://galois.com/blog/2020/12/proofs-should-repair-themselves/

\begin{quote}
We have \textit{reason to think} such proof repair is tractable. Rather than trying to synthesize a complete proof from nothing---a problem known to be immensely difficult---we 
start from a correct proof of fairly similar software. We will be attempting proof reconstruction \textit{within a known neighborhood}.
\end{quote}
The proof engineer credited my proof repair work on Twitter, % TODO link to Twitter
but noted that there ought to be much more work in this space.

I agree.
Actually, I want to take that a step further:
proof repair is just a small piece of what will carry us to the next era of proof engineering.
And that era will be one in which programmers of all skill levels across all domains can develop and maintain verified systems---the era of
\textit{proof engineering for all}.

\section*{The Next Era: Proof Engineering for All}

% 2030? could do decades for each of this lol. I guess only if energy
% Do just 2-3 sentences per project.

Proof engineering has come a long way, but it is still accessible mostly to experts, and perhaps the occasional practitioner.
Proof repair has made proof engineering easier for experts, and a bit easier for practitioners.
But there is a lot more that we as a community can do to bring proof engineering to all: not just experts and practitioners,
but also software engineers and potential users from in other domains.

I conclude with a discussion of 12 future project ideas building up to the next era of proof engineering for all.
If any of these ideas inspire you, please work with me to bring them to life.

\subsection*{Experts}

In the future, maintaining proofs ought to be seamless for expert proof engineers.
Ideally, this means technologies---like proof repair---that automate everything but the creative parts of maintenance tasks,
leaving the creativity to the experts.
But the maintenance technologies of the future ought to reach all proof assistants,
be much more powerful,
and produce natural proof scripts in the end.

\paragraph{All Proof Assistants}
Proof repair ought to be accessible across all proof assistants.
The techniques from this thesis should be simple to apply to proof assistants with similar foundations to Coq, like Agda,
and potentially Lean.\footnote{Lean assumes an axiom that is incompatible with univalence.
It is not quite clear to me what that assumption would mean for implementing the \toolnamec transformation in Lean.
Everything else should carry over.}
But Isabelle/HOL, for example, lacks proof terms and is based on a classical logic.
One path toward proof repair for Isabelle/HOL may be to reify proof terms using
Isabelle/HOL-Proofs, apply a transformation based on the Transfer~\cite{Huffman2013} package for proof reuse, and then (as in \toolnamec) decompile those terms to automation that does 
not apply Transfer or refer to the old datatype.
Similar approaches may work for other proof assistants.

\paragraph{More Power} 
Proof repair ought to be much more powerful than it is right now.
%For example, the repair tools of the present have only limited support for proof assistant version updates;
%the repair tools of the future ought to run automatically in response to version updates, and ought to support fundamental changes in the proof term
%or proof script language.
The repair tools of the future ought to run automatically in response to proof assistant version updates.
%, and ought to support
%fundamental changes in the proof term or proof script language.
They ought to break down large changes into smaller pieces---perhaps by drawing on work in change and 
dependency management~\cite{873647, Autexier:2010:CMH:1986659.1986663, Celik:2017:IRP:3155562.3155588} to identify changes, then use the factoring transformation
to break those changes into smaller parts.
And they ought to support an even broader and more practical class of changes than they do now,
like arbitrary relations.

\paragraph{Natural Proofs}
Proof repair tools ought to produce repaired proof scripts that are natural for expert proof engineers,
regardless of style.
Toward this end, I have just begun a promising project with \kl{RanDair} and some collaborators at UMass Amherst
on integrating the prototype decompiler with the machine learning proof synthesis tool TacTok~\cite{10.1145/3428299} to rank tactic hints.
Producing more natural proofs than the prototype decompiler using fixed training data seems feasible.
More difficult---but desirable---is to train the decompiler to produce scripts that match the style of the proof engineer using the tool.

\subsection*{Practitioners}

In the future, developing and maintaining proofs ought to be much easier for practitioners to use.
Proof engineering is just starting to reach practitioners, and my proof repair tools have been a small part of that.
Workflow integration has been central to that effort, but there is a long way to go.
The repair tools of the future ought to include much more automation,
integrate smoothly into IDEs and CI tools,
and continually improve in response to feeedback from user studies.

\paragraph{More Automation}
Proof repair tools ought to include much more automation.
They ought to elegantly support repair over large libraries when many changes occur at once,
while imposing little additional effort on the proof engineer.
They ought to be simple to extend with new optimizations, all while preserving correctness.
One promising path to these ends is integrating the \toolnamec transformation with \textit{e-graphs}~\cite{egraph1},
a data structure %that is used in the constraint solver and rewrite system communities 
for managing equivalences
built with these kinds of problems in mind.
E-graphs were recently adapted to express path equality in cubical~\cite{egraph6}---a perfect fit for the \toolnamec transformation.
E-graphs in other proof assistants, like those in Lean~\cite{selsam:lean}, could help with similar automation for repair tools for other proof assistants.

\paragraph{Smooth Integration}
Proof repair tools ought to integrate smoothly into IDEs like Proof General~\cite{proofgeneral},
or into continuous integration (CI) systems like Travis~\cite{travis}.
CI support hinges on the ability to break large changes into smaller pieces.
At the level of the IDE, recording changes during development may help circumvent this problem.
The infrastructure from the \textsc{REPLica} user study may be a good start for IDE integration.
Program repair tools like CatchUp!~\cite{Henkel:2005:CCR:1062455.1062512} with existing IDE integration may provide inspiration
for both infrastructure and user experience.

\paragraph{User Feedback} 
Proof repair tools ought to continually adapt to feedback from the proof engineers who use them.
This means user studies not just of proof engineers using proof assistants (like the \textsc{REPLica} user study),
but also of proof engineers using proof repair tools.
The same principle applies to other proof engineering tools.
The \textsc{REPLica} user study paper discusses many important challenges of and potential solutions to conducting user studies
of proof engineers, and can serve as inspiration for the design.

\subsection*{Software Engineers}

In the future, proof engineering ought to be accessible to software engineers.
Even with good proof engineering technologies, it is not always economically feasible or even desirable for software engineers to formally verify 
an entire system using a proof assistant.
This makes a strong case for \textit{mixed methods verification}:
verification using multiple techniques while guaranteeing that their composition preserves correctness.
I advocated for this in the survey paper~\cite{PGL-045}, and I implemented one case of this at Galois: using
\sysnamelong to help a proof engineer interoperate between a constraint solver and Coq.
The repair tools of the future ought to similarly integrate with tools familiar to software engineers,
help software engineers infer specifications,
and assist software engineers in redesiging code for verification.

\paragraph{Familiar Tools} A continuum from testing to verification, tools to help with that.

\paragraph{Specification Inference} Analysis to infer specs (TA1).

\paragraph{Tool-Assisted Redesign} Tool-assisted development to follow good design principles for verificattion (James Wilcox conversation, final REPLICA takeaway).

\subsection*{New Domains}

Unifying theme: collaboration, new abstractions for new domains). Some examples:

\paragraph{Machine Learning} Fairification \& other ML correctness properties. Some stuff here but more.

\paragraph{Cryptography} Lots of stuff here but not thinking broadly enough. What about cryptographic proof systems? ZK and beyond. Recall email thread.

\paragraph{Something Else} Look for more in survey paper, email, DARPA TAs, Twitter. Healthcare perhaps? % https://twitter.com/TaliaRinger/status/1373747841944883201



%% Acknowledgments
\begin{acks}                            %% acks environment is optional
                                        %% contents suppressed with 'anonymous'
  %% Commands \grantsponsor{<sponsorID>}{<name>}{<url>} and
  %% \grantnum[<url>]{<sponsorID>}{<number>} should be used to
  %% acknowledge financial support and will be used by metadata
  %% extraction tools.
% TODO Gaetan for sure
% TODO call out Anders and Conor specially, given depth of feedback
% TODO Jasper
This paper has really been a community effort.
Anders M\"ortberg and Conor McBride gave us \textit{hours} worth of detailed feedback that was instrumental to writing this paper.
Nicolas Tabareau helped us understand the need to port definitional to propositional equalities.
Valentin Robert gave us feedback on usability that informed tool design.
Ga\"{e}tan Gilbert, James Wilcox, and Jasper Hugunin all at some point helped us write Coq proofs;
let this be a record that we owe Ga\"{e}tan a beer, and we owe James boba.

And of course, we thank our shepherd Gerwin Klein, and we thank all of our reviewers.
We got other wonderful feedback on the paper from 
Cyril Cohen, Tej Chajed, Ben Delaware, Jacob Van Geffen, Janno, James Wilcox, Chandrakana Nandi, 
Martin Kellogg, Audrey Seo, James Decker,
Ben Kushigian, John Regehr, and Justus Adam.
The Coq developers have for \textit{years} given us frequent and efficient feedback on plugin APIs for tool implementation.
The programming languages community on Twitter (yes, seriously) has also been essential to this effort.
Especially during a pandemic. 
And we'd like to extend a special thank you to Talia's mentor, Derek Dreyer.

We have also gotten a lot of feedback on future ideas that we are excited to pursue.
Carlo Angiuli has helped us understand some beautiful theory beneath our implementation 
(spoiler: we believe the analogy connecting \lstinline{DepConstr} and \lstinline{DepElim} to constructors and eliminators
has formal meaning in terms of initial algebras), and we are excited to integrate these insights into future papers
and use them to generalize our insights.
Alex Polozov helped us sketch out ideas for future work with the decompiler.
And we got wonderful feedback on e-graph integration for future work from 
Max Willsey, Chandrakana Nandi, Remy Wang, Zach Tatlock, Bas Spitters, Steven Lyubomirsky, Andrew Liu, Mike He, Ben Kushigian, 
Gus Smith, and Bill Zorn.


This material is based upon work supported by the \grantsponsor{GS100000001}{National Science Foundation}{http://dx.doi.org/10.13039/100000001} Graduate Research Fellowship under Grant No.~\grantnum{GS100000001}{DGE-1256082}. Any opinions, findings, and conclusions or recommendations expressed in this material are those of the authors and do not necessarily reflect the views of the National Science Foundation. % TODO PEO?
\end{acks}


%% Bibliography
\balance
\bibliography{paper.bib}


%% Appendix
%\appendix
%\section{Appendix}

%Text of appendix \ldots

\end{document}
