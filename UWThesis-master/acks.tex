
% ACKS
\cleardoublepage
\pdfbookmark[1]{Acknowledgments}{acknowledgments}
\chapter*{Acknowledgments}
I've always believed the acknowledgments section to be one of the most important parts of a paper.
But there's never enough room to thank everyone I want to thank.
Now that I have the chance, though, I'm having trouble figuring out where I should begin.

\paragraph{Lymor}
In the past, when I had trouble figuring out where to begin, I just followed my sister, \intro{Lymor}.
I followed her to Maryland for college, and when I got there, she told me to do research.
I had no idea what ``doing research'' meant, but she told me that if I wanted to go to graduate school, I needed to do it.
Honestly, I had no idea what graduate school was, either, but I did know it was something she was doing, and it sounded cool.
So I did research.

This is the kind of thing that I think people often don't get to write about in acknowledgments sections.
I don't think I could have had the opportunity to write this if not for her.
How else was I supposed to know what research was, or that I should do it?
That it would help me get into graduate school?
Draw me to the world of programs and proofs?
Instill in me the dream to become a professor?
I start at Illinois in October, and I owe a lot of that to her.

So thanks, \kl{Lymor}, for being such an amazing big sister $\heartsuit \heartsuit \heartsuit$.

\paragraph{Undergraduate Mentors \& Advisors}
Once I got to Maryland, I did research with two wonderful math professors:
\intro{Kasso Okoudjou} and \intro{Larry Washington}.
These experiences were fun forays into the worlds of linear algebra and cryptology.
They also helped me build the skills I needed to succeed in graduate school.

It took a couple of years at Maryland before I found my way from math into computer science (CS).
I'm honestly thankful that I was afraid enough of physics and statistics to instead take CS to satisfy a breadth requirement for my math degree!
But I'm also thankful that the undergraduate math advisor \intro{Ida Chan} talked me into taking the more advanced sequence, so that it wasn't a dead end,
and I could major in it later on. And I'm thankful that, shortly after, I had a chance to attend the second ever
\intro{Google} Computer Science Summer Institute (CSSI), where for the first time I felt empowered---so much that when I returned to Maryland,
I decided to try to minor in CS. And I'm thankful that when I tried to minor in CS,
\intro{Brandi K. Adams} talked me into picking it up as a second major instead.

So thanks, \kl{Kasso} and \kl{Larry}, for the wonderful research opportunities.
And thanks, \kl{Ida}, \kl{Brandi}, and everyone at \kl{Google} CSSI, for giving me the confidence to major in CS.

\paragraph{Jeff}
A pivotal semester for me at Maryland came during my senior year, when I took \intro{Jeff Foster}'s advanced undergraduate programming languages class.
That whole class was amazing and got me super into programming languages.
Like Jeff spent one of the lectures talking about the Curry-Howard Correspondence,
which relates programs and proofs to one another.
I thought it was the coolest thing ever, even though I didn't really understand it that well at the time.
It must have planted a seed or something, because I kept coming back to it again and again years later.
And this thesis really gets to the soul of Curry-Howard, treats proofs just like the programs they really are,
transforms them and evolves them over time.
I'm obsessed.

Jeff is more than a great teacher, though---he's also a really great mentor.
Like, the once-in-a-lifetime kind of mentor you're lucky to meet, who gives you selfless advice and helps you get to know \textit{yourself} better,
who is patient and kind and teaches you all of the things you'd never even known to ask.
It was thanks to Jeff that, one day in 2013, I found myself at a programming languages conference in Seattle talking to my eventual advisor.
And it was thanks to Jeff that I knew anything at all about how to apply to and choose a graduate school.
I actually spoke to Jeff every so often all throughout graduate school up to the very 
end---even talking to him about the faculty job search.

So thanks, \kl{Jeff}, for your fantastic teaching, advocacy, connections, and selfless advice over the last decade.
Your students are lucky to learn from and work with you.

\paragraph{Industry Mentors \& Managers}
During my time at Maryland, I worked as a software engineering intern at two companies:
\intro{Carr Astronautics} and \intro{Amazon}.
I continued to work as a software engineer at Amazon for three years after I graduated.
My adventures as a software engineer at these two companies helped me learn to motivate useful research problems,
build good tools, and collaborate with other people---skills that were essential to this thesis work.

At Amazon, my mentor \intro{Musachy Barroso} and my manager \intro{Ernesto Gonzales} made my experiences so much fun
that it was hard to leave. So it's maybe not surprising that I found myself back there for a research internship with
\intro{Serdar Tasiran} and \intro{Daniel Schwartz-Narbonne} later on---and that was super valuable, too.

So thanks, all of my mentors at \kl{Carr Astronautics} and \kl{Amazon}---especially \kl{Musachy} and \kl{Ernesto}---for
the amazing real life software engineering experiences. And thanks, \kl{Serdar} and \kl{Daniel}, for the adventures in industrial research.

\paragraph{Dan}
After three years at Amazon, I left to work with my advisor \intro{Dan Grossman} at the University of Washington.
Dan is the best advisor.
I'm really happy that I had a chance to work with him.
It was a good work dynamic for me, since he always gave me the autonomy that I needed to explore the problems that I love.
He did help me push myself a few times---just only when I really, really needed to be pushed,
and never any more than I needed.

Dan is also maybe the wisest and most patient person I've ever met, and I mean that.
All through graduate school, he always gave me great advice---advice that I on many occasions rejected at first because I didn't want to hear it.
When I rejected his advice, he just kind of patiently talked me through it.
He recognized that I'm my own person, and so accepted that sometimes I just wouldn't listen in the end.
But I often did listen in the end.
And when I did, I'd often find myself looking back months later, thinking ``oh yeah, he was right.''

This thesis is a good example of that.
Dan told me many times that I should actually put a lot of effort into this thesis, because people actually do read these,
and they can be really useful.
I fought that advice because I'd always thought of the thesis as some weird academic hazing ritual.
But this past week I've spoken to potential students at my next job, and invariably I've ended up sending them a copy of this thesis.
It's a really useful source of my work that has historical context and background information.
So yeah, he was right. I apologize in advance if his future students have to read this, now.

Above all, though, it's important to note that Dan thought of the acronyms for three of my tools:

\begin{enumerate}
\item \textsc{AUDACIOUS}: Android User-Driven Access Control In Only User Space,
\item \textsc{PUMPKIN PATCH}: Proof Updater Mechanically Passing Knowledge Into New Proofs, Assisting The Coq Hacker, and
\item \textsc{DEVOID}: Dependent Equivalences Via Ornamenting Inductive Definitions.
\end{enumerate}
This gave me a endless stream of jokes to tap into for my research talks all throughout graduate school.

So thanks, \kl{Dan}, for the patience, wisdom, and dad jokes. I feel ready to be a professor and a dad.\footnote{I'm not actually going to be a dad.}

\paragraph{PLSE}
The most amazing thing about my graduate school experience by far was being a part of the \intro{PLSE} lab.
I didn't realize what an absolute blessing it was to be a part of PLSE until one day the pandemic hit and the lab suddenly closed.
If I could see everyone in PLSE right now, I would hug all of them.
What an amazing group of people, always there to support each other,
to read papers, to give feedback, to inspire new ideas, to chat,
to be just amazing friends---I love PLSE.

One of the first people that I met in PLSE was \intro{Chandrakana Nandi}, and she was there for me throughout my entire graduate school journey.
It's not just that she gave me feedback on basically all of my papers and talks (she did).
But also, she did things like---she sent me donuts when I was working on my PLDI paper recently.
She helped me through the hardest year of my life.
She even visited me in the hospital, once.
She's just great.
The most genuine and kind friend I could have had by my side for this journey. Really.

You know who else from PLSE is amazing? \intro{Zach Tatlock}.
Zach worked with me a survey paper, taught me Coq, and inspired the original problem that got me interested in this thesis work.
That's all fine and dandy, but he had to take that a step further and literally spend an hour of his week every week
for an entire year helping me through a really hard time.
Just because he's a kind and caring person, not because he got anything out of it.
It's not just that I wouldn't have made it through graduate school without Zach's help;
I honestly don't think I would \textit{be here today} without Zach's help.

The same goes for \intro{Alex Polozov}, who overlapped with me in PLSE for just one year,
but ended up being one of the best friends I could have ever asked for.
Alex gave me really great advice when life was really hard.
Some of that advice saved my life.
That's cool; you don't find friends like that just anywhere.

\intro{Sarah Chasins} also gave me incredible advice all throughout graduate school; she is honestly the best listener I have ever met.
My first year mentor \intro{John Toman} humored my probably very weird early graduate school questions.
My cohort---\kl{Chandra}, \intro{Chenglong Wang}, \intro{Amanda Swearngin}, \intro{Jared Roesch}, 
\intro{Sam Elliott}, and \intro{Bill Zorn}---was so much fun to share this journey with.
My seniors---especially \kl{Sarah}, \kl{Alex Polozov}, \intro{Doug Woos}, \intro{James Wilcox}, \intro{Stuart Pernsteiner}, 
\intro{Konstantin Weitz}, and \intro{Joe Redmond}---were such wonderful role models and friends to me throughout this journey.
My juniors---especially \intro{Max Willsey}, \intro{Martin Kellog}, \intro{Alex Sanchez-Stern},
\intro{Gus Smith}, % Gus Smith
\intro{Ben Kushigian}, % Ben Kushigian
\intro{Steven Lyubomirsky}, % Steven Lyubomirsky
\intro{Jacob Van Geffen}, % Jacob Van Geffen
\intro{Marisa Kirisame}, % Marisa Kirisame
\intro{Remy Wang}, % Remy Wang
\intro{Melissa Hovik},
\intro{Rashmi Mudduluru},
\intro{Sorawee Porncharoenwase},
and \intro{Krzysztof Drewniak}---helped me so much throughout, too.

Above all, though, my students and research mentees---\intro{Jasper Hugunin}, \intro{Taylor Blau}, \intro{RanDair Porter}, and \intro{Nate Yazdani}---brought
so much light and joy to my graduate school experience.
They are the reason I'm so excited to become a professor.

So thank you so much \kl{Chandra}, \kl{Zach}, \kl{Alex Polozov}, \kl{Sarah}, \kl{John}, \kl{Chenglong}, \kl{Sam}, \kl{Amanda}, \kl{Jared}, \kl{Doug}, \kl{James},
\kl{Stuart}, \kl{Konstantin}, \kl{Joe}, \kl{Max}, \kl{Martin}, \kl{Alex Sanchez-Stern}, \kl{Gus}, \kl{Ben},
\kl{Steven}, \kl{Jacob}, \kl{Marisa}, \kl{Remy}, \kl{Melissa}, \kl{Rashmi}, \kl{Oak}, \kl{Krzysztof},
\kl{Jasper}, \kl{Taylor}, \kl{RanDair}, \kl{Nate},
and every single person I've ever overlapped with in \kl{PLSE}.
I miss all of you, and I wish all of you nothing but joy and success for the rest of your lives.
Please come visit me in Illinois!

\paragraph{My Friends (WIP)}
I went through a really difficult time during graduate school,
and there is a small group of friends whose help was life-saving.
Those friends are 
\intro{Roy Or-El},
\intro{Vikram Iyer},
\intro{Esther Jang},
\intro{Anna Kornfeld Simpson},
\intro{Mark Velednitsky},
\intro{Sarah Chasins},
the entire \intro{UCSD ProgSys} lab, % TODO list
\intro{Marcela Mendoza},
\intro{Anne Spencer Ross},
\intro{Laura Chick}, % Maxim?
and any fuzzy animals\footnote{May Boris' memory be a blessing.} I interacted with that year.
You are the best.

WIP: Tony Chick. David Lasky. Ezgi. Erica. Danielle.
Chris Maines.
Care committee takeover.

And I cannot begin to imagine my life over these years had I not met
\intro{Qi Cheng}, \intro{Ellie Berry}, \intro{Mer Joyce}, \intro{Wade Gordon}, \intro{Jasper Tran O'Leary},
and \intro{Misha Kolmogorov}. % TODO more
Nothing but love to all of you.

\paragraph{My Coauthors (WIP)}
\intro{Karl Palmskog}.
\intro{Alex Sanchez-Stern},

\paragraph{My Community (WIP)}
Cyril Cohen, Tej Chajed, Ben Delaware, Janno,
James Decker, Bas Spitters,
Derek Dreyer,
Alexandra Silva,
the Coq community (Emilio J. Gallego Arias, Enrico Tassi, Ga\"{e}tan Gilbert, Maxime D\'{e}n\`{e}s,
Matthieu Sozeau, Vincent Laporte, Th\'{e}o Zimmermann, Jason Gross, Nicolas Tabareau, Cyril Cohen, Pierre-Marie P\'{e}drot, Yves Bertot, Tej Chajed, Ben Delaware, Janno),
\intro{Michael Shulman} (feels like univalence, like categorical coherence, like an endofunctor).
Valentin Robert.
\intro{Anders M\"ortberg}, Conor McBride, \intro{Carlo Angiuli}, Bas Spitters, Jon Sterling,
PL Twitter, other Twitter friends,
Edward Z. Yang, Robert Rand.
Matthew Dwyer at Virginia, Nathanael from Twitter, Matt Might, UVA folks, Quinn Wilton (all for medical device ideas).
Jonathan Aldrich.
David van Horn.
Michael Hicks. % TODO probably many more
Rebecca Turner from Twitter.
Ymir Vigfusson from Twitter (typo).
Benjamin Lipp from Twitter (typo).
Daniel-Nikpayuk from Twitter (typo).

\paragraph{My Writing Buddy (WIP)}
Michelle Lee.

\paragraph{My Running Buddies (WIP)}
Club Northwest.

\paragraph{My Puppy (WIP)}
Belle.

\paragraph{My Grandparents (WIP)}

\paragraph{My Parents (WIP)}

% https://twitter.com/16kbps/status/1379266732611796993?s=20 ack this person (Rebecca Turner)
% Does Jeff Dean get an ack? Ask him I guess. Timnit Emily Meg and so on?
% survey paper acks
% See Twitter and FB
