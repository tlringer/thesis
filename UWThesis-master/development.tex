\section{Proof Development}
\label{sec:mot-dev}

\begin{figure}
   \lstinputlisting[firstline=1, lastline=3]{often/listswap.tex}
\caption{The \lstinline{list} datatype in Coq, from the Coq standard library.}
\label{fig:list}
\end{figure}

The first thing that we need is our \lstinline{list} datatype.
A \lstinline{list} in Coq (Figure~\ref{fig:list}) is an inductive datatype that is either empty (the \lstinline{nil} constructor), or the result
of placing an element in front of another \lstinline{list} (the \lstinline{cons} constructor).

\begin{figure}
\begin{lstlisting}
TODO
\end{lstlisting}
\caption{The \lstinline{length} function for \lstinline{list} in Coq, from the Coq standard library.}
\label{fig:length}
\end{figure}

Also inside of the standard library is the \lstinline{length} function (Figure~\ref{fig:length}).
The \lstinline{length} of the empty list \lstinline{nil} is \lstinline{0}, and the length of any other list
is just the successor of (one plus) the result of recursively calling \lstinline{length} on everything but the first element of the list.	

\paragraph{Program}
then zip.

\paragraph{Specification}
then the spec, choice of specs, note here.

\paragraph{Proof}
finally this, interactive process, automation, underlying proof term with highlighting to corresponding tactics,
how Coq checks this for the expected type.

