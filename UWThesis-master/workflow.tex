\subsection{The Workflow}
\label{sec:mot-workflow}

Before we can write our small verified program, we need some datatypes and functions that we can find in the Coq standard library (Section~\ref{sec:prelims}).
We can then use these datatypes and functions to write the \lstinline{zip} function and show that it preserves its length (Section~\ref{sec:verif}).

\subsubsection{Preliminaries}
\label{sec:prelims}

To prove that the list zip function preserves its length,
we need the \lstinline{list} datatype and the \lstinline{length} function.
To write the \lstinline{length} function, we need the \lstinline{nat} datatype of unary natural numbers.
All of these can be found in the Coq standard library.

\begin{figure}
\begin{minipage}{0.30\textwidth}
\begin{lstlisting}
Inductive nat :=
| O : nat
| S : nat $\rightarrow$ nat.
\end{lstlisting}
\end{minipage}
\hfill
\begin{minipage}{0.68\textwidth}
\begin{lstlisting}
nat_rect :
  $\forall$ (P : nat $\rightarrow$ Type),
    P O $\rightarrow$ (* base case *)
    ($\forall$ (n : nat),
      (* inductive case *)
      P n $\rightarrow$ P (S n)) $\rightarrow$
    $\forall$ (n : nat), P n.
\end{lstlisting}
\end{minipage}
\caption{The type of natural numbers \lstinline{nat} in Coq defined inductively by its two constructors (left), and the type of the corresponding eliminator or induction principle \lstinline{nat_rect} that Coq generates (right).}
\label{fig:nat}
\end{figure}

Each of \lstinline{nat} and \lstinline{list} in Gallina is an \textit{inductive type}:
it is defined by its \textit{constructors} (ways of constructing a term with that type).
A \lstinline{nat} (Figure~\ref{fig:nat}, left), for example,
is either \lstinline{0} or the successor \lstinline{S} of another \lstinline{nat};
these are the two constructors of \lstinline{nat}.

Every inductive type in Gallina comes equipped with an \textit{eliminator} or induction principle
that the proof engineer can use to write functions and proofs about the datatype.
For example, the eliminator for \lstinline{nat} (Figure~\ref{fig:nat}, right) is the standard induction principle for natural numbers,
which Coq calls \lstinline{nat_rect}. % TODO footnote about nat_ind, nat_rec, nat_rect
This eliminator states that a statement \lstinline{P} (called the inductive \textit{motive}) about the natural numbers
holds for every number if it holds for \lstinline{O} in the base case and, in the inductive case,
assuming it holds for some \lstinline{n}, it also holds for the successor \lstinline{S n}.

\begin{figure}
\begin{minipage}{0.44\textwidth}
\begin{lstlisting}
Inductive list {T : Type} :=
| nil : list T
| cons :
    T $\rightarrow$ list T $\rightarrow$ list T.
\end{lstlisting}
\end{minipage}
\hfill
\begin{minipage}{0.54\textwidth}
\begin{lstlisting}
list_rect :
  $\forall$ {T : Type} (P : list T $\rightarrow$ Type),
    P nil $\rightarrow$ (* base case *)
    ($\forall$ (t : T) (tl : list T),
      (* inductive case *)
      P tl $\rightarrow$ P (cons t tl)) $\rightarrow$
    $\forall$ (l : list T), P l.
\end{lstlisting}
\end{minipage}
\caption{The type of polymorphic lists \lstinline{list} in Coq defined inductively by its two constructors (left), and the type of the corresponding eliminator or induction principle \lstinline{list_rect} that Coq generates (right). The curly brace notation means that the type parameter \lstinline{T} is implicit in applications.}
\label{fig:list}
\end{figure}

 % TODO footnote about nat_ind, nat_rec, nat_rect

A \lstinline{list} (Figure~\ref{fig:list}, left) is similar to a \lstinline{nat}, 
but with two differences: \lstinline{list} is polymorphic over some type \lstinline{T} (so we can have a list of natural numbers,
for example, written \lstinline{list nat}), and the second constructor adds a new element of the type \lstinline{T} to the front of the list.
Otherwise, \lstinline{list} also has two constructors, \lstinline{nil} and \lstinline{cons}, where \lstinline{nil} represents the empty list,
and \lstinline{cons} sticks a new element in front of any existing list.
Similarly, the eliminator for \lstinline{list} (Figure~\ref{fig:list}, right) looks like the eliminator for \lstinline{nat} 
but with an argument corresponding to the parameter \lstinline{T} over which \lstinline{list} is polymorphic,
and with an additional argument corresponding to the new element in the inductive case.

One interesting thing about the types of these eliminators \lstinline{list_rect} and \lstinline{nat_rect}
include universal quantification over all inputs, written $\forall$.
Gallina's type system is expressive enough to include universal quantification over inputs---I will explain how
in Section~\ref{sec:mot-theory}.

\begin{figure}
\begin{minipage}{0.42\textwidth}
\begin{lstlisting}
Fixpoint length {T} l :=
  match l with
  | nil => O
  | cons t tl =>
      S (length tl)
  end.
\end{lstlisting}
\end{minipage}
\hfill
\begin{minipage}{0.54\textwidth}
\begin{lstlisting}
Definition length {T} l :=
  list_rect T (fun _ => nat)
    O
    (fun t tl (length_tl : nat) =>
      S length_tl)
    l.
\end{lstlisting}
\end{minipage}
\caption{The list \lstinline{length} function, defined both by pattern matching and recursion (left) and using the eliminator \lstinline{list_rect} (right).}
\label{fig:length}
\end{figure}

We can use these eliminators to write functions and proofs, like the \lstinline{length} function we will need for our proof development (Figure~\ref{fig:length}, right).
More standard is to write functions using pattern matching and guarded recursion, like the \lstinline{length} function from the Coq standard library (Figure~\ref{fig:length}, left).
Both of these two functions behave the same way, but the function on the left is perhaps a bit easier to understand from a traditional programming background:
the \lstinline{length} of the empty list \lstinline{nil} is \lstinline{0}, and the length of any other list
is just the successor (\lstinline{S}) of the result of recursively calling \lstinline{length} on everything but the first element of the list.
Indeed, \lstinline{list_rect}---like all eliminators in Coq---is just a constant that refers to a function itself defined using pattern matching and guarded recursion.
In fact, eliminators are equally expressive to pattern matching and guarded recursion~\cite{TODO}. % the general translation is hard and Coq sucks at eliminator gen though

For the sake of this thesis, however, I will assume \intro{primitive eliminators}: eliminators that are a part of the core syntax and theory itself,
and that do not reduce to terms that use pattern matching and guarded recursion.
Likewise, when I show Gallina code, I will always use functions that apply eliminators rather than pattern matching, like the \lstinline{length} function from Figure~\ref{fig:length}
on the right.
I remove the \lstinline{Definition} and \lstinline{Fixpoint} keywords, since everything from here on out is a \lstinline{Definition}.
To handle practical code that uses pattern matching and guarded recursion,
I preprocesss the code using a tool by my coauthor Nate Yazdani (more about this later). % TODO make this more formal, can say limitations, can say generates length above, wahtever
In the rest of the paper, I skip this preprocessing step in examples, but I describe it more in the implementation section later.

\subsubsection{List Zip Preserves Length}
\label{sec:verif}

With \lstinline{nat}, \lstinline{list}, and \lstinline{length} defined, we can now write our small verified program.
We start by writing the \lstinline{zip} program, then specify what it means to preserve its length, and then finally
write an interactive proof that shows that specification actually holds.
Coq checks this proof and lets us now that our proof is correct, so our \lstinline{zip} function is verified.

\begin{figure}
\begin{lstlisting}
zip {T$_1$} {T$_2$} (l$_1$ : list T$_1$) (l$_2$ : list T$_2$) : list (T$_1$ * T$_2$) :=
  list_rect (fun _ : list T$_1$ => list T$_2$ $\rightarrow$ list (T$_1$ * T$_2$))
    (fun _ => nil)
    (fun t$_1$ tl$_1$ (zip_tl$_1$ : list T$_2$ $\rightarrow$ list (T$_1$ * T$_2$)) l$_2$ =>
      list_rect (fun _ : list T$_2$ => list (T$_1$ * T$_2$))
        nil
        (fun t$_2$ tl$_2$ (_ : list (T$_1$ * T$_2$)) =>
          cons (t$_1$, t$_2$) (zip_tl$_1$ tl$_2$))
        l$_2$)
    l$_1$
    l$_2$.
\end{lstlisting}
\caption{The list \lstinline{zip} function, taken from an existing tool~\cite{TODO} % TODO hs-to-coq cite
and translated to use eliminators.}
\label{fig:zip}
\end{figure}

\paragraph{Program} The list \lstinline{zip} function is in Figure~\ref{fig:zip}.
It takes as arguments two lists \lstinline{l}$_1$ and \lstinline{l}$_2$ of possibly different types \lstinline{T}$_1$ and \lstinline{T}$_2$, and zips them together into
a list of pairs \lstinline{(T}$_1$\lstinline{ * T}$_2$\lstinline{)}.
For example, if the input lists are: % TODO put this in a figure tbh, like from the slides

\begin{lstlisting}
(* [1; 2; 3; 4] *)
l$_1$ := cons 1 (cons 2 (cons 3 (cons 4 nil))).

(* ["x"; "y"; "z"] *)
l$_2$ := cons "x" (cons "y" (cons "z" nil)).
\end{lstlisting}
then \lstinline{zip} applied to those two lists returns:

\begin{lstlisting}
(* [(1, "x"); (2, "y"); (3, "z")] *)
cons (1, "x") (cons (2, "y") (cons (3, "z") nil)).
\end{lstlisting}
It is worth noting that the implementation of \lstinline{zip} has to make some decision with
what to do with the extra \lstinline{4} at the end---that is, how \lstinline{zip} behaves when the input
lists are different lengths.
The decision that this implementation makes is to just ignore those extra elements.

Otherwise, the implementation is fairly standard.
If \lstinline{l}$_1$ is \lstinline{nil} (base case of the outer induction),
\lstinline{zip} returns \lstinline{nil}.
Otherwise (inductive case of the outer induction),
if \lstinline{l}$_2$ is \lstinline{nil} (base case of the inner induction),
\lstinline{zip} returns \lstinline{nil}.
If \lstinline{l}$_2$ is anything else (inductive case of the inner induction),
\lstinline{zip} combines the first two elements of each list into a pair (\lstinline{(t}$_1$\lstinline{, t}$_2$\lstinline{)}),
then sticks that in front of (using \lstinline{cons}) the result of recursively calling \lstinline{zip} on the tails of each list
(\lstinline{zip_tl}$_1$ \lstinline{tl}$_2$). % TODO a bit hard to understand probably, may need more explanation

\begin{figure}
\begin{minipage}{0.49\textwidth}
\begin{lstlisting}
Theorem zip_preserves_length :
  $\forall$ {T$_1$} {T$_2$} (l$_1$ : list T$_1$) (l$_2$ : list T$_2$),
    length l$_1$ = length l$_2$ $\rightarrow$
    length (zip l$_1$ l$_2$) = length l$_1$.
\end{lstlisting}
\end{minipage}
\hfill
\begin{minipage}{0.49\textwidth}
\begin{lstlisting}
Theorem zip_preserves_length :
  $\forall$ {T$_1$} {T$_2$} (l$_1$ : list T$_1$) (l$_2$ : list T$_2$),
    length (zip l$_1$ l$_2$) = min (length l$_1$) (length l$_2$).
\end{lstlisting}
\end{minipage}
\caption{Two possible specifications of a proof that \lstinline{zip} preserves the length of the input lists.}
\label{fig:zip-pres}
\end{figure}

\paragraph{Specification} Once we have written our \lstinline{zip} function, we can then specify what we 
want to prove about it: that the \lstinline{zip} function preserves the lengths of the inputs \lstinline{l}$_1$ and \lstinline{l}$_2$.
We do this by defining a type \lstinline{zip_preserves_length} (Figure~\ref{fig:zip-pres}, left), which in Coq we state as a \lstinline{Theorem}.
This theorem takes advantage of Gallina's rich type system to quantify over all possible input lists \lstinline{l}$_1$ and \lstinline{l}$_2$.
It says that if the lengths of the inputs are the same, then the length of the output is the same as the lengths of the inputs.
Our proof will soon show that this type is inhabited, and so this statement is true.

It is worth noting that this step of choosing a specification is a bit of an art---we have some freedom when we choose our specification.
We could just as well have chosen a different version of \lstinline{zip_preserves_length} (Figure~\ref{fig:zip-pres}, right)
that states that the length of the output is the \textit{minimum} of the lengths of the inputs (using \lstinline{min} from the Coq standard library).
This is also true for our \lstinline{zip} implementation, and in fact it is stronger---it implies the original theorem as well.
But regardless of which version we choose, we then get to the fun part of actually writing our proof.

\paragraph{Proof} As I mentioned earlier, it is possible to write proofs directly in Gallina---but this can be difficult.
Instead, it is more common to write proofs interactively using the tactic language Ltac.
Each tactic is effectively a search procedure for a proof term, given the context and goals at each step of the proof.
The way that this works is, after we state the theorem that we want to prove:

\begin{lstlisting}
Theorem zip_preserves_length :
  $\forall$ {T$_1$} {T$_2$} (l$_1$ : list T$_1$) (l$_2$ : list T$_2$),
    length l$_1$ = length l$_2$ $\rightarrow$
    length (zip l$_1$ l$_2$) = length l$_1$.
\end{lstlisting}
we then add one more word:

\begin{lstlisting}
Proof.
\end{lstlisting}
then step down past that word inside of an IDE. % TODO expand acronym? or something?
The IDE then drops into an interactive proof mode.
In that proof mode, it tracks the context of the proof so far, along with the goal we want to prove.
After each tactic we type and step past, Coq responds by refining the goal into some subgoal
and updating the context.
We continue this until no goals remain.
The survey paper~\cite{PGL-045} has a good overview of the tactic language in Coq and in other proof assistants,
plus different interfaces and IDEs for writing proofs interactively and screenshots of them in action.

\begin{figure}
\begin{lstlisting}
Theorem zip_preserves_length :
  $\forall$ {T$_1$} {T$_2$} (l$_1$ : list T$_1$) (l$_2$ : list T$_2$),
    length l$_1$ = length l$_2$ $\rightarrow$
    length (zip l$_1$ l$_2$) = length l$_1$.
Proof.
  intros T1 T2 l1.
  induction l$_1$ as [|t$_1$ tl$_1$ IHtl$_1$].
  - auto.
  - intros l2. induction l2 as [|t$_2$ tl$_2$ IHtl$_2$].
    + intros H. auto.
    + intros H. simpl. rewrite IHtl$_1$; auto.
Defined.
\end{lstlisting}
\begin{lstlisting}
zip_preserves_length :
  $\forall$ {T$_1$} {T$_2$} (l$_1$ : list T$_1$) (l$_2$ : list T$_2$),
    length l$_1$ = length l$_2$ $\rightarrow$
    length (zip l$_1$ l$_2$) = length l$_1$
:=
fun (T$_1$ T$_2$ : Type) (l$_1$ : list T$_1$) (l$_2$ : list T$_2$) =>
  list_rect
    (fun (l$_1$ : list T$_1$) =>
      $\forall$ (l$_2$ : list T$_2$),
        length l$_1$ = length l$_2$ $\rightarrow$
        length (zip l$_1$ l$_2$) = length l$_1$)
    (fun (l$_2$ : list T$_2$) _ => eq_refl)
    (fun (t$_1$ : T$_1$) (tl$_1$ : list T$_1$)
         (IHtl$_1$ :
           $\forall$ (l$_2$ : list T2),
             length tl$_1$ = length l$_2$ $\rightarrow$
             length (zip tl$_1$ l$_2$) = length tl$_1$)
         (l$_2$ : list T$_2$) =>
      list_rect
        (fun (l$_2$ : list T$_2$) =>
          length (cons t$_1$ tl$_1$) = length l$_2$ $\rightarrow$
          length (zip (cons t$_1$ tl$_1$) l$_2$) = length (cons t$_1$ tl$_1$))
        (fun (H : length T$_1$ (cons t$_1$ tl$_1$) = length nil) => eq_sym H)
        (fun (t$_2$ : T$_2$) (tl$_2$ : list T$_2$) _ (H : length (cons t$_1$ tl$_1$) = length (cons t$_2$ tl$_2$)) =>
          eq_ind_r
            (fun (n : nat) => S n = S (length tl$_1$))
            eq_refl
            (IHtl$_1$ tl$_2$ (eq_add_S (length tl$_1$) (length tl$_2$) H)))
        l2)
  l1
  l2.
\end{lstlisting}
\caption{A proof script (left) and corresponding proof term (right) in Coq that shows that the list zip function preserves its length.}
\label{fig:zip-proof}
\end{figure}

Figure~\ref{fig:zip-proof} shows a proof script for this theorem (left), along with the corresponding proof term (right).
As we can see, the proof term is quite complicated.
Thankfully, the details do not matter to us, since we can write the high-level proof script on the left instead.
Even though this proof script is still a bit manual for the sake of demonstration,
it is much simpler than the low-level proof term.

To write the proof script on the left of Figure~\ref{fig:zip-proof}, we start by stepping past \lstinline{Proof} in our IDE.
After this, our initial context (above the line) is empty, and our initial goal (below the line) is the original theorem:

\begin{lstlisting}
______________________________________(1/1)
$\forall$ {T$_1$} {T$_2$} (l$_1$ : list T$_1$) (l$_2$ : list T$_2$),
  length l$_1$ = length l$_2$ $\rightarrow$
  length (zip l$_1$ l$_2$) = length l$_1$.
\end{lstlisting}
We start this proof with the introduction tactic \lstinline{intros}:

\begin{lstlisting}
  intros T$_1$ T$_2$ l$_1$.
\end{lstlisting}
This is essentially the equivalent of the natural language proof strategy ``assume arbitrary \lstinline{T}$_1$, \lstinline{T}$_2$, and \lstinline{l}$_1$.''
That is, it moves the universally quantified arguments from our goal into our context:

\begin{lstlisting}
T$_1$ : Type
T$_2$ : Type
l$_1$ : list T$_1$
______________________________________(1/1)
$\forall$ (l$_2$ : list T$_2$),
  length l$_1$ = length l$_2$ $\rightarrow$
  length (zip l$_1$ l$_2$) = length l$_1$.
\end{lstlisting}
From this state, we can induct over the input list (choosing names for variables Coq introduces in the inductive case):

\begin{lstlisting}
  induction l$_1$ as [|t$_1$ tl$_1$ IHtl$_1$].
\end{lstlisting}
This breaks into two subgoals and subcontexts: one for the base case and one for the inductive case.
The base case:
\begin{lstlisting}
T$_1$ : Type
T$_2$ : Type
______________________________________(1/2)
$\forall$ l$_2$ : list T$_2$,
  length nil = length l$_3$ $\rightarrow$
  length (zip nil l$_2$) = length nil.
\end{lstlisting}
holds by reflexivity, which the \lstinline{auto} tactic takes care of.

In the inductive case:

\begin{lstlisting}
T$_1$ : Type
T$_2$ : Type
t$_1$ : T$_1$
tl$_1$ : list T$_1$
IHtl$_1$ : $\forall$ l$_2$ : list T$_2$,
        length tl$_1$ = length l$_2$ $\rightarrow$
        length (zip tl$_1$ l$_2$) = length tl$_1$
______________________________________(2/2)
$\forall$ l$_2$ : list T2,
  length (cons t$_1$ tl$_1$) = length l$_2$ $\rightarrow$
  length (zip (cons t$_1$ tl$_1$) l$_2$) = length (cons t$_1$ tl$_1$).
\end{lstlisting}
we again use \lstinline{intros} and \lstinline{induction}, this time to induct over \lstinline{l}$_2$.
This again produces two subgoals: one for the base case and one for the inductive case.
The base case has an absurd hypothesis, which we introduce as \lstinline{H} and then use \lstinline{auto} to show our conclusion holds.
The inductive case holds by simplification and rewriting by the inductive hypothesis \lstinline{IHtl}$_1$.

After this, no goals remain, so our proof is done; we can write \lstinline{Defined}.
What happens when we write \lstinline{Defined} is that Coq produces the proof term on the right of Figure~\ref{fig:zip-proof}. % TODO syntax highlighting, maybe explain
It then checks the type of that term and ensures that it is exactly the theorem we have stated.
Since it is, Coq lets us know that our proof is correct, so our \lstinline{zip} function is verified.
Thankfully, though, we never have to write the low-level proof term ourselves; we see proofs as these high-level proof csripts.

Some correspondence between the proof script and proof term may already be clear.
For example, every call to \lstinline{induction} in the proof script shows up as an application of the eliminator \lstinline{list_rect}
in the proof term.
In Section~\ref{TODO}, I will explain this connection in more detail by introducing a prototype decompiler from proof terms back up to proof scripts.
This decompiler makes it possible for \sysname to work over highly structured Gallina terms,
but produce proof scripts that the proof engineer can use going forward.

These tactics are a form of \textit{proof automation}, and they do indeed make proof development easier than writing raw proof terms.
Unfortunately, this flavor of proof automation a bit naive when it comes to \textit{maintaining} proofs in the face of change.
Proof repair is a new form of proof automation that understands how proofs evolve over time.
But it is exactly the rich structure of the type theory beneath Gallina that makes a principled approach to proof repair possible.



