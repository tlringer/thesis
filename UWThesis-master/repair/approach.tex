\section{Approach}

% New (outline for now), plus parts of PUMPKIN PATCH Motivating the Core

In the example from Section~\ref{sec:patch-motivating}, we can see how the example change in one proof carries enough information
to fix other proofs broken by the same change (namely the rewrite by \lstinline{INR\_IPR}).
So a tool can extract that, generalize it, and use it to fix other proofs broken by the same change.

The key insight behind \sysname's approach is that this is true more generally.
To use \sysname, the programmer modifies a single proof script to provide an \textit{example} of how to adapt a proof to a change.
\sysname extracts that information into a \textit{patch candidate}---which is localized to the context of the example, but not enough to fix other proofs broken by the change.
It then generalizes that candidate into a \textit{reusable patch}: a function
that can be used to fix other broken proofs broken by the same change, which \sysname defines as a Coq term.

In other words, looking back to the thesis statement, the information shows up in the difference between versions of the example patched proof.
\sysname can extract and generalize that information.
Application works with hint databases or is manual.
Here is the system diagram for \sysname.
The \sysname repository contains a detailed user guide.

As mentioned earlier, \sysname does this using a combination of semantic differencing and program transformations.
Differencing looks at the difference between versions of the example patched proof for this information, and finds the candidate.
Then, program transformations modify that candidate to produce the reusable proof patch.

And of course all of this happens over proof terms, since tactics might hide necessary in information.
Of course this is hard to see on the example from Section~\ref{sec:patch-motivating},
since we were lucky enough to see the difference in tactics here.
Let's look at a toy example for which that isn't true.

\begin{figure*}
\begin{minipage}{0.50\textwidth}
\lstset{language=coq, aboveskip=0pt, belowskip=0pt}
\lstinputlisting[firstline=1, lastline=1]{repair/introex.tex}
\lstinputlisting[backgroundcolor=\color{orange!35},firstline=2,lastline=2]{repair/introex.tex}
\lstinputlisting[firstline=3, lastline=8]{repair/introex.tex}
\lstinputlisting[firstline=1, lastline=4]{repair/introexterm.tex}
\lstinputlisting[backgroundcolor=\color{orange!35},firstline=5,lastline=5]{repair/introexterm.tex}
\lstinputlisting[firstline=6, lastline=9]{repair/introexterm.tex}
\end{minipage}
\hfill
\begin{minipage}{0.48\textwidth}
\lstset{language=coq, aboveskip=0pt, belowskip=0pt}
\lstinputlisting[firstline=9, lastline=9]{repair/introex.tex}
\lstinputlisting[backgroundcolor=\color{orange!35},firstline=10,lastline=10]{repair/introex.tex}
\lstinputlisting[firstline=11, lastline=16]{repair/introex.tex}
\lstinputlisting[firstline=11, lastline=14]{repair/introexterm.tex}
\lstinputlisting[backgroundcolor=\color{orange!35},firstline=15,lastline=15]{repair/introexterm.tex}
\lstinputlisting[firstline=16, lastline=19]{repair/introexterm.tex}
\end{minipage}
\caption{Two proofs with different conclusions (top) and the
corresponding proof terms (bottom) with relevant type information. We highlight the change in theorem conclusion and
the difference in terms that corresponds to a patch.}
\label{fig:example}
\end{figure*}

To motivate this workflow, consider using \sysname to search the proofs in
Figure~\ref{fig:example} for a patch between conclusions.
Except we will show a place where the lemma is actually applied.
Note that the tactics don't change even though the terms do---and even though the change could break other proofs.

So what do we do?
We invoke the plugin using \lstinline{old} and \lstinline{new} as the example change:

\begin{lstlisting}[language=ml4]
  Patch Proof old new as patch.
\end{lstlisting}
\sysname first determines the type that a patch from \lstinline{new} to \lstinline{old} should have.
To determine this, it semantically \textit{diffs} the types and finds this goal type (line 2):

\begin{lstlisting}[language=coq]
  (@\ltacforall@) n m p, n <= m -> m <= p -> (@\diff{n <= p}@) -> (@\diff{n <= p + 1}@)
\end{lstlisting}
It then breaks each inductive proof into cases and determines an intermediate goal type for the candidate.
%recursively searches each case for a candidate.
In the base case, for example, it \textit{diffs} the types and determines that a candidate
between the base cases of \lstinline{new} and \lstinline{old} should have this type (lines 11 and 12):

\begin{lstlisting}
  (@\diff{(fun p0 => n <= p0)}@) m -> (@\diff{(fun p0 => n <= p0 + 1)}@) m
\end{lstlisting}
It then \textit{diffs} the terms (line 13) for such a candidate:

\begin{lstlisting}
  fun n m p H0 H1 =>(@\vspace{-0.04cm}@)
    (fun (H : n <= m) => (@\diff{le\_plus\_trans n m 1}@) H)(@\vspace{-0.04cm}@)
  : (@\ltacforall@) n m p, n <= m -> m <= p -> (@\diff{n <= m}@) -> (@\diff{n <= m + 1}@)
\end{lstlisting}

This candidate is close, but it is not yet a patch. This candidate
maps base case to base case (it is applied to \lstinline{m}); the patch should map conclusion to conclusion (it should
be applied to \lstinline{p}).

This is where the transformations come in.
There are four:

\begin{enumerate}
\item \textit{Patch specialization} to arguments
\item \textit{Patch abstraction} of arguments or functions
\item \textit{Patch inversion} to reverse a patch
\item \textit{Lemma factoring} to break a term into parts
\end{enumerate}

Here, \sysname \textit{abstracts} this candidate by \lstinline{m} (line 11), which lifts it out of the base case:

\begin{lstlisting}
  fun (@\diff{n0}@) n m p H0 H1 =>(@\vspace{-0.04cm}@)
    (fun (H : n <= (@\diff{n0}@)) => le_plus_trans n (@\diff{n0}@) 1 H)(@\vspace{-0.04cm}@)
  : (@\ltacforall@) (@\diff{n0}@) n m p, n <= m -> m <= p -> n <= (@\diff{n0}@) -> n <= (@\diff{n0}@) + 1
\end{lstlisting}
\sysname then \textit{specializes} this candidate to \lstinline{p} (line 16), the argument
to the conclusion of \lstinline{le_ind}. This produces a patch:

\begin{lstlisting}
  patch n m p H0 H1 := (@\vspace{-0.04cm}@)
    (fun (H : n <= (@\diff{p}@)) => le_plus_trans n (@\diff{p}@) 1 H)(@\vspace{-0.04cm}@)
  : (@\ltacforall@) n m p, n <= m -> m <= p -> n <= (@\diff{p}@) -> n <= (@\diff{p}@) + 1
\end{lstlisting}
The user can then use \lstinline{patch} to fix other broken proofs.
For example, given a proof that applies \lstinline{old}, the user can use \lstinline{patch} to prove the same conclusion
by applying \lstinline{new}:

\begin{lstlisting}[language=coq]
  (@\succeeds{apply old.}@)(@\vspace{-0.1cm}@)
  (@\diff{apply patch.}@) (@\succeeds{apply new.}@)
\end{lstlisting}
This can happen automatically through hint databases.

This simple example uses only two transformations. The other transformations help turn candidates
into patches in similar ways. We discuss all of this in detail later.

\paragraph{Configuration}

The components come together to form a proof patch finding procedure:

\begin{algorithm}
\begin{algorithmic}[1]
\renewcommand{\thealgorithm}{}
\footnotesize
\caption{\footnotesize{find\_patch(term, term', direction)}}
    \STATE \diff{\textit{diff} types of term and term' for goals}
    \STATE \diff{\textit{diff} term and term' for candidates}
    \IF{there are candidates}
      \STATE \diff{\textit{factor}, \textit{abstract}, \textit{specialize}, and/or \textit{invert} candidates}
      \IFRETURN{there are patches}{patches}
    \ENDIF
    \RETURN failure
\end{algorithmic}
\label{alg:patching}	
\end{algorithm}
\sysname infers a \textit{configuration} from the example change.
This configuration customizes the highlighted lines for an entire class of changes:
It determines what to diff on lines 1 and 2,
and how to use the components on line 4.

For example, to find a patch for Figure~\ref{fig:example},
\sysname used the configuration for changes in conclusions of two proofs
that induct over the same hypothesis. Given two such
proofs:

%We describe one
%such procedure in this section; we descibe three more
%in Section~\ref{sec:case}.
\begin{lstlisting}[language=coq]
  (@\ltacforall@) x, H x -> (@\diff{P}@) x(@\vspace{-0.1cm}@)
  (@\ltacforall@) x, H x -> (@\diff{P'}@) x
\end{lstlisting}
\sysname searches for a patch with this type:

\begin{lstlisting}[language=coq]
  (@\ltacforall@) x, H x -> (@\diff{P'}@) x -> (@\diff{P}@) x
\end{lstlisting}
using this configuration:

\begin{algorithm}
\footnotesize
\begin{algorithmic}[1]
%  Is it really necessary to say that an algorithm might do initialization?
%  \STATE Build trees for search
  %\STATE build trees for proofs
  %\REPEAT
    \STATE \textit{diff} conclusion types for goals
    \STATE \textit{diff} conclusion terms for candidates
    \STATE \textbf{if} there are candidates \textbf{then}
    \STATE \hspace*{1em} \textit{abstract} and then \textit{specialize} candidates
\end{algorithmic}
\end{algorithm}

Later we will see real-world examples that demonstrate more configurations.


