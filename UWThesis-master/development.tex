\section{Proof Development}
\label{sec:mot-dev}

% TODO table with simplified theory languages and the implementation languages, plus a summary of some differences covered in implementation sections

I already briefly introduced the Coq workflow in the introduction.
Here I am going to go into a bit more detail on the example and talk about foundations as relevant to this thesis.
I am not going to teach you all of Coq in this thesis;
good sources for learning Coq from scratch include the books Certified Programming with Dependent Types~\cite{chlipala:cpdt}
and Software Foundations~\cite{software-foundations}.
More about this can also be found in the survey paper.

The introduction mentioned that proofs are written as proof scripts in a high-level language of tactics,
and that those proof scripts compile down to proof terms in a low-level language.
Technically, proofs can be written entirely as proof terms; tactics just make it easier to write proofs and offer abstraction.
We'll get to tactics soon (Section~\ref{sec:tactics}), but let's start with terms (Section~\ref{sec:mot-coq}).

\section{The Coq Proof Assistant}
\label{sec:mot-coq}

This thesis focuses on proof developments done in Coq.
I already briefly introduced the Coq workflow in the introduction.
Here I am going to go into a bit more detail and talk about foundations as relevant to this thesis.
I am not going to teach you all of Coq in this thesis;
good sources for learning Coq from scratch include the books Certified Programming with Dependent Types~\cite{chlipala:cpdt}
and Software Foundations~\cite{software-foundations}.

Workflow from intro, but with specific languages named (Ltac, Gallina); refer to a diagram

What tactics really are, and what compiling them means

Gallina and its foundations

CoC

\begin{figure}
   \lstinputlisting[firstline=1, lastline=3]{often/listswap.tex}
\caption{The \lstinline{list} datatype in Coq, from the Coq standard library.}
\label{fig:list}
\end{figure}

Inductive types, theory (CIC) and practice (Gallina); constructors and eliminators.
For example, a \lstinline{list} in Coq (Figure~\ref{fig:list}) is an inductive datatype that is 
either empty (the \lstinline{nil} constructor), or the result
of placing an element in front of another \lstinline{list} (the \lstinline{cons} constructor).
Mention the induction principle for \lstinline{list}.

\begin{figure}
\begin{lstlisting}
Definition length (T : Type) : list T $\rightarrow$ nat := fix length l :=
  match l with
   | nil => O
   | _ :: l' => S (length l')
  end.
\end{lstlisting}
\caption{The \lstinline{length} function for \lstinline{list} in Coq, from the Coq standard library.}
\label{fig:length}
\end{figure}

Once we have inductive types, we can write functions and proofs about them, like the \lstinline{length} function (Figure~\ref{fig:length}).
The \lstinline{length} of the empty list \lstinline{nil} is \lstinline{0}, and the length of any other list
is just the successor (\lstinline{S}) of the result of recursively calling \lstinline{length} on everything but the first element of the list.
Note we can also write this using eliminators (show), and in fact that reduces to pattern matching.
When we reason about theory we think about eliminators.

Conventions in this thesis, including using induction principles/eliminators instead of pattern matching, and assuming primitive.
Infinite universe hierarchy---mostly can ignore in this thesis, though matters in implementation.


\subsection{Tactics: Ltac}
\label{sec:tactics}

In Coq, those tactics are in a language called Ltac.
Each tactic is effectively a search procedure for a proof term, given the context and goals at each step of the proof.

Tactics can be used to write simple proofs and functions too.
So if we'd like, for example, we could write \lstinline{length} using tactics:

\begin{lstlisting}
TODO
\end{lstlisting}
But it's much more common to use this for proofs, since controlling the details of the term can be hard (important for functions),
but for a proof it may be enough to just have any term with the correct type.

Here is one possible proof of \lstinline{zip_preserves_length} using very simple tactics:
\begin{lstlisting}
Lemma zip_length : zip_preserves_length.
Proof.
   intros a b l1.
   induction l1 as [|_ _ IHtl1]; auto.
   induction l2 as [|_ _ IHtl2]; intros H; auto.
   simpl. rewrite IHtl1; auto.
Qed.
\end{lstlisting}
This compiles to the thing we saw before, which then type checks as expected.
But we never have to see the low-level proof term; we can reason about the high-level proof script instead.
% TODO save the better version for later so that we can show how good automation can get around some of the repair stuff

Mention decompiler, explain, tease implementation section.



