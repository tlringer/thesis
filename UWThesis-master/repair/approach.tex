\section{Approach}
\label{sec:pumpkin-approach}

Proof repair by example takes advantage of the fact that an example patch to a broken proof can carry enough 
information (like the rewrite by \lstinline{INR_IPR}) to fix other proofs broken by the same change (like the broken
client proofs). \sysname implements a prototype of this.

To use \sysname (Section~\ref{sec:pumpkin-workflow}), the proof engineer modifies a single proof script to provide 
an \intro{example patched proof}: an example of how to adapt a proof to a change.
\sysname extracts the information that example carries into a \intro{patch candidate}: a function that describes the change 
in the example patched proof,
but that is localized to the context of the example, and not yet enough to fix other proofs broken by the change.
\sysname then generalizes that candidate into a \kl{reusable patch}: a function
that can be used to fix other broken proofs broken by the same change, which \sysname defines as a Gallina term.

The \sysname prototype focuses on finding reusable patches to proofs in response to certain changes in the content of programs and specifications (Section~\ref{sec:pumpkin-scope}).
It does this using a combination of \kl{semantic differencing} and \kl{proof term transformations}:
Differencing (Section~\ref{sec:pumpkin-spec-diff}) looks at the difference between versions of the example patched proof for this information, and finds the candidate.
The proof term transformations (Section~\ref{sec:pumpkin-spec-trans}) then modify that candidate to produce the reusable proof patch.
All of this happens over proof terms in Gallina, since tactics may hide necessary information. % as I will soon show.

In other words, looking back to the thesis statement, the information corresponding to changes in programs and specifications
shows up in the difference between versions of the example patched proof.
\sysname can extract and generalize that information.
\sysname has only preliminary support for automatically \textit{applying} the extracted
and generalized information (see Section~\ref{sec:pumpkin-impl}),
though the later \toolnamec extension addresses this limitation (Chapter~\ref{chapt:pi}).

\subsection{Workflow: Repair by Example}
\label{sec:pumpkin-workflow}

The interface to \sysname is exposed to the proof engineer as a \intro{command}.
Commands in Coq are similar to \kl{tactics}, except that they can occur outside of the context of proofs, and they can define new terms.
Plugins like \sysname can extend Coq with new commands.
In this case, \sysname extends Coq with a new command called \lstinline{Patch Proof}, with the syntax:

\begin{lstlisting}
  Patch Proof old_proof new_proof as patch_name. 
\end{lstlisting}
where \lstinline{old_proof} and \lstinline{new_proof} are the old and new versions of the \kl{example patched proof},
and \lstinline{patch_name} is the desired name of the \kl{reusable proof patch}.\footnote{Section~\ref{sec:pumpkin-impl} describes an alternative interface for \sysname with Git integration.} This invokes the \sysname plugin, which searches for a reusable proof patch
and defines it as a new term if successful.
All terms that \sysname defines are type checked in the end, so \sysname does not extend the \kl{TCB}.

\begin{figure}
%\begin{algorithm}
\begin{algorithmic}
\renewcommand{\thealgorithm}{}
\footnotesize
\STATE \hspace{-0.5cm} \footnotesize{find\_patch(old\_proof, new\_proof) :=}
    \STATE \diff{\kl{diff} types of old\_proof and new\_proof for \kl{goals}}
    \STATE \diff{\kl{diff} terms old\_proof and new\_proof for \kl{candidates}}
    \IF{there are \kl{candidates}}
      \STATE \diff{\kl{transform} \kl{candidates}}
      \IFRETURN{there are \kl{patches}}{\kl{patches}}
    \ENDIF
    \RETURN failure
\end{algorithmic}
%\end{algorithm}
\caption{Search procedure for a reusable proof patch in \sysname.}
\label{alg:patching}	
\end{figure}

When the proof engineer calls \lstinline{Patch Proof}, this invokes the proof patch search procedure in Figure~\ref{alg:patching}.
The search procedure starts by differencing the \textit{types} of \lstinline{old_proof} and \lstinline{new_proof}
(that is, the theorems they prove).
The result that it finds is the \intro{goal type}: the type that the reusable proof patch should have.
It then differences the \textit{terms} \lstinline{old_proof} and \lstinline{new_proof} directly to identify patch candidates,
which are themselves proof terms.
Finally, it transforms those patch candidates directly into a reusable patch.
If it finds a reusable patch with the goal type, it succeeds and defines it as a term.

\begin{figure*}
\begin{minipage}{0.50\textwidth}
\lstset{language=coq, aboveskip=0pt, belowskip=0pt}
\lstinputlisting[firstline=1, lastline=4]{repair/introex.tex}
\lstinputlisting[backgroundcolor=\color{orange!35},firstline=5,lastline=5]{repair/introex.tex}
\lstinputlisting[firstline=6, lastline=11]{repair/introex.tex}
\lstinputlisting[firstline=1, lastline=7]{repair/introexterm.tex}
\lstinputlisting[backgroundcolor=\color{orange!35},firstline=8,lastline=8]{repair/introexterm.tex}
\lstinputlisting[firstline=9, lastline=13]{repair/introexterm.tex}
\end{minipage}
\hfill
\begin{minipage}{0.50\textwidth}
\lstset{language=coq, aboveskip=0pt, belowskip=0pt}
\lstinputlisting[firstline=12, lastline=15]{repair/introex.tex}
\lstinputlisting[backgroundcolor=\color{orange!35},firstline=16,lastline=16]{repair/introex.tex}
\lstinputlisting[firstline=17, lastline=22]{repair/introex.tex}
\lstinputlisting[firstline=15, lastline=21]{repair/introexterm.tex}
\lstinputlisting[backgroundcolor=\color{orange!35},firstline=22,lastline=22]{repair/introexterm.tex}
\lstinputlisting[firstline=23, lastline=27]{repair/introexterm.tex}
\end{minipage}
\caption{Two proofs with different conclusions (top) and the
corresponding proof terms (bottom). Highlighted lines correspond to
the change in theorem conclusion (top) and the difference in terms that correspond to a patch (bottom).}
\label{fig:example}
\end{figure*}

To demonstrate this workflow, consider the change from the theorem \lstinline{old} to the slightly stronger theorem \lstinline{new} in Figure~\ref{fig:example}.
Changing \lstinline{old} to \lstinline{new} can break proofs that used to successfully apply \lstinline{old}:

\begin{lstlisting}[language=coq]
  (@\succeeds{apply old.}@)
\end{lstlisting}
so that they fail after migrating to \lstinline{new}:

\begin{lstlisting}[language=coq]
  (@\fails{apply new.}@)
\end{lstlisting}
When we call:

\begin{lstlisting}
  Patch Proof old new as patch.
\end{lstlisting}
\sysname invokes the search procedure, which differences \lstinline{old} and \lstinline{new} to infer the \kl{goal type} for the \kl{patch}.
Here, it infers the following goal:

\begin{lstlisting}[language=coq]
  (@\ltacforall@) n m p,
    n <= m $\rightarrow$ m <= p $\rightarrow$ (@\diff{n <= p}@) $\rightarrow$ (@\diff{n <= p + 1}@)
\end{lstlisting}
which maps from the conclusion of \lstinline{new} back to the conclusion of \lstinline{old} in a common context.
It then differences the terms \lstinline{old} and \lstinline{new} to identify \kl{candidate proof patches} (Section~\ref{sec:pumpkin-spec-diff}),
then tranforms those candidates to a \kl{reusable proof patch} with that type (Section~\ref{sec:pumpkin-spec-trans}),
which it defines as a new term \lstinline{patch}.
This is something that we can use to fix other proofs broken by this change, either by applying it with traditional \kl{proof automation}:

\begin{lstlisting}[language=coq]
  (@\diff{apply patch.}@) (@\succeeds{apply new.}@)
\end{lstlisting}
or by using the automation in Section~\ref{sec:pumpkin-impl}.

\subsection{Scope: Changes in Content}
\label{sec:pumpkin-scope}

The search procedure in Figure~\ref{alg:patching} searches for patches to proofs broken by changes in the \textit{content} of programs and specifications.
For example, \sysname can support the change in Figure~\ref{fig:example}, since some of the content (the conclusion of the theorem) changes,
but the structure remains identical.
In general, the \sysname prototype does not support any changes that add, remove, or rearrange any hypotheses.
The \toolnamec extension to \sysnamelong in Chapter~\ref{chapt:pi} supports a broad class of changes
in datatypes that may change in these ways.

The search procedure can be instantiated to different classes of change in the content of programs and specifications.
Thus, before running the search procedure, \sysname infers an \intro{instance} of the search 
procedure from the example change.\footnote{In the original 2018 \sysnamelong paper, I called this a \textit{configuration}.
In this chapter, I rename configuration to \kl{instance} everywhere to avoid overloading this term,
since I unfortunately used this word again for something different when developing \toolnamec.}
This instance customizes the highlighted lines for an entire class of changes:
it determines what to diff on lines 1 and 2,
and what transformations to run to achieve what goal on line 4.

Figure~\ref{fig:example} used the instance for a change in the conclusion of a theorem.
Given two such proofs of theorems:

%We describe one
%such procedure in this section; we descibe three more
%in Section~\ref{sec:case}.
\begin{lstlisting}[language=coq]
  (@\ltacforall@) x, H x $\rightarrow$ (@\diff{P}@) x(@\vspace{-0.1cm}@)
  (@\ltacforall@) x, H x $\rightarrow$ (@\diff{Q}@) x
\end{lstlisting}
for any \lstinline{x}, \lstinline{H}, \lstinline{P}, and \lstinline{Q},
\sysname searches for a patch with goal type:

\begin{lstlisting}[language=coq]
  (@\ltacforall@) x, H x $\rightarrow$ (@\diff{Q}@) x $\rightarrow$ (@\diff{P}@) x
\end{lstlisting}

In total, the \sysname prototype currently implements six instances of the search procedure.
Section~\ref{sec:pumpkin-impl-procedure} explains these instances,
and Section~\ref{sec:pumpkin-results} demonstrates some of them on real proof developments.

\subsection{Differencing: Candidates from Examples}
\label{sec:pumpkin-spec-diff}

\kl{Differencing} operates over terms and types. Differencing tactics
would be insufficient, since tactics and hints may mask information helpful to finding patches.
For example, for the change in Figure~\ref{fig:example},
the tactics in the proofs of \lstinline{old} and \lstinline{new} are identical, even 
though the proof terms they compile down to are not. %\footnote{Since this is a simple example, replaying an existing tactic happens to work. There
%are additional examples in the repository (\lstinline{Cex.v}).} % TODO what? What did this even mean? lmao
This is why differencing looks at the change in terms to extract the patch candidates.

In the end, differencing identifies the semantic difference between the old and new versions of the proof terms for the example patched proof.
At a high level, the semantic difference is the difference between two terms that corresponds to the 
difference between their types (see Section~\ref{sec:pumpkin-diff}).
The details of the semantic difference and where differencing looks to find it vary by instance of the search procedure.

Consider a simplified version of the example in Figure~\ref{fig:example}, 
looking only at the base case (line 19):

\begin{lstlisting}[language=coq]
  old_proof := (@\diff{le\_plus\_trans n m 1}@) H : (@\diff{n <= m + 1}@).
  new_proof := H : (@\diff{n <= m}@).
\end{lstlisting}
For this change, \sysname uses the instance for changes in conclusions:

\begin{algorithm}
\footnotesize
\begin{algorithmic}[1]
%  Is it really necessary to say that an algorithm might do initialization?
%  \STATE Build trees for search
  %\STATE build trees for proofs
  %\REPEAT
    \STATE \kl{diff} \diff{theorem conclusions} of old\_proof and new\_proof for \kl{goals}
    \STATE \kl{diff} \diff{function bodies} of old\_proof and new\_proof for \kl{candidates}
    \STATE \textbf{if} there are \kl{candidates} \textbf{then}
    \STATE \hspace*{1em} \kl{transform} \kl{candidates}
\end{algorithmic}
\end{algorithm}
When this instance of the search procedure is invoked,
semantic differencing first identifies the difference in their types, here the respective \kl{motives} (line 18):

\begin{lstlisting}
  (@\diff{(fun p0 => n <= p0 + 1)}@)
  (@\diff{(fun p0 => n <= p0)}@)
\end{lstlisting}
applied to \lstinline{m} (line 17).
This produces the \kl{candidate} \kl{goal type}:

\begin{lstlisting}
  (@\diff{n <= m}@) $\rightarrow$ (@\diff{n <= m + 1}@)
\end{lstlisting}
Differencing then diffs the terms for a function that has that type, here (line 19):

\begin{lstlisting}[language=coq]
  fun (H : n <= m) => (@\diff{le\_plus\_trans n m 1}@) H
\end{lstlisting}
This is a \kl{patch candidate}.
This candidate is close, but it is not yet a \kl{reusable patch}. In particular, this candidate
maps base case to base case (it is applied to \lstinline{m}); the patch should map conclusion to conclusion (it should
be applied to \lstinline{p}).
This is where the proof term transformations will come in.

\paragraph{Summary}
In summary, differencing has the following specification:

% TODO may need to improve this, just want something here for now
\begin{itemize}
\item \textbf{Inputs}: the \kl{example patched proof} given by \lstinline{old_proof} and \lstinline{new_proof}, a set of options corresponding to the \kl{instance} of the search procedure \lstinline{instance}, and a final \kl{goal type} \lstinline{goal}, assuming:
\begin{itemize}
\item the change from \lstinline{old_proof} to \lstinline{new_proof} is in the class of changes supported by \lstinline{instance}.
\end{itemize}
\item \textbf{Outputs}: a list of terms \lstinline{candidates} of \kl{patch candidates}, and a \kl{candidate} \kl{goal type} \lstinline{candidate_goal}, guaranteeing:
\begin{itemize}
\item each term in \lstinline{candidates} has type \lstinline{candidate_goal}.
\end{itemize}
\end{itemize}
\sysname infers the instance of the procedure and the candidate goal type from the change in the example patched proof itself, so the proof engineer does not have to provide this information.
\sysname could in theory infer the wrong instance or the wrong candidate goal type, but this would not sacrifice soundness---it would mean only that
the patch procedure would either fail to produce a patch, or produce a patch that is not useful in the end.
All terms that \sysname produces type check in the end. % TODO hyphen or no hyphen?

\subsection{Transformations: Patches from Candidates}
\label{sec:pumpkin-spec-trans}

Differencing produces \kl{patch candidates} that are localized to a particular context according to the inferred goal for that change,
but do not yet generalize to other contexts.
The \kl{transformations} are what take each candidate and try to modify it to produce a term that \textit{does} generalize to other contexts.
If they succeed, \sysname has found a \kl{reusable patch}.

Consider once more the example in Figure~\ref{fig:example}.
The candidate patch that differencing found has this type:

\begin{lstlisting}[language=coq]
  candidate :=
    fun (H : n <= m) => (@\diff{le\_plus\_trans n m 1}@) H
  : (@\diff{n <= m}@) $\rightarrow$
    (@\diff{n <= m + 1}@).
\end{lstlisting}
in an environment with fixed \lstinline{n} and \lstinline{m}.
The reusable patch that \sysname is looking for, however, should have this type:

\begin{lstlisting}[language=coq]
  (@\ltacforall@) n m p,
    n <= m $\rightarrow$
    m <= p $\rightarrow$
    (@\diff{n <= p}@) $\rightarrow$
    (@\diff{n <= p + 1}@).
\end{lstlisting}
as this is the \kl{goal} that \sysname inferred for this \kl{instance}.
The transformations that \sysname runs will attempt to transform the candidate
into a patch with that type.

The details of which transformations to run vary by instance.
There are four transformations that turn candidates into reusable patches:

\begin{enumerate}
\item \textit{patch} \intro{specialization} to arguments,
\item \textit{patch} \intro{generalization}\footnote{In the original paper, I called this \textit{patch abstraction}. But I later learned that \kl{generalization} has a technical meaning in the automated theorem proving world, and that the technical meaning coincides beautifully with the technical meaning in the world of proof assistants. So I renamed it for this thesis.} of arguments or functions,
\item \textit{patch} \intro{inversion} to reverse a patch, and
\item \textit{lemma} \intro{factoring} to break a term into parts.
\end{enumerate}
Each instance chooses among these transformations strategically based on the structure of the proof term.

For Figure~\ref{fig:example}, we can instantiate \textit{transform} with two transformations:

\begin{algorithm}
\footnotesize
\begin{algorithmic}[1]
%  Is it really necessary to say that an algorithm might do initialization?
%  \STATE Build trees for search
  %\STATE build trees for proofs
  %\REPEAT
    \STATE \kl{diff} conclusions of the theorems of old\_proof and new\_proof for \kl{goals}
    \STATE \kl{diff} bodies of the proof terms for \kl{candidates}
    \STATE \textbf{if} there are \kl{candidates} \textbf{then}
    \STATE \hspace*{1em} \diff{\kl{generalize} and then \kl{specialize}} \kl{candidates}
\end{algorithmic}
\end{algorithm}
That is, first, \sysname \kl{generalizes} the candidate by \lstinline{m} (line 17), which lifts it out of the base case:

\begin{lstlisting}
  fun (@\diff{n0}@) n m p H0 H1 =>
    (fun (H : n <= (@\diff{n0}@)) => le_plus_trans n (@\diff{n0}@) 1 H)
  : (@\ltacforall@) (@\diff{n0}@) n m p,
      n <= m $\rightarrow$
      m <= p $\rightarrow$
      n <= (@\diff{n0}@) $\rightarrow$
      n <= (@\diff{n0}@) + 1.
\end{lstlisting}
\sysname then \kl{specializes} this generalized candidate to \lstinline{p} (line 23), the argument
to the conclusion of \lstinline{le_ind}. This produces a patch:

\begin{lstlisting}
  patch n m p H0 H1 :=
    (fun (H : n <= (@\diff{p}@)) => le_plus_trans n (@\diff{p}@) 1 H)
  : (@\ltacforall@) n m p,
      n <= m $\rightarrow$
      m <= p $\rightarrow$
      n <= (@\diff{p}@) $\rightarrow$
      n <= (@\diff{p}@) + 1.
\end{lstlisting}
which has the goal type, so \sysname is done.

This simple example uses only two transformations.
The other transformations help turn candidates into patches in similar ways, all guided by
the structure of the proof term.
I will describe these transformations more in Section~\ref{sec:pumpkin-trans}.
%and present real-world examples that demonstrate more instances in Section~\ref{sec:pumpkin-results}.

\paragraph{Summary}
In summary, the transformations together have the following specification:

% TODO I hate this, should make it clearer, for both summaries so far
\begin{itemize}
\item \textbf{Inputs}: the inputs and outputs of differencing, assuming:
\begin{itemize}
\item the assumptions and guarantees from differencing hold.
\end{itemize}
\item \textbf{Outputs}: a term \lstinline{patch} that is the \kl{reusable proof patch}, guaranteeing:
\begin{itemize}
\item \lstinline{patch} has the inferred final \kl{goal type} \lstinline{goal} for the change.
\end{itemize}
\end{itemize}
When these transformations fail, or when the list of candidates that differencing returns is empty,
\sysname simply fails to return a patch.
As with differencing, it is possible that a mistake in the implementation
leads to a final goal type is not useful to the proof engineer,
but this cannot sacrifice soundness:
every patch \sysname produces type checks with the goal type in the end.

