\section{Automation, Reimagined}

Traditional proof automation considers only the current state of
theorems, proofs, and definitions. This is a missed opportunity: Verification projects 
are rarely static. Like other software, these projects evolve over time.

Currently, the burden of change largely falls on programmers.
This does not have to be true. Proof automation
can view theorems, proofs, and definitions as fluid entities:
When a proof or specification changes,
a tool can search the difference between the old and new versions
for a \emph{reusable patch} that can fix broken proofs.

\paragraph{Status Quo} Experienced Coq programmers use design principles and custom tactics to make proofs
resilient to change. These techniques are useful for large proof developments, but they 
%require expertise. Even then, they 
place the burden of change on the programmer. This can be problematic
when change occurs outside of the programmer's control. % How to ward off Adam Chlipala here?

Consider a commit from the upcoming Coq 8.7 release~\cite{coq87commit}. 
This commit redefined injection from integers to reals (Figure~\ref{fig:izr}).
This change broke 18 proofs in the standard library. 

The Coq developer who committed the change fixed the broken proofs, then
made an additional 12 commits to address the change in \lstinline{coq-contribs},
a regression suite of projects that the Coq developers maintain as versions change.
Many of these changes were simple. For example, the developer wrote a lemma that describes the change:

\lstset{language=coq, aboveskip=3pt, belowskip=3pt}
\begin{lstlisting}
    Lemma INR_IPR : (@\ltacforall@)p, INR (Pos.to_nat p) = IPR p.
\end{lstlisting}

The developer then used this lemma to fix broken proofs within the standard library. 
For example, one proof broke on this line:

\begin{lstlisting}
    (@\fails{rewrite Pos2Nat.inj\_sub by trivial.}@)
\end{lstlisting}

It succeeded with the lemma:

\begin{lstlisting}
    rewrite (@\diff{<- 3!INR\_IPR}@), (@\succeeds{Pos2Nat.inj\_sub by trivial.}@)
\end{lstlisting}

These changes are outside-facing: Coq users have to make similar changes to their own proofs when
they update to Coq 8.7. The Coq developer can update some tactics to account for this, but it is 
impossible to account for every tactic that users could use.
Furthermore, while the developer responsible for the changes knows about
the lemma that describes the change, the Coq user does not. The Coq user must determine
how the definition has changed and how to address the change, perhaps by reading documentation or by
talking to the developers. 

\paragraph{Our Vision} When a user updates Coq, a tool can determine
that the definition has changed, then analyze changes in the standard library and in \lstinline{coq-contribs}
that resulted from the change in definition (in this case, rewriting by the lemma).
It can extract a reusable patch from those changes, which it can automatically apply within broken user proofs.
The user never has to consider how the definition has changed.

% TODO talk about the search space

% This isn't an example we actually handle yet, but I think that's OK for this section
% Mostly just had dependency issues trying to pull this code and so on, since I'm no on 8.7
% But it's topical and outside-facing
