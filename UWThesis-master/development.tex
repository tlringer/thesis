\section{Proof Development}
\label{sec:mot-dev}

Before I motivate proof maintenance and repair, it helps to understand \intro{proof development} to begin with.
In the introduction, I briefly explained the workflow for using a \kl{proof assistant} to develop a verified system,
noting that the proof engineer:

\begin{enumerate}
\item implements a \kl{program},
\item \kl{specifies} what it means for the \kl{program} to be correct, and
\item \kl{proves} that the \kl{program} satisfies the \kl{specification}.
\end{enumerate}
In the \kl{Coq} proof assistant, proof engineers implement programs in a rich functional programming language \intro{Gallina}.
In fact, it is possile to use Gallina to write the program, the specification, \textit{and} the proof---but writing the proof in Gallina can be challenging.
Instead, proof engineers typically use Gallina to write only the program and specification,
and write the proof interactively.
I alluded to this when I explained the typical proof development workflow in Coq:

\begin{quote}
To write a proof, the proof engineer passes Coq high-level search procedures called \kl{tactics} (like \lstinline{induction}), and Coq responds to each tactic
by refining the current goal to some subgoal (like the goal for the base case). This loop of tactics and goals 
continues until no goals remain, at which point the proof engineer has constructed a sequence of tactics called a \kl{proof script}---the
high-level representation.
To check the proof, Coq compiles that proof script down to a low-level representation called a \kl{proof term},
then uses its \kl{kernel} to check that the proof term has the expected type.
If the term has the expected type, Coq lets the proof engineer know in the end.
\end{quote}
The low-level language of proof terms in Coq is \kl{Gallina}---the same rich functional programming language proof engineers use to write programs and specifications.
The high-level language of proof scripts in Coq is a language called \intro{Ltac} that I will soon describe.

In this thesis, I will not teach you all of Coq.\footnote{Good sources 
for learning more about Coq include the books Certified Programming with Dependent Types~\cite{chlipala:cpdt}
and Software Foundations~\cite{software-foundations}, and the survey paper \kl{QED at Large}.}
What I will do is motivate this workflow on an example (Section~\ref{sec:mot-workflow})
and explain the theory beneath (Section~\ref{sec:mot-theory}).

\subsection{The Workflow}
\label{sec:mot-workflow}

Before we can write our small verified program,
we need the \lstinline{list} datatype and the \lstinline{length} function.
To write the \lstinline{length} function, we need the \lstinline{nat} datatype of unary natural numbers.
All of these can be found in the Coq standard library.

Each of \lstinline{nat} and \lstinline{list} in Gallina is an \textit{inductive type}:
it is defined by its \textit{constructors} (ways of constructing a term with that type)
or, dually, its \textit{eliminators} (ways of eliminating or inducting over a term with that type).
A \lstinline{nat}, for example:

\begin{lstlisting}
Inductive nat :=
| O : nat
| S : nat $\rightarrow$ nat.
\end{lstlisting}
is either \lstinline{0} or the successor \lstinline{S} of another \lstinline{nat};
these are the two constructors of \lstinline{nat}.
The eliminator this yields is the standard induction principle for natural numbers,
which Coq calls \lstinline{nat_rect}: % TODO footnote about nat_ind, nat_rec, nat_rect

\begin{lstlisting}
nat_rect :
  $\forall$ (P : nat $\rightarrow$ Type), (* motive *)
    P O $\rightarrow$ (* base case *)
    ($\forall$ (n : nat), P n $\rightarrow$ P (S n)) $\rightarrow$ (* inductive case *)
    $\forall$ (n : nat), P n.
\end{lstlisting}
This eliminator states that a statement \lstinline{P} (called the inductive \textit{motive}) about the natural numbers
holds if it holds for \lstinline{O} in the base case and, in the inductive case,
assuming it holds for some \lstinline{n}, it also holds for the successor \lstinline{S n}.

In Gallina, each eliminator like \lstinline{nat_rect} is just a function defined by
pattern matching and recursion.
TODO left off here.

\begin{figure}
   \lstinputlisting[firstline=1, lastline=3]{often/listswap.tex}
\caption{The \lstinline{list} datatype in Gallina, from the Coq standard library.}
\label{fig:list}
\end{figure}

Consider, for example, the \lstinline{list} datatype (Figure~\ref{fig:list}). % TODO explain
Coq automatically generates an eliminator for \lstinline{list} in Gallina:

\begin{lstlisting}
TODO list_rect
\end{lstlisting}
But this eliminator is not primitive in Gallina.
Rather, \lstinline{list_rect} is just a constant that defines another term,
and this term uses pattern matching and recursion:

\begin{lstlisting}
TODO list_rect body
\end{lstlisting}

For example, the length function in Coq is implemented using pattern matching and guarded recursion:

\begin{lstlisting}
Definition length (T : Type) : list T $\rightarrow$ nat := fix length l :=
  match l with
   | nil => O
   | _ :: l' => S (length l')
  end.
\end{lstlisting}
The \lstinline{length} of the empty list \lstinline{nil} is \lstinline{0}, and the length of any other list
is just the successor (\lstinline{S}) of the result of recursively calling \lstinline{length} on everything but the first element of the list.

With that in mind, we can now write our small verified program.

\begin{figure}
\begin{lstlisting}
Definition zip {a} {b} : list a -> list b -> list (a * b)%type :=
  fix zip arg_0__ arg_1__
        := match arg_0__, arg_1__ with
           | nil, _bs => nil
           | _as, nil => nil
           | cons a as_, cons b bs => cons (pair a b) (zip as_ bs)
  end. (* TODO convert to eliminators *)
\end{lstlisting}
\caption{The list \lstinline{zip} function from an old version of hs-to-coq~\cite{TODO}, up to renaming (TODO just say everything is, and when you introduce foundations just before this use induction principles and briefly mention fix-to-elim).}
\label{fig:zip}
\end{figure}

\paragraph{Program} Figure~\ref{fig:zip}.

\begin{figure}
\begin{lstlisting}
TODO zip_preserves_length, one version
\end{lstlisting}
\caption{TODO}
\label{fig:zip-pres}
\end{figure}

\begin{figure}
\begin{lstlisting}
TODO zip_preserves_length, another version
\end{lstlisting}
\caption{TODO}
\label{fig:zip-pres-alt}
\end{figure}

\paragraph{Specification} Figure~\ref{fig:zip-pres} or Figure~\ref{zip-pres-alt}, note art, stronger and so on, will come back to this.
Also this is where a lot of the power of the language comes in, worth noting.

\begin{figure}
\begin{lstlisting}
TODO zip_preserves_length proof
\end{lstlisting}
\caption{TODO}
\label{fig:zip-pres-proof}
\end{figure}

\paragraph{Proof} 
In Coq, those tactics are in a language called Ltac.
Each tactic is effectively a search procedure for a proof term, given the context and goals at each step of the proof.
Here is one possible proof of \lstinline{zip_preserves_length} using very simple tactics:
\begin{lstlisting}
Lemma zip_length : zip_preserves_length.
Proof.
   intros a b l1.
   induction l1 as [|_ _ IHtl1]; auto.
   induction l2 as [|_ _ IHtl2]; intros H; auto.
   simpl. rewrite IHtl1; auto.
Qed.
\end{lstlisting}
This compiles to the thing we saw before, which then type checks as expected.
But we never have to see the low-level proof term; we can reason about the high-level proof script instead.
% TODO save the better version for later so that we can show how good automation can get around some of the repair stuff
% TODO walk through and so on

Mention decompiler, explain, tease implementation section.






\subsection{The Theory Beneath}
\label{sec:mot-theory}

\begin{figure*}
\small
\begin{grammar}
<i> $\in \mathbbm{N}$, <v> $\in$ Vars, <s> $\in$ \{ Prop, Set, Type<i> \}

<t> ::= <v> \hspace{0.06cm} | \hspace{0.06cm} <s> \hspace{0.06cm} | \hspace{0.06cm} $\Pi$ (<v> : <t>) . <t> \hspace{0.06cm} | \hspace{0.06cm} $\lambda$ (<v> : <t>) . <t> \hspace{0.06cm} | \hspace{0.06cm} <t> <t>
\end{grammar}
\vspace{-0.3cm}
\caption{Syntax for CoC$_\omega$ with (from left to right) variables, sorts, dependent types, functions, and application.}
\label{fig:coc-syntax}
\end{figure*}

The type theory that Gallina implements is CIC$_{\omega}$, or the Calculus of Inductive Constructions.
CIC$_{\omega}$ is based on the Calculus of Constructions (CoC), a variant of the lambda calculus with polymorphism (types that dependent on types) and dependent types (types that depend on terms)~\cite{coquand:inria-00076024}.
CoC with an infinite universe hierarchy is called CoC$_{\omega}$. % TODO need to explain or defer
The syntax for CoC$_{\omega}$ is in Figure~\ref{fig:coc-syntax}.

\begin{figure*}
\small
\begin{grammar}
<t> ::= ... | \hspace{0.06cm} Ind (<v> : <t>)\{<t>,\ldots,<t>\} \hspace{0.06cm} | \hspace{0.06cm} Constr (<i>, <t>) \hspace{0.06cm} | \hspace{0.06cm} Elim(<t>, <t>)\{<t>,\ldots,<t>\}
\end{grammar}
\vspace{-0.3cm}
\caption{CIC$_\omega$ is CoC$_\omega$ with inductive types, inductive constructors, and eliminators.}
\label{fig:cic-syntax}
\end{figure*}

The syntax for CIC$_{\omega}$ is in Figure~\ref{fig:cic-syntax}), building on syntax from an existing paper~\cite{Timany2015FirstST}.
CIC$_{\omega}$ extends CoC$_{\omega}$ with inductive types~\cite{inductive}.
Inductive types are defined solely by their constructors (ways of constructing or making a term with that type)
and eliminators (ways of eliminating or inducting over a term with that type).

Consider the inductive type \lstinline{nat} of unary natural numbers:

\begin{lstlisting}
TODO nat in CIC$_{\omega}$
\end{lstlisting}
A unary natural number is either \lstinline{0} or the successor \lstinline{S} of another natural number;
these are the two constructors of \lstinline{nat}.
The eliminator this yields is the standard induction principle for natural numbers:

\begin{lstlisting}
TODO eliminator for the above
\end{lstlisting}
which states that a proposition about the natural numbers holds if it holds for \lstinline{0} in the base case and, in the inductive case,
assuming it holds for some \lstinline{n}, it also holds for the successor \lstinline{S n}.

\begin{figure}
\begin{lstlisting}
TODO
\end{lstlisting}
\caption{The \lstinline{list} datatype in CIC$_{\omega}$}
\label{fig:list-theory}
\end{figure}

The \lstinline{list} datatype we will need for our proof is in Figure~\ref{fig:list-theory}, and is quite similar to \lstinline{nat}.
There are two differences: \lstinline{list} is polymorphic over some type \lstinline{T} (so we can have a list of natural numbers,
for example, written \lstinline{list nat}), and the second constructor adds a new element of the type \lstinline{T} to the front of the list.
Otherwise, \lstinline{list} also has two constructors, \lstinline{nil} and \lstinline{cons}, where \lstinline{nil} represents the empty list,
and \lstinline{cons} sticks a new element in front of any existing list.
The eliminator for \lstinline{list} is similarly similar to the eliminator for \lstinline{nat}:

\begin{lstlisting}
TODO eliminator for list in CIC$_{\omega}$
\end{lstlisting}

\begin{figure}
\begin{lstlisting}
TODO
\end{lstlisting}
\caption{The \lstinline{length} function for \lstinline{list} in CIC$_{\omega}$.}
\label{fig:length-theory}
\end{figure}

Once we have inductive types like \lstinline{nat} and \lstinline{list}, we can write functions and proofs about them, like the \lstinline{length} function (Figure~\ref{fig:length-theory}).
The \lstinline{length} of the empty list \lstinline{nil} is \lstinline{0}, and the length of any other list
is just the successor (\lstinline{S}) of the result of recursively calling \lstinline{length} on everything but the first element of the list.
% TODO frame this using induction instead

\paragraph{Bridging Theory and Practice}

Gallina implements CIC$_{\omega}$, but with a few important differences.
More information is on the website, % https://coq.github.io/doc/v8.9/refman/language/gallina-specification-language.html, noting it's 8.9
but two differences are relevant to repair:
The first is that Gallina has constants that define terms---later on, this will help with building optimizations for repair tools.
The second notable difference has to do with eliminators.
The grammar for CIC$_{\omega}$ we just saw represents eliminators explicitly---that is, it has \intro{primitive eliminators} that do
not reduce down to any other term in the language.
Eliminators in Gallina are not primitive.
Rather, they are constants that reduce to statements using pattern matching and guarded recursion.

Consider, for example, the \lstinline{list} datatype from Figure~\ref{fig:list}.
Coq automatically generates an eliminator for \lstinline{list} in Gallina:

\begin{lstlisting}
TODO list_rect
\end{lstlisting}
But this eliminator is not primitive in Gallina.
Rather, \lstinline{list_rect} is just a constant that defines another term,
and this term uses pattern matching and recursion:

\begin{lstlisting}
TODO list_rect body
\end{lstlisting}

Throughout, in this thesis, when I present the theory, I assume \kl{primitive eliminators}.
Likewise, in Gallina, I always use functions that apply eliminators rather than pattern matching
and guarded recursion (the two are equivalent~\cite{TODO}).
To handle practical code that uses pattern matching and guarded recursion,
I preprocesss the code using a tool by my coauthor Nate Yazdani (more about this later).

For example, though the length function in Coq is implemented using pattern matching and guarded recursion,
using the preprocessing tool, I convert that function to a form using eliminators instead:

\begin{lstlisting}
TODO
\end{lstlisting}
That behaves the same way, though there are some technicalities about equalities not preserved by this preprocessing step.
In the rest of the paper, I skip this preprocessing step in examples, but I describe it more in the implementation section later.



