\section{Transformation}

% From Motivating the Core. Honestly, maybe inversion example should go earlier in Approach.

\paragraph{Patch Specialization} The tool should be able to specialize a patch candidate to specific arguments as determined by
the differences in terms. To find a patch for Figure~\ref{fig:example}, for example, \sysname
must specialize the patch candidate to \lstinline{p} to produce the final patch.

\paragraph{Patch Abstraction} A tool should be able to abstract patch candidates of this form by the common argument:

\begin{lstlisting}[language=coq]
  candidate : P' t -> P t(@\vspace{-0.08cm}@)
  candidate_abs : (@\ltacforall@) (@\diff{t0}@), P' (@\diff{t0}@) -> P (@\diff{t0}@)
\end{lstlisting}
and it should be able to abstract patch candidates of this form by the common function:

\begin{lstlisting}[language=coq]
  candidate : P t' -> P t(@\vspace{-0.08cm}@)
  candidate_abs : (@\ltacforall@) (@\diff{P0}@), (@\diff{P0}@) t' -> (@\diff{P0}@) t
\end{lstlisting} 

This is necessary because the tool may find candidates in an applied form.
For example, when searching for a patch between the proofs in Figure~\ref{fig:example},
\sysname finds a candidate in the difference of base cases. To produce a patch, 
\sysname must abstract the candidate by the argument \lstinline{m}.
Abstracting candidates is not always possible; abstraction will necessarily be a collection of heuristics.

\paragraph{Patch Inversion} The tool should be able to invert a patch candidate.
This is necessary to search for isomorphisms.
It is also necessary to search for implications between propositionally
equal types, since candidates may appear in the wrong direction.
For example, consider two list lemmas (we write \lstinline{length} as \lstinline{len}):

\begin{lstlisting}[language=coq]
  old : (@\ltacforall@) l' l, len (l' ++ l) = len l' + len l(@\vspace{-0.08cm}@)
  new : (@\ltacforall@) l' l, len (l' ++ l) = len l' + len (rev l)
\end{lstlisting} 

If \sysname searches the difference in proofs of these lemmas for a patch from the 
conclusion of \lstinline{new} to the conclusion of \lstinline{old},
it may find a candidate \emph{backwards}:

\begin{lstlisting}
  candidate l' l (H : (@\diff{old}@) l' l) :=(@\vspace{-0.04cm}@)
    eq_ind_r ... (@\diff{(rev\_length l)}@)(@\vspace{-0.04cm}@)
  : (@\ltacforall@) l' l, (@\diff{old}@) l' l -> (@\diff{new}@) l' l
\end{lstlisting}
The component can invert this to get the patch: %from \lstinline{new} to \lstinline{old}:

\begin{lstlisting}
  patch l' l (H : (@\diff{new}@) l' l) :=(@\vspace{-0.04cm}@)
    eq_ind_r ... (@\diff{(eq\_sym (rev\_length l))}@)(@\vspace{-0.04cm}@)
  : (@\ltacforall@) l' l, (@\diff{new}@) l' l -> (@\diff{old}@) l' l
\end{lstlisting}
We can then use this patch to port proofs.
For example, if we add this patch to a hint database~\cite{hints},
we can port this proof:

\begin{lstlisting}
  Theorem app_rev_len : (@\ltacforall@) l l',(@\vspace{-0.04cm}@)
    len (rev (l' ++ l)) = len (rev l) + len (rev l').(@\vspace{-0.04cm}@)
  Proof.(@\vspace{-0.04cm}@)
    intros. rewrite rev_app_distr. (@\succeeds{apply old.}@)(@\vspace{-0.04cm}@)
  Qed.
\end{lstlisting}
to this proof:

\begin{lstlisting}
  Theorem app_rev_len : (@\ltacforall@) l l',(@\vspace{-0.04cm}@)
    len (rev (l' ++ l)) = len (rev l) + len (rev l').(@\vspace{-0.04cm}@)
  Proof.(@\vspace{-0.04cm}@)
    intros. rewrite rev_app_distr. (@\succeeds{apply new.}@)(@\vspace{-0.04cm}@)
  Qed.
\end{lstlisting}

Rewrites like \lstinline{candidate} are \textit{invertible}:
We can invert any rewrite in one direction by rewriting in the opposite direction.
In contrast, it is not possible to invert the patch \sysname
found for Figure~\ref{fig:example}.
Inversion will necessarily sometimes fail, since not all terms are invertible.
%Not all terms are invertible. It is not possible, for example, to invert the patch 
%Inversion will necessarily sometimes fail.

\paragraph{Lemma Factoring} The tool should be able to factor a term into a sequence of lemmas.
This can help break other problems, like abstraction, into smaller subproblems.
It is also necessary to invert certain terms.
Consider inverting an arbitrary sequence of two rewrites:

\begin{lstlisting}
  t := (@\diff{eq\_ind\_r G ...}@) ((@\diff{eq\_ind\_r F ...}@))
\end{lstlisting}
We can view \lstinline{t} as a term that composes two functions:

\begin{lstlisting}
  g := (@\diff{eq\_ind\_r G ...}@)(@\vspace{-0.04cm}@)
  f := (@\diff{eq\_ind\_r F ...}@)(@\vspace{-0.04cm}@)
  t := g $\circ$ f
\end{lstlisting}
The inverse of \lstinline{t} is the following:

\begin{lstlisting}
  t$\inv$ := f$\inv$ $\circ$ g$\inv$
\end{lstlisting}
To invert \lstinline{t}, \sysname identifies the factors \lstinline{[f; g]}, 
inverts each factor to \lstinline{[f}$\inv$\lstinline{; g}$\inv$\lstinline{]}, 
then folds and applies the inverse factors in the opposite direction.

plus parts of PUMPKIN PATCH Inside the Core, Testing Boundaries, Future Work

How the four transformations work in detail

Limitations and whether they're addressed in later tools yet or not
