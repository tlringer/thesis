\section{Conclusions \& Future Work}
\label{sec:concl}

We presented \toolnameb: a tool for searching for and lifting across algebraic ornaments in Coq.
\toolnameb is the first tool to lift across ornaments in a non-embedded dependently typed language,
and to automatically infer certain kinds of ornaments from types alone.
Our algorithms give efficient transport across equivalences arising
from algebraic ornaments; our case study demonstrates that such automation can make lifted
terms smaller and faster as part of an incremental workflow.

\subparagraph*{Future Work.} 
A future version may support other ornaments beyond algebraic ornaments,
with additional user interaction as needed; this may help support, for example,
the ornament between \lstinline{nat} and \lstinline{list}, where \lstinline{list}
has a new element in the \lstinline{cons} case.
A future version may loosen restrictions on input types to support
adding constructors while preserving inductive structure, recursive references under products,
and coinductive types. Integrating with \textsc{Pumpkin Patch}~\cite{ringer2018adapting} 
may help remove restrictions \toolnameb makes about the hypotheses of \B.
\lstinline{Preprocess} currently supports only certain fixpoints;
a more general translation may help \toolnameb support more terms, and discussions
with Coq developers suggest that the implementation of such a translation
building on work from the equations~\cite{sozeau:equations} plugin is in progress.
Extending \toolnameb to generate proofs of coherence conditions for lifted terms may increase user confidence.
Proofs that the commands that \toolnameb implements satisfy their specifications may also increase user confidence.
Better automating the recovery of user-friendly types may improve user experience.

