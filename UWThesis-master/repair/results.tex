\section{Results}
\label{sec:pumpkin-results}

% TODO? Needed: key technical results (in intro though)

% below: case studies

To show how \sysname could save work for proof engineers on real proof developments, I used the \sysname prototype on three \kl{case studies} to emulate three motivating scenarios from real-world proof developments:

\begin{enumerate}
\item \textbf{Updating definitions} within a project \\
(CompCert, Section~\ref{sec:compcert})
\item \textbf{Porting definitions} between libraries \\
(Software Foundations, Section~\ref{sec:foundations})
\item \textbf{Updating proof assistant versions} \\
(Coq Standard Library, Section~\ref{sec:coq})
\end{enumerate}
The code I chose for these scenarios demonstrated different classes of changes.
For each case, I describe how \sysname configures the procedure to use differencing and transformations for that class of changes.
My experiences with these scenarios suggest that patches are useful and that both differencing and the transformations 
are effective and flexible.

\paragraph{Identifying Changes} I identified commits from popular projects that
demonstrated each scenario.
I emulated each scenario as follows:

\begin{enumerate}
\item \textit{Replay} an example proof update for \sysname.
\item \textit{Search} the example for a patch using \sysname.
\item \textit{Apply} the patch to fix a different broken proof.
\end{enumerate}
My goal was to simulate incremental use of a repair tool,
at the level of a small change or a commit that follows best practices.
I favored commits with changes that I could isolate.
When isolating examples for \sysname, I replayed changes from the bottom up,
as if I was making the changes myself.
This means that I did not always make the same change as the user. For example,
the real change from Section~\ref{sec:compcert} updated multiple definitions;
I updated only one.

\sysname does not yet handle structural changes like adding constructors or parameters, 
so I focused on changes that preserve structure, like modifying constructors.
Chapter~\ref{chapt:pi} describes an extension to \sysnamelong that supports changes in structure.

\begin{figure*}
\begin{minipage}{0.49\textwidth}
\lstset{language=coq, aboveskip=0pt,belowskip=0pt}
\lstinputlisting[firstline=1, lastline=3]{repair/compcert.tex}
\lstinputlisting[backgroundcolor=\color{orange!35},firstline=4,lastline=4]{repair/compcert.tex}
\lstinputlisting[firstline=5, lastline=5]{repair/compcert.tex}
\end{minipage}
\hfill
\begin{minipage}{0.49\textwidth}
\lstset{language=coq, aboveskip=0pt,belowskip=0pt}
\lstinputlisting[firstline=7, lastline=9]{repair/compcert.tex}
\lstinputlisting[backgroundcolor=\color{orange!35},firstline=10,lastline=10]{repair/compcert.tex}
\lstinputlisting[firstline=11, lastline=11]{repair/compcert.tex}
\end{minipage}
\caption[Caption for LOF]{Old (left) and new (right) definitions of \lstinline{int} in CompCert.}
\label{fig:int}
\end{figure*}

\subsection{Updating Definitions}
\label{sec:compcert}

Coq programmers sometimes make changes to definitions that break proofs
within the same project. To emulate this use case, 
I identified a CompCert commit~\cite{compcertcommit}
with a breaking change to \lstinline{int} (Figure~\ref{fig:int}). %The commit message notes that the new representation is more efficient.
I used \sysname to find a patch that corresponds to the change in \lstinline{int}.
The patch \sysname found fixed broken inductive proofs.

\paragraph{Replay} I used the proof of \lstinline{unsigned_range} as the example for \sysname.
The proof failed with the new \lstinline{int}:

\lstset{language=coq, aboveskip=3pt,belowskip=3pt}
\begin{lstlisting}[language=coq]
  Theorem unsigned_range:
    (@\ltacforall@) (i : int), 0 <= unsigned i < modulus.
  Proof.
    intros i. induction i using int_ind; auto(@\fails{.}@)
\end{lstlisting}
I replayed the change to \lstinline{unsigned_range}:

\begin{lstlisting}[language=coq]
    intros i. induction i using int_ind. (@\diff{simpl. omega}@)(@\succeeds{.}@)
\end{lstlisting}

\paragraph{Search} I used \sysname to search the example for a patch that corresponds to the change in \lstinline{int}. It found
a patch with this type:

\begin{lstlisting}[language=coq]
   (@\ltacforall@) (z : Z), (@\diff{-1 < z < modulus}@) $\rightarrow$ (@\diff{0 <= z < modulus}@)
\end{lstlisting}

\paragraph{Apply} After changing the definition of \lstinline{int}, the proof of the
theorem \lstinline{repr_unsigned} failed on the last tactic:

\begin{lstlisting}[language=coq]
  Theorem repr_unsigned:
    (@\ltacforall@) (i : int), repr (unsigned i) = i.
  Proof.
    $\ldots$ apply Zmod_small; auto(@\fails{.}@)
\end{lstlisting}
Manually trying \lstinline{omega}---the tactic which helped us in the proof of \lstinline{unsigned_range}---did not
succeed.
I added the patch that \sysname found to a hint database.
The proof of the theorem \lstinline{repr_unsigned} then went through:

\begin{lstlisting}[language=coq]
  $\ldots$ apply Zmod_small; auto(@\succeeds{.}@)
\end{lstlisting}

\subsubsection*{Instance}

This scenario used the search procedure instance for changes in constructors of an inductive type.
Given such a change:

\begin{lstlisting}[language=coq]
  Inductive (@\diff{T}@) := $\ldots$ | C : $\ldots$ $\rightarrow$ (@\diff{H}@) $\rightarrow$ T
  Inductive (@\diff{T'}@) := $\ldots$ | C : $\ldots$ $\rightarrow$ (@\diff{H'}@) $\rightarrow$ T'
\end{lstlisting}
\sysname searches in the difference between two inductive proofs of theorems:

\begin{lstlisting}[language=coq]
  (@\ltacforall@) (t : (@\diff{T}@)), P t
  (@\ltacforall@) (t : (@\diff{T'}@)), P t
\end{lstlisting}
for an isomorphism\footnote{If \sysname finds just one implication, it returns that.} between the constructors:

\begin{lstlisting}[language=coq]
  $\ldots$ $\rightarrow$ (@\diff{H}@) $\rightarrow$ (@\diff{H'}@)
  $\ldots$ $\rightarrow$ (@\diff{H'}@) $\rightarrow$ (@\diff{H}@)
\end{lstlisting}
The proof engineer can apply these patches within the inductive case that corresponds to the constructor \lstinline{C}
to fix other broken proofs that induct over the changed type. 
\sysname uses this search procedure instance for changes in constructors:

\begin{algorithm}
\footnotesize
\begin{algorithmic}[1]
    \STATE \kl{diff} inductive constructors for \kl{goals}
    \STATE \kl{diff} and \kl{transform} to recursively search for changes in conclusions of the corresponding case of the proof
    \STATE \textbf{if} there are \kl{candidates} \textbf{then}
    \STATE \hspace*{1em} try to \kl{invert} the \kl{patch} to find an isomorphism 
\end{algorithmic}
\end{algorithm}

\subsection{Porting Definitions}
\label{sec:foundations}

\begin{figure*}
\begin{minipage}{0.49\textwidth}
\lstset{language=coq, aboveskip=0pt,belowskip=0pt} % blindFS / User A
\lstinputlisting[firstline=2, lastline=5]{repair/bintonat.tex}
\lstinputlisting[backgroundcolor=\color{orange!35},firstline=6,lastline=6]{repair/bintonat.tex}
\lstinputlisting[firstline=7, lastline=7]{repair/bintonat.tex}
\lstinputlisting[backgroundcolor=\color{orange!35},firstline=8,lastline=8]{repair/bintonat.tex}
\lstinputlisting[firstline=9, lastline=9]{repair/bintonat.tex}
\end{minipage}
\hfill
\begin{minipage}{0.49\textwidth}
\lstset{language=coq, aboveskip=0pt,belowskip=0pt}  % marshall / user B
\lstinputlisting[firstline=12, lastline=15]{repair/bintonat.tex}
\lstinputlisting[backgroundcolor=\color{orange!35},firstline=16,lastline=17]{repair/bintonat.tex}
\lstinputlisting[firstline=18, lastline=18]{repair/bintonat.tex}
\lstinputlisting[backgroundcolor=\color{orange!35},firstline=19,lastline=20]{repair/bintonat.tex}
\lstinputlisting[firstline=21, lastline=21]{repair/bintonat.tex}
\end{minipage}
\caption[Caption for LOF]{Definitions of \lstinline{bin_to_nat} for Users A (left) and B (right). Note that \lstinline{bin_to_nat} uses pattern matching and fixpoints
rather than \kl{primitive eliminators}, unlike most of the terms in this thesis.}
\label{fig:bintonat}
\end{figure*}

\lstset{language=coq, aboveskip=3pt,belowskip=3pt}

Proof engineers sometimes port theorems and proofs to use definitions from different libraries.
To simulate this, I used \sysname to port two solutions~\cite{usera, userb}
to an exercise in Software Foundations to each use the other solution's definition of the fixpoint \lstinline{bin_to_nat} (Figure~\ref{fig:bintonat}).
I demonstrate one direction; the opposite was similar.

\paragraph{Replay} I used the proof of \lstinline{bin_to_nat_pres_incr} from User A as the example for \sysname.
User A cut an inline lemma in an inductive case and proved it using a rewrite:

\lstset{language=coq, aboveskip=3pt,belowskip=3pt}
\begin{lstlisting}[language=coq]
  assert ((@\ltacforall@) a, S (a + S (@\diff{(a + 0)}@)) = S (S (a + (@\diff{(a + 0)}@)))).
  - $\ldots$ (@\diff{rewrite $\leftarrow$ plus\_n\_O.}@) rewrite $\rightarrow$ plus_comm.
\end{lstlisting} % blindFS / user A
When I ported the solution by User A to use User B's definition of \lstinline{bin_to_nat}, 
the application of this inline lemma failed. I changed the conclusion of the lemma 
and removed the corresponding rewrite:

\begin{lstlisting}[language=coq]
  assert ((@\ltacforall@) a, S (a + S a) = S (S (a + a))).
  - $\ldots$ rewrite $\rightarrow$ plus_comm.
\end{lstlisting} % blindFS / user A, adapted

\paragraph{Search} I used \sysname to search for a patch that corresponds to the change in \lstinline{bin_to_nat}.
It found an isomorphism:

\begin{lstlisting}[language=coq]
  (@\ltacforall@) P b, P (@\diff{(bin\_to\_nat b)}@) $\rightarrow$ P (@\diff{(bin\_to\_nat b + 0)}@)
  (@\ltacforall@) P b, P (@\diff{(bin\_to\_nat b + 0)}@) $\rightarrow$ P (@\diff{(bin\_to\_nat b)}@)
\end{lstlisting}

\paragraph{Apply} After porting to User B's definition, a rewrite in the proof of the theorem
\lstinline{normalize_correctness} failed:

\begin{lstlisting}[language=coq]
  Theorem normalize_correctness:
    (@\ltacforall@) b, nat_to_bin (bin_to_nat b) = normalize b.
  Proof.
    $\ldots$ (@\fails{rewrite $\rightarrow$ plus\_0\_r.}@)
\end{lstlisting}
Attempting the obvious patch from the difference in tactics---rewriting by \lstinline{plus_n_O}---failed.
Applying the patch that \sysname found fixed the broken proof:

\begin{lstlisting}[language=coq]
   $\ldots$ (@\diff{apply patch\_inv.}@) (@\succeeds{rewrite $\rightarrow$ plus\_0\_r.}@)
\end{lstlisting}

In this case, since I ported User A's definition to a simpler 
definition,\footnote{User A uses \lstinline{*}; User B uses \lstinline{+}. 
For arbitrary \lstinline{n}, the term \lstinline{2 * n} reduces to \lstinline{n + (n + 0)}, which does not reduce any further.}
\sysname found a patch that was not the most natural patch.
The natural patch would be to remove the rewrite, just as I removed a different rewrite from the example proof.
This did not occur when I ported User B's definition,
which suggests that in the future, a proof repair tool may help inform novice users which definition is simpler:
It can factor the proof,  
then inform the user if two factors are inverses.
Tactic-level changes do not provide enough information to determine this; the tool must have a semantic
understanding of the terms.
My Magic tutorial plugin\footnote{\url{https://github.com/uwplse/magic}} % TODO static link
implements a prototype of the functionality needed for this, based on lessons from this case study.

\subsubsection*{Instance}

This scenario used the search procedure instance for changes in cases of a fixpoint.
Given such a change:

\begin{lstlisting}[language=coq]
  Fixpoint (@\diff{f}@) $\ldots$ := $\ldots$ | g (@\diff{x}@)
  Fixpoint (@\diff{f'}@) $\ldots$ := $\ldots$ | g (@\diff{x'}@)
\end{lstlisting}
\sysname searches in the difference between two versions of proofs of the theorems:

\begin{lstlisting}[language=coq]
  (@\ltacforall@) $\ldots$, P ((@\diff{f}@) $\ldots$)
  (@\ltacforall@) $\ldots$, P ((@\diff{f'}@) $\ldots$)
\end{lstlisting}
for an isomorphism that corresponds to the change:

\begin{lstlisting}[language=coq]
  (@\ltacforall@) P, P (@\diff{x}@) $\rightarrow$ P (@\diff{x'}@)
  (@\ltacforall@) P, P (@\diff{x'}@) $\rightarrow$ P (@\diff{x}@)
\end{lstlisting}
The proof engineer can apply these patches to fix other broken proofs about the fixpoint.

The key feature that differentiates these from the patches we have encountered so far is that
these patches hold for \emph{all} \lstinline{P}; for changes in fixpoint cases, the procedure generalizes
candidates by \lstinline{P}, not by its arguments.
\sysname uses this search procedure instance for changes in fixpoint cases:

\begin{algorithm}
\footnotesize
\begin{algorithmic}[1]
    \STATE \kl{diff} fixpoint cases for goals
    \STATE \kl{diff} and \kl{transform} to recursively search within an intermediate lemma for a change in conclusions
    \STATE \textbf{if} there are \kl{candidates} \textbf{then}
    \STATE \hspace*{1em} \kl{specialize} and \kl{factor} the \kl{candidate} \\
           \hspace*{1em} \kl{generalize} the factors by functions \\
           \hspace*{1em} try to \kl{invert} the \kl{patch} to find an isomorphism 
\end{algorithmic}
\end{algorithm}

For the prototype, I require the user to cut the intermediate lemma explicitly and to 
pass its type and arguments.
In the future, an improved semantic differencing component
can infer both the intermediate lemma and the arguments: it can search
within the proof for some proof of a function that is applied
to the fixpoint.

\begin{figure*}
\begin{minipage}{0.48\textwidth}
\lstset{language=coq, aboveskip=0pt,belowskip=0pt}
\lstinputlisting[firstline=1, lastline=2]{repair/divide.tex}
\end{minipage}
\hfill
\begin{minipage}{0.48\textwidth}
\lstset{language=coq, aboveskip=0pt,belowskip=0pt}
\lstinputlisting[firstline=4, lastline=5]{repair/divide.tex}
\end{minipage}
\caption[Caption for LOF]{Old (left) and new (right) definitions of \lstinline{divide} in Coq.}
\label{fig:divide}
\end{figure*}

\lstset{language=coq, aboveskip=3pt,belowskip=3pt}

\subsection{Updating Proof Assistant Versions}
\label{sec:coq}

Coq sometimes makes changes to its standard library that break
backwards compatibility.
To test the plausibility of using a patch finding tool for proof assistant version updates,
I identified a breaking change in the Coq standard library~\cite{coq84commit}.
The commit changed the definition of \lstinline{divide} prior to the Coq 8.4 release (Figure~\ref{fig:divide}).
The change broke 46 proofs in the standard library.
I used \sysname to find an isomorphism that corresponds to the change in \lstinline{divide}.
The isomorphism \sysname found fixed broken proofs.

\paragraph{Replay} I used the proof of \lstinline{mod_divide} as the example for \sysname.
The proof broke with the new \lstinline{divide}:

\lstset{language=coq, aboveskip=3pt,belowskip=3pt}
\begin{lstlisting}[language=coq]
  Theorem mod_divide:
    (@\ltacforall@) a b, b~=0 $\rightarrow$ (a mod b == 0 $\leftrightarrow$ (divide b a)).
  Proof.
    $\ldots$ (@\fails{rewrite (div\_mod a b Hb) at 2.}@)
\end{lstlisting}
I replayed changes to \lstinline{mod_divide}:

\begin{lstlisting}[language=coq]
    $\ldots$ (@\diff{rewrite mul\_comm. symmetry.}@)
    (@\succeeds{rewrite (div\_mod a b Hb) at 2.}@)
\end{lstlisting}

\paragraph{Search} I used \sysname to search within the example patched proof for a patch
that corresponds to the change in \lstinline{divide}.
It found an isomorphism:

\begin{lstlisting}[language=coq]
   (@\ltacforall@) r p q, (@\diff{p * r = q}@) $\rightarrow$ (@\diff{q = r * p}@)
   (@\ltacforall@) r p q, (@\diff{q = r * p}@) $\rightarrow$ (@\diff{p * r = q}@)
\end{lstlisting}

\paragraph{Apply} The proof of the theorem \lstinline{Zmod_divides} broke after rewriting by the changed theorem \lstinline{mod_divide}:

\begin{lstlisting}[language=coq]
  Theorem Zmod_divides:
    (@\ltacforall@)a b, b<>0 $\rightarrow$ (a mod b = 0 $\leftrightarrow$ $\exists$ c, a = b * c).
  Proof.
    $\ldots$ split; intros (c,Hc); exists c; (@\fails{auto.}@)
\end{lstlisting}
Adding the patches \sysname found to a hint database made the proof go through:

\begin{lstlisting}[language=coq]
    $\ldots$ split; intros (c,Hc); exists c; (@\succeeds{auto.}@)
\end{lstlisting}

\subsubsection*{Instance}

This scenario used the search procedure instance for changes in dependent arguments to constructors.
\sysname searches within the difference between two versions of a proof that apply the same constructor
to different dependent arguments:

\begin{lstlisting}[language=coq]
   $\ldots$ (C ((@\diff{P}@) x)) $\ldots$
   $\ldots$ (C ((@\diff{P'}@) x)) $\ldots$
\end{lstlisting}
for an isomorphism between the arguments:

\begin{lstlisting}[language=coq]
   (@\ltacforall@) x, (@\diff{P}@) x $\rightarrow$ (@\diff{P'}@) x
   (@\ltacforall@) x, (@\diff{P'}@) x $\rightarrow$ (@\diff{P}@) x
\end{lstlisting}
The proof engineer can apply these patches to patch proofs that apply the constructor (in this case study,
to fix broken proofs that instantiate \lstinline{divide} with some specific \lstinline{r}).

So far, we have encountered changes of this form as arguments to an 
induction principle; in this case, the change is an argument to a constructor.
A patch between arguments to an induction principle maps
directly between conclusions of the new and old theorem without
induction; a patch between constructors does not.
For example, for \lstinline{divide}, we can find a patch with this form:

\begin{lstlisting}[language=coq]
   (@\ltacforall@)x, (@\diff{P}@) x $\rightarrow$ (@\diff{P'}@) x
\end{lstlisting}
However, without using the induction principle for \lstinline{exists}, we can't use that patch to prove this:

\begin{lstlisting}[language=coq]
   ($\exists$ x, (@\diff{P}@) x) $\rightarrow$ ($\exists$ x, (@\diff{P'}@) x)
\end{lstlisting}
This changes the goal type that semantic differencing determines.

\sysname uses this search procedure instance for changes in constructor arguments:

\begin{algorithm}
\footnotesize
\begin{algorithmic}[1]
    \STATE \kl{diff} constructor arguments for \kl{goals}
    \STATE \kl{diff} and \textit{transform} to recursively search within those arguments for changes in conclusions
    \STATE \textbf{if} there are candidates \textbf{then}
    \STATE \hspace*{1em} \kl{generalize} the \kl{candidate} \\
           \hspace*{1em} \kl{factor} and try to \kl{invert} the \kl{patch} to find an isomorphism
\end{algorithmic}
\end{algorithm}
For the prototype, the model of constructors for the semantic differencing component is limited,
so \sysname asks the user to provide the type of the change in argument (to guide line 2).
Extending semantic differencing may help remove this restriction.




