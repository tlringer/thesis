\section{Implementation}
\label{sec:impl}

The \toolnameb Coq plugin implements the algorithms from Section~\ref{sec:alg};
the link to the code is in \textbf{Supplement Material}. 
\toolnameb cannot produce an ill-typed term, since Coq type checks all terms that plugins produce
and rejects ill-typed terms.
The implementations of 
\lstinline{Find ornament} (\href{http://github.com/uwplse/ornamental-search/blob/itp+equiv/plugin/src/automation/search.ml}{\lstinline{search.ml}}) 
and 
\lstinline{Lift} (\href{http://github.com/uwplse/ornamental-search/blob/itp+equiv/plugin/src/automation/search.ml}{\lstinline{lift.ml}}) 
are mostly the same as the algorithms,
but with changes to address implementation challenges that scale the algorithms to a Coq tool for proof engineers.
This section describes a sample of these changes from each of three categories: 
addressing differences between Coq and the type theory that the algorithms assume (Section~\ref{sec:langdiff}), 
optimizing for efficiency (Section~\ref{sec:opteff}),
and improving usability (Section~\ref{sec:autouse}).

\subsection{Addressing Language Differences}
\label{sec:langdiff}

\subparagraph*{Fixpoints.}
Coq implements eliminators in terms of pattern matching and fixpoints. 
To handle terms that use these features, \toolnameb includes a \lstinline{Preprocess} command that translates these terms into equivalent eliminator applications.
This command can preprocess a definition (like \lstinline{zip} from Section~\ref{sec:example}) or an entire module
(like \lstinline{List}, as shown in \href{http://github.com/uwplse/ornamental-search/blob/itp+equiv/plugin/coq/examples/ListToVect.v}{\lstinline{ListToVect.v}}) for lifting.
It currently supports fixpoints that are structurally recursive on only immediate substructures.
To translate such a fixpoint, it first extracts a motive, then generates each case
by partially reducing the function's body under a hypothetical context for the constructor arguments.
This is enough to preprocess \lstinline{List}; Section~\ref{sec:concl} discusses possible extensions.

\subparagraph*{Non-Primitive Projections.} By default, projections in Coq are non-primitive. That is, this:
\begin{lstlisting}
$\forall$ (T : Type) (v : $\Sigma$ (n : nat).vector T n), v = $\exists$ ($\pi_l$ v) ($\pi_r$ v).
\end{lstlisting}
cannot be proven by reflexivity alone (see \href{http://github.com/uwplse/ornamental-search/blob/itp+equiv/plugin/coq/examples/Projections.v}{\lstinline{Projections.v}}).
Therefore, \toolnameb must pack terms like \lstinline{v} into existentials; otherwise, lifting
will sometimes fail. This is why the type of \codeauto{\lstinline{zip_with_is_zipV_p}} 
in the example from Section~\ref{sec:example} packs \lstinline{v}$_1$ and \lstinline{v}$_2$.
For the sake of performance and readability of lifted code, \toolnameb is strategic about when it packs.

\subparagraph*{Constants.}
Because Coq has constants, the implementation of \textsc{Normalize} 
refolds~\cite{boutillier:tel-01054723} after normalizing.
That is, it acts like the \lstinline{simpl}
tactic in Coq, but with special support for sigma types.
For example, to lift the \lstinline{cons} constructor of a list, after normalizing the promotion of \lstinline{cons t l},
\toolnameb substitutes the projections of the promotion of \lstinline{l} for their normal forms,
which determines and saves the following fact:
\begin{lstlisting}
$\forall$ {T} (l : list T), $\uparrow$ (cons t l) = $\exists$ (S ($\uparrow_{I_B}$ l)) (consV ($\uparrow_{I_B}$ l) t ($\uparrow_{B}$ l)).
\end{lstlisting}
Refolding helps produce more readable lifted code.
It also improves lifting performance,
since it occurs just once for each constructor.

\subsection{Optimizing for Efficiency}
\label{sec:opteff}

\subparagraph*{Delayed Reduction.}
When lifting eliminators, \toolnameb computes a list of arguments and delays reduction.
It computes this list backwards, storing the new indices that inductive hypotheses
refer to as it recurses. This removes the call to \lstinline{new} in the premise of \textsc{Drop-Index}.

\subparagraph*{Lazy $\eta$-Expansion.}
The lifting algorithm assumes that all terms are fully \smallmath{$\eta$}-expanded. Sometimes,
however, \smallmath{$\eta$}-expansion is not necessary.
For efficiency, rather than fully \smallmath{$\eta$}-expand ahead of time, \toolnameb \smallmath{$\eta$}-expands lazily, 
only when it is necessary for correctness.

\subparagraph*{Caching.} 
To prevent extra recursion, \toolnameb caches the outputs of search,
as well as lifted constants, inductive types, and constructors.
Since these are constants, lookup is low-cost.

\subsection{Improving Usability}
\label{sec:autouse}

\subparagraph*{Correctness Proofs.}
\toolnameb has options (used in \href{http://github.com/uwplse/ornamental-search/blob/itp+equiv/plugin/coq/examples/Example.v}{\lstinline{Example.v}}) that tell search to generate proofs that its outputs are correct, thereby increasing confidence in and
usefulness of those outputs.
The proof of \lstinline{coherence} is reflexivity.
The intuition behind 
the automation to prove \lstinline{section} and \lstinline{retraction} (\href{http://github.com/uwplse/ornamental-search/blob/itp+equiv/plugin/src/automation/equivalence.ml}{\lstinline{equivalence.ml}}) 
is that
\lstinline{promote} and \lstinline{forget} map along corresponding constructors, so inductive cases preserve equalities.
Thus, each inductive case of these proofs is generated by a fold that rewrites each recursive reference,
with reflexivity as identity.


\subparagraph*{Unpacking.}
\toolnameb includes an \lstinline{Unpack} command (used in \href{http://github.com/uwplse/ornamental-search/blob/itp+equiv/plugin/coq/examples/Example.v}{\lstinline{Example.v}})
that unpacks packed types in functions and proofs. 
This way, users may access unpacked terms without writing boilerplate code.
For simple functions, this command packs arguments and projects
results. It splits higher-order functions into two functions. For proofs that use equality, it applies 
one lemma convert to dependent equality, and one
lemma to deal with non-primitive projections.

\subparagraph*{User-Friendly Types.}
\href{http://github.com/uwplse/ornamental-search/blob/itp+equiv/plugin/coq/examples/Example.v}{\lstinline{Example.v}} describes how the user can recover user-friendly types
after unpacking. For example, to recover a function with an output of type \lstinline{vector T n},
the user lifts a proof that the length of the output of the unlifted \lstinline{list} version of that function is \lstinline{n}, then rewrites by that 
lifted proof. The intuition behind this is that this equivalence holds:
\begin{lstlisting}
{ l : list T & length l = n } $\simeq$ vector T n
\end{lstlisting}
Recovering a user-friendly type for a proof relating these functions is more complex, since
it necessitates reasoning at some point about equalities between equalities. For some index types like \lstinline{nat}, this follows simply
from the fact that the type forms an h-set~\cite{univalent2013homotopy}:
all proofs of equality between the same two terms of that type are equal.
There is preliminary work on determining a general methodology for deriving user-friendly types for proofs
that does not rely on any properties of the index type. The idea is to use the adjunction condition
along with the proof of \lstinline{coherence} by reflexivity; see \href{http://github.com/uwplse/ornamental-search/issues/39}{GitHub issue \#39} for the status of this work.

