\section{Transformation}
\label{sec:pi-trans}

% PUMPKIN Pi, plus DEVOID example

\begin{figure*}
\begin{mathpar}
\mprset{flushleft}
\small
\hfill\fbox{$\Gamma$ $\vdash$ $t$ $\Uparrow$ $t'$}\vspace{-0.3cm}\\

\inferrule[Dep-Elim]
  { \Gamma \vdash a \Uparrow b \\ \Gamma \vdash p_{a} \Uparrow p_b \\ \Gamma \vdash \vec{f_{a}}\phantom{l} \Uparrow \vec{f_{b}} }
  { \Gamma \vdash \mathrm{DepElim}(a,\ p_{a}) \vec{f_{a}} \Uparrow \mathrm{DepElim}(b,\ p_b) \vec{f_{b}} }

\inferrule[Dep-Constr]
{ \Gamma \vdash \vec{t}_{a} \Uparrow \vec{t}_{b} } %\\ TODO must we explicitly lift A to B if we want to handle parameters/indices?
{ \Gamma \vdash \mathrm{DepConstr}(j,\ A)\ \vec{t}_{a} \Uparrow \mathrm{DepConstr}(j,\ B)\ \vec{t}_{b}  }

\inferrule[Eta]
  { \\ }
  { \Gamma \vdash \mathrm{Eta}(A) \Uparrow \mathrm{Eta}(B) }

\inferrule[Iota]
  { \Gamma \vdash q_A \Uparrow q_B \\ \Gamma \vdash \vec{t_A} \Uparrow \vec{t_B} }
  { \Gamma \vdash \mathrm{Iota}(j,\ A,\ q_A)\ \vec{t_A} \Uparrow \mathrm{Iota}(j,\ B,\ q_B)\ \vec{t_B} }

\inferrule[Equivalence]
  { \\ }
  { \Gamma \vdash A\ \Uparrow B }

\inferrule[Constr]
{ \Gamma \vdash T \Uparrow T' \\ \Gamma \vdash \vec{t} \Uparrow \vec{t'} }
{ \Gamma \vdash \mathrm{Constr}(j,\ T)\ \vec{t} \Uparrow \mathrm{Constr}(j,\ T')\ \vec{t'} }

\inferrule[Ind]
  { \Gamma \vdash T \Uparrow T' \\ \Gamma \vdash \vec{C} \Uparrow \vec{C'}  }
  { \Gamma \vdash \mathrm{Ind} (\mathit{Ty} : T) \vec{C} \Uparrow \mathrm{Ind} (\mathit{Ty} : T') \vec{C'} }

%% Application
\inferrule[App]
 { \Gamma \vdash f \Uparrow f' \\ \Gamma \vdash t \Uparrow t'}
 { \Gamma \vdash f t \Uparrow f' t' }

\inferrule[Elim] % TODO wait why do we have c here when it clearly refers to the term we eliminate over? um
  { \Gamma \vdash c \Uparrow c' \\ \Gamma \vdash Q \Uparrow Q' \\ \Gamma \vdash \vec{f} \Uparrow \vec{f'}}
  { \Gamma \vdash \mathrm{Elim}(c, Q) \vec{f} \Uparrow \mathrm{Elim}(c', Q') \vec{f'}  }

% Lamda
\inferrule[Lam]
  { \Gamma \vdash t \Uparrow t' \\ \Gamma \vdash T \Uparrow T' \\ \Gamma,\ t : T \vdash b \Uparrow b' }
  {\Gamma \vdash \lambda (t : T).b \Uparrow \lambda (t' : T').b'}

% Product
\inferrule[Prod]
  { \Gamma \vdash t \Uparrow t' \\ \Gamma \vdash T \Uparrow T' \\ \Gamma,\ t : T \vdash b \Uparrow b' }
  {\Gamma \vdash \Pi (t : T).b \Uparrow \Pi (t' : T').b'}

\inferrule[Var]
  { v \in \mathrm{Vars} }
  {\Gamma \vdash v \Uparrow v}

%\inferrule[Sort]
%  { \\ }
%  {\Gamma \vdash s \Uparrow s}
\end{mathpar}
\vspace{-0.3cm}
\caption{Transformation for transporting terms across $A \simeq B$ with configuration \lstinline{((DepConstr, DepElim), (Eta, Iota))}.}
\label{fig:final}
\end{figure*}

\begin{figure*}
\begin{minipage}{0.49\textwidth}
\begin{lstlisting}
(* 1: original term *)(@\vspace{-0.04cm}@)
$\lambda$ (T : Type) (l m : Old.list T) .(@\vspace{-0.04cm}@)
 Elim(l, $\lambda$(l: Old.list T).Old.list T $\rightarrow$ Old.list T)) {(@\vspace{-0.04cm}@)
   ($\lambda$ m . m),(@\vspace{-0.04cm}@)
   ($\lambda$ t _ IHl m . Constr(1, Old.list T) t (IHl m))(@\vspace{-0.04cm}@)
 } m.(@\vspace{-0.04cm}@)
(@\vspace{-0.10cm}@)
(* 2: after unifying (@\texttt{with}@) configuration *)(@\vspace{-0.04cm}@)
$\lambda$ (T : Type) (l m : (@\codediff{A}@)) .(@\vspace{-0.04cm}@)
 (@\codediff{DepElim}@)(l, $\lambda$(l: (@\codediff{A}@)).(@\codediff{A}@) $\rightarrow$ (@\codediff{A}@))) {(@\vspace{-0.04cm}@)
   ($\lambda$ m . m)(@\vspace{-0.04cm}@)
   ($\lambda$ t _ IHl m . (@\codediff{DepConstr}@)(1, (@\codediff{A}@)) t (IHl m))(@\vspace{-0.04cm}@)
 } m.(@\vspace{-0.04cm}@)
\end{lstlisting}
\end{minipage}
\hfill
\begin{minipage}{0.49\textwidth}
\begin{lstlisting}
(* 4: reduced to final term *)(@\vspace{-0.04cm}@)
$\lambda$ (T : Type) (l m : New.list T) .(@\vspace{-0.04cm}@)
 Elim(l, $\lambda$(l: New.list T).New.list T $\rightarrow$ New.list T)) {(@\vspace{-0.04cm}@)
   ($\lambda$ t _ IHl m . Constr(0, New.list T) t (IHl m)),(@\vspace{-0.04cm}@)
   ($\lambda$ m . m)(@\vspace{-0.04cm}@)
 } m.(@\vspace{-0.04cm}@)
(@\vspace{-0.10cm}@)
(* 3: after transforming *)(@\vspace{-0.04cm}@)
$\lambda$ (T : Type) (l m : (@\codediff{B}@)) .(@\vspace{-0.04cm}@)
 (@\codediff{DepElim}@)(l, $\lambda$(l: (@\codediff{B}@)).(@\codediff{B}@) $\rightarrow$ (@\codediff{B}@))) {(@\vspace{-0.04cm}@)
   ($\lambda$ m . m)(@\vspace{-0.04cm}@)
   ($\lambda$ t _ IHl m . (@\codediff{DepConstr}@)(1, (@\codediff{B}@)) t (IHl m))(@\vspace{-0.04cm}@)
 } m.(@\vspace{-0.04cm}@)
\end{lstlisting}
\end{minipage}
\vspace{-0.3cm}
\caption{Swapping cases of the append function, counterclockwise, the input term: 1) unmodified, 2) unified with the configuration, 3) ported to the updated type, and 4) reduced to the output.}
\label{fig:appswap1}
\end{figure*}

At the heart of \toolnamec is a configurable proof term transformation for transporting
proofs across equivalences~\href{https://github.com/uwplse/pumpkin-pi/blob/v2.0.0/plugin/src/automation/lift/lift.ml}{\circled{4}}.
The transformation takes as input the configuration.
This section introduces the transformation that builds on that.
Section~\ref{sec:pi-implementation} describes the additional work needed to implement this transformation.

Figure~\ref{fig:final} shows the proof term transformation $\Gamma \vdash t \Uparrow t'$ that forms the core of \toolnamec.
The transformation is parameterized over equivalent types \Aa and \B (\textsc{Equivalence})
as well as the configuration.
It assumes $\eta$-expanded functions.
It implicitly constructs an updated context $\Gamma'$ in which to interpret $t'$, but this is not needed for computation.

The proof term transformation is (perhaps deceptively) simple by design:
it moves the bulk of the work into the configuration,
and represents the configuration explicitly.
This configuration either comes from automatic configuration (like the search procedure in the previous section),
or directly from the proof engineer.
Of course, in both of these cases, typical proof terms in Coq do not apply these configuration
terms explicitly.
\toolnamec does some additional work using \textit{unification heuristics} to get real proof terms into this format before running the transformation.
It then runs the proof term transformation, which transports proofs across the equivalence that corresponds to the configuration.

\subsection{Unification Heuristics}
The transformation does not fully describe the search procedure for transforming terms that \toolnamec implements.
Before running the transformation, \toolnamec \textit{unifies} subterms with particular \Aa (fixing parameters and indices),
and with applications of configuration terms over \Aa. 
The transformation then transforms configuration terms over \Aa
to configuration terms over \B.
Reducing the result produces the output term defined over \B.

Figure~\ref{fig:appswap1} shows this with the list append function \lstinline{++} from Section~\ref{sec:overview}.
To update \lstinline{++} (top left), \toolnamec unifies \lstinline{Old.list T} with \Aa, and \lstinline{Constr} and \lstinline{Elim}
with \lstinline{DepConstr} and \lstinline{DepElim} (bottom left).
After unification, the transformation recursively substitutes \B
for \Aa, which moves \lstinline{DepConstr} and \lstinline{DepElim}
to construct and eliminate over the updated type (bottom right).
This reduces to a term with swapped constructors and cases over \lstinline{New.list T} (top right).

In this case, unification is straightforward.
This can be more challenging when configuration terms are dependent.
This is especially pronounced with definitional \lstinline{Eta} and \lstinline{Iota},
which typically are implicit (reduced) in real code.
To handle this, \toolnamec implements custom \textit{unification heuristics} for each search procedure
that unify subterms with applications of configuration terms, and that instantiate parameters and dependent indices in those subterms~\href{https://github.com/uwplse/pumpkin-pi/blob/v2.0.0/plugin/src/automation/lift/liftconfig.ml}{\circled{6}}. % liftconfig.ml
The transformation in turn assumes that all existing parameters and indices are determined and instantiated
by the time it runs.

\toolnamec falls back to Coq's unification for manual configuration and when these custom heuristics fail.
When even Coq's unification is not enough, \toolnamec relies on proof engineers to provide hints
in the form of annotations~\href{https://github.com/uwplse/pumpkin-pi/blob/v2.0.0/plugin/coq/nonorn.v}{\circled{5}}.

\iffalse
\begin{figure}
\begin{lstlisting}
(@\vspace{-0.4cm}@)
$\uparrow$$\phantom{_{I_{B}}}$(@\hspace{0.06cm}@) {$\vec{i_a}$ : $\vec{\mathrm{X}_A}$} := promote $\vec{i_a}$. $\phantom{separations}$ $\downarrow$$\phantom{_{I_{B}}}$(@\hspace{0.01cm}@) {$\vec{i_b}$ : $\vec{\mathrm{X}_B}$} := forget $\vec{i_b}$.(@\vspace{-0.04cm}@)
$\pi_{I_B}$ {$\vec{i_a}$ : $\vec{\mathrm{X}_A}$} := indexer $\vec{i_a}$. $\phantom{separations}$ $\exists_{I_B}$ {$\vec{i_b}$ : $\vec{\mathrm{X}_B}$} ($b$ : $B$ $\vec{i_b}$) := $\exists$ $\vec{i_b}$[off] b.(@\vspace{-0.02cm}@)
$\uparrow_{B}$(@\hspace{0.11cm}@) := $\pi_{r}$ $\circ$ $\uparrow$. $\phantom{separationseparationsepara}$(@\hspace{0.02cm}@) $\downarrow_{A}$(@\hspace{0.11cm}@) := $\downarrow$(@\hspace{0.37cm}@) $\circ$ $\exists_{I_B}$.(@\vspace{-0.02cm}@)
$\uparrow_{I_B}$ := $\pi_{l}$(@\hspace{0.04cm}@)  $\circ$ $\uparrow$. $\phantom{separationseparationsepar}$(@\hspace{0.18cm}@) $\downarrow_{I_B}$ := $\pi_{I_B}$ $\circ$ $\downarrow_{A}$.(@\vspace{-0.02cm}@)
\end{lstlisting}
\vspace{-0.3cm}
\caption{Common definitions for the core lifting algorithm.}
\label{fig:convenience}
\end{figure}

\subparagraph*{Common Definitions.} The common definitions (Figure~\ref{fig:convenience}) %instantiate the equivalence with the types
%\lstinline{from} and \lstinline{to}, which represent the two equivalent types that we are lifting from and to, respectively. 
define some useful syntax:
\smallmath{$\uparrow$} applies \lstinline{promote}, \smallmath{$\downarrow$} applies
\lstinline{forget}, and \smallmath{$\pi_{I_B}$} applies \lstinline{indexer}.
\smallmath{$\exists_{I_B}$} packs a term of type \B\
into an existential with the index at the appropriate offset. \smallmath{$\uparrow_{B}$} and \smallmath{$\uparrow_{I_B}$} promote and then project;
\smallmath{$\downarrow_{A}$} packs and forgets, and \smallmath{$\downarrow_{I_B}$} packs, forgets, and then 
applies \lstinline{indexer} to project the index.

\paragraph{Lifting Eliminators}

The \smallmath{$\Gamma$ $\vdash$ $t$ $\Uparrow_E$ $t'$} judgment (Figure~\ref{fig:ind}) defines rules for lifting the motive and case of an
eliminator, changing the \textit{domain of induction} from \Aa\ to \B.
The intuition is that any term of type \Aa\ is the result of forgetting some term of type packed \B.
Then, since \Aa\ and \B\ have the same inductive structure, we can lift the eliminator of \Aa\ to the eliminator of \B,
and move that forgetfulness \textit{inside of each case}.
For example,
the following terms are propositionally equal:\vspace{-0.2cm}\\\\
\begin{minipage}{0.39\textwidth}
\begin{lstlisting}
Elim((@\codediff{($\downarrow_{A}$ b)}@), p$_A$){
  f$_{\mathrm{nil}}$,
  ($\lambda$(t$_l$:T)(l:list T)(IH$_l$:p$_A$ l).
    f$_{\mathrm{cons}}$ t$_l$ l IH$_l$)
}
\end{lstlisting}
\end{minipage}
\hspace{-0.2cm}
\begin{minipage}{0.61\textwidth}
\begin{lstlisting}
Elim(b, $\lambda$(n:nat)(v:vector T n).p$_A$ (@\codediff{($\downarrow_{A}$ v)}@)){
  f$_{\mathrm{nil}}$,
  ($\lambda$(n:nat)(t$_v$:T)(v:vector T n)(IH$_v$:p$_A$ (@\codediff{($\downarrow_{A}$ v)}@)).
     f$_{\mathrm{cons}}$ t$_v$ (@\codediff{($\downarrow_{A}$ v)}@) IH$_v$)
}
\end{lstlisting}
\end{minipage}

\begin{figure}
\begin{mathpar}
\mprset{flushleft}
\small
\hfill\fbox{$\Gamma$ $\vdash$ $(t,\ T)$ $\Uparrow_{E_x}$ $t'$}\vspace{-0.4cm}\\

\inferrule[Drop-Index]
{ \mathrm{new}\ n\ b \\ \Gamma,\ n : t \vdash (f,\ b) \Uparrow_{E_x} b' }
{ \Gamma \vdash (f,\ \Pi (n : t) . b) \Uparrow_{E_x} \lambda (n : t) . b' }

\inferrule[Forget-Arg]
{ \Gamma \vdash \vec{i} : \vec{\mathrm{X}_B} \\ \Gamma,\ n : B\ \vec{i} \vdash ((f\ (\downarrow_A n))_\beta,\ b) \Uparrow_{E_x} b' }
{ \Gamma \vdash (f,\ \Pi (n : B\ \vec{i}) . b) \Uparrow_{E_x} \lambda (n : B\ \vec{i}) . b' }

\inferrule[Arg] % if we get rid of fall through, needs: \neg\ (\mathrm{new}\ n\ b) \\ \Gamma \vdash T \not\defeq B, and should be \Pi (n : T\ \vec{i}) 
{ \Gamma,\ n : t \vdash((f\ n)_\beta,\ b) \Uparrow_{E_x} b' }
{ \Gamma \vdash (f,\ \Pi (n : t) . b) \Uparrow_{E_x} \lambda (n : t) . b' }

\inferrule[Concl]
{ \\ }
{ \Gamma \vdash (t,\ p_B\ \vec{y}) \Uparrow_{E_x} t}

\hfill\phantom{sorrysorrysorrysorrysorrysorrysorrysorrysorrysorrysorrysorrysorrysorrysorrys}\fbox{$\Gamma$ $\vdash$ $t$ $\Uparrow_E$ $t'$}\vspace{-0.4cm}\\

% Property
\inferrule[Motive]
  { \Gamma \vdash p_A : \mathrm{P}_A }
  { \Gamma \vdash p_A \Uparrow_E \lambda(\vec{i}:\vec{\mathrm{X}_B})(b:B\ \vec{i}).(p_A\ (\mathrm{deindex}\ \vec{i})\ (\downarrow_A b))_\beta }\hspace{-0.1cm}

\inferrule[Case]
{ \Gamma \vdash p_A : \mathrm{P}_A \\
  \Gamma \vdash f_i : \mathrm{E}_{A_i}\ p_A \\\\
  \Gamma \vdash p_A \Uparrow_E p_B \\
  \hspace{-0.3cm} \Gamma \vdash (f_i,\  \mathrm{E}_{B_i}\ p_B) \Uparrow_{\mathrm{E}_{\mathrm{x}}} f_i' }
{ \Gamma \vdash f_i \Uparrow_E f_i' }

\end{mathpar}	
\vspace{-0.4cm}
\caption{Lifting eliminators.}
\label{fig:ind}
\end{figure}

The induction rules implement this transformation. 
\textsc{Case} lifts a case of the eliminator by first recursively lifting the motive,
then using the lifted motive to compute the type of the new case, and then using that type to
compute the body of the new case. In the example above, when lifting the inductive case,
it first recursively lifts the motive \smallmath{$p_A$} using \textsc{Motive}, which drops the index, packs and forgets the argument of type \B,
and then \smallmath{$\beta$}-reduces the result, eliminating references to \B.
This produces the new motive:
\begin{lstlisting}
$\lambda$(n:nat)(v:vector T n).p$_A$ ($\downarrow_{A}$ v)
\end{lstlisting}
which \textsc{Case} then uses to compute the type of the inductive case of the eliminator for \B:
\begin{lstlisting}
$\Pi$(t$_v$:T)(n:nat)(v:vector T n)(IH$_v$:p$_A$ ($\downarrow_{A}$ v)).p$_A$ ($\downarrow_{A}$ (consV t$_v$ (S n) v))
\end{lstlisting}
The \smallmath{$\Gamma$ $\vdash$ $(t,\ T)$ $\Uparrow_{E_{x}}$ $t'$} judgment then uses that type to compute
the lifted function body. It computes this in a similar way to \textsc{Motive}, except that there are as many
indices to drop and arguments to pack and forget as there are inductive hypotheses, and these do not occur in predictable
places, so more rules are involved. This computes the new function:
\begin{lstlisting}
$\lambda$(n:nat)(t$_v$:T)(v:vector T n)(IH$_v$:p$_A$ ($\downarrow_{A}$ v)).f$_{\mathrm{cons}}$ t$_v$ ($\downarrow_{A}$ v) IH$_v$
\end{lstlisting}

\begin{figure}
\begin{mathpar}
\mprset{flushleft}  
\small
\hfill\fbox{$\Gamma$ $\vdash$ $t$ $\Uparrow_C$ $t'$}\vspace{-0.55cm}\\

\inferrule[Normalize]
{ \\ } 
{ \Gamma \vdash \mathrm{Constr}(j,\ A)\ \vec{x} \Uparrow_C (\uparrow (\mathrm{Constr}(j,\ A)\ \vec{x}))_{\beta\delta\iota} }
\end{mathpar}
\vspace{-0.5cm}
\caption{Lifting constructors.}
\label{fig:construct}
\end{figure}

\paragraph{Lifting Constructors}
The \smallmath{$\Gamma$ $\vdash$ $t$ $\Uparrow_C$ $t'$} judgment (Figure~\ref{fig:construct}) lifts applications of constructors of \Aa\ to applications of constructors of \B. 
This judgment computes one step of the promotion, leaving the recursive lifting of the arguments to the final algorithm. Using the same types, 
in the base case:
\begin{lstlisting}
$\uparrow$ nil $\defeq$ $\exists$ O nilV
\end{lstlisting}
and in the inductive case:
\begin{lstlisting}
$\uparrow$ (cons t l) $\defeq$ $\exists$ (S ($\uparrow_{I_B}$ l)) (consV ($\uparrow_{I_B}$ l) t ($\uparrow_{B}$ l))
\end{lstlisting}
This derivation consists of only one rule: 
\textsc{Normalize}, which normalizes the promotion of the constructor.
This is guaranteed to succeed because the application of the constructor is fully \smallmath{$\eta$}-expanded.
The core algorithm later internalizes the promotion functions in the result.

\paragraph{Core Lifting Algorithm}
The core algorithm (Figure~\ref{fig:final}) builds on these intermediate judgments.
The interesting derivations for correctness are the first six:
\textsc{Lift-Elim} and \textsc{Lift-Constr} use the judgments for lifting eliminators and constructors of \Aa.
\textsc{Internalize} internalizes the explicit \lstinline{promote} functions from the lifted constructors to recursive applications of the algorithm.
\textsc{Retraction} and \textsc{Coherence} use the respective properties of the ornamental promotion isomorphism metatheoretically:
the first to drop the explicit \lstinline{forget} functions from the lifted eliminators, and the second 
to lift the \lstinline{indexer} to a projection (in the forgetful direction, \textsc{Section} replaces \textsc{Retraction}).
Finally, \textsc{Equivalence} lifts \Aa\ along the equivalence to packed \B.
The remaining derivations recurse predictably.

\begin{figure}
\begin{mathpar}
\mprset{flushleft}
\small
\hfill\fbox{$\Gamma$ $\vdash$ $t$ $\Uparrow$ $t'$}\vspace{-0.3cm}\\

%% Eliminating B_A
\inferrule[Lift-Elim]
  { \Gamma \vdash \vec{i} : \vec{\mathrm{X}_A} \\ \Gamma \vdash a : A\ \vec{i} \\\\ 
    \Gamma \vdash p_{a} \Uparrow_E p' \\  \Gamma \vdash \vec{f_{a}}\phantom{l} \Uparrow_E \vec{f}'  \\\\
    \Gamma \vdash p' \Uparrow p_b \\ \Gamma \vdash \vec{f}' \Uparrow \vec{f_b} \\ \Gamma \vdash a \Uparrow b_{\Sigma} }
  { \Gamma \vdash \mathrm{Elim}(a,\ p_{a}) \vec{f_{a}} \Uparrow \mathrm{Elim}(\pi_r\ b_{\Sigma},\ p_b) \vec{f_{b}} }

\inferrule[Lift-Constr]
{ \Gamma \vdash \vec{i} : \vec{\mathrm{X}_A} \\ \Gamma \vdash \mathrm{Constr}(j,\ A)\ \vec{t}_{a} : A\ \vec{i} \\\\
  \Gamma \vdash \mathrm{Constr}(j,\ A)\ \vec{t}_{a} \Uparrow_C t' \\\\ \Gamma \vdash t' \Uparrow t'' } % \\ \Gamma \vdash i_b \Uparrow i_b' \\\\ }
{ \Gamma \vdash \mathrm{Constr}(j,\ A)\ \vec{t}_{a} \Uparrow t'' }

\inferrule[Internalize] % can get around this by just moving constr rule inside, if desired; maybe do because it's kind of weird
{ \Gamma \vdash a \Uparrow b_{\Sigma} }
{ \Gamma \vdash\ \uparrow\ a \Uparrow b_{\Sigma} }

\inferrule[Retraction]
{ \Gamma \vdash\ b_{\Sigma} \Uparrow b_{\Sigma}' }
{ \Gamma \vdash\ \downarrow\ b_{\Sigma} \Uparrow b_{\Sigma}' }

\inferrule[Coherence]
 { \Gamma \vdash \vec{i} : \vec{\mathrm{X}_A} \\ \Gamma \vdash a : A\ \vec{i} \\ \Gamma \vdash a \Uparrow b_{\Sigma} }
 { \Gamma \vdash \pi_{I_B}\ a \phantom{,} \Uparrow (\pi_l\ b_{\Sigma})_\beta}

%% From-Type
\inferrule[Equivalence]
  { \Gamma \vdash \vec{i} : \vec{\mathrm{X}_A} }
  { \Gamma \vdash A\ \vec{i} \Uparrow \Sigma (n : (I_B\ \vec{i})_\beta) . B\ (\mathrm{index}\ n\ \vec{i}) }

\inferrule[Constr]
{ \Gamma \vdash T \Uparrow T' \\ \Gamma \vdash \vec{t} \Uparrow \vec{t'} }
{ \Gamma \vdash \mathrm{Constr}(j,\ T)\ \vec{t} \Uparrow \mathrm{Constr}(j,\ T')\ \vec{t'} }

\inferrule[Ind]
  { \Gamma \vdash T \Uparrow T' \\ \Gamma \vdash \vec{C} \Uparrow \vec{C'}  }
  { \Gamma \vdash \mathrm{Ind} (\mathit{Ty} : T) \vec{C} \Uparrow \mathrm{Ind} (\mathit{Ty} : T') \vec{C'} }

\inferrule[Elim]
  { \Gamma \vdash c \Uparrow c' \\ \Gamma \vdash Q \Uparrow Q' \\ \Gamma \vdash \vec{f} \Uparrow \vec{f'}}
  { \Gamma \vdash \mathrm{Elim}(c, Q) \vec{f} \Uparrow \mathrm{Elim}(c', Q') \vec{f'}  }

%% Application
\inferrule[App]
 { \Gamma \vdash f \Uparrow f' \\ \Gamma \vdash t \Uparrow t'}
 { \Gamma \vdash f t \Uparrow f' t' }

% Lamda
\inferrule[Lam]
  { \Gamma \vdash T \Uparrow T' \\ \Gamma,\ t : T \vdash b \Uparrow b' }
  {\Gamma \vdash \lambda (t : T).b \Uparrow \lambda (t : T').b'}

% Product
\inferrule[Prod]
  { \Gamma \vdash T \Uparrow T' \\ \Gamma,\ t : T \vdash b \Uparrow b' }
  {\Gamma \vdash \Pi (t : T).b \Uparrow \Pi (t : T').b'}
\end{mathpar}
\vspace{-0.5cm}
\caption{Core lifting algorithm.}
\label{fig:final}
\end{figure}
\fi

\subsection{Specifying a Correct Transformation}

The implementation of this transformation in \toolnamec produces a term that Coq type checks, and so does not
add to the trusted computing base.
As \toolnamec is an engineering tool, there is no need to formally prove the transformation correct, though doing so would be satisfying.
The goal of such a proof would be to show that % the transformation preserves equality up to transport along the equivalence $A \simeq B$,
%while no longer referring to the old specification.
%That is, we need that 
if $\Gamma \vdash t \Uparrow t'$,
then $t$ and $t'$ are equal up to transport, and $t'$ refers to \B in place of \Aa.
%This is the same as the correctness criterion for the program transformation from \textsc{Devoid} that this is based on,
%with the transformation generalized to handle other equivalences beyond the class that \textsc{Devoid} supports.
The key steps in this transformation that make this possible are porting terms along the configuration % corresponding
%to a particular equivalence 
(\textsc{Dep-Constr}, \textsc{Dep-Elim}, \textsc{Eta}, and \textsc{Iota}).
%The rest is straightforward.
For metatheoretical reasons, without additional axioms, a proof of this theorem in Coq can only be approximated~\cite{tabareau2017equivalences}.
It would be possible to generate per-transformation proofs of correctness, but this does not serve an engineering need.

\subsection{Limitations}

(Limitations and whether they're addressed in other tools yet.)
