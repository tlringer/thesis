\section{Related Work}

Our work builds upon prior research in proof reuse, proof automation, proof engineering, refactoring, differencing \& incremental computation,
programming by example, and program repair.

\paragraph{Proof Reuse} Our approach reimagines the problem of proof reuse in the context of proof automation.
While we focus on changes that occur over time, traditional proof reuse techniques can help
improve our approach.
Existing work in proof reuse focuses on transferring proofs between isomorphisms,
either through extending the type system~\cite{Barthe:2001:TIP:646793.704711} or through an automatic method~\cite{Magaud2002}.
This is later generalized and implemented in Isabelle~\cite{Huffman2013} and Coq~\cite{ZimmermannH15, tabareau:hal-01559073};
later methods can also handle implications. 
%Transfer tactics apply these functions but do not infer them, while our approach
%infers these functions but does not apply them.
Integrating a transfer tactic with a proof patch finding tool will create an end-to-end
tool that can both find patches and apply them automatically.

Proof reuse for extended inductive types~\cite{Boite2004} adapts proof obligations
to structural changes in inductive types. Later work~\cite{Mulhern06proofweaving} proposes a method
to generate proofs for new constructors. These approaches may be useful when extending the differencing
component to handle structural changes. Existing work in theorem reuse and proof generalization~\cite{Felty1994, pons00, Johnsen2004} abstracts existing proofs for reusability, and may be useful
for improving the abstraction component.
Our work focuses on the components critical to searching for patches; these complementary approaches
can drive improvements to the components.

\paragraph{Proof Automation} We address a missed opportunity in proof automation for ITP: searching
for patches that can fix broken proofs.
This is complementary to existing automation techniques. Nonetheless, there is a wealth
of work in proof automation that makes proofs more resilient to change. %; we discuss a sample.
Powerful tactics like \lstinline{crush}~\cite{chlipala:cpdt} can make
proofs more resilient to changes. 
Hammers like Isabelle's sledgehammer~\cite{Blanchette2013} can make proofs agnostic to some low-level changes.
Recent work~\cite{coqhammer} paves the way for a hammer in Coq.
%by translating a substantial subset of CIC into untyped
%first-order logic. 
Even the most powerful tactics cannot address all changes;
our hope is to open more possibilities for automation.

Powerful project-specific tactics~\cite{chlipala:cpdt, Chlipala2013} can help prevent low-level maintenance tasks.
Writing these tactics requires good engineering~\cite{Gonthier2011} and domain-specific knowledge,
and these tactics still sometimes break in the face of change.
%Furthermore, these tactics still sometimes break in the face of change. %, and when they do, they are difficult to debug.
A future patching tool may be able to repair tactics; the debugging process
for adapting a tactic is not too dissimilar to providing an example to a tool.

Rippling~\cite{rippling} is a technique for automating inductive proofs that uses restricted rewrite rules to
guide the inductive hypothesis toward the conclusion; this may guide improvements to the
differencing, abstraction, and specialization components.
The abstraction and factoring components address specific classes of unification problems;
recent developments to higher-order unification~\cite{Miller:2012:PHL:2331097} may help
improve these components.
Lean~\cite{selsam:lean} introduces the first congruence closure algorithm for dependent type theory that
relies only on the Uniqueness of Identity Proofs (UIP) axiom. While UIP is not fundamental to Coq,
it is frequently assumed as an axiom; when it is, it may be tractable to use a similar algorithm to improve the tool.

GALILEO~\cite{bundyreasoning} repairs faulty physics theories
in the context of a classical higher-order logic (HOL); there is preliminary work extending this
style of repair to mathematical proofs. 
Knowledge-sharing methods~\cite{tgck-cicm14} can adapt some proofs across different representations of HOL.
These complementary approaches may guide extensions to support decidable domains and classical logics.

\paragraph{Proof Engineering} Existing proof engineering work addresses brittleness
by planning for changes~\cite{proof-eng} and designing theorems and proofs that make maintenance less of an issue.
Design principles for specific domains (such as formal metatheory~\cite{Aydemir2008, Delaware2013POPL, Delaware2013ICFP})
can make verification more tractable. CertiKOS~\cite{certikos} introduces the idea of a deep specification to
ease verification of large systems.
Ornaments~\cite{Dagand17jfp, Williams:2014:OP:2633628.2633631}
separate the computational and logical components of a datatype, and may
make proofs more resilient to datatype changes.
These design principles and frameworks are complementary to our approach.
Even when programmers use informed design principles,
changes outside of the programmer's control can break proofs;
our approach addresses these changes.

There is a small body of work on change and dependency management for verification,
both to evaluate impact of potential changes and maximize reuse~\cite{873647, Autexier:2010:CMH:1986659.1986663}
and to optimize build performance~\cite{Celik:2017:IRP:3155562.3155588}.
These approaches may help isolate changes, which is necessary to identify future benchmarks, integrate
with CI systems, and fully support version updates.

\paragraph{Refactoring} Our approach is close in spirit to refactoring~\cite{Mens:2004:SSR:972215.972286}.
The Haskell refactoring tool HaRe~\cite{HaRe} automatically lifts definitions, and may be useful
for improving abstraction.
There is a growing body of work on refactoring in the context of ITP~\cite{Whiteside2011, Bourke_DKK_12}.
The IDE CoqPIE~\cite{Roe2016} and the verification platform Why3~\cite{Bobot2013} can
both adapt Coq proofs to simple syntactic changes.
It may be possible to use our lemma factoring component to improve proof refactoring tools.
Proof refactoring tools are semantics-preserving; unlike these tools,
our approach handles semantic changes.

\paragraph{Differencing \& Incremental Computation} Existing work in differencing and incremental computation may help 
improve our semantic differencing component.
Type-directed diffing~\cite{Miraldo:2017:TDS:3122975.3122976}
finds differences in algebraic data types.
Semantics-based change impact analysis~\cite{Autexier:2010:SCI:1860559.1860580} models semantic differences
between documents.
Differential assertion checking~\cite{differential-assertion-checking-2} analyzes different
versions of a program for relative correctness with respect to a specification.
Incremental $\lambda$-calculus~\cite{Cai:2014:TCH:2594291.2594304} introduces a general model for program changes.
All of these may be useful for improving semantic differencing.

\paragraph{Programming by Example} Our approach generalizes an example that the programmer provides.
This is similar to programming by example, a subfield of 
program synthesis~\cite{DBLP:journals/ftpl/GulwaniPS17}. 
This field addresses different challenges in different logics,
but may drive solutions to similar problems in a dependently typed language.

\paragraph{Program Repair} Adapting proofs to changes is essentially program repair
for dependently typed languages. 
Program repair tools for 
languages with non-dependent type 
systems~\cite{Pei:2014:APR:2731750.2731779, Long:2016:APG:2837614.2837617, Le:2017:SSS:3106237.3106309, Mechtaev:2016:ASM:2884781.2884807, Monperrus2015} 
may have applications in the context of a dependently typed language.
Similarly, our work may have applications within program repair in these languages:
Future applications of our approach may repurpose it to repair programs for functional languages.



