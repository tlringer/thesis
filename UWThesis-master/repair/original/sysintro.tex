\section{Generating Reusable Patches}
\label{sec:sys}

We identify five components that are key to finding reusable patches.
We implement these components in a prototype Coq plugin, which we call
\sysnamelong (Proof Updater Mechanically Passing Knowledge Into New Proofs, Assisting The Coq Hacker), or \sysname.\footnote{\url{http://github.com/uwplse/PUMPKIN-PATCH/tree/cpp18}}

We focus the \sysname prototype on the proof assistant Coq~\cite{coq}.
In Coq, each theorem is a type, and a proof of that theorem is a term that inhabits that type. 
Rather than write proof terms directly, users write proof scripts in a high-level tactic language;
Coq then uses these scripts to guide search for a proof term.
%We discuss supporting other proof assistants in Section~\ref{sec:future}.

The \sysname repository contains 
a detailed user guide.
To use \sysname, the programmer modifies a single proof script to provide an \textit{example} of how to adapt a proof to a
change. \sysname generalizes the example adaptation into a \textit{reusable patch}: a function
that can be used to fix other broken proofs, which \sysname defines as a Coq term.
We focus on the problem of \textit{finding} these patches;
we discuss how to extend \sysname to automatically \textit{apply} patches it finds in Section~\ref{sec:future}.

We motivate the core components (Section~\ref{sec:coreintro}) and describe how they
compose to find patches (Section~\ref{sec:composeintro}). We find patches for 
real code in Section~\ref{sec:case}, and we describe our implementation
in Section~\ref{sec:core}.

\begin{figure*}[ht]
\begin{minipage}{0.50\textwidth}
\lstset{language=coq, aboveskip=0pt, belowskip=0pt}
\lstinputlisting[firstline=1, lastline=1]{repair/introex.tex}
\lstinputlisting[backgroundcolor=\color{orange!35},firstline=2,lastline=2]{repair/introex.tex}
\lstinputlisting[firstline=3, lastline=8]{repair/introex.tex}
\lstinputlisting[firstline=1, lastline=4]{repair/introexterm.tex}
\lstinputlisting[backgroundcolor=\color{orange!35},firstline=5,lastline=5]{repair/introexterm.tex}
\lstinputlisting[firstline=6, lastline=9]{repair/introexterm.tex}
\end{minipage}
\hfill
\begin{minipage}{0.48\textwidth}
\lstset{language=coq, aboveskip=0pt, belowskip=0pt}
\lstinputlisting[firstline=9, lastline=9]{repair/introex.tex}
\lstinputlisting[backgroundcolor=\color{orange!35},firstline=10,lastline=10]{repair/introex.tex}
\lstinputlisting[firstline=11, lastline=16]{repair/introex.tex}
\lstinputlisting[firstline=11, lastline=14]{repair/introexterm.tex}
\lstinputlisting[backgroundcolor=\color{orange!35},firstline=15,lastline=15]{repair/introexterm.tex}
\lstinputlisting[firstline=16, lastline=19]{repair/introexterm.tex}
\end{minipage}
\vspace{-.4cm}
\setlength{\belowcaptionskip}{-4pt}
\caption{Two proofs with different conclusions (top) and the
corresponding proof terms (bottom) with relevant type information. We highlight the change in theorem conclusion and
the difference in terms that corresponds to a patch.}
\label{fig:example}
\end{figure*}

\subsection{Motivating the Core}
\label{sec:coreintro}

We identify and isolate
five core components critical to finding reusable patches:

\begin{enumerate}
\item \textit{Semantic differencing} between terms
\item \textit{Patch specialization} to arguments
\item \textit{Patch abstraction} of arguments or functions
\item \textit{Patch inversion} to reverse a patch
\item \textit{Lemma factoring} to break a term into parts
\end{enumerate}

The semantic differencing component finds the difference between two terms, which produces a \textit{candidate} for
a reusable patch. %However, this candidate may not be the final patch the tool is looking for.
 %: It may be too general or too specific,
%for example. 
The other components modify the candidate to try to produce a patch.
To motivate this workflow, consider using \sysname to search the proofs in
Figure~\ref{fig:example} for a patch between conclusions. We invoke the plugin using \lstinline{old} and \lstinline{new}
as the example change:

\begin{lstlisting}[language=ml4]
    Patch Proof old new as patch.
\end{lstlisting}

\sysname first determines the type that a patch from \lstinline{new} to \lstinline{old} should have.
To determine this, it semantically \textit{diffs} the types and finds this goal type (line 2):

\begin{lstlisting}[language=coq]
    (@\ltacforall@) n m p, n <= m -> m <= p -> (@\diff{n <= p}@) -> (@\diff{n <= p + 1}@)
\end{lstlisting}

It then breaks each inductive proof into cases and determines an intermediate goal type for the candidate.
%recursively searches each case for a candidate.
In the base case, for example, it \textit{diffs} the types and determines that a candidate
between the base cases of \lstinline{new} and \lstinline{old} should have this type (lines 11 and 12):

\begin{lstlisting}
    (@\diff{(fun p0 => n <= p0)}@) m -> (@\diff{(fun p0 => n <= p0 + 1)}@) m
\end{lstlisting}

It then \textit{diffs} the terms (line 13) for such a candidate:

\begin{lstlisting}
    fun n m p H0 H1 =>(@\vspace{-0.04cm}@)
      (fun (H : n <= m) => (@\diff{le\_plus\_trans n m 1}@) H)(@\vspace{-0.04cm}@)
    : (@\ltacforall@) n m p, n <= m -> m <= p -> (@\diff{n <= m}@) -> (@\diff{n <= m + 1}@)
\end{lstlisting}

This candidate is close, but it is not yet a patch. This candidate
maps base case to base case (it is applied to \lstinline{m}); the patch should map conclusion to conclusion (it should
be applied to \lstinline{p}).
\sysname \textit{abstracts} this candidate by \lstinline{m} (line 11), which lifts it out of the base case:

\begin{lstlisting}
    fun (@\diff{n0}@) n m p H0 H1 =>(@\vspace{-0.04cm}@)
      (fun (H : n <= (@\diff{n0}@)) => le_plus_trans n (@\diff{n0}@) 1 H)(@\vspace{-0.04cm}@)
    : (@\ltacforall@) (@\diff{n0}@) n m p, n <= m -> m <= p -> n <= (@\diff{n0}@) -> n <= (@\diff{n0}@) + 1
\end{lstlisting}

%This produces a candidate that is closer: It is not yet a patch, but applying it to \lstinline{p} (line 16),
%the argument to the conclusion of \lstinline{le_ind}, will produce a patch.
\sysname then \textit{specializes} this candidate to \lstinline{p} (line 16), the argument
to the conclusion of \lstinline{le_ind}. This produces a patch:

\begin{lstlisting}
    patch n m p H0 H1 := (@\vspace{-0.04cm}@)
      (fun (H : n <= (@\diff{p}@)) => le_plus_trans n (@\diff{p}@) 1 H)(@\vspace{-0.04cm}@)
    : (@\ltacforall@) n m p, n <= m -> m <= p -> n <= (@\diff{p}@) -> n <= (@\diff{p}@) + 1
\end{lstlisting}

The user can then use \lstinline{patch} to fix other broken proofs.
For example, given a proof that applies \lstinline{old}, the user can use \lstinline{patch} to prove the same conclusion
by applying \lstinline{new}:

\begin{lstlisting}[language=coq]
    (@\succeeds{apply old.}@)(@\vspace{-0.1cm}@)
    (@\diff{apply patch.}@) (@\succeeds{apply new.}@)
\end{lstlisting}

This simple example uses only three components. The other components help turn candidates
into patches in similar ways. We discuss all five components in more detail in the
rest of this section.

\paragraph{Semantic Differencing} The tool should be able to identify the semantic difference between terms.
The semantic difference is the difference between two terms that corresponds to
the difference between their types. Consider the base case terms in Figure~\ref{fig:example} (line 13):

\begin{lstlisting}[language=coq]
    (@\diff{le\_plus\_trans n m 1}@) H : (@\diff{n <= m + 1}@)(@\vspace{-0.1cm}@)
    (@\phantom{le\_plus\_trans n m 1}@) H : (@\diff{n <= m}@)
\end{lstlisting}

The semantic differencing component first identifies the difference in their types, or the \textit{goal type}:

\begin{lstlisting}
    (@\diff{n <= m}@) -> (@\diff{n <= m + 1}@)
\end{lstlisting}

It then finds a difference in terms that has that type:

\begin{lstlisting}[language=coq]
    fun (H : n <= m) => (@\diff{le\_plus\_trans n m 1}@) H
\end{lstlisting}

%This is the core of any patching procedure: The terms that it finds are candidates for reusable patches.
%The other components modify these candidates to find true patches.
 % (for example, in this case, the abstraction component  may generalize this function to work with nu

% TODO this erases type differencing stuff for fixpoints, but maybe that's OK
% Also, doesn't talk about zooming into functions and so on
This is the \emph{candidate} for a reusable patch that the other components modify to find a patch.

Differencing operates over terms and types. Differencing tactics
is insufficient, since tactics and hints may mask patches
(line 5).\footnote{Since this is a simple example, replaying an existing tactic happens to work. There
are additional examples in the repository (\lstinline{Cex.v}).}
Furthermore, differencing is aware of the semantics of terms and types.
Simply exploring the syntactic difference %between terms 
makes it hard to identify
which changes are meaningful.
For example, in the inductive case (line 14), the inductive hypothesis
changes:

\begin{lstlisting}[language=coq]
    ... (IHle : (@\diff{n <= m0 + 1}@)) ...(@\vspace{-0.08cm}@)
    ... (IHle : (@\diff{n <= m0}@)) ...
\end{lstlisting}

However, the type of \lstinline{IHle} changes for \emph{any} two inductive proofs over \lstinline{le}
with different conclusions. A syntactic differencing component 
may identify this change as a candidate.
Our semantic differencing component knows that it can ignore this change.

\paragraph{Patch Specialization} The tool should be able to specialize a patch candidate to specific arguments as determined by
the differences in terms. To find a patch for Figure~\ref{fig:example}, for example, \sysname
must specialize the patch candidate to \lstinline{p} to produce the final patch.

%Consider searching two proofs of theorems with hypotheses for a patch between conclusions:

%\begin{lstlisting}[language=coq]
 %   (@\ltacforall@) n m p, n <= m -> m <= p -> n <= p + 1
  %  (@\ltacforall@) n m p, n <= m -> m <= p -> n <= p
%\end{lstlisting}

%Semantic differencing determines the goal type:

%\begin{lstlisting}[language=coq]
 %   (@\ltacforall@) n m p, n <= m -> m <= p -> n <= p -> n <= p + 1
%\end{lstlisting}

%A patch of this type may not show up explicitly in the difference of terms.
%The tool may instead find an abstract patch candidate:

%\begin{lstlisting}[language=coq]
 %   fun n m p H0 H1 =>
  %    (fun (@\diff{n0}@) (H : n <= (@\diff{n0}@)) => le_plus_trans n (@\diff{n0}@) 1 H)
   % : (@\ltacforall@) n m p (@\diff{n0}@),
   %   n <= m -> m <= p -> n <= (@\diff{n0}@) -> n <= (@\diff{n0}@) + 1
%\end{lstlisting}

%This does not quite have the correct type, but it \emph{does} have the correct type
%if we specialize it to \lstinline{p}:

%\begin{lstlisting}[language=coq]
 %   fun n m p H0 H1 =>
  %    (fun (H : n <= (@\diff{p}@)) => le_plus_trans n (@\diff{p}@) 1 H)
   % : (@\ltacforall@) n m p,
    %  n <= m -> m <= p -> n <= (@\diff{p}@) -> n <= (@\diff{p}@) + 1
%\end{lstlisting}

\paragraph{Patch Abstraction} A tool should be able to abstract patch candidates of this form by the common argument:

\begin{lstlisting}[language=coq]
    candidate : P' t -> P t(@\vspace{-0.08cm}@)
    candidate_abs : (@\ltacforall@) (@\diff{t0}@), P' (@\diff{t0}@) -> P (@\diff{t0}@)
\end{lstlisting}
and it should be able to abstract patch candidates of this form by the common function:

\begin{lstlisting}[language=coq]
    candidate : P t' -> P t(@\vspace{-0.08cm}@)
    candidate_abs : (@\ltacforall@) (@\diff{P0}@), (@\diff{P0}@) t' -> (@\diff{P0}@) t
\end{lstlisting} 

This is necessary because the tool may find candidates in an applied form.
For example, when searching for a patch between the proofs in Figure~\ref{fig:example},
\sysname finds a candidate in the difference of base cases. To produce a patch, 
\sysname must abstract the candidate by the argument \lstinline{m}.
Abstracting candidates is not always possible; abstraction will necessarily be a collection of heuristics.

\paragraph{Patch Inversion} The tool should be able to invert a patch candidate.
This is necessary to search for isomorphisms. %in both directions (that is, to find isomorphisms) between propositionally equal types.
It is also necessary to search for implications between propositionally
equal types, since candidates may appear in the wrong direction.
For example, consider two list lemmas (we write \lstinline{length} as \lstinline{len}):

\begin{lstlisting}[language=coq]
    old : (@\ltacforall@) l' l, len (l' ++ l) = len l' + len l(@\vspace{-0.08cm}@)
    new : (@\ltacforall@) l' l, len (l' ++ l) = len l' + len (rev l)
\end{lstlisting} 

If \sysname searches the difference in proofs of these lemmas for a patch from the 
conclusion of \lstinline{new} to the conclusion of \lstinline{old},
it may find a candidate \emph{backwards}: %, as a patch from \lstinline{old} to \lstinline{new}:

\begin{lstlisting}
    candidate l' l (H : (@\diff{old}@) l' l) :=(@\vspace{-0.04cm}@)
      eq_ind_r ... (@\diff{(rev\_length l)}@)(@\vspace{-0.04cm}@)
   : (@\ltacforall@) l' l, (@\diff{old}@) l' l -> (@\diff{new}@) l' l
\end{lstlisting}

The component can invert this to get the patch: %from \lstinline{new} to \lstinline{old}:

\begin{lstlisting}
    patch l' l (H : (@\diff{new}@) l' l) :=(@\vspace{-0.04cm}@)
      eq_ind_r ... (@\diff{(eq\_sym (rev\_length l))}@)(@\vspace{-0.04cm}@)
   : (@\ltacforall@) l' l, (@\diff{new}@) l' l -> (@\diff{old}@) l' l
\end{lstlisting}

We can then use this patch to port proofs.
For example, if we add this patch to a hint database~\cite{hints},
we can port this proof:

\begin{lstlisting}
    Theorem app_rev_len : (@\ltacforall@) l l',(@\vspace{-0.04cm}@)
      len (rev (l' ++ l)) = len (rev l) + len (rev l').(@\vspace{-0.04cm}@)
    Proof.(@\vspace{-0.04cm}@)
      intros. rewrite rev_app_distr. (@\succeeds{apply old.}@)
\end{lstlisting}
to this proof:

\begin{lstlisting}
      intros. rewrite rev_app_distr. (@\succeeds{apply new.}@)
\end{lstlisting}

Rewrites like \lstinline{candidate} are \textit{invertible}:
We can invert any rewrite in one direction by rewriting in the opposite direction.
In contrast, it is not possible to invert the patch \sysname
found for Figure~\ref{fig:example}.
Inversion will necessarily sometimes fail, since not all terms are invertible.
%Not all terms are invertible. It is not possible, for example, to invert the patch 
%Inversion will necessarily sometimes fail.

\paragraph{Lemma Factoring} The tool should be able to factor a term into a sequence of lemmas.
This can help break other problems, like abstraction, into smaller subproblems.
It is also necessary to invert certain terms.
Consider inverting an arbitrary sequence of two rewrites:

\begin{lstlisting}
    t := (@\diff{eq\_ind\_r G ...}@) ((@\diff{eq\_ind\_r F ...}@))
\end{lstlisting}

We can view \lstinline{t} as a term that composes two functions:

\begin{lstlisting}
    g := (@\diff{eq\_ind\_r G ...}@)(@\vspace{-0.04cm}@)
    f := (@\diff{eq\_ind\_r F ...}@)(@\vspace{-0.04cm}@)
    t := g $\circ$ f
\end{lstlisting}

The inverse of \lstinline{t} is the following:

\begin{lstlisting}
    t$\inv$ := f$\inv$ $\circ$ g$\inv$
\end{lstlisting}

To invert \lstinline{t}, \sysname identifies the factors \lstinline{[f; g]}, 
inverts each factor to \lstinline{[f}$\inv$\lstinline{; g}$\inv$\lstinline{]}, 
then folds and applies the inverse factors in the opposite direction.

\subsection{Composing Components}
\label{sec:composeintro}

% Note about tactics?

The components come together to form a proof patch finding procedure:

\begin{algorithm}
\begin{algorithmic}[1]
\renewcommand{\thealgorithm}{}
\footnotesize
\caption{\footnotesize{find\_patch(term, term', direction)}}
    \STATE \diff{\textit{diff} types of term and term' for goals}
    \STATE \diff{\textit{diff} term and term' for candidates}
    \IF{there are candidates}
      \STATE \diff{\textit{factor}, \textit{abstract}, \textit{specialize}, and/or \textit{invert} candidates}
      \IFRETURN{there are patches}{patches}
    \ENDIF
    \IF{direction is forwards}
      \STATE find\_patch(term', term, backwards) for candidates
      \STATE \textit{factor} and \textit{invert} candidates
      \IFRETURN{there are patches}{patches}
    \ENDIF
    \RETURN failure
\end{algorithmic}
\label{alg:patching}	
\end{algorithm}

\begin{figure*}[ht]
\begin{minipage}{0.48\textwidth}
\lstset{language=coq, aboveskip=0pt,belowskip=0pt}
\lstinputlisting[firstline=1, lastline=1]{repair/compcert.tex}
\lstinputlisting[backgroundcolor=\color{orange!35},firstline=2,lastline=2]{repair/compcert.tex}
\end{minipage}
\hfill
\begin{minipage}{0.48\textwidth}
\lstset{language=coq, aboveskip=0pt,belowskip=0pt}
\lstinputlisting[firstline=4, lastline=4]{repair/compcert.tex}
\lstinputlisting[backgroundcolor=\color{orange!35},firstline=5,lastline=5]{repair/compcert.tex}
\end{minipage}
\vspace{-.3cm}
%\setlength{\belowcaptionskip}{-4pt}
\caption[Caption for LOF]{Old (left) and new (right) definitions of \lstinline{int} in CompCert.}
\label{fig:int}
\end{figure*}

\sysname infers a \textit{configuration} from the example change.
This configuration customizes the highlighted lines for an entire class of changes:
It determines what to diff on lines 1 and 2,
and how to use the components on line 4.

For example, to find a patch for Figure~\ref{fig:example},
\sysname used the configuration for changes in conclusions of two proofs
that induct over the same hypothesis. Given two such
proofs:

%We describe one
%such procedure in this section; we descibe three more
%in Section~\ref{sec:case}.
\begin{lstlisting}[language=coq]
    (@\ltacforall@) x, H x -> (@\diff{P}@) x(@\vspace{-0.1cm}@)
    (@\ltacforall@) x, H x -> (@\diff{P'}@) x
\end{lstlisting}
\sysname searches for a patch with this type:

\begin{lstlisting}[language=coq]
    (@\ltacforall@) x, H x -> (@\diff{P'}@) x -> (@\diff{P}@) x
\end{lstlisting}
using this configuration:

\begin{algorithm}
\footnotesize
\begin{algorithmic}[1]
%  Is it really necessary to say that an algorithm might do initialization?
%  \STATE Build trees for search
  %\STATE build trees for proofs
  %\REPEAT
    \STATE \textit{diff} conclusion types for goals
    \STATE \textit{diff} conclusion terms for candidates
    \STATE \textbf{if} there are candidates \textbf{then}
    \STATE \hspace*{1em} \textit{abstract} and then \textit{specialize} candidates
\end{algorithmic}
\end{algorithm}

Section~\ref{sec:case} describes real-world examples that demonstrate more configurations.

%Scenarios from real-world examples in Section~\ref{sec:case} demonstrate more configurations.
%We discuss more configurations in Section~\ref{sec:case},
%and we describe our implementation in Section~\ref{sec:core}.

%We show three more procedures
%in Section~\ref{sec:case}.

%The type of this term takes the conclusion of \lstinline{new} to the conclusion of \lstinline{old} (line 5):

%\begin{lstlisting}
%    n <= p -> n <= p + 1
%\end{lstlisting}

%TODO If both passes fail, the algorithm defaults to returning a function that always produces the old term.



