\section{Motivating Example: Porting a Library}
\label{sec:example}

\toolnameb is a plugin for Coq 8.8; it can be found in the repository linked to as \textbf{Supplement Material} under the abstract of this paper.
To see how it works, consider an example using the types from Figure~\ref{fig:listvect},
the code for which is in \href{http://github.com/uwplse/ornamental-search/blob/itp+equiv/plugin/coq/examples/Example.v}{\lstinline{Example.v}}.
In this example, we lift two list zip functions and a proof of a theorem relating them from the Haskell CoreSpec library~\cite{hstocoqv}:
\begin{lstlisting}
zip {T$_1$ T$_2$}: list T$_1$ $\rightarrow$ list T$_2$ $\rightarrow$ list (T$_1$ * T$_2$).
zip_with {T$_1$ T$_2$ T$_3$} (f : T$_1$ $\rightarrow$ T$_2$ $\rightarrow$ T$_3$): list T$_1$ $\rightarrow$ list T$_2$ $\rightarrow$ list T$_3$.
zip_with_is_zip {T$_1$ T$_2$}: $\forall$(l$_1$:list T$_1$)(l$_2$:list T$_2$), zip_with pair l$_1$ l$_2$ = zip l$_1$ l$_2$.
\end{lstlisting}
\toolnameb runs a preprocessing step before lifting, which we describe in Section~\ref{sec:impl}; we assume this step has already run.
We use the \codeauto{cyan} background color to denote tool-produced terms and the names that refer to them.
We run \toolnameb to lift functions and proofs from lists to vectors, but it can also lift in the opposite direction.

\subparagraph*{Step 1: Search.}
We first use \toolnameb's \lstinline{Find ornament} command to search for the relation between lists and vectors:
\begin{lstlisting}
Find ornament list vector.
\end{lstlisting}
This produces functions which together form an equivalence (denoted \smallmath{$\simeq$}):
\begin{lstlisting}
list T $\simeq$ $\Sigma$ (n : nat).vector T n
\end{lstlisting}

\subparagraph*{Step 2: Lift.}
We then lift our functions and proofs
along that equivalence using \toolnameb's \lstinline{Lift} command.
For example, to lift \lstinline{zip}, we run the command:
\begin{lstlisting}
Lift list vector in zip as (@\codeauto{zipV\_p}@).
\end{lstlisting}
This produces a function with this type:
\begin{lstlisting}
(@\codeauto{zipV\_p}@) {T$_1$ T$_2$} : $\Sigma$ n.vector T$_1$ n $\rightarrow$ $\Sigma$ n.vector T$_2$ n $\rightarrow$ $\Sigma$ n.vector (T$_1$ * T$_2$) n.
\end{lstlisting}
that behaves like \lstinline{zip}, but whose body no longer refers to lists.
We lift our proof similarly:
\begin{lstlisting}
Lift list vector in zip_with_is_zip as (@\codeauto{zip\_with\_is\_zipV\_p}@).
\end{lstlisting}
This produces a proof of the analogous result (denoting projections by \smallmath{$\pi_{l}$} and \smallmath{$\pi_{r}$}):
\begin{lstlisting}
(@\codeauto{zip\_with\_is\_zipV\_p}@) {T$_1$ T$_2$} : $\forall$ (v$_1$ : $\Sigma$ n.vector T$_1$ n) (v$_2$ : $\Sigma$ n.vector T$_2$ n),
  (@\codeauto{zip\_withV\_p}@) pair ($\exists$ ($\pi_l$ v$_1$) ($\pi_r$ v$_1$)) ($\exists$ ($\pi_l$ v$_2$) ($\pi_r$ v$_2$)) =
  (@\codeauto{zipV\_p} ($\exists$ ($\pi_l$ v$_1$) ($\pi_r$ v$_1$)) ($\exists$ ($\pi_l$ v$_2$) ($\pi_r$ v$_2$)).
\end{lstlisting}
that no longer refers to lists, \lstinline{zip}, or \lstinline{zip_with} in any way.

\subparagraph*{Step 3: Unpack.}
The lifted terms operate over vectors whose lengths are \textit{packed} inside of a sigma type.
While this lets \lstinline{Lift} provide strong theoretical guarantees, it can make it difficult to interface with the lifted code.
We can recover \textit{unpacked} terms using \toolnameb's \lstinline{Unpack} command.
For example, to unpack \codeauto{\lstinline{zipV_p}}, we run the command:
\begin{lstlisting}
Unpack $\codeauto{zipV\_p}$ as $\codeauto{zipV}$.
\end{lstlisting}
This produces functions and proofs that operate directly over vectors, like \codeauto{\lstinline{zipV}}:
\begin{lstlisting}
(@\codeauto{zipV}@) {T$_1$ T$_2$} {n$_1$} (v$_1$ : vector T$_1$ n$_1$) {n$_2$} (v$_2$ : vector T$_2$ n$_2$) : 
  vector (T$_1$ * T$_2$) ($\pi_l$ ((@\codeauto{zipV\_p}@) ($\exists$ n$_1$ v$_1$) ($\exists$ n$_2$ v$_2$))).
\end{lstlisting}
and \codeauto{\lstinline{zip\_with\_is\_zipV}}:
\begin{lstlisting}
(@\codeauto{zip\_with\_is\_zipV}@) : $\forall$ {T$_1$ T$_2$} {n$_1$} (v$_1$ : vector T$_1$ n$_1$) {n$_2$} (v$_2$ : vector T$_2$ n$_2$),
  eq_dep _ _ _ ((@\codeauto{zip\_withV}@) pair v$_1$ v$_2$) _ ((@\codeauto{zipV}@) v$_1$ v$_2$).
\end{lstlisting}

\subparagraph*{Step 4: Interface.}
For any two inputs of the same length, \codeauto{\lstinline{zipV}} and \codeauto{\lstinline{zipV_with}} contain proofs
that the output has the same length as the inputs.
However, the types obscure this information.
\href{http://github.com/uwplse/ornamental-search/blob/itp+equiv/plugin/coq/examples/Example.v}{\lstinline{Example.v}} explains how to recover more user-friendly types, like that of \lstinline{zipV_uf}:
\begin{lstlisting}
zipV_uf {T$_1$ T$_2$} {n} : vector T$_1$ n $\rightarrow$ vector T$_2$ n $\rightarrow$ vector (T$_1$ * T$_2$) n.
\end{lstlisting}
and that of \lstinline{zip_withV_uf}:
\begin{lstlisting}
zip_withV_uf {T$_1$ T$_2$ T$_3$} (f : T$_1$ $\rightarrow$ T$_2$ $\rightarrow$ T$_3$) {n} : 
  vector T$_1$ n $\rightarrow$ vector T$_2$ n $\rightarrow$ vector T$_3$ n.
\end{lstlisting}
which both restrict input lengths.
We can then use our lifted functions and proofs in client code.
For example, we can write a different version of Coq's
\lstinline{BVand} function for bitvectors:
\begin{lstlisting}
BVand {n} (v$_1$ : vector bool n) (v$_2$ : vector bool n) : vector bool n := 
  zip_withV_uf andb v$_1$ v$_2$.
\end{lstlisting}
By working over lists, we are able to reason about only the interesting pieces, thinking about indices only
when relevant; in contrast, when writing proofs over vectors, even simple theorems
can generate tricky proof obligations.
With \toolnameb, the programmer can use the lifted functions and proofs
to interface with code that uses vectors, then switch back to lists when vectors are unmanageable. 
In essence, ornaments form the glue between these types.


