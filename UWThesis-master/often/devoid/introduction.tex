\section{Introduction}

Indexed inductive types make it possible to internalize data into the type level, eliminating the need for certain functions and proofs.
Consider, for example, a theorem from the Coq standard library~\cite{stdlib}
which states that mapping a function over lists preserves length:
\begin{lstlisting}
map_length T$_1$ T$_2$ (f : T$_1$ $\rightarrow$ T$_2$) : $\forall$ (l : list T), length (List.map f l) = length l.
\end{lstlisting}
One way to eliminate the need for this theorem is to internalize the length of a list into its type,
creating a dependently typed vector (Figure~\ref{fig:listvect}).
The map function for vectors in Coq's standard library, for example, carries a proof that it preserves length:
\begin{lstlisting}
Vector.map {T$_1$} {T$_2$} (f : T$_1$ $\rightarrow$ T$_2$) : $\forall$ (n : nat) (v : vector T$_1$ n), vector T$_2$ n.
\end{lstlisting}
so that a theorem like \lstinline{map_length} is no longer necessary.

Unfortunately, for all of the benefits they bring, indexed inductive types are notoriously difficult to use.
Dependently typed vectors, for example, impose proof obligations about their lengths on the user;
these can quickly spiral out of control.
In recent coq-club threads asking for advice on how to use dependently typed vectors, experts called them
``not suitable for extended use''~\cite{emiliodom} and noted that ``almost no one should be using [them] for anything''~\cite{adamchlipala}.

\begin{figure}
\begin{minipage}{0.35\textwidth}
   \lstinputlisting[firstline=1, lastline=4]{often/devoid/listvect.tex}
\end{minipage}
\hfill
\begin{minipage}{0.65\textwidth}
   \lstinputlisting[firstline=6, lastline=9]{often/devoid/listvect.tex}
\end{minipage}
\vspace{-0.3cm}
\caption{A \lstinline{vector} (right) is a \lstinline{list} (left) indexed by its length (highlighted in \codediff{orange}).}
\label{fig:listvect}
\end{figure}

We show how \textit{proof reuse}---reusing existing proofs to derive new proofs---can tackle many of the challenges posed
by indexed inductive types, allowing the user to move between unindexed and indexed versions of a type 
(for example, lists and vectors)
and reap the benefits of indexed types without many of the costs.
We focus in particular on the benefits of this approach in deriving functions and proofs for
fully-determined indexed types, when the index
is a fold over the unindexed version (such as the length of a list).
In our approach, the user writes functions and proofs over the unindexed version,
and a tool then automatically \textit{lifts} those functions and proofs to the indexed version.
The user can then switch back to working with the unindexed version by running the tool in the opposite direction.
In that way, the user can use lists when lists are convenient, and vectors when vectors are convenient.

Our approach uses \textit{ornaments}~\cite{mcbride}, which express relations between types that preserve
inductive structure, and which enable lifting of functions and proofs along those relations. 
Recent work introduced ornaments to a subset of ML and was heavily focused on automatically
lifting functions~\cite{Williams2017};
until now, such an approach was not available in a dependently typed language.
Existing implementations of ornaments in dependently typed languages work only in embedded languages,
and have little to no automation~\cite{ko2016programming, mcbride, dagand2014transporting}.

Our main contribution is a Coq plugin for automatic function and proof reuse
using ornaments. 
Our plugin \toolnameb (Dependent Equivalences Via Ornamenting Inductive Definitions) 
works directly on Coq code, rather than on an embedded language.
\toolnameb automates lifting functions and proofs along
\textit{algebraic ornaments}~\cite{mcbride}, a particular class of ornaments that represent fully-determined indexed types
like lists and vectors.
\toolnameb implements an algorithm to search for ornaments between these types---to the best of our knowledge, the first
search algorithm for ornaments---and an algorithm to lift functions and proofs along the ornaments it discovers.

We motivate (Section~\ref{sec:example}), specify (Section~\ref{sec:spec}), and formalize (Section~\ref{sec:alg}) 
the search and lifting algorithms that \toolnameb implements (Section~\ref{sec:impl}).
A comparison to a more general proof reuse approach (Section~\ref{sec:case}) demonstrates the benefits of using ornaments: \toolnameb
imposes less of a proof burden on the user, and produces smaller terms and faster functions.

