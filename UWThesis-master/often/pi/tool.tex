\section{Implementation}
\label{sec:impl}

The transformation and mini decompiler abstract many of the challenges
of building a tool for proof engineers. % many details needed to build a tool that reaches proof engineers.
%and the mini decompiler abstracts a lot of the details that make Ltac so useful to proof engineers---and so painful to 
%reason about automatically.
This section describes how we solved some of these challenges.
%for both the transformation (Section~\ref{sec:implementation}) and the decompiler (Section~\ref{sec:second}).
%Section~\ref{sec:discussion} describes some remaining challenges and our plans to address them. % in the future.

\subsection{Implementing the Transformation}
\label{sec:implementation}

\mysubsubsec{Termination}
When a subterm unifies with a configuration term, this suggests that \toolnamec \textit{can}
transform the subterm, but it does not necessarily mean that it \textit{should}.
In some cases, doing so would result in nontermination.
For example, if \B is a refinement of \Aa, then we can always run \textsc{Equivalence}
over and over again, forever.
%\textsc{Devoid} ruled out this case by simply prohibiting the case where \B refers to \A, but we found it sometimes
%useful to support this case.
We thus include some simple termination checks in our code~\href{https://github.com/uwplse/pumpkin-pi/blob/silent/plugin/src/automation/lift/liftrules.ml}{\circled{12}}. % liftrules.ml

\mysubsubsec{Intent}
Even when termination is guaranteed, whether to transform a subterm depends on intent.
That is, \toolnamec automates the case of porting \textit{every} \Aa to \B,
but proof engineers sometimes wish to port only \textit{some} $A$s to $B$s.
\toolnamec has some support for this using an interactive workflow~\href{https://github.com/uwplse/pumpkin-pi/blob/silent/plugin/coq/minimal_records.v}{\circled{13}},
with plans for automatic support in the future. % minimal_records.v, but show this
%We helped the proof engineer do this by interacting with \toolnamec using a particular workflow.

\mysubsubsec{From CIC$_{\omega}$ to Coq}
The implementation~\href{https://github.com/uwplse/pumpkin-pi/blob/silent/plugin/src/automation/lift/lift.ml}{\circled{4}} % lift.ml
of the transformation handles language differences to scale from CIC$_{\omega}$ to Coq.
We use the existing \lstinline{Preprocess}~\cite{Ringer2019} command to turn pattern matching and fixpoints into 
eliminators.
We handle refolding of constants in constructors using \lstinline{DepConstr}.

\mysubsubsec{Reaching Real Proof Engineers}
Many of our design decisions in implementing \toolnamec were informed by our partnership with
an industrial proof engineer (Section~\ref{sec:search}).
For example, the proof engineer rarely had the patience to wait more than ten seconds
for \toolnamec to port a term,
so we implemented optional aggressive caching, even caching intermediate subterms that
we encounter in the course of running the transformation~\href{https://github.com/uwplse/pumpkin-pi/blob/silent/plugin/src/cache/caching.ml}{\circled{14}}. % TODO caching.ml
We also added a cache to tell \toolnamec not to $\delta$-reduce certain terms~\href{https://github.com/uwplse/pumpkin-pi/blob/silent/plugin/src/cache/caching.ml}{\circled{14}}.

 % caching.ml
%These caches are implemented in \href{https://github.com/uwplse/pumpkin-pi/blob/master/plugin/src/cache/caching.ml}{caching.ml}.
%or recurse into certain modules.
% set certain terms or modules as opaque to \toolnamec, to prevent unnecessary $\delta$-reduction.

The experiences of proof engineers also inspired new features.
For example, we implemented a search procedure to generate custom eliminators %(\href{https://github.com/uwplse/pumpkin-pi/blob/master/plugin/src/automation/search/smartelim.ml}{smartelim.ml})
to help reason about types like $\Sigma$\lstinline{(l : list T).length l = n}
by reasoning separately about the projections~\href{https://github.com/uwplse/pumpkin-pi/blob/silent/plugin/src/automation/search/smartelim.ml}{\circled{15}}. %smartelim.ml
We added informative error messages~\href{https://github.com/uwplse/pumpkin-pi/blob/silent/plugin/src/lib/ornerrors.ml}{\circled{22}} to help the proof engineer distinguish between user errors and bugs. % TODO link to errors
These features helped with workflow integration. % tactic decompiler helped with integration into proof engineering workflows.

\begin{table*}
\small
  \begin{tabular}{|l|l|l|l|l|}
    \hline
    \textbf{Class} & \textbf{Config.} & \textbf{Examples} & \textbf{Repair Tools} & \textbf{Search Tools} \\
    \hline
    \multirow[t]{2}{*}{Algebraic Ornaments} & \multirow[t]{2}{*}{Auto} & List to Packed Vector, hs-to-coq \href{https://github.com/uwplse/pumpkin-pi/blob/silent/plugin/coq/examples/Example.v}{\circled{3}} % Example.v
    & \toolnamec, \textsc{Devoid}, UP & \toolnamec, \textsc{Devoid} \\
    & & List to Packed Vector, Standard Library \href{https://github.com/uwplse/pumpkin-pi/blob/silent/plugin/coq/examples/ListToVect.v}{\circled{16}} % ListToVect.v
    & \toolnamec, \textsc{Devoid}, UP & \toolnamec, \textsc{Devoid} \\
    \hline
    Unpack Sigma Types & Auto & Vector of Particular Length, hs-to-coq \href{https://github.com/uwplse/pumpkin-pi/blob/silent/plugin/coq/examples/Example.v}{\circled{3}} % Example.v
    & \toolnamec, UP & \toolnamec \\
    \hline
    \multirow[t]{3}{*}{Tuples \& Records} & \multirow[t]{3}{*}{Auto} & Simple Records \href{https://github.com/uwplse/pumpkin-pi/blob/silent/plugin/coq/minimal_records.v}{\circled{13}} % minimal_records.v 
     & \toolnamec, UP & \toolnamec \\
    & & Parameterized Records \href{https://github.com/uwplse/pumpkin-pi/blob/silent/plugin/coq/more_records.v}{\circled{17}} % more_records.v
    & \toolnamec, UP & \toolnamec \\
    & & Industrial Use \href{https://github.com/Ptival/saw-core-coq/tree/dump-wip}{\circled{18}} %(\href{https://github.com/Ptival/saw-core-coq/tree/dump-wip}{saw-core-coq})
    & \toolnamec, UP & \toolnamec \\
    \hline
    \multirow[t]{3}{*}{Permute Constructors} & \multirow[t]{3}{*}{Auto} & List, Standard Library \href{https://github.com/uwplse/pumpkin-pi/blob/silent/plugin/coq/Swap.v}{\circled{1}}
    & \toolnamec, UP & \toolnamec \\
     & & Modifying a PL, \textsc{REPLica} Benchmark \href{https://github.com/uwplse/pumpkin-pi/blob/silent/plugin/coq/Swap.v}{\circled{1}} % Swap.v 
     & \toolnamec, UP  & \toolnamec \\
    & & Large Ambiguous Enum \href{https://github.com/uwplse/pumpkin-pi/blob/silent/plugin/coq/Swap.v}{\circled{1}} % Swap.v
    & \toolnamec, UP & \toolnamec \\
    \hline
    Add new Constructors & Mixed & PL Extension, \textsc{REPLica} Benchmark \href{https://github.com/uwplse/pumpkin-pi/blob/silent/plugin/coq/playground/add_constr.v}{\circled{19}} % (\href{https://github.com/uwplse/pumpkin-pi/blob/master/plugin/coq/playground/add_constr.v}{add_constr.v})
    & \toolnamec & \toolnamec (partial) \\
    \hline
    Factor out Constructors & Manual & External Example \href{https://github.com/uwplse/pumpkin-pi/blob/silent/plugin/coq/playground/constr_refactor.v}{\circled{2}} % (\href{https://github.com/uwplse/pumpkin-pi/blob/master/plugin/coq/playground/constr_refactor.v}{constr_refactor.v}) 
    & \toolnamec, UP & None \\
    \hline
    Permute Hypotheses & Manual & External Example \href{https://github.com/uwplse/pumpkin-pi/blob/silent/plugin/coq/playground/flip.v}{\circled{20}} %(\href{https://github.com/uwplse/pumpkin-pi/blob/master/plugin/coq/playground/flip.v}{flip.v}) 
    & \toolnamec, UP & None \\
    \hline
    \multirow[t]{2}{*}{Change Inductive Structure} & \multirow[t]{2}{*}{Manual} & Unary to Binary, Classic Benchmark \href{https://github.com/uwplse/pumpkin-pi/blob/silent/plugin/coq/nonorn.v}{\circled{5}} %(\href{https://github.com/uwplse/pumpkin-pi/blob/master/plugin/coq/nonorn.v}{nonorn.v})
     & \toolnamec, Magaud and Bertot & None \\
     & & Vector to Finite Set, External Example \href{https://github.com/uwplse/pumpkin-pi/blob/silent/plugin/coq/playground/fin.v}{\circled{21}} % (\href{https://github.com/uwplse/pumpkin-pi/blob/master/plugin/coq/playground/fin.v}{fin.v}) 
     & \toolnamec & None \\
    \hline
  \end{tabular}
\vspace{0.05cm}
  \caption{Some changes using \toolnamec (left to right): class of changes, kind of configuration, examples, and Coq tools we know of that support repair along (Repair) or automatic proof of (Search) the equivalence corresponding to each example. Tools considered are \textsc{Devoid}~\cite{Ringer2019}, the Univalent Parametricity (UP) white-box transformation~\cite{tabareau2019marriage}, and a classic tool from 2000~\cite{magaud2000changing}. \toolnamec is the only one that supports tactic suggestions.
More nuanced comparisons to these and more are in Section~\ref{sec:related}.}
\vspace{-0.5cm}
\label{fig:changes}
\end{table*}

\subsection{Implementing the Decompiler}
\label{sec:second}

\mysubsubsec{From Qtac to Ltac}
The mini decompiler assumes more predictable versions of \lstinline{rewrite} and \lstinline{induction}
than those in Coq. \textbf{Decompile} includes additional logic to reason about these tactics~\href{https://github.com/uwplse/pumpkin-pi/blob/silent/plugin/src/coq-plugin-lib/src/coq/decompiler/decompiler.ml}{\circled{11}}. % decompiler.ml
For example, Qtac assumes that there is only one \lstinline{rewrite} direction. Ltac has two rewrite directions,
and so the decompiler infers the direction from the motive.

Qtac also assumes that both tactics take the inductive motive explicitly,
while in Coq, both tactics infer the motive automatically.
Consequentially, Coq sometimes fails to infer the correct motive.
% infers the wrong motive, % without manipulation of goals and hypotheses,
%or fails to infer a motive at all.
%This is especially common for \lstinline{rewrite}, which is purely syntactic.
To handle induction, the decompiler strategically uses \lstinline{revert} to manipulate the goal
so that Coq can better infer the motive.
To handle rewrites, it uses \lstinline{simpl} to refold the goal before rewriting.
Neither of these approaches is guaranteed to work, so the proof engineer may sometimes need to tweak the suggested proof script appropriately.
Even if we pass Coq's induction principle an explicit motive, Coq still sometimes fails due
to unrepresented assumptions.
Long term, using another tactic like \lstinline{change} or \lstinline{refine} before applying these tactics
may help with cases for which Coq cannot infer the correct motive.

\mysubsubsec{From CIC$_{\omega}$ to Coq}
Scaling the decompiler to Coq introduces let bindings, which are generated by 
tactics like \lstinline{rewrite in}, \lstinline{apply in}, and \lstinline{pose}.
\textbf{Decompile} implements~\href{https://github.com/uwplse/pumpkin-pi/blob/silent/plugin/src/coq-plugin-lib/src/coq/decompiler/decompiler.ml}{\circled{11}} % decompiler.ml
support for \lstinline{rewrite in} and \lstinline{apply in} similarly to how it supports
\lstinline{rewrite} and \lstinline{apply}, except that it ensures that the unmanipulated hypothesis does not occur in the body of the let expression,
it swaps the direction of the rewrite, and it recurses into any generated subgoals.
In all other cases, it uses \lstinline{pose}, a catch-all for let bindings.

\mysubsubsec{Pretty Printing}
After decompiling proof terms, \textbf{Decompile} pretty prints the result~\href{https://github.com/uwplse/pumpkin-pi/blob/silent/plugin/src/coq-plugin-lib/src/coq/decompiler/decompiler.ml}{\circled{11}}.
Like the mini decompiler, \textbf{Decompile} represents its output using a predefined grammar of Ltac tactics,
albeit one larger that is larger than Qtac, and that also includes tacticals.
It maintains the recursive proof structure for formatting. %, then uses that to print proofs of subgoals using bullet points.
%It displays the resulting proof script to the proof engineer, who can modify it as needed.
%It includes scripts that automate the process of printing all of these tactic proofs to a Coq file,
%in case the proof engineer does not want an interactive workflow.
\toolnamec keeps all output terms from \textbf{Transform} in the Coq environment in case the decompiler does not succeed.
Once the proof engineer has the new proof, she can remove the old one.


