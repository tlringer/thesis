\subsection{The Theory Beneath}
\label{sec:mot-theory}

\begin{figure*}
\small
\begin{grammar}
<i> $\in \mathbbm{N}$, <v> $\in$ Vars, <s> $\in$ \{ Prop, Set, Type<i> \}

<t> ::= <v> \hspace{0.06cm} | \hspace{0.06cm} <s> \hspace{0.06cm} | \hspace{0.06cm} $\Pi$ (<v> : <t>) . <t> \hspace{0.06cm} | \hspace{0.06cm} $\lambda$ (<v> : <t>) . <t> \hspace{0.06cm} | \hspace{0.06cm} <t> <t>
\end{grammar}
\vspace{-0.3cm}
\caption{Syntax for CoC$_\omega$ with (from left to right) variables, sorts, dependent types, functions, and application.}
\label{fig:coc-syntax}
\end{figure*}

The type theory that Gallina implements is CIC$_{\omega}$, or the Calculus of Inductive Constructions.
CIC$_{\omega}$ is based on the Calculus of Constructions (CoC), a variant of the lambda calculus with polymorphism (types that dependent on types) and dependent types (types that depend on terms)~\cite{coquand:inria-00076024}.
CoC with an infinite universe hierarchy is called CoC$_{\omega}$. % TODO need to explain or defer
The syntax for CoC$_{\omega}$ is in Figure~\ref{fig:coc-syntax}.
Note that whereas in Gallina we represent universal quantification over terms or types with $\forall$, here we represent it with $\Pi$, as is standard.

\begin{figure*}
\small
\begin{grammar}
<t> ::= ... | \hspace{0.06cm} Ind (<v> : <t>)\{<t>,\ldots,<t>\} \hspace{0.06cm} | \hspace{0.06cm} Constr (<i>, <t>) \hspace{0.06cm} | \hspace{0.06cm} Elim(<t>, <t>)\{<t>,\ldots,<t>\}
\end{grammar}
\vspace{-0.3cm}
\caption{CIC$_\omega$ is CoC$_\omega$ with inductive types, inductive constructors, and \kl{primitive eliminators}.}
\label{fig:cic-syntax}
\end{figure*}

The syntax for CIC$_{\omega}$ is in Figure~\ref{fig:cic-syntax}), building on syntax from an existing paper~\cite{Timany2015FirstST};
the type theory is standard and omitted. % TODO
CIC$_{\omega}$ extends CoC$_{\omega}$ with inductive types~\cite{inductive}.
As in Gallina, inductive types are defined by their constructors and eliminators.
Consider the inductive type \lstinline{nat} of unary natural numbers that we saw in Figure~\ref{fig:nat},
this time in CIC$_{\omega}$: % TODO explain sorts

\begin{lstlisting}
Ind (nat : Set) {
  nat,
  nat $\rightarrow$ nat
}
\end{lstlisting}
where the \lstinline{O} constructor type is the zeroth element in the list, and the \lstinline{S} constructor type is the first element.
Accordingly, the terms:

\begin{lstlisting}
Constr (O, nat)
\end{lstlisting}
and:

\begin{lstlisting}
Constr (1, nat)
\end{lstlisting}
refer to the constructors \lstinline{O} and \lstinline{S}, respectively.

As in Gallina, \lstinline{nat} comes associated with an eliminator.
Unlike in Gallina, here we truly assume \kl{primitive eliminators}---that these eliminators
do not reduce at all.
Instead, we represent them explicitly with the \lstinline{Elim} construct.
Thus, to eliminate over a natural number \lstinline{n} with motive \lstinline{P},
we write: % TODO check

\begin{lstlisting}
Elim (n, P) {
  f$_O$,
  f$_S$
}
\end{lstlisting}
where functions:

\begin{lstlisting}
f$_O$ : P O
\end{lstlisting}
and:

\begin{lstlisting}
f$_S$ : $\Pi$ (n : nat) . $\Pi$ (IHn : P n) . P (S n)
\end{lstlisting}
prove the base and inductive cases, respectively.
When \lstinline{n}, \lstinline{P}, \lstinline{f}$_O$, and \lstinline{f}$_S$ are arbitrary,
this statement has the same type as \lstinline{nat_rect} in Gallina.

Gallina implements CIC$_{\omega}$, but with a few important differences.
More information is on the website, % https://coq.github.io/doc/v8.9/refman/language/gallina-specification-language.html, noting it's 8.9
but two differences are relevant to repair:
The first is that Gallina lacks \kl{primitive eliminators}, as we mentioned earlier.
The second notable difference is that Gallina has constants that define terms---later on, this will help with building optimizations for repair tools.

Otherwise, a proof repair tool for Gallina can harness the power of CIC$_{\omega}$. This type theory is fairly simple,
but $\Pi$ makes it possible to quantify over both terms and types,
so that we can state powerful theorems and prove that they hold.
Inductive types make it possible to write proofs by induction.
Both of these constructs mean that terms in Gallina are extremely structured,
and as we will soon see, that structure makes a proof repair tool's job much easier.

But this structure can be difficult for proof engineers to work with,
which is why proof engineers typically rely on the tactics we saw in Section~\ref{sec:mot-workflow}.
Tactics more generally are a form of \textit{proof automation}, and this proof automation makes it much
simpler to develop proofs to begin with.
But it turns out this proof automation is a bit naive when it comes to \textit{maintaining} proofs
as programs and specifications change over time.
Proof repair is a new form of proof automation for maintaining proofs: it uses the rich type information carried by proof terms
to automatically fix broken proofs in response to change.


