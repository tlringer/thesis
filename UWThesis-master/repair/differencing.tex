\section{Differencing}
\label{sec:pumpkin-diff}

% From Motivating the Core. Some of this might move into the Approach.

The first step finds this goal type:

\begin{lstlisting}[language=coq]
  (@\ltacforall@) n m p, n <= m $\rightarrow$ m <= p $\rightarrow$ (@\diff{n <= p}@) $\rightarrow$ (@\diff{n <= p + 1}@)
\end{lstlisting}
It then breaks each inductive proof into cases and determines an intermediate goal type for the candidate.
%recursively searches each case for a candidate.
In the base case, for example, it \textit{diffs} the types and determines that a candidate
between the base cases of \lstinline{new} and \lstinline{old} should have this type (lines 11 and 12):

\begin{lstlisting}
  (@\diff{(fun p0 => n <= p0)}@) m $\rightarrow$ (@\diff{(fun p0 => n <= p0 + 1)}@) m
\end{lstlisting}
It then \textit{diffs} the terms (line 13) for such a candidate:

\begin{lstlisting}
  fun n m p H0 H1 =>(@\vspace{-0.04cm}@)
    (fun (H : n <= m) => (@\diff{le\_plus\_trans n m 1}@) H)(@\vspace{-0.04cm}@)
  : (@\ltacforall@) n m p, n <= m -> m <= p -> (@\diff{n <= m}@) -> (@\diff{n <= m + 1}@)
\end{lstlisting}
This candidate is close, but it is not yet a patch. This candidate
maps base case to base case (it is applied to \lstinline{m}); the patch should map conclusion to conclusion (it should
be applied to \lstinline{p}). % TODO redundant for now

Note again it's over terms since tactics would have been insufficient,
and it understands and is guided by semantics! This is cool and should probably open this section.
Differencing is aware of and guided by the semantics of Coq's rich proof term language Gallina---that is, it is a \textit{semantic differencing} algorithm.
In other words, it takes advantage of Gallina's rich structure to guide the differencing process,
then uses Gallina's rich type system to ensure that it finds a correct candidate patch in the end.
I will explain this in more detail in Section~\ref{sec:pumpkin-diff}.
Simply exploring the syntactic difference %between terms 
makes it hard to identify
which changes are meaningful.
For example, in the inductive case (line 14), the inductive hypothesis
changes:

\begin{lstlisting}[language=coq]
  ... (IHle : (@\diff{n <= m0 + 1}@)) ...(@\vspace{-0.08cm}@)
  ... (IHle : (@\diff{n <= m0}@)) ...
\end{lstlisting}

However, the type of \lstinline{IHle} changes for \emph{any} two inductive proofs over \lstinline{le}
with different conclusions. A syntactic differencing component 
may identify this change as a candidate.
Our semantic differencing component knows that it can ignore this change.

note somewhere that it actually recursively diffs

Plus parts of Inside the Core, Testing Boundaries, Future Work

How differencing works in detail

Limitations and whether they're addressed in later tools yet or not

