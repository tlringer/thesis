\chapter{Introduction}

Motivation for verifying systems

Era of scale---enter proof engineering~\cite{PGL-045}

Looking back (Social Processes~\cite{DeMillo1977}), development has come a long way, but maintenance is still hard! And this is a problem in practice!

But missed opportunity: automation doesn't understand that proofs evolve

So we build automation that does, and we call this proof repair. Proof repair shows that there is reason to believe that verifying a modified system should often, in practical use cases, be easier than verifying the original the first time around.

Or, in other words (thesis statement): Changes in programs, specifications, and proofs carry information that a tool can extract, generalize, and apply to fix other proofs broken by the same change. A tool that automates this can save work for proof engineers relative to reference manual repairs in practical use cases.

Key technical bit: differencing and program transformations, taking advantage of the rich and structured language proofs are written in.

We implement this in a tool suite for Coq, get some sweet results.

Pave path to the next era of verification

% Dan: It might be good either in 1 or in both 3 and 4 to lay out the key /results/ -- you have _design_ of the transformation, 
% the highly non-trivial _implementation_ in Coq in a way that doesn't extend the TCB, and you have substantial case studies to _evaluate_.
% You have a bit of formalism and metatheory, but that is more for exposition than for specific results (am I right?) -- you are 
% guided by the theory to build tools that actually work in this unforgiving domain -- "hacks won't work".

\chapter{Reading Guide}

% Dan: Instead, I strongly advocate an explicit section in Chapter 1 that lays this all out for the reader: what prior publications are being leaned on, where does text from those reappear, and most importantly, where -- in explicit sections of forthcoming chapters -- are expanded explanations and additional data.  Voila, your thesis is now both a self-contained coherent document and a useful 'appendix' for people who have already read large parts of what is to follow.

