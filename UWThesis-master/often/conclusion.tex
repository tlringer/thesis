\section{Conclusion}
\label{sec:pi-concl}

The \toolnamec plugin extends the \sysnamelong plugin suite with support
for a broad class of changes in datatypes.
It also supports patch application in a principled manner,
is built with workflow integration including tactics in mind,
and can save and in fact already has saved work for proof engineers in practical use cases.
At this point, it is fair to say that my \kl{thesis} holds:

\begin{quote}
Changes in programs, specifications, and proofs can carry information that a tool can extract, generalize, and apply to fix other proofs broken by the same change (Sections~\ref{sec:pumpkin-approach},~\ref{sec:pumpkin-diff},~\ref{sec:pumpkin-trans},~\ref{sec:pi-approach},~\ref{sec:pi-diff}, and~\ref{sec:pi-trans}). A tool that automates this (Sections~\ref{sec:pumpkin-impl} and~\ref{sec:pi-implementation}) can save work for proof engineers relative to reference manual repairs in 
practical use cases (Sections~\ref{sec:pumpkin-results} and~\ref{sec:pi-results}).
\end{quote}
And so there really is \kl{reason to believe}.

I will talk about what that means for proof engineers---and what I believe the next era of verification can look like---in Chapter~\ref{chapt:conclusions}.
But first, I will back up a bit and talk about related work.
