\section{Reading Guide}
\label{sec:guide}

% Dan: Instead, I strongly advocate an explicit section in Chapter 1 that lays this all out for the reader: what prior publications are being leaned on, where does text from those reappear, and most importantly, where -- in explicit sections of forthcoming chapters -- are expanded explanations and additional data.  Voila, your thesis is now both a self-contained coherent document and a useful 'appendix' for people who have already read large parts of what is to follow.

This thesis assumes some background in \kl{proof engineering}, \intro{type theory}, and (to a lesser extent) the \kl{Coq} \kl{proof assistant}. 
I strongly encourage readers of all backgrounds who would like more context to better understand this thesis
look to my survey paper on proof engineering~\cite{PGL-045}, which includes a detailed list of resources
and is available for free on my website: \url{https://dependenttyp.es}.

I recommend that readers with less background on proof engineering, dependent type theory, or Coq
take time to digest Chapter~\ref{chapt:mot} before moving on---though I recommend that even Coq experts read Chapter~\ref{chapt:mot}!
Chapters~\ref{ch:example} and~\ref{chapt:pi} get rather technical, so it is normal not to understand every detail.
You may always contact me with questions.

\subsection*{Previously Published Material}

While this thesis is self-contained, it centers material from two previously published papers:

\begin{itemize}
\item \textbf{Talia Ringer}, Nathaniel Yazdani, John Leo, and Dan Grossman. \intro{Adapting Proof Automation to Adapt Proofs}~\cite{ringer2018adapting}. Certified Programs and Proofs. 2018.
\item \textbf{Talia Ringer}, RanDair Porter, Nathaniel Yazdani, John Leo, and Dan Grossman. \intro{Proof Repair across Type Equivalences}~\cite{Ringer2021}. Programming Languages Design and Implementation. 2021.
\end{itemize}
It also includes material from three other papers:

\begin{itemize}
\item \textbf{Talia Ringer}, Nathaniel Yazdani, John Leo, and Dan Grossman. \intro{Ornaments for Proof Reuse in Coq}~\cite{Ringer2019}. Interactive Theorem Proving. 2019.
\item \textbf{Talia Ringer}, Alex Sanchez-Stern, Dan Grossman, and Sorin Lerner. \intro{\toolname: REPL Instrumentation for Coq Analysis}~\cite{replica}. Certified Programs and Proofs. 2020.
\item \textbf{Talia Ringer}, Karl Palmskog, Ilya Sergey, Milos Gligoric, and Zachary Tatlock. \intro{QED at Large: A Survey of Engineering of Formally Verified Software}~\cite{PGL-045}. Foundations and Trends® in Programming Languages: Vol. 5: No. 2-3, pp 102-281. 2019. 
\end{itemize}
The following is a map from each of these papers to corresponding sections,
along with an explanation of what is new in this thesis and what is omitted.
All of these papers are free on my website.

\paragraph{}\hspace{-0.6cm}
\kl{Adapting Proof Automation to Adapt Proofs}:
The bulk of Chapter~\ref{ch:example} comes from this paper,
though the content is significantly reorganized and reframed.
The introduction and conclusion of Chapter~\ref{ch:example} are fresh.
Sections~\ref{sec:pumpkin-approach}, \ref{sec:pumpkin-diff}, \ref{sec:pumpkin-trans}, and~\ref{sec:pumpkin-impl}
include content beyond the original paper.
Chapter~\ref{sec:related} includes some related work from this paper,
and Chapter~\ref{chapt:conclusions} includes some future work from this paper.

\paragraph{}\hspace{-0.6cm}
\kl{Proof Repair across Type Equivalences}:
Parts of the introduction and Section~\ref{sec:mot-theory} come from this paper.
The bulk of Chapter~\ref{chapt:pi} comes from this paper,
though the content is likewise reorganized and reframed.
The conclusion of Chapter~\ref{chapt:pi} is fresh.
Sections~\ref{sec:pi-approach}, \ref{sec:pi-diff}, \ref{sec:pi-trans}, and~\ref{sec:pi-implementation}
all include content beyond the original paper.
Chapter~\ref{sec:related} includes some related work from this paper,
and Chapter~\ref{chapt:conclusions} includes some future work from this paper.

\paragraph{}\hspace{-0.6cm}
\kl{Ornaments for Proof Reuse in Coq}:
The example from Section~\ref{sec:mot-dev} comes from this paper, though most of the text is new.
Parts of Section~\ref{sec:mot-theory} also come from this paper.
Section~\ref{sec:pi-diff} shows a simplified version of the differencing algorithm (originally called the search algorithm) from this paper.
Section~\ref{sec:eval} includes the evaluation from this paper with additional context.
Chapter~\ref{sec:related} includes related work from this paper.
This thesis retires the name of the tool from this paper (\textsc{Devoid})
in favor of the name of the generalized version of the tool from \kl{Proof Repair across Type Equivalences} (\toolnamec).

\paragraph{}\hspace{-0.6cm}
\contour[2]{violet}{\kl{\textsc{REPLica}}}:
Section~\ref{sec:irl} includes a few samples of this paper,
and the abstract includes a few sentences from this paper.

\paragraph{}\hspace{-0.6cm}
\kl{QED at Large}:
Chapter~\ref{chapt:mot} includes a few samples of this survey paper.
Chapter~\ref{sec:related} includes a large amount of related work from this paper.

\subsubsection*{Authorship Statements}

The material in this thesis draws on work that I did with
four student and postdoctoral coauthors: \kl{Nathaniel Yazdani}, \kl{RanDair Porter}, \kl{Alex Sanchez-Stern},
and \intro{Karl Palmskog}.
Below is a summary of the contributions of each of those coauthors,
indexed for later reference.
The contributions of my faculty and professional coauthors---\intro{John Leo}, \kl{Dan Grossman}, \kl{Zach Tatlock},
\intro{Ilya Sergey}, \intro{Milos Gligoric}, and \intro{Sorin Lerner}---were of course also extremely valuable:

\paragraph{Nathaniel Yazdani}
I worked with \kl{Nate} starting from when he was an undergraduate student.
He contributed conceptually to all three proof repair papers his name appears on,
helped with a number of the case studies,
implemented important features on the critical path to success,
and did some of the writing about his contributions.
His contributions include:

\begin{enumerate}
\item a tool for preprocessing proof developments into a format suitable for repair,
\item higher-order transformations for applying proof term transformations over entire libraries, and
\item a key early insight about equality.
\end{enumerate}
All three of these were necessary to scale proof repair to help real proof engineers in practical scenarios.

\paragraph{RanDair Porter}
\kl{RanDair} joined the project as an undergraduate.
He implemented a prototype decompiler from proof terms to proof scripts,
and wrote a description of the behavior of the decompiler that I built on in the corresponding paper.
This decompiler was necessary for integrating proof repair tools with real proof engineering workflows,
and it continues to inspire exciting work.

\paragraph{Alex Sanchez-Stern}
\kl{Alex} worked with me as a PhD student on the \contour[2]{violet}{\kl{\textsc{REPLica}}} user study of proof engineers. % during my visit \kl{UCSD}.
He designed, implemented, and evaluated one of the two analyses in the user study paper.
He also helped substantially in building the infrastructure necessary to deploy the user study,
and wrote large sections of the paper.
The user study and paper would not have happened without him.

\paragraph{Karl Palmskog}
\kl{Karl} was a postdoctoral researcher when he joined me on the \kl{QED at Large} survey paper.
He wrote entire chapters of the survey paper.
I could not have written that paper without him.

\subsubsection*{Pronouns}

In this thesis, I use ``I'' to refer to work that I did,
even though of course no work happens in a vacuum.
I use the names of my coauthors like ``\kl{Nate}'' or ``\kl{RanDair}'' to refer to work that they did,
when I operated primarily in an advisory role.
When I collaborated with my coauthors, I name them and myself, like ``\kl{Nate} and I,''
and then (when not ambiguous) I use ``we'' thereafter.
Throughout, I also use mathematical ``we'' to mean both myself and the reader.

When I discuss a rhetorical proof engineer who does not actually exist,
like ``the proof engineer,'' I always use ``she''---this is a small attempt
to seed the world with data that counteracts stereotypes. 
When preserving anonymity of a particular person, I always use singular ``they.''
Otherwise, I use the person's pronouns.

\subsubsection*{Historical Note}

At the time of writing, the Coq community is considering renaming the Coq proof assistant.
There is a chance that the name of the proof assistant will be different for future readers.



