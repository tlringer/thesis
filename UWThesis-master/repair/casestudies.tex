\section{Case Studies: Changes in the Wild}
\label{sec:case}

We used the \sysname prototype to emulate three motivating scenarios from real-world code:

\begin{enumerate}
\item \textbf{Updating definitions} within a project \\
(CompCert, Section~\ref{sec:compcert})
\item \textbf{Porting definitions} between libraries \\
(Software Foundations, Section~\ref{sec:foundations})
\item \textbf{Updating proof assistant versions} \\
(Coq Standard Library, Section~\ref{sec:coq})
\end{enumerate}

The code we chose for these scenarios demonstrated different classes of changes.
For each case, we describe how \sysname configures the procedure to use the core components for that class of changes.
Our experiences with these scenarios suggest that patches are useful and that the components 
are effective and flexible.

%For each scenario, we describe the experience using \sysname
%as well as the algorithm and the core components the scenario used.

%For each use case, %we identified how to adapt \sysname.
%In doing so, 
%we identified which core components were useful.
%In total, we used all five core components, and we did not need any new components.
%We identified useful extensions to two components:
%semantic search and patch abstraction.

% Also new ways to combine them, new and unforeseen uses to existing components
% Lemma detection is subcase of goal detection

%\vspace{-0.1cm}
\paragraph{Identifying Changes} We identified Git commits from popular Coq projects that
demonstrated each scenario.
These commits updated proofs in response to breaking changes. %that broke as a result of some change.
%which we could use as an example for \sysname.
We emulated each scenario as follows:

\begin{enumerate}
\item \textit{Replay} an example proof update for \sysname
\item \textit{Search} the example for a patch using \sysname
\item \textit{Apply} the patch to fix a different broken proof
\end{enumerate}

%In each use case, we built new algorithms from the core components of 
%\sysname. In doing so, we identified any extensions necessary to improve
%the core components to handle new uses cases.

Our goal was to simulate incremental use of a patch finding tool,
at the level of a small change or a commit that follows best practices.
We favored commits with changes that we could
isolate. When isolating examples for \sysname, we replayed changes from the bottom up,
as if we were making the changes ourselves.
This means that we did not always make the same change as the user. For example,
the real change from Section~\ref{sec:compcert} updated multiple definitions;
we updated only one.

\sysname is a proof-of-concept and does not yet handle some kinds of proofs.
In each scenario, we made minor modifications to proofs so that we could use \sysname (for example,
using induction instead of destruction).
\sysname does not yet handle structural changes like adding constructors or parameters, 
so we focused on changes that preserve shape, like modifying constructors.
We discuss supporting these features and whether they may necessitate other components in Section~\ref{sec:future}.
%Supporting these features is a matter of engineering, which we leave to future work 

\begin{figure*}[ht]
\begin{minipage}{0.48\textwidth}
\lstset{language=coq, aboveskip=0pt,belowskip=0pt} % blindFS / User A
\lstinputlisting[firstline=2, lastline=4]{repair/bintonat.tex}
\lstinputlisting[backgroundcolor=\color{orange!35},firstline=5,lastline=6]{repair/bintonat.tex}
\lstinputlisting[firstline=7, lastline=7]{repair/bintonat.tex}
\end{minipage}
\hfill
\begin{minipage}{0.48\textwidth}
\lstset{language=coq, aboveskip=0pt,belowskip=0pt}  % marshall / user B
\lstinputlisting[firstline=10, lastline=12]{repair/bintonat.tex}
\lstinputlisting[backgroundcolor=\color{orange!35},firstline=13,lastline=14]{repair/bintonat.tex}
\lstinputlisting[firstline=15, lastline=15]{repair/bintonat.tex}
\end{minipage}
\vspace{-.35cm}
%\setlength{\belowcaptionskip}{-4pt}
\caption[Caption for LOF]{Definitions of \lstinline{bin_to_nat} for Users A (left) and B (right).}
\label{fig:bintonat}
\end{figure*}

\lstset{language=coq, aboveskip=3pt,belowskip=3pt}

%\vspace{-0.1cm}
%\subsection{Patching Proofs\\\textit{\small{CompCert}}}
\subsection{Updating Definitions}
\label{sec:compcert}

Coq programmers sometimes make changes to definitions that break proofs
within the same project. To emulate this use case, 
we identified a CompCert commit~\cite{compcertcommit}
with a breaking change to \lstinline{int} (Figure~\ref{fig:int}). %The commit message notes that the new representation is more efficient.
We used \sysname to find a patch that corresponds to the change in \lstinline{int}.
The patch \sysname found fixed broken inductive proofs.

% TODO still need to ask Xavier, not sure if this depends on other definitions also changing


%to find a patch between versions of a changed constructor of an inductive type.
%This patched broken inductive proofs.
%In doing so, we used all five of the core components---four of them without any changes.

%\vspace{-0.1cm}
\paragraph{Replay} We used the proof of \lstinline{unsigned_range} as the example for \sysname.
The proof failed with the new \lstinline{int}:

\lstset{language=coq, aboveskip=3pt,belowskip=3pt}
\begin{lstlisting}[language=coq]
   Theorem unsigned_range:(@\vspace{-0.04cm}@)
     (@\ltacforall@)(i : int), 0 <= unsigned i < modulus.(@\vspace{-0.04cm}@)
   Proof.(@\vspace{-0.04cm}@)
     intros i. induction i using int_ind; auto(@\fails{.}@)
\end{lstlisting}

We replayed the change to \lstinline{unsigned_range}:

\begin{lstlisting}[language=coq]
     intros i. induction i using int_ind. (@\diff{simpl. omega}@)(@\succeeds{.}@)
\end{lstlisting}

%\vspace{-0.1cm}
\paragraph{Search} We used \sysname to search the example for a patch that corresponds to the change in \lstinline{int}. It found
a patch with this type:

\begin{lstlisting}[language=coq]
   (@\ltacforall@) z : Z,(@\vspace{-0.04cm}@) (@\diff{-1 < z < modulus}@) -> (@\diff{0 <= z < modulus}@)
\end{lstlisting}

%\vspace{-0.1cm}
\paragraph{Apply} After changing the definition of \lstinline{int}, the proof of the
theorem \lstinline{repr_unsigned} failed on the last tactic:

\begin{lstlisting}[language=coq]
   Theorem repr_unsigned:(@\vspace{-0.04cm}@)
     (@\ltacforall@)(i : int), repr (unsigned i) = i.(@\vspace{-0.04cm}@)
   Proof.(@\vspace{-0.04cm}@)
     ... apply Zmod_small; auto(@\fails{.}@)
\end{lstlisting}

Manually trying \lstinline{omega}---the tactic which helped us in the proof of \lstinline{unsigned_range}---did not
succeed.
We added the patch that \sysname found to a hint database.
The proof of the theorem \lstinline{repr_unsigned} then went through:

\begin{lstlisting}[language=coq]
    ... apply Zmod_small; auto(@\succeeds{.}@)
\end{lstlisting}

%\vspace{-0.1cm}
\subsubsection{Core Components}

%\paragraph{The Problem} The algorithm for changes in constructors of inductive
%types needed to produce patches with a form we had not yet encountered.
%The algorithm for changes in conclusions of proofs that induct over the same
%type produced patches with this form:

%\begin{lstlisting}[language=coq]
 %  (@\ltacforall@)(x : T), P x -> P' x
%\end{lstlisting}

This scenario used the configuration for changes in constructors of an inductive type.
Given such a change: %of an inductive type:

\begin{lstlisting}[language=coq]
   Inductive (@\diff{T}@) := ... | C : ... -> (@\diff{H}@) -> T(@\vspace{-0.08cm}@)
   Inductive (@\diff{T'}@) := ... | C : ... -> (@\diff{H'}@) -> T'
\end{lstlisting}
\sysname searches two inductive proofs of theorems:

\begin{lstlisting}[language=coq]
   (@\ltacforall@) (t : (@\diff{T}@)), P t(@\vspace{-0.08cm}@)
   (@\ltacforall@) (t : (@\diff{T'}@)), P t
\end{lstlisting}
for an isomorphism\footnote{If \sysname finds just one implication, it returns that.} between the constructors:

\begin{lstlisting}[language=coq]
   ... -> (@\diff{H}@) -> (@\diff{H'}@)(@\vspace{-0.1cm}@)
   ... -> (@\diff{H'}@) -> (@\diff{H}@)
\end{lstlisting}

The user can apply these patches within the inductive case that corresponds to the constructor \lstinline{C}
to fix other broken proofs that induct over the changed type. 
\sysname uses this configuration for changes in constructors:

\begin{algorithm}
\footnotesize
\begin{algorithmic}[1]
%  Is it really necessary to say that an algorithm might do initialization?
%  \STATE Build trees for search
  %\STATE build trees for proofs
  %\REPEAT
    \STATE \textit{diff} inductive constructors for goals
    \STATE use \textit{all components} to recursively search for changes in conclusions of the corresponding case of the proof
    \STATE \textbf{if} there are candidates \textbf{then}
    \STATE \hspace*{1em} try to \textit{invert} the patch to find an isomorphism 
\end{algorithmic}
\end{algorithm}

%We found it natural to use this patch within
%inductive proofs; if necessary, we can construct a patch that maps between types instead of constructors
%by applying the patch within the induction principle.
% TODO do we ever have to generalize by P? Why or why not?

%\paragraph{The Extension} It was simple to adapt \sysname to handle changes in the constructor of an inductive type. 
%We extended the type search algorithm to find search goals that correspond to changes in constructors.
%We used the other four components without any changes.

%\subsection{Porting Definitions\\\textit{\small{Software Foundations}}}

\subsection{Porting Definitions}
\label{sec:foundations}

Coq programmers sometimes port theorems and proofs to use definitions
from different libraries.
To simulate this, we used \sysname to port two solutions~\cite{usera, userb}
to an exercise in Software Foundations to each use the other solution's definition of the fixpoint \lstinline{bin_to_nat} (Figure~\ref{fig:bintonat}).
We demonstrate one direction; the opposite was similar.
%We used \sysname to find a patch between versions of a fixpoint;
%this patched inductive proofs about \lstinline{bin_to_nat}.
%We used \sysname to port User A's proofs to User B's definition and to port User B's proofs to User A's definition.

%In doing so, we used all five components---four without any changes.

%We also combined components in ways that we had not yet encountered,
%which may It combined components in ways that we had not yet encountered,
%which may help us improve our algorithms. It also demonstrated a new use case for abstraction
%that we had not considered; we were able to adapt \sysname to handle this easily.

% With this framing we lose the "why not tactics," but notably all of these are not obvious from tactics

%\vspace{-0.1cm}
\paragraph{Replay} We used the proof of \lstinline{bin_to_nat_pres_incr} from User A as the example for \sysname.
User A cut an inline lemma in an inductive case and proved it using a rewrite:

\begin{lstlisting}[language=coq]
   assert ((@\ltacforall@)a, S (a + S (@\diff{(a + 0)}@)) = S (S (a + (@\diff{(a + 0)}@)))).(@\vspace{-0.04cm}@)
   (@\ltacba@) ... (@\diff{rewrite <- plus\_n\_O.}@) rewrite -> plus_comm.
\end{lstlisting} % blindFS / user A

When we ported User A's solution to use User B's definition of \lstinline{bin_to_nat}, 
the application of this inline lemma failed. We changed the conclusion of the inline lemma 
and removed the corresponding rewrite:

\begin{lstlisting}[language=coq]
   assert ((@\ltacforall@)a, S (a + S a) = S (S (a + a))).(@\vspace{-0.04cm}@)
   (@\ltacba@) ... rewrite -> plus_comm.
\end{lstlisting} % blindFS / user A, adapted

%\vspace{-0.1cm}
\paragraph{Search} We used \sysname to search the example
for a patch that corresponds to the change in \lstinline{bin_to_nat}.
It found an isomorphism:

\begin{lstlisting}[language=coq]
   (@\ltacforall@)P b, P (@\diff{(bin\_to\_nat b)}@) -> P (@\diff{(bin\_to\_nat b + 0)}@)(@\vspace{-0.08cm}@)
   (@\ltacforall@)P b, P (@\diff{(bin\_to\_nat b + 0)}@) -> P (@\diff{(bin\_to\_nat b)}@)
\end{lstlisting}

%The only semantic difference between the two definitions of \lstinline{bin_to_nat} was that in 
%each recursive case,
%User A had one occurrence of \lstinline{bin_to_nat b' + 0} where User B had \lstinline{bin_to_nat b'}.

%This found an abstracted version of the term produced by the rewrite by \lstinline{plus_n_O},
%and inverted the abstract patch so that it can be used in both directions.

%\sysname was able to find a patch for this proof by searching the difference in proof terms of \lstinline{bin_to_nat_pres_incr}.
%In particular, each proof of \lstinline{bin_to_nat_pres_incr} passes through an inline lemma. \sysname found a patch
%between the conclusions of the inline lemmas, which had the type:

%\begin{lstlisting}[language=coq]
%   (@\ltacforall@)a,
%     S (a + S (@\remove{(}@)a (@\remove{+ 0)}@)) = S (S (a + (@\remove{(}@)a (@\remove{+ 0)}@))) ->
%     S (a + S (a + 0)) = S (S (a + (a + 0)))
%\end{lstlisting}

%In other words, for a specific property \lstinline{P}, this patch had the type:

%\begin{lstlisting}[language=coq]
 %  (@\ltacforall@)a, P a -> P (a + 0)
%\end{lstlisting}
%
%\sysname then applied that patch to the argument the lemma was applied to in each theorem:

%\begin{lstlisting}[language=coq]
%   (@\ltacforall@)b, P (bin_to_nat b) -> P (bin_to_nat b + 0)
%\end{lstlisting}

%\sysname then abstracted the result to work for any property:

%\begin{lstlisting}[language=coq]
%   (@\ltacforall@)P b, P (bin_to_nat b) -> P (bin_to_nat b + 0)
%\end{lstlisting}

%Finally, \sysname inverted the result to find the patch in the opposite direction:

%\begin{lstlisting}[language=coq]
%   (@\ltacforall@)P b, P (bin_to_nat b + 0) -> P (bin_to_nat b)
%\end{lstlisting}

%\vspace{-0.1cm}
\paragraph{Apply} After porting to User B's definition, a rewrite in the proof of the theorem
\lstinline{normalize_correctness} failed:

\begin{lstlisting}[language=coq]
   Theorem normalize_correctness:
     (@\ltacforall@)b, nat_to_bin (bin_to_nat b) = normalize b.
   Proof.
     ... (@\fails{rewrite -> plus\_0\_r.}@)
\end{lstlisting}

Attempting the obvious patch from the difference in tactics---rewriting by \lstinline{plus_n_O}---failed.
Applying the patch that \sysname found fixed the broken proof:

\begin{lstlisting}[language=coq]
     ... (@\diff{apply patch\_inv.}@) (@\succeeds{rewrite -> plus\_0\_r.}@)
\end{lstlisting}

% TODO shorten this:

In this case, since we ported User A's definition to a simpler 
definition,\footnote{User A uses \lstinline{*}; User B uses \lstinline{+}. 
For arbitrary \lstinline{n}, the term \lstinline{2 * n} reduces to \lstinline{n + (n + 0)}, which does not reduce any further.}
\sysname found a patch that was not the most natural patch.
The natural patch would be to remove the rewrite, just as we removed a different rewrite from the example proof.
This did not occur when we ported User B's definition,
which suggests that in the future, a patch finding tool may help inform novice users which definition is simpler:
It can factor the proof, %into a sequence of lemmas, 
then inform the user if two factors are inverses. %, who can remove the admissable tactics.
Tactic-level changes do not provide enough information to determine this; the tool must have a semantic
understanding of the terms.

%\paragraph{User A $\rightarrow$ User B}


%\paragraph{User B $\rightarrow$ User A}

%Porting in the opposite direction worked similarly. 
%User B proved \lstinline{bin_to_nat_pres_incr} through a more general auxiliary lemma:

%\begin{lstlisting}[language=coq]
 %  (@\ltacforall@)a b, S a + S b = S (S (a + b)).
%\end{lstlisting} % marshall / user B

%To port user B's proof, we added a rewrite by \lstinline{plus_n_O}. \sysname searched the difference of those auxiliary lemmas,
%applied the patch, and abstracted and inverted it accordingly.

%Both the patch and the inverse patch were useful for two later proofs. In both inductive cases
%of a lemma \lstinline{bin_to_nat_plus}, we were able to patch each broken line:

%\begin{lstlisting}[language=coq]
%   (@\fails{reflexivity.}@)
%\end{lstlisting}

%By applying the patch:

%\begin{lstlisting}[language=coq]
 %  simpl. apply patch. (@\succeeds{auto.}@)
%\end{lstlisting}

%Similarly, in the proof of \lstinline{nat_to_bin_bin_to_nat}, we were able to fix two cases that were broken in the same way:

%\begin{lstlisting}[language=coq]
%   (@\fails{rewrite -> nat\_to\_bin\_Sn\_Sn.}
%\end{lstlisting}

%By applying the the inverse patch before destructing:

%\begin{lstlisting}[language=coq]
 %  apply patch_inv.
  % destruct (bin_to_nat b').
   %(@\ltacba@) reflexivity.
   %(@\ltacba@) (@\succeeds{rewrite -> nat\_to\_bin\_Sn\_Sn.}@)
%\end{lstlisting}

%\vspace{-0.1cm}
\subsubsection{Core Components}

This scenario used the configuration for changes in cases of a fixpoint.
Given such a change:

\begin{lstlisting}[language=coq]
   Fixpoint (@\diff{f}@) ... := ... | g (@\diff{x}@)(@\vspace{-0.1cm}@)
   Fixpoint (@\diff{f'}@) ... := ... | g (@\diff{x'}@)
\end{lstlisting}
\sysname searches two proofs of theorems:

\begin{lstlisting}[language=coq]
   (@\ltacforall@) ..., P ((@\diff{f}@) ...)(@\vspace{-0.1cm}@)
   (@\ltacforall@) ..., P ((@\diff{f'}@) ...)
\end{lstlisting}
for an isomorphism that corresponds to the change:

\begin{lstlisting}[language=coq]
   (@\ltacforall@)P, P (@\diff{x}@) -> P (@\diff{x'}@)(@\vspace{-0.1cm}@)
   (@\ltacforall@)P, P (@\diff{x'}@) -> P (@\diff{x}@)
\end{lstlisting}

The user can apply these patches to fix other broken proofs about the fixpoint.

The key feature that differentiates these from the patches we have encountered so far is that
these patches hold for \emph{all} \lstinline{P}; for changes in fixpoint cases, the procedure abstracts
candidates by \lstinline{P}, not by its arguments.
\sysname uses this configuration for changes in fixpoint cases:

\begin{algorithm}
\footnotesize
\begin{algorithmic}[1]
%  Is it really necessary to say that an algorithm might do initialization?
%  \STATE Build trees for search
  %\STATE build trees for proofs
  %\REPEAT
    \STATE \textit{diff} fixpoint cases for goals
    \STATE use \textit{all components} to recursively search an intermediate lemma for a change in conclusions
    \STATE \textbf{if} there are candidates \textbf{then}
    \STATE \hspace*{1em} \textit{specialize} and \textit{factor} the candidate \\
           \hspace*{1em} \textit{abstract} the factors by functions \\
           \hspace*{1em} try to \textit{invert} the patch to find an isomorphism 
\end{algorithmic}
\end{algorithm}

%The procedure configures Procedure~\ref{alg:patching} to look for differences in fixpoint cases
%on line 1, to recursively search an intermediate lemma for a change in conclusions on line 2,
%and to specialize, factor, abstract, then try to invert the patch on line 4.
%This composes all five components.

% We thus abstract candidates not by the arguments to \lstinline{P}, but by
%the function \lstinline{P}.
%The procedure composes all five components:

%\begin{algorithm}
%\footnotesize
%\begin{algorithmic}[1]
 % \STATE \textit{diff} the fixpoints for the change in case
 % \STATE \textit{diff} the proof terms for an intermediate lemma
 % \STATE use Procedure~\ref{alg:alg1} (\textit{all components}) to search the difference in proofs of the lemma for a patch in conclusions
 % \STATE \textit{specialize} the patch to the lemma arguments
 % \STATE \textit{factor} and \textit{abstract} the patch function
 % \STATE \textit{invert} the patch
 % \IF{there is a patch and/or inverse}
 %   \RETURN patch and/or inverse
 % \ENDIF
 % \RETURN failure
%\end{algorithmic}
%\caption{Changes in Fixpoint Cases}
%\label{alg:alg3}	
%\end{algorithm}

For the prototype, we require the user to cut the intermediate lemma explicitly and to 
pass its type and arguments. %to the tool.
In the future, an improved semantic differencing component
can infer both the intermediate lemma and the arguments: It can search
within the proof for some proof of a function that is applied
to the fixpoint.

%\paragraph{The Extension} We improved two of the components:
%We extended type search to handle changes in fixpoints, and we extended
%abstraction to abstract functions as opposed to only arguments.
%We also found that the components interacted in a new way:
%Normally, we factor the patch before inversion; in this case,
%we factored the patch before abstraction. 
%This may be useful more generally; given a patch of the form \lstinline{g} $\circ$ \lstinline{f}, 
%it may make sense to abstract its factors separately, since abstracting \lstinline{f} changes the argument to \lstinline{g}.

%We were able to use the other three components without any changes.
%To accomplish this,

% TODO shorten above

\begin{figure*}[ht]
\begin{minipage}{0.48\textwidth}
\lstset{language=coq, aboveskip=0pt,belowskip=0pt}
\lstinputlisting[firstline=1, lastline=1]{repair/divide.tex}
\end{minipage}
\hfill
\begin{minipage}{0.48\textwidth}
\lstset{language=coq, aboveskip=0pt,belowskip=0pt}
\lstinputlisting[firstline=3, lastline=3]{repair/divide.tex}
\end{minipage}
\vspace{-.3cm}
%\setlength{\belowcaptionskip}{-4pt}
\caption[Caption for LOF]{Old (left) and new (right) definitions of \lstinline{divide} in Coq.}
\label{fig:divide}
\end{figure*}

\lstset{language=coq, aboveskip=3pt,belowskip=3pt}

%\subsection{Updating Language Versions\\\textit{\small{Coq Standard Library}}}
\subsection{Updating Proof Assistant Versions}
\label{sec:coq}

Coq sometimes makes changes to its standard library that break
backwards-compatibility. %This is an ideal use-case for finding proof patches: 
%When the client updates Coq, a tool can automatically search commits to the standard library or to \lstinline{coq-contribs}
%for patches.
%In this case, the example comes from the Coq developers, not from the library client---the 
%client never has to look at the changes that the Coq developers make.
To test the plausibility of using a patch finding tool for proof assistant version updates,
we identified a breaking change in the Coq standard library~\cite{coq84commit}.
The commit changed the definition of \lstinline{divide} prior to the Coq 8.4 release (Figure~\ref{fig:divide}).
%The change addressed a compatibility problem between natural numbers and integers
%in the standard library prior to the release of Coq 8.4.
The change broke 46 proofs in the standard library.
We used \sysname to find an isomorphism that corresponds to the change in \lstinline{divide}.
The isomorphism \sysname found fixed broken proofs.
%The adaptation to \sysname composed all five components without any changes to the components;
%we identified extensions to two components that may improve the algorithm further.

%\vspace{-0.1cm}
\paragraph{Replay} We used the proof of \lstinline{mod_divide} as the example for \sysname.
The proof broke with the new \lstinline{divide}:

\begin{lstlisting}[language=coq]
   Theorem mod_divide:(@\vspace{-0.04cm}@)
     (@\ltacforall@) a b, b~=0 -> (a mod b == 0 <-> (divide b a)).(@\vspace{-0.04cm}@)
   Proof.(@\vspace{-0.04cm}@)
     ... (@\fails{rewrite (div\_mod a b Hb) at 2.}@)
\end{lstlisting}

We replayed changes to \lstinline{mod_divide}:

\begin{lstlisting}[language=coq]
     ... (@\diff{rewrite mul\_comm. symmetry.}@)(@\vspace{-0.04cm}@)
     (@\succeeds{rewrite (div\_mod a b Hb) at 2.}@)
\end{lstlisting}

%\vspace{-0.1cm}
\paragraph{Search} We used \sysname to search the example for a patch
that corresponds to the change in \lstinline{divide}.
It found an isomorphism:

\begin{lstlisting}[language=coq]
    (@\ltacforall@)r p q, (@\diff{p * r = q}@) -> (@\diff{q = r * p}@)(@\vspace{-0.08cm}@)
    (@\ltacforall@)r p q, (@\diff{q = r * p}@) -> (@\diff{p * r = q}@)
\end{lstlisting}

%\vspace{-0.1cm}
\paragraph{Apply} The proof of the theorem \lstinline{Zmod_divides} broke after rewriting by the changed theorem \lstinline{mod_divide}:

\begin{lstlisting}[language=coq]
    Theorem Zmod_divides:(@\vspace{-0.04cm}@)
      (@\ltacforall@)a b, b<>0 -> (a mod b = 0 <-> (@\ltacexists@)c, a = b * c).(@\vspace{-0.04cm}@)
    Proof.(@\vspace{-0.04cm}@)
      ... split; intros (c,Hc); exists c; (@\fails{auto.}@)
\end{lstlisting}

%The \lstinline{exists c} tactic instantiated \lstinline{divide} with \lstinline{c}.
%Consequentially, 
Adding the patches \sysname found to a hint database made the proof go through:

\begin{lstlisting}[language=coq]
      ... split; intros (c,Hc); exists c; (@\succeeds{auto.}@)
\end{lstlisting}

% TODO, how many other proofs? Test

%\vspace{-0.1cm}
\subsubsection{Core Components}

This scenario used the configuration for changes in dependent arguments to constructors.
\sysname searches two proofs that apply the same constructor
to different dependent arguments:

\begin{lstlisting}[language=coq]
    ... (C ((@\diff{P}@) x)) ...(@\vspace{-0.1cm}@)
    ... (C ((@\diff{P'}@) x)) ...
\end{lstlisting}
for an isomorphism between the arguments:

\begin{lstlisting}[language=coq]
    (@\ltacforall@) x, (@\diff{P}@) x -> (@\diff{P'}@) x(@\vspace{-0.1cm}@)
    (@\ltacforall@) x, (@\diff{P'}@) x -> (@\diff{P}@) x
\end{lstlisting}

The user can apply these patches to patch proofs that apply the constructor (in this case study,
to fix broken proofs that instantiate \lstinline{divide} with some specific \lstinline{r}).

%This is a change in conclusions, but with one key difference:
So far, we have encountered changes of this form as arguments to an 
induction principle; in this case, the change is an argument to a constructor.
A patch between arguments to an induction principle maps
directly between conclusions of the new and old theorem without
induction; a patch between constructors does not.
%This means that the semantic differencing component must determine \emph{not} to attempt to produce a patch
%that maps conclusion to conclusion.
For example, for \lstinline{divide}, we can find a patch with this form:

\begin{lstlisting}[language=coq]
   (@\ltacforall@)x, (@\diff{P}@) x -> (@\diff{P'}@) x
\end{lstlisting}

However, without using the induction principle for \lstinline{exists}, we can't use that patch to prove this:

\begin{lstlisting}[language=coq]
   ((@\ltacexists@)x, (@\diff{P}@) x) -> ((@\ltacexists@) x, (@\diff{P'}@) x)
\end{lstlisting}

%The type \lstinline{divide} that changed states an existence property.
%The proof of \lstinline{mod_divide} proved the existence property
%using the constructor \lstinline{ex_intro}:

%\begin{lstlisting}[language=coq]
 %   Inductive ex (A : Type) (P : A -> Prop) : Prop :=
  %    ex_intro : (@\ltacforall@)x : A, P x -> (@\ltacexists@)x, P x
%\end{lstlisting}

%This constructor takes a property \lstinline{P} and a proof \lstinline{P x} for a specific \lstinline{x}.
%This means that if we find a patch between
%\lstinline{P x} and \lstinline{P' x}, we can
%try to abstract it to a patch between \lstinline{P} and \lstinline{P'}---
%just like we do when we find patches between inductive proofs.

%However, the similarity ends there. When we find such a patch between inductive proofs,
%it gets us directly between conclusions of the new and old theorem without any need
%for induction. In contrast, when we find a patch between arguments to a constructor,
%we cannot directly use this to map between all occurrences of the type. 

This changes the goal type that semantic differencing determines.
\sysname uses this configuration for changes in constructor arguments:

\begin{algorithm}
\footnotesize
\begin{algorithmic}[1]
%  Is it really necessary to say that an algorithm might do initialization?
%  \STATE Build trees for search
  %\STATE build trees for proofs
  %\REPEAT
    \STATE \textit{diff} constructor arguments for goals
    \STATE use \textit{all components} to recursively search those arguments for changes in conclusions
    \STATE \textbf{if} there are candidates \textbf{then}
    \STATE \hspace*{1em} \textit{abstract} the candidate \\
           \hspace*{1em} \textit{factor} and try to \textit{invert} the patch to find an isomorphism
\end{algorithmic}
\end{algorithm}
%Like in the first use case, we find this applied patch natural to use; in this case, too, we can 
%produce a more general patch by applying the induction principle.

For the prototype, the model of constructors for the semantic differencing component is limited,
so we ask the user to provide the type of the change in argument (to guide line 2).
We can extend semantic differencing to remove this restriction.

% improve one of the core components of \sysname to remove this restriction:
%We can extend semantic differencing so that it has a notion of constructors, and not just induction principles.
%%Then, when it encounters a difference of proofs through constructors, it will know how to determine the goal type
%and what argume where to search for the patch, what the goal is,
%and what arguments to abstract within the patch---much like semantic differencing currently does
%for proofs through induction principles.




