\section{Differencing}
\label{sec:pumpkin-diff}

% From Motivating the Core. Some of this might move into the Approach.

The tool should be able to identify the semantic difference between terms.
The semantic difference is the difference between two terms that corresponds to
the difference between their types. Consider the base case terms in Figure~\ref{fig:example} (line 13):

\begin{lstlisting}[language=coq]
  (@\diff{le\_plus\_trans n m 1}@) H : (@\diff{n <= m + 1}@)(@\vspace{-0.1cm}@)
  (@\phantom{le\_plus\_trans n m 1}@) H : (@\diff{n <= m}@)
\end{lstlisting}
The semantic differencing component first identifies the difference in their types, or the \textit{goal type}:

\begin{lstlisting}
  (@\diff{n <= m}@) -> (@\diff{n <= m + 1}@)
\end{lstlisting}
It then finds a difference in terms that has that type:

\begin{lstlisting}[language=coq]
  fun (H : n <= m) => (@\diff{le\_plus\_trans n m 1}@) H
\end{lstlisting}
This is the \emph{candidate} for a reusable patch that the other components modify to find a patch.

Differencing operates over terms and types. Differencing tactics
is insufficient, since tactics and hints may mask patches
(line 5).\footnote{Since this is a simple example, replaying an existing tactic happens to work. There
are additional examples in the repository (\lstinline{Cex.v}).}
Furthermore, differencing is aware of the semantics of terms and types.
Simply exploring the syntactic difference %between terms 
makes it hard to identify
which changes are meaningful.
For example, in the inductive case (line 14), the inductive hypothesis
changes:

\begin{lstlisting}[language=coq]
  ... (IHle : (@\diff{n <= m0 + 1}@)) ...(@\vspace{-0.08cm}@)
  ... (IHle : (@\diff{n <= m0}@)) ...
\end{lstlisting}

However, the type of \lstinline{IHle} changes for \emph{any} two inductive proofs over \lstinline{le}
with different conclusions. A syntactic differencing component 
may identify this change as a candidate.
Our semantic differencing component knows that it can ignore this change.

Plus parts of Inside the Core, Testing Boundaries, Future Work

How differencing works in detail

Limitations and whether they're addressed in later tools yet or not

