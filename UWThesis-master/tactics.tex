\subsection{Tactics: Ltac}
\label{sec:tactics}

In Coq, those tactics are in a language called Ltac.
Each tactic is effectively a search procedure for a proof term, given the context and goals at each step of the proof.

Tactics can be used to write simple proofs and functions too.
So if we'd like, for example, we could write \lstinline{length} using tactics:

\begin{lstlisting}
TODO
\end{lstlisting}
But it's much more common to use this for proofs, since controlling the details of the term can be hard (important for functions),
but for a proof it may be enough to just have any term with the correct type.

Here is one possible proof of \lstinline{zip_preserves_length} using very simple tactics:
\begin{lstlisting}
Lemma zip_length : zip_preserves_length.
Proof.
   intros a b l1.
   induction l1 as [|_ _ IHtl1]; auto.
   induction l2 as [|_ _ IHtl2]; intros H; auto.
   simpl. rewrite IHtl1; auto.
Qed.
\end{lstlisting}
This compiles to the thing we saw before, which then type checks as expected.
But we never have to see the low-level proof term; we can reason about the high-level proof script instead.
% TODO save the better version for later so that we can show how good automation can get around some of the repair stuff

Mention decompiler, explain, tease implementation section.

