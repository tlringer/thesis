\section{Motivating Example}
\label{sec:overview}

\begin{figure*}
\begin{minipage}{0.50\textwidth}
   \lstinputlisting[firstline=1, lastline=3]{often/listswap.tex}
\end{minipage}
\hfill
\begin{minipage}{0.49\textwidth}
   \lstinputlisting[firstline=5, lastline=7]{often/listswap.tex}
\end{minipage}
\caption{A change from the old version (left) to the new version (right) of \lstinline{list}.
Recall that \lstinline{list} is an inductive datatype that is either empty (the \lstinline{nil} constructor), or the result
of placing an element in front of another \lstinline{list} (the \lstinline{cons} constructor). The change swaps these
constructors (\codediff{orange}).}
\label{fig:listswap}
\end{figure*}

\begin{figure*}
\begin{lstlisting}
  Lemma rev_app_distr {A} :
    (@\ltacforall@) (x y : list A),
      rev (x ++ y) = rev y ++ rev x.
  Proof. (@\color{gray}{(* by induction over x and y *)}@)
    induction x as [| a l IHl].
    (@\color{gray}{(* x nil: *)}@) induction y as [| a l IHl].
    (@\color{gray}{(* y nil: *)}@) simpl. auto.
    (@\color{gray}{(* y cons *)}@) simpl. rewrite app_nil_r; auto.
    (@\color{gray}{(* both cons: *)}@) intro y. simpl.
    rewrite (IHl y). rewrite app_assoc; trivial.
  Defined.
\end{lstlisting}
\caption{The proof of the lemma \lstinline{rev_app_distr} from the Coq standard library. Comments mine for clarity.}
\label{fig:rev-app-distr}
\end{figure*}

Consider a simple example of using \toolnamec: repairing proofs after swapping the two constructors of the \lstinline{list} datatype (Figure~\ref{fig:listswap}).
This is inspired by a similar change from a user study of proof engineers (Section~\ref{sec:pi-results}).
Even such a simple change can cause trouble, as in the proof of the lemma \lstinline{rev_app_distr} from the
Coq standard library (Figure~\ref{fig:rev-app-distr}).
This lemma says that appending (\lstinline{++}) two lists and reversing (\lstinline{rev}) the result behaves the same as appending
the reverse of the second list onto the reverse of the first list.
The proof script works by induction over the input lists \lstinline{x} and \lstinline{y}:
In the base case for both \lstinline{x} and \lstinline{y}, the result holds by reflexivity.
In the base case for \lstinline{x} and the inductive case for \lstinline{y}, the result follows from the existing lemma \lstinline{app_nil_r}.
Finally, in the inductive case for both \lstinline{x} and \lstinline{y}, the result follows by the inductive hypothesis
and the existing lemma \lstinline{app_assoc}.

When we change the \lstinline{list} type, this proof no longer works.
%This theorem statement \lstinline{rev_app_distr} defined over the old version of \lstinline{list} is our \textit{old specification}.
%When we change the \lstinline{list} type, we get the \textit{new specification}.
%But the \textit{old proof} or tactic script no longer works with this new specification.
To repair this proof with \toolnamec, we run this command:

\begin{lstlisting}
  Repair Old.list New.list in rev_app_distr.
\end{lstlisting}
assuming the old and new list types from Figure~\ref{fig:listswap} are in modules \lstinline{Old} and \lstinline{New}.
This suggests a proof script that succeeds (in $\codeauto{light blue}$ to denote \toolnamec produces it automatically):

\begin{lstlisting}[backgroundcolor=\color{cyan!30}]
  Proof. (@\color{gray}{(* by induction over x and y *)}@)
    intros x. induction x as [a l IHl| ]; intro y0.
    - (@\color{gray}{(* both cons: *)}@) simpl. rewrite IHl. simpl.
      rewrite app_assoc. auto.
    - (@\color{gray}{(* x nil: *)}@) induction y0 as [a l H| ].
      + (@\color{gray}{(* y cons: *)}@) simpl. rewrite app_nil_r. auto.
      + (@\color{gray}{(* y nil: *)}@) auto.
  Defined.
\end{lstlisting}
where the dependencies (\lstinline{rev}, \lstinline{++}, \lstinline{app_assoc}, and \lstinline{app_nil_r}) have
also been updated automatically~\href{https://github.com/uwplse/pumpkin-pi/blob/v2.0.0/plugin/coq/Swap.v}{\circled{1}}. % Swap.v, TODO update links
If we would like, we can manually modify this to something that more closely matches the style of the original proof script:

\begin{lstlisting}
  Proof. (@\color{gray}{(* by induction over x and y *)}@)
    induction x as [a l IHl|].
    (@\color{gray}{(* both cons: *)}@) intro y. simpl.
    rewrite (IHl y). rewrite app_assoc; trivial.
    (@\color{gray}{(* x nil: *)}@) induction y as [a l IHl|].
    (@\color{gray}{(* y cons: *)}@) simpl. rewrite app_nil_r; auto.
    (@\color{gray}{(* y nil: *)}@) simpl. auto.
  Defined.
\end{lstlisting}
We can even repair the entire list module from the Coq standard library all at once by running the \lstinline{Repair module}
command~\href{https://github.com/uwplse/pumpkin-pi/blob/v2.0.0/plugin/coq/Swap.v}{\circled{1}}. % Swap.v
When we are done, we can get rid of \lstinline{Old.list}. % entirely.

The key to success is taking advantage of Coq's structured proof term language:
Recall that Coq compiles every \kl{proof script} to a \kl{proof term} in the rich functional programming language
%Gallina that is based on the calculus of inductive  constructions---\toolnamec repairs that term.
\kl{Gallina}---\toolnamec repairs that term.
\toolnamec then decompiles the repaired proof term (with optional hints from the original proof script) back 
to a suggested proof script that the proof engineer can maintain.
%Here, \toolnamec transforms the proof term Coq compiles \lstinline{rev_app_distr} to,
%and then decompiles that transformed proof term to the proof script in light blue.

In contrast, updating the poorly structured proof script directly would not be straightforward.
Even for the simple proof script above, grouping tactics by line, there are $6! = 720$ permutations of this proof script.
It is not clear which lines to swap since these tactics do not have a semantics beyond the searches their evaluation performs.
Furthermore, just swapping lines is not enough: even for such a simple change, we must also swap
arguments, so that:

\begin{lstlisting}
    induction x as [|(@\codediff{a l IHl}@)].
\end{lstlisting}
becomes:

\begin{lstlisting}
    induction x as [(@\codediff{a l IHl}@)|].
\end{lstlisting}
%Handling even swapping constructors this way would require a search procedure that would not generalize to other changes.
\kl{Valentin Robert}'s thesis~\cite{robert2018} describes the challenges of repairing tactics in detail.
\toolnamec's approach circumvents this challenge.

%\toolnamec's approach circumvents this challenge. % by transforming proof terms, then decompiling the transformed proof term to a tactic script.
%By decompiling the transformed proof term, \toolnamec is able to suggest a tactic script in the end.
%As later sections show, this approach is much more general than just permuting constructors.


