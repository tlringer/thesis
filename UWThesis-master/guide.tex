\section{Reading Guide}

% Dan: Instead, I strongly advocate an explicit section in Chapter 1 that lays this all out for the reader: what prior publications are being leaned on, where does text from those reappear, and most importantly, where -- in explicit sections of forthcoming chapters -- are expanded explanations and additional data.  Voila, your thesis is now both a self-contained coherent document and a useful 'appendix' for people who have already read large parts of what is to follow.

How to read this thesis.
Chapters and sections.

Expected reader background \& where to find more info.

Segue.

\subsubsection*{Map}

This thesis centers material from two previously published papers:

\begin{itemize}
\item \textbf{Talia Ringer}, Nathaniel Yazdani, John Leo, and Dan Grossman. Adapting Proof Automation to Adapt Proofs~\cite{ringer2018adapting}. CPP 2018.
\item \textbf{Talia Ringer}, RanDair Porter, Nathaniel Yazdani, John Leo, and Dan Grossman. Proof Repair Across Type Equivalences~\cite{Ringer2021}. PLDI 2021.
\end{itemize}
It sprinkles in material from three other papers:

\begin{itemize}
\item \textbf{Talia Ringer}, Nathaniel Yazdani, John Leo, and Dan Grossman. Ornaments for Proof Reuse in Coq~\cite{Ringer2019}. ITP 2019.
\item \textbf{Talia Ringer}, Alex Sanchez-Stern, Dan Grossman, and Sorin Lerner. \textsc{REPLica}: REPL Instrumentation for Coq Analysis~\cite{replica}. CPP 2020.
\item \textbf{Talia Ringer}, Karl Palmskog, Ilya Sergey, Milos Gligoric, and Zachary Tatlock. QED at Large: A Survey of Engineering of Formally Verified Software~\cite{PGL-045}. Foundations and Trends® in Programming Languages: Vol. 5: No. 2-3, pp 102-281. 2019. 
\end{itemize}
Below is a map from each of these papers to corresponding parts of the thesis,
along with an explanation of what is new in this thesis and what is omitted.
All of these papers can be found for free on my website.\footnote{\url{https://dependenttyp.es}}

\paragraph{Adapting Proof Automation to Adapt Proofs}
The bulk of Chapter~\ref{ch:example} comes from this paper,
though it is significantly reorganized and reframed.
In addition, the introducton and conclusion of Chapter~\ref{ch:example} are fresh content.
Sections~\ref{sec:pumpkin-approach}, \ref{sec:pumpkin-diff}, \ref{sec:pumpkin-trans}, and~\ref{sec:pumpkin-impl}
all include additions and elaborations not found in the original paper.
Chapter~\ref{sec:related} includes some related work from this paper,
and Chapter~\ref{chapt:conclusions} includes some future work from this paper.

\paragraph{Proof Repair Across Type Equivalences}
Parts of the introduction and Section~\ref{sec:mot-theory} come from this paper.
The bulk of Chapter~\ref{chapt:pi} comes from this paper,
though it is likewise reorganized and reframed.
In addition, the conclusion of Chapter~\ref{chapt:pi} is fresh content.
Sections~\ref{sec:pi-approach}, \ref{sec:pi-diff}, \ref{sec:pi-trans}, and~\ref{sec:pi-implementation}
all include additions and elaborations not found in the original paper.
Chapter~\ref{sec:related} includes some related work from this paper,
and Chapter~\ref{chapt:conclusions} includes some future work from this paper.

\paragraph{Ornaments for Proof Reuse in Coq}
The example from Section~\ref{sec:mot-dev} comes from this paper, though most of the text is new.
Parts of Section~\ref{sec:mot-theory} also come from this paper.
Section~\ref{sec:pi-diff} uses a simplified version of the search algorithm from this paper as an example.
Section~\ref{sec:eval} includes the evaluation from this paper with some additional context.
Chapter~\ref{sec:related} includes a small amount of related work from this paper.

\paragraph{\textsc{REPLica}}
Section~\ref{sec:irl} includes a few samples of this paper.

\paragraph{QED at Large}
Chapter~\ref{chapt:mot} includes a few samples of this paper.
Chapter~\ref{sec:related} includes a large amount of related work from this paper.

\subsubsection*{Authorship Statements}

Authorship statements for included paper materials, to credit coauthors. D
Description of who they are and how we collaborated.
Indexed for links: \intro{RanDair Porter}, \intro{Nathaniel Yazdani} (Nate),
\intro{Karl Palmskog}, and so on.

\subsubsection*{Pronouns}

In this thesis, I use ``I'' to refer to work that I did as part of my thesis work,
even though of course no work happens in a vacuum.
I use the names of my coauthors like ``\kl{Nate}'' or ``\kl{RanDair}'' when referring to work that my coauthors did,
when I was operating primarily in an advisory role.
When I collaborated with my coauthors, I name them and myself, like ``\kl{Nate} and I,''
and then (when not ambiguous) I use ``we'' thereafter.
Throughout, I also use mathematical ``we'' to mean both myself and the reader.

When I discuss a rhetorical proof engineer who does not actually exist,
like ``the proof engineer,'' I always use ``she''---this is a small attempt
to seed the world with data that counteracts stereotypes. 
When preserving anonymity of a particular person, I always use singular ``they.''
Otherwise, I use the pronouns that the person prefers.



