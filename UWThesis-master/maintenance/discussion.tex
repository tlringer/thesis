\section{Conclusions \& Future Work}
\label{sec:discussion}

We built \toolname, a Coq plugin that remotely collects fine-grained
proof development data in a way that is decoupled from the UI.
Using \toolname, we collected data on changes to proofs, programs,
and specifications at a level of granularity not previously
seen for an ITP.
Visualization and analysis of this data revealed evidence in our study
population of development patterns, at times confirming folk knowledge---like 
that discovering  the correct program or specification was often a
conversation with the proof itself, even for experts---and providing
useful insights and benchmarks for tools.

The infrastructure that we have built and the data that we have collected
using it are publicly available.
We hope to see it used as benchmarks for the improvement of
proof engineering tools, and as data for future studies.
We hope to see the infrastructure that we have built reused or
adapted to other ITPs for future studies.

\paragraph{Three Wishes}
We would like for our experiences building and using \toolname
to help the community conduct more studies of proof development processes.
We thus conclude with three wishes, the fulfillment of which would help
address the challenges that we encountered along the way:

\begin{enumerate}
\item Better abstraction of user environments (Section~\ref{sec:wish1})
\item More information about user interaction (Section~\ref{sec:wish2})
\item More users (Section~\ref{sec:wish3})
\end{enumerate}
We discuss how to grant each wish both at the level of study design
and at the level of ITP design in order to facilitate future studies of 
this kind.

\subsection{Wish: Better Abstraction of User Environments}
\label{sec:wish1}

In order to cast as broad of a net as possible for potential users, we
designed \toolname to be independent of the UI, 
the version of Coq, and the build system. Coq's plugin infrastructure
and interaction model offered a promising avenue for this,
%Together with the Coq developers, we were able to extend the plugin API
%with a hook to listen to Coq's REPL.
and Coq's resource file infrastructure meant that users could load the plugin
universally with just one LOC in one location.
All of this gave us the impression of independence from the details of
the UI, ITP version, and build system.

We later found that, while this infrastructure was useful,
the impression of total independence was somewhat of an illusion; 
all of these details mattered.
In particular, while \toolname itself was UI-independent, the analyses
that we ran did not achieve full independence.
For example, we found that depending on the version of Coq and what
event triggers a cancellation, different UIs use different mechanisms
for cancellation (recall these mechanisms from Section~\ref{sec:plugin}). 
Our analyses had to deal with all of these mechanisms.

In addition, partway through the study, we received a bug 
report that noted that one common build system for Coq compiles files by
default using a flag that disables loading the Coq resource file.
This is one possible explanation for the lack of data that some users sent.
To remedy this, we must ask future users of \toolname
to compile all of their Coq projects without using this flag;
we have updated the \toolname documentation to account for this. 

\paragraph{The Study Designer} Without help from the ITP designer,
the study designer cannot achieve full abstraction from these details.
The study designer may, however, work around the lack of full abstraction.
We recommend testing the study infrastructure with 
many different environments for many different scenarios.
It may help to identify potential users early and survey their
development environments before even beginning to test,
covering likely scenarios in advance.
It may also help to build in a short trial period on the final users
before the final data collection begins, thereby
covering their particular development environments.
The latter has the additional benefit of giving the study designer
time to discover and discuss with users possible confounding
variables for analysis, like development style or project phase.

We had the foresight to test \toolname with \lstinline{coqtop}, 
Proof General, and CoqIDE, but we did not anticipate User 5's custom
UI, which treated failures differently from the others.
We also did not anticipate the behavior of different
build systems on Coq's resource file, since we tested only one
build configuration.
We could have avoided both of these issues with our recommendations.
Instead, we had to work around both issues after deployment.

\paragraph{The ITP Designer} 
Most of the power to grant this wish is in the hands of the ITP designer.
Full abstraction over development environments may not be possible, but the
ITP designer may design the interaction model with this as a goal.
The REPL and state machine already strive for this---and come
close to achieving it---but their current implementations in Coq fall short.
Improvements to abstraction here may be minimal, like
providing fewer mechanisms for UIs to accomplish the same thing,
or providing a way to guarantee that a plugin is always loaded for all
build systems.

For a more radical approach, the ITP designer may look to other 
interaction models.
For example, Isabelle/HOL has recently moved away from its REPL,
instead favoring the Prover IDE (PIDE)~\cite{Wenzel2012} framework
as its interaction model.
With PIDE, UIs and ITPs communicate using an asynchronous protocol to manage
versions of a document~\cite{Wenzel2014}, annotating the document with 
new information when it becomes available~\cite{Wenzel2012}.
Personal communication suggests that there is an ongoing attempt to 
to centralize functionality like the build system into PIDE.
While centralization may limit the potential for customization by different
UIs, it may make studies of fine-grained proof development data easier,
since the instrumentation may occur at a higher level,
guaranteeing more uniform interaction.

\subsection{Wish: More Information about User Interaction}
\label{sec:wish2}

By observing state machine messages through the plugin API,
we were able to log data at high enough granularity to reconstruct
hundreds of interactive sessions. This helped us identify incremental
changes to tactics and terms that are difficult to gather from other sources.

The hooks that we designed together with the Coq developers, however, 
were not perfect.
There was no way for us to use those hooks to listen to the messages
that Coq sent back to the UI.
Making this change would have required modifying Coq again, and by the time
we realized this, it was too late.
Listening to those responses would have given us more insight into user
behavior.
It would have also helped us distinguish between failures and stepping up.
Instead, we had to distinguish between these using IDE-specific heuristics,
which left us with no automatic way to distinguish failures from stepping up
for User 5's custom UI.

We also found that once we had collected granular data, 
analyzing it was sometimes difficult. 
For example, sometimes users defined a term, stepped up and changed
an earlier term, then stepped back down and changed the original term.
For such a change, our visualization script constructed 3 consecutive commits;
to determine that the original term had changed, we had to look at 
the difference between the 1st and the 3rd commits.
Sometimes, changes occurred over tens of commits, or over several sessions.
Thus, manual analysis of changes for Q2 was tedious, and involved
not only inspecting thousands of diffs but also understanding each 
session well enough to track term information over time.

We considered automating the analysis for Q2, but we found
automatically classifying changes in terms to be prohibitively difficult.
While part of this was due to the inherent difficulty of the problem,
part of this was also due to missing information:
When the UI backed up to an earlier state, we did not know what was below 
the command or tactic corresponding to that state in the file. 
So, when a user copied and pasted a term and then modified both versions,
or made several changes at once to a term before stepping 
down below it, even manually deciding whether it was the same term 
that had changed or a new term entirely proved to be difficult.

\paragraph{The Study Designer}
To grant this wish, we recommend that the study designer run a beta test, complete with analysis, as we did---and that they do so early,
as we did not.
That way, there is enough time to address needs discovered during testing,
for example by communicating with ITP designers.
It may also help to use multiple methods for data collection,
for example by augmenting the collected data with information from the IDE
to track the state of definitions in the file below the location to
which the user has stepped.

The beta testing phase was extremely useful to us; as a result we
reworked our server infrastructure to receive and easily
analyze larger amounts of data, and we tested data backup
infrastructure and network failure resilience code. 
We also discovered and reported a 
bug\footnote{\url{http://github.com/coq/coq/issues/8989}} in CoqIDE and 
Proof General that had made it impossible for us to tell which 
sessions corresponded to which files; the fix was marked as critical 
and backported to Coq 8.9, so only the analyses for Users 5 (custom UI), 8 (Coq 8.8), and 10 (Coq 8.8) were impacted.
However, while we discovered the lack of response information in the plugin hooks
during this period, we did not have enough time to coordinate with Coq developers on
changes to the hooks. 
Leaving more time between testing and the final study would have helped us.

\paragraph{The ITP Designer} 
By implementing hooks like the one that we helped implement for Coq,
the ITP designer can make it easier for researchers to study 
proof development processes. 
To grant this wish, we recommend that these hooks expose not just 
request information, but also response information, 
especially error information.

Augmenting the interaction model with more information
about the file may also help.
For example,
the interaction model could expose a simple and uniform way for UIs
to track that a definition in a new state corresponds to a definition 
in a previous state (that is, that the user did not remove that
definition from the file and replace it with something entirely new). 
This could make it much simpler for an analysis to track changes to a definition over time,
and to choose the granularity with which to inspect changes.
There is some tension between this and the wish for abstraction from
Section~\ref{sec:wish1}, but it may be worth weighing the tradeoffs.

\subsection{Wish: More Users}
\label{sec:wish3}

We attempted to recruit a large and representative sample of Coq proof
engineers.
In our attempts to recruit users, we contacted programming
languages groups at several universities that were known to be working
with Coq, and emailed \lstinline{coq-club} with our project
pitch. We included a promotional video in the hopes of
attracting as many users as possible.

In the end, however, we were able to recruit only 12 users,
all of whom were intermediate through expert users with at least two years
of Coq experience. We received interactive sessions for just 8 of 12
of these users. While this data was rich and granular, with just 8 users,
we were not able to reach broad conclusions about proof engineers more generally.

Part of this was likely due to the demographics of the
community: Compared to other programming environments,
Coq has only a small number of users,
many of whom have or are pursuing graduate degrees.
However, part of this may have been due to
deterrents in our screening process, or poor incentives for our users.

\paragraph{The Study Designer}
The study designer may fulfill this in part by casting as broad of a net
as possible, carefully considering any deterrents in screening criteria.
It may help to use welcoming language to encourage potential users
who are unsure if they qualify.
To reach beginner users, it may help to recruit students.
To reach a more international audience, it may help to distribute
promotional materials and consent forms in multiple languages.
It may also help to consider incentives that appeal to proof engineers.
However, until the ITP user community grows significantly, it will continue 
to be difficult to conduct large-scale studies of proof engineering.

We reached users from different institutions, but
most of our users had similar levels of expertise. We suspect that this was
in part because we asked for users to have at least a year of 
experience using Coq, so as to avoid mixing in data from
users learning Coq for the first time.
We also required fluency in English to ensure that users understood
the consent forms.
Both of these may have deterred users.
One potential user who did not participate noted that we did not make
it clear in our recruitment materials that data from occasional rather
than frequent Coq users was still useful to us.
The same potential user noted that we could have engaged more
with community leaders, thereby giving others in the community
more incentive to participate.
We considered monetary incentives, but ultimately decided they were less
tempting to the community than appeals to improving tools.
Perhaps this was misguided or attracted a more advanced population,
and perhaps we could have considered a different incentive.

\paragraph{The ITP Designer}
The ITP designer may help reduce the costs of participating in these studies. 
We suspect that the primary gains here will come from continuing to 
break down barriers to using plugins and other outside tooling. 
Some examples of this include reducing the brittleness of plugin APIs 
over time, improving build and distribution systems, making it possible 
to prove that a plugin does not interact with a kernel in a way that 
could compromise soundness of the system, 
and providing more support for users who port proof developments
and tools between ITP versions.

Of course, continuing to improve the ITP itself may continue 
to expand the community to reach more users. 
This may cause a positive feedback loop, helping to collect more data 
in order to drive further improvements to the ITP, in turn continuing to help
the community grow.


