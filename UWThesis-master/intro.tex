\chapter{Introduction}

Motivation for verifying systems

Era of scale---enter proof engineering~\cite{PGL-045}.
The seL4~\cite{Klein2009} verified operating system kernel, for example,
is the effort of a team of proof engineers spanning more than
a million lines of proof, costing over 20 person-years.
Given a famous 1977 critique of verification~\cite{DeMillo1977} (emphasis ours):

\begin{quote}
\textit{A sufficiently fanatical researcher}
might be willing to devote \textit{two or 
three years} to verifying a significant 
piece of software if he could be 
assured that the software would remain stable.
\end{quote}
we could argue that, over 40 years, either verification has become easier,
or researchers have become more fanatical. Unfortunately, not all has changed (emphasis still ours):

\begin{quote}
But real-life programs need to 
be maintained and modified. 
There is \textit{no reason to believe} that verifying a modified program is any 
easier than verifying the original the 
first time around.
\end{quote}
To this day, maintaining verified programs can be challenging: it means keeping not just the programs, but also specifications and proofs about those programs up-to-date.
This remains so difficult that sometimes, even experts give up in the face of change~\cite{replica}. % TODO rather than cite self, refer to section?

But missed opportunity: automation doesn't understand that proofs evolve

So we build automation that does, and we call this proof repair.
Proof repair shows that there \textit{is} reason to believe that verifying a modified system should often, in practical use cases, be easier than verifying the original the first time around,
even when proof engineers do not follow good development processes,
or when change occurs outside of proof engineers' control~\cite{PGL-045}.

Or, in other words (thesis statement): Changes in programs, specifications, and proofs carry information that a tool can extract, generalize, and apply to fix other proofs broken by the same change. A tool that automates this can save work for proof engineers relative to reference manual repairs in practical use cases.

Key technical bit: differencing and program transformations, taking advantage of the rich and structured language proofs are written in.

We implement this in a tool suite for Coq, get some sweet results.

Pave path to the next era of verification

% Dan: It might be good either in 1 or in both 3 and 4 to lay out the key /results/ -- you have _design_ of the transformation, 
% the highly non-trivial _implementation_ in Coq in a way that doesn't extend the TCB, and you have substantial case studies to _evaluate_.
% You have a bit of formalism and metatheory, but that is more for exposition than for specific results (am I right?) -- you are 
% guided by the theory to build tools that actually work in this unforgiving domain -- "hacks won't work".

\chapter{Reading Guide}

% Dan: Instead, I strongly advocate an explicit section in Chapter 1 that lays this all out for the reader: what prior publications are being leaned on, where does text from those reappear, and most importantly, where -- in explicit sections of forthcoming chapters -- are expanded explanations and additional data.  Voila, your thesis is now both a self-contained coherent document and a useful 'appendix' for people who have already read large parts of what is to follow.

