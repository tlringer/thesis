\section{Introduction}

%Proof assistants allow programmers to verify everything from distributed systems~\cite{wilcox:verdi} and compilers~\cite{leroy:compcert} to
%formalizations of mathematics~\cite{DBLP:journals/corr/BauerGLSSS16}. 

Proof automation makes verification more accessible to programmers, but it is often intractable
without programmer guidance. In \textit{interactive theorem proving} (ITP), the programmer guides the 
proof search process. The guidance reduces the search space to make proof automation more tractable.
This in turns helps the programmer, who does not have to manually verify the entire system.

%Interactive theorem provers allow programmers to guide proof search and interactively 
%verify everything from distributed systems~\cite{wilcox:verdi} and compilers~\cite{leroy:compcert} to
%formalizations of mathematics~\cite{DBLP:journals/corr/BauerGLSSS16}. In many of these theorem provers, programmers use

%In these languages, programmers write precise specifications of programs
%and prove implementations correct with respect to the specifications.

Despite automation, programming in these proof assistants is brittle: Even a minor change to a definition or theorem
can break many dependent proofs. This is a major source of development inefficiency in proof assistants
based on dependent type theory~\cite{proof-eng, Aydemir2008, Delaware2013ICFP}.

Traditional proof automation does not consider how proofs, definitions, and theorems change over time.
Instead, it is driven by the state of the current proof, sometimes with supplementary
information from other proofs, definitions, and theorems (as in hint databases~\cite{hints} and rippling~\cite{rippling}).
This puts the burden of dealing with brittleness on the programmer. 

We present a new approach to proof automation that accounts for breaking changes. %:
%searching the difference in versions of a proof, definition, or theorem. 
%We reimagine the problem of proof reuse~\cite{Boite2004, Mulhern06proofweaving} 
%in the context of traditional automation for ITP. 
In our approach, the programmer guides proof search 
by providing an \textit{example} of how to adapt proofs to changes in definitions or theorems.
A tool then generalizes the example adaptation into a \textit{reusable patch} that the programmer can use to fix
other broken proofs.

In doing so, we chart a path for a future %of ITP 
that moves the burden of brittleness
away from the programmer and into proof automation. 
Programmers typically address brittleness through design principles that make proofs
resilient to change~\cite{proof-eng, Aydemir2008, Delaware2013ICFP}, or through program-specific proof automation~\cite{chlipala:cpdt}.
These techniques, while useful, have limitations: 
Planning a verification effort around future change is challenging, and program-specific automation requires specialized knowledge.
The programmer's ability to anticipate likely changes determines the robustness of both techniques in
the face of change. %The robustness of both techniques in the face of change is determined by the programmer's ability to anticipate likely changes. 
Even then, many breaking changes are outside of the programmer's control. Updating proof assistant
versions, for example, can break proofs regardless of planning or automation~\cite{verdicommit}.

\begin{figure*}[ht]
\begin{minipage}{0.48\textwidth}
\centering
\lstset{language=coq, aboveskip=0pt, belowskip=0pt}
\lstinputlisting[firstline=1, lastline=3]{repair/izr.tex}
\lstinputlisting[backgroundcolor=\color{orange!35},firstline=4,lastline=5]{repair/izr.tex}
\lstinputlisting[firstline=6, lastline=6]{repair/izr.tex}
\end{minipage}
\hfill
\begin{minipage}{0.48\textwidth}
\centering
\lstset{language=coq, aboveskip=0pt, belowskip=0pt}
\lstinputlisting[firstline=8, lastline=10]{repair/izr.tex}
\lstinputlisting[backgroundcolor=\color{orange!35},firstline=11,lastline=12]{repair/izr.tex}
\lstinputlisting[firstline=13, lastline=13]{repair/izr.tex}
\end{minipage}
\vspace{-.3cm}
%\setlength{\belowcaptionskip}{-2pt}
\caption[Caption]{Old (left) and new (right) definitions of \lstinline{IZR} in Coq.
The old definition applies injection from naturals to reals and conversion of positives to
naturals; the new definition applies injection from positives to reals.}
\label{fig:izr}
\end{figure*}


We identify a set of core components that are critical to searching an example for patches in Coq.
We use the components to build %four different 
a procedure for finding patches %for non-structural changes 
in a Coq plugin as a 
proof-of-concept.
Case studies on real projects like CompCert~\cite{leroy:compcert} and the Coq standard library
suggest that patches are useful for realistic scenarios, and that it is simple to compose the components 
to handle different classes of changes.
We test the boundaries of the components on a suite of tests
and show how our findings can help build the future we envision.

In summary, we contribute the following:

\begin{enumerate}
%\item We develop new foundations for proof automation in ITP that account for breaking changes. % TODO reword
\item We identify a set of core components that are key to searching for patches.
\item We demonstrate that these core components are useful on real changes in Coq code.
\item We explore how to drive a future that moves the burden of change in ITP away from the programmer and into automated tooling.
\end{enumerate}

% TODO fix contributions

% evaluation stuff in contributions too

% SOMEWHERE: we focus on non-structural changes



