\section{Proof Development}
\label{sec:mot-dev}

Before I motivate proof maintenance and repair, it helps to understand \intro{proof development} to begin with.
In the introduction, I briefly explained the workflow for using a \kl{proof assistant} to develop a verified system,
noting that the proof engineer:

\begin{enumerate}
\item implements a \kl{program},
\item \kl{specifies} what it means for the \kl{program} to be correct, and
\item \kl{proves} that the \kl{program} satisfies the \kl{specification}.
\end{enumerate}
In the \kl{Coq} proof assistant, proof engineers implement programs in a rich functional programming language \intro{Gallina}.
In fact, it is possile to use Gallina to write the program, the specification, \textit{and} the proof---but writing the proof in Gallina can be challenging.
Instead, proof engineers typically use Gallina to write only the program and specification,
and write the proof interactively.
I alluded to this when I explained the typical proof development workflow in Coq:

\begin{quote}
To write a proof, the proof engineer passes Coq high-level search procedures called \kl{tactics} (like \lstinline{induction}), and Coq responds to each tactic
by refining the current goal to some subgoal (like the goal for the base case). This loop of tactics and goals 
continues until no goals remain, at which point the proof engineer has constructed a sequence of tactics called a \kl{proof script}---the
high-level representation.
To check the proof, Coq compiles that proof script down to a low-level representation called a \kl{proof term},
then uses its \kl{kernel} to check that the proof term has the expected type.
If the term has the expected type, Coq lets the proof engineer know in the end.
\end{quote}
The low-level language of proof terms in Coq is \kl{Gallina}---the same rich functional programming language proof engineers use to write programs and specifications.
The high-level language of proof scripts in Coq is a language called \intro{Ltac} that I will soon describe.

In this thesis, I will not teach you all of Coq.\footnote{Good sources 
for learning more about Coq include the books Certified Programming with Dependent Types~\cite{chlipala:cpdt}
and Software Foundations~\cite{software-foundations}, and the survey paper \kl{QED at Large}.}
What I will do is motivate this workflow on an example (Section~\ref{sec:mot-workflow})
and explain the theory beneath (Section~\ref{sec:mot-theory}).

\subsection{The Workflow}
\label{sec:mot-workflow}

% TODO move this back to pattern matching!!! since eliminator stuff will come later
For a moment, let us assume some primitives from the Coq standard library: the type \lstinline{nat} of natural numbers,
the type \lstinline{list} of polymorphic lists, and the \lstinline{length} function that computes the length of a list as a natural number.
We start by writing the list \lstinline{zip} program, then specify what it means to preserve its length, and then finally
write an interactive proof that shows that specification actually holds.
Coq checks this proof and lets us now that our proof is correct, so our \lstinline{zip} function is verified.

\begin{figure}
\begin{lstlisting}
zip {T$_1$} {T$_2$} (l$_1$ : list T$_1$) (l$_2$ : list T$_2$) : list (T$_1$ * T$_2$) :=
  match l$_1$, l$_2$ with
  | nil, _ => nil
  | _, nil => nil
  | cons t$_1$ tl$_1$, cons t$_2$ tl$_2$ => cons (t$_1$, t$_2$) (zip tl$_1$ tl$_2$)
  end.
\end{lstlisting}
\caption{The list \lstinline{zip} function, taken from an existing tool~\cite{TODO}.} % TODO hs-to-coq cite
\label{fig:zip}
\end{figure}

\paragraph{Program} The list \lstinline{zip} function is in Figure~\ref{fig:zip}.
It takes as arguments two lists \lstinline{l}$_1$ and \lstinline{l}$_2$ of possibly different types \lstinline{T}$_1$ and \lstinline{T}$_2$, and zips them together into
a list of pairs \lstinline{(T}$_1$\lstinline{ * T}$_2$\lstinline{)}.
For example, if the input lists are: % TODO put this in a figure tbh, like from the slides

\begin{lstlisting}
(* [1; 2; 3; 4] *)
l$_1$ := cons 1 (cons 2 (cons 3 (cons 4 nil))).

(* ["x"; "y"; "z"] *)
l$_2$ := cons "x" (cons "y" (cons "z" nil)).
\end{lstlisting}
then \lstinline{zip} applied to those two lists returns:

\begin{lstlisting}
(* [(1, "x"); (2, "y"); (3, "z")] *)
cons (1, "x") (cons (2, "y") (cons (3, "z") nil)).
\end{lstlisting}
It is worth noting that the implementation of \lstinline{zip} has to make some decision with
what to do with the extra \lstinline{4} at the end---that is, how \lstinline{zip} behaves when the input
lists are different lengths.
The decision that this implementation makes is to just ignore those extra elements.

Otherwise, the implementation is fairly standard.
If \lstinline{l}$_1$ is \lstinline{nil} (first case),
or if \lstinline{l}$_2$ is \lstinline{nil} (second case), 
then \lstinline{zip} returns \lstinline{nil}.
Otherwise,
\lstinline{zip} combines the first two elements of each list into a pair (\lstinline{(t}$_1$\lstinline{, t}$_2$\lstinline{)}),
then sticks that in front of (using \lstinline{cons}) the result of recursively calling \lstinline{zip} on the tails of each list
(\lstinline{zip} \lstinline{tl}$_1$ \lstinline{tl}$_2$). % TODO a bit hard to understand probably, may need more explanation

\begin{figure}
\begin{minipage}{0.49\textwidth}
\begin{lstlisting}
Theorem zip_preserves_length :
  $\forall$ {T$_1$} {T$_2$} (l$_1$ : list T$_1$) (l$_2$ : list T$_2$),
    length l$_1$ = length l$_2$ $\rightarrow$
    length (zip l$_1$ l$_2$) = length l$_1$.
\end{lstlisting}
\end{minipage}
\hfill
\begin{minipage}{0.49\textwidth}
\begin{lstlisting}
Theorem zip_preserves_length :
  $\forall$ {T$_1$} {T$_2$} (l$_1$ : list T$_1$) (l$_2$ : list T$_2$),
    length (zip l$_1$ l$_2$) = min (length l$_1$) (length l$_2$).
\end{lstlisting}
\end{minipage}
\caption{Two possible specifications of a proof that \lstinline{zip} preserves the length of the input lists.}
\label{fig:zip-pres}
\end{figure}

\paragraph{Specification} Once we have written our \lstinline{zip} function, we can then specify what we 
want to prove about it: that the \lstinline{zip} function preserves the lengths of the inputs \lstinline{l}$_1$ and \lstinline{l}$_2$.
We do this by defining a type \lstinline{zip_preserves_length} (Figure~\ref{fig:zip-pres}, left), which in Coq we state as a \lstinline{Theorem}.
This theorem takes advantage of \kl{Gallina}'s rich type system to quantify over all possible input lists \lstinline{l}$_1$ and \lstinline{l}$_2$.
It says that if the lengths of the inputs are the same, then the length of the output is the same as the lengths of the inputs.
Our proof will soon show that this type is inhabited, and so this statement is true.

It is worth noting that this step of choosing a specification is a bit of an art---we have some freedom when we choose our specification.
We could just as well have chosen a different version of \lstinline{zip_preserves_length} (Figure~\ref{fig:zip-pres}, right)
that states that the length of the output is the \textit{minimum} of the lengths of the inputs (using \lstinline{min} from the Coq standard library).
This is also true for our \lstinline{zip} implementation, and in fact it is stronger---it implies the original theorem as well.
But regardless of which version we choose, we then get to the fun part of actually writing our proof.

\paragraph{Proof} As I mentioned earlier, it is possible to write proofs directly in \kl{Gallina}---but this can be difficult.
Instead, it is more common to write proofs interactively using the tactic language \kl{Ltac}.
Each tactic in \kl{Ltac} is effectively a search procedure for a proof term, given the context and goals at each step of the proof.
The way that this works is, after we state the theorem that we want to prove:

\begin{lstlisting}
Theorem zip_preserves_length :
  $\forall$ {T$_1$} {T$_2$} (l$_1$ : list T$_1$) (l$_2$ : list T$_2$),
    length l$_1$ = length l$_2$ $\rightarrow$
    length (zip l$_1$ l$_2$) = length l$_1$.
\end{lstlisting}
we then add one more word:

\begin{lstlisting}
Proof.
\end{lstlisting}
then step down past that word inside of an IDE. % TODO expand acronym? or something?
The IDE then drops into an interactive proof mode.
In that proof mode, it tracks the context of the proof so far, along with the goal we want to prove.
After each tactic we type and step past, Coq responds by refining the goal into some subgoal
and updating the context.
We continue this until no goals remain.
The survey paper~\cite{PGL-045} has a good overview of the tactic language in Coq and in other proof assistants,
plus different interfaces and IDEs for writing proofs interactively and screenshots of them in action.

In this case, after stepping past \lstinline{Proof} in our IDE, 
our initial context (above the line) is empty, and our initial goal (below the line) is the original theorem:

\begin{lstlisting}
______________________________________(1/1)
$\forall$ {T$_1$} {T$_2$} (l$_1$ : list T$_1$) (l$_2$ : list T$_2$),
  length l$_1$ = length l$_2$ $\rightarrow$
  length (zip l$_1$ l$_2$) = length l$_1$.
\end{lstlisting}
We can start this proof with the introduction tactic \lstinline{intros}:

\begin{lstlisting}
  intros T$_1$ T$_2$ l$_1$.
\end{lstlisting}
This is essentially the equivalent of the natural language proof strategy ``assume arbitrary \lstinline{T}$_1$, \lstinline{T}$_2$, and \lstinline{l}$_1$.''
That is, it moves the universally quantified arguments from our goal into our context:

\begin{lstlisting}
T$_1$ : Type
T$_2$ : Type
l$_1$ : list T$_1$
______________________________________(1/1)
$\forall$ (l$_2$ : list T$_2$),
  length l$_1$ = length l$_2$ $\rightarrow$
  length (zip l$_1$ l$_2$) = length l$_1$.
\end{lstlisting}
From this state, we can induct over the input list (choosing names for variables Coq introduces in the inductive case):

\begin{lstlisting}
  induction l$_1$ as [|t$_1$ tl$_1$ IHtl$_1$].
\end{lstlisting}
This breaks into two subgoals and subcontexts: one for the base case and one for the inductive case.
The base case:
\begin{lstlisting}
T$_1$ : Type
T$_2$ : Type
______________________________________(1/2)
$\forall$ l$_2$ : list T$_2$,
  length nil = length l$_3$ $\rightarrow$
  length (zip nil l$_2$) = length nil.
\end{lstlisting}
holds by reflexivity, which the \lstinline{auto} tactic takes care of.

In the inductive case:

\begin{lstlisting}
T$_1$ : Type
T$_2$ : Type
t$_1$ : T$_1$
tl$_1$ : list T$_1$
IHtl$_1$ : $\forall$ l$_2$ : list T$_2$,
        length tl$_1$ = length l$_2$ $\rightarrow$
        length (zip tl$_1$ l$_2$) = length tl$_1$
______________________________________(2/2)
$\forall$ l$_2$ : list T2,
  length (cons t$_1$ tl$_1$) = length l$_2$ $\rightarrow$
  length (zip (cons t$_1$ tl$_1$) l$_2$) = length (cons t$_1$ tl$_1$).
\end{lstlisting}
we again use \lstinline{intros} and \lstinline{induction}, this time to induct over \lstinline{l}$_2$.
This again produces two subgoals: one for the base case and one for the inductive case.
The base case has an absurd hypothesis, which we introduce as \lstinline{H} and then use \lstinline{auto} to show our conclusion holds.
The inductive case holds by simplification and rewriting by the inductive hypothesis \lstinline{IHtl}$_1$.

After this, no goals remain, so our proof is done; we can write \lstinline{Defined}.
What happens when we write \lstinline{Defined} is that Coq compiles that proof script down to a proof term.
It then checks the type of that term and ensures that it is exactly the theorem we have stated.
Since it is, Coq lets us know that our proof is correct, so our \lstinline{zip} function is verified.

\begin{figure}
\begin{lstlisting}
Theorem zip_preserves_length :
  $\forall$ {T$_1$} {T$_2$} (l$_1$ : list T$_1$) (l$_2$ : list T$_2$),
    length l$_1$ = length l$_2$ $\rightarrow$
    length (zip l$_1$ l$_2$) = length l$_1$.
Proof.
  intros T1 T2 l1. (@\codesimb{induction l$_1$}@) as [|t$_1$ tl$_1$ IHtl$_1$].
  - (@\codesima{auto.}@)
  - intros l2. (@\codesimc{induction l2}@) as [|t$_2$ tl$_2$ IHtl$_2$].
    + (@\codesimd{intros H. auto.}@)
    + (@\codesime{intros H. simpl. rewrite IHtl$_1$; auto.}@)
Defined.
\end{lstlisting}
\begin{lstlisting}
zip_preserves_length :
  $\forall$ {T$_1$} {T$_2$} (l$_1$ : list T$_1$) (l$_2$ : list T$_2$),
    length l$_1$ = length l$_2$ $\rightarrow$
    length (zip l$_1$ l$_2$) = length l$_1$
:=
fun (T$_1$ T$_2$ : Type) (l$_1$ : list T$_1$) (l$_2$ : list T$_2$) =>
  (@\codesimb{list\_rect}@) (fun (l$_1$ : list T$_1$) => $\ldots$)
    (@\codesima{(fun (l$_2$ : list T$_2$) \_ => eq\_refl)}@)
    (fun (t$_1$ : T$_1$) (tl$_1$ : list T$_1$) (IHtl$_1$ : $\ldots$) (l$_2$ : list T$_2$) =>
      (@\codesimc{list\_rect}@) (fun (l$_2$ : list T$_2$) => $\ldots$)
        (@\codesimd{(fun (H : $\ldots$) => eq\_sym H)}@)
        (fun (t$_2$ : T$_2$) (tl$_2$ : list T$_2$) (IHtl$_2$ : $\ldots$) =>
          (@\codesime{fun (H : $\ldots$) => eq\_ind\_r $\ldots$ eq\_refl (IHtl$_1$ $\ldots$)}@))
        (@\codesimc{l$_2$}@))
  (@\codesimb{l$_1$}@)
  l$_2$.
\end{lstlisting}
\caption{A proof script (top) and corresponding proof term (bottom) in Coq that shows that the list zip function preserves its length.
Some details of the proof term are omitted for simplicity.
Corresponding parts of the proof are highlighted in the same color and annotated with the same number; the rest is boilerplate.}
\label{fig:zip-proof}
\end{figure}

Figure~\ref{fig:zip-proof} shows the resulting proof script for this theorem (top), along with the corresponding proof term (bottom).
As we can see, the proof term is quite complicated---I will explain what it means soon, in Section~\ref{sec:mot-theory}.
The important thing to note for now is that  the details of this low-leve proof term do not matter to us,
since we can write the high-level proof script on the top instead.
Even though this proof script is still a bit manual for the sake of demonstration,
it is much simpler than the low-level proof term.

Still, there is a correspondence between the proof script and the proof term, highlighted in the same color in Figure~\ref{fig:zip-proof}.
For example, every call to \lstinline{induction} in the proof script shows up as an application of a function \lstinline{list_rect},
which is the induction principle for \lstinline{list}.
In Section~\ref{TODO}, I will explain this connection in more detail by introducing a prototype decompiler from proof terms back up to proof scripts.
This decompiler makes it possible for \sysname to work over highly structured \kl{Gallina} terms,
but produce proof scripts that the proof engineer can use going forward.

Writing proofs using tactics does indeed proof development easier than writing raw proof terms.
But these highly structured proof terms carry a lot of information that is lost at the level of tactics.
It is exactly that rich structure---the type theory beneath \kl{Gallina}---that makes a principled approach to proof repair possible.







\subsection{The Theory Beneath}
\label{sec:mot-theory}

Now that we have written a small proof development in Coq, let us take a step back and look at the theoretical foundations that make this possible.
While proof scripts help humans like us write proofs, it is thanks to the proof terms these compile down to that Coq is able to check our proofs for us.
These proof terms are in the rich functional programming language Gallina (Section~\ref{sec:gallina}).
Gallina implements a rich type theory called the \intro{Calculus of Inductive Constructions} (Section~\ref{sec:cic}).
This rich type theory makes it possible to write programs, specifications, and proofs in Coq,
and have a small part of Coq---the \kl{kernel}---check those proofs in the end.

\subsubsection{Gallina}
\label{sec:gallina}

Coq's \kl{proof term} language \kl{Gallina} is a rich functional programming language with support for writing proofs.
To see the power of Gallina, let us dissect our proof that \lstinline{zip} preserves its length.

Our proof development about \lstinline{zip} uses the \lstinline{nat} and \lstinline{list} datatypes,
as well as the \lstinline{length} function.
It also uses the equality type \lstinline{=}.
All of these can be found inside of the Coq standard library.
But dissecting them already points to many important features of Gallina:
\intro{inductive types} and \intro{intensionality}.
All of these arise in our proof development, and will continue to arise throughout this thesis.

\paragraph{Inductive Types}

Each of \lstinline{nat} and \lstinline{list} in Gallina is an \kl{inductive type}:
it is defined by its \intro{constructors} (ways of constructing a term with that type).
A \lstinline{nat} (Figure~\ref{fig:nat}, left), for example,
is either \lstinline{0} or the successor \lstinline{S} of another \lstinline{nat};
these are the two constructors of \lstinline{nat}.

\begin{figure}
\begin{minipage}{0.30\textwidth}
\begin{lstlisting}
Inductive nat :=
| (@\codesimb{O : nat}@)
| (@\codesima{S : nat $\rightarrow$ nat}@).
\end{lstlisting}
\end{minipage}
\hfill
\begin{minipage}{0.68\textwidth}
\begin{lstlisting}
nat_rect :
  (@\ltacforall@) (P : nat $\rightarrow$ Type),
    (@\codesimb{P O}@) $\rightarrow$
    (@\codesima{(\ltacforall (n : nat), P n $\rightarrow$ P (S n))}@) $\rightarrow$
    (@\ltacforall@) (n : nat), P n.
\end{lstlisting}
\end{minipage}
\caption{The type of natural numbers \lstinline{nat} in Coq defined inductively by its two \kl{constructors} (left), and the type of the corresponding \kl{eliminator} or induction principle \lstinline{nat_rect} that Coq generates (right).}
\label{fig:nat}
\end{figure}

Every inductive type in Gallina comes equipped with an \intro{eliminator} (also called an induction principle)
that the proof engineer can use to write functions and proofs about the datatype.
For example, the eliminator for \lstinline{nat} (Figure~\ref{fig:nat}, right) is the standard induction principle for natural numbers,
which Coq calls \lstinline{nat_rect}.\footnote{For technical reasons that are not important to this thesis,
Coq actually defines \textit{three} eliminators: \lstinline{nat_rect}, \lstinline{nat_rec}, and \lstinline{nat_ind}.
For the sake of this thesis, it is sufficient to imagine that these are all the same.}
This eliminator states that a statement \lstinline{P} (called the inductive \intro{motive}) about the natural numbers
holds for every number if it holds for \lstinline{O} in the base case and, in the inductive case,
assuming the \intro{inductive hypothesis} that it holds for some \lstinline{n}, it also holds for the successor \lstinline{S n}.

\begin{figure}
\begin{minipage}{0.44\textwidth}
\begin{lstlisting}
Inductive list {T : Type} :=
| (@\codesimb{nil : list T}@)
| (@\codesimans{cons :}@)(@\vspace{-0.04cm}@)
    (@\codesima{T $\rightarrow$ list T $\rightarrow$ list T.}@)
\end{lstlisting}
\end{minipage}
\hfill
\begin{minipage}{0.55\textwidth}
\begin{lstlisting}
list_rect {T : Type} :
  (@\ltacforall@) (P : list T $\rightarrow$ Type),
    (@\codesimb{P nil}@) $\rightarrow$
    (@\codesimans{((\ltacforall (t : T) (tl : list T),}@)(@\vspace{-0.04cm}@)
      (@\codesima{P tl $\rightarrow$ P (cons t tl))}@) $\rightarrow$
    (@\ltacforall@) (l : list T), P l.
\end{lstlisting}
\end{minipage}
\caption{The type of polymorphic lists \lstinline{list} in Coq defined inductively by its two \kl{constructors} (left), and the type of the corresponding \kl{eliminator} or induction principle \lstinline{list_rect} that Coq generates (right).}
\label{fig:list}
\end{figure}

 % TODO footnote about nat_ind, nat_rec, nat_rect

A \lstinline{list} (Figure~\ref{fig:list}, left) is similar to a \lstinline{nat}, 
but with two differences: \lstinline{list} is polymorphic over some type \lstinline{T} (so we can have a list of natural numbers,
for example, written \lstinline{list nat}), and the second constructor adds a new element of the type \lstinline{T} to the front of the list.
Otherwise, \lstinline{list} also has two constructors, \lstinline{nil} and \lstinline{cons}, where \lstinline{nil} represents the empty list,
and \lstinline{cons} sticks a new element in front of any existing list.
Similarly, the eliminator for \lstinline{list} (Figure~\ref{fig:list}, right) looks like the eliminator for \lstinline{nat} 
but with an argument corresponding to the parameter \lstinline{T} over which \lstinline{list} is polymorphic,
and with an additional argument corresponding to the new element in the inductive case.

One interesting thing about the types of these eliminators \lstinline{list_rect} and \lstinline{nat_rect}
include universal quantification over all inputs, written \ltacforall.
Gallina's type system is expressive enough to include universal quantification over inputs, as we will soon see.

\begin{figure}
\begin{minipage}{0.42\textwidth}
\begin{lstlisting}
Fixpoint length {T} l :=
  match l with
  | (@\codesimb{nil => O}@)
  | (@\codesimans{cons t tl =>}@)(@\vspace{-0.04cm}@)
      (@\codesima{S (length tl)}@)
  end.
\end{lstlisting}
\end{minipage}
\hfill
\begin{minipage}{0.54\textwidth}
\begin{lstlisting}
Definition length {T} l :=
  list_rect T (fun _ => nat)
    (@\codesimb{O}@)
    (@\codesimans{(fun t tl (length\_tl : nat) =>}@)(@\vspace{-0.04cm}@)
      (@\codesima{S length\_tl)}@)
    l.
\end{lstlisting}
\end{minipage}
\caption{The list \lstinline{length} function, defined by pattern matching and recursion (left) and using the eliminator \lstinline{list_rect} (right).}
\label{fig:length}
\end{figure}

We can use these eliminators to write not just proofs, but also functions, like the \lstinline{length} function (Figure~\ref{fig:length}, right).
For functions, though, it is more standard to instead use pattern matching and recursion with a 
syntactic guard condition,\footnote{Coq does not support arbitrary recursion. The syntactic guard on recursion forces recursive functions in Coq to terminate. \kl{QED at Large} explains ways to reason about possibly nonterminating functions in spite of this.}
like the \lstinline{length} function from the Coq standard library (Figure~\ref{fig:length}, left).
Both of these functions behave the same way, but the function on the left is perhaps a bit easier to understand from a traditional programming background:
the \lstinline{length} of the empty list \lstinline{nil} is \lstinline{0}, and the length of any other list
is the successor (\lstinline{S}) of the result of recursively calling \lstinline{length} on everything but the first element of the list.
Indeed, \lstinline{list_rect}---like all eliminators in Coq---is a constant that refers to a function itself defined using pattern matching and recursion
with a syntactic guard condition.
In fact, eliminators are equally expressive to pattern matching and recursion with a syntactic guard~\cite{recursion-elimination, jesper}.\footnote{What may be confusing, though, is that \textit{Coq's} automatically 
generated eliminators are \textit{not} as expressive Coq's pattern matching and recursion, though it \textit{is} possible to manually define sufficiently expressive eliminators in Coq. This is an artifact of Coq's history.}

\begin{figure}
\begin{lstlisting}
zip {T$_1$} {T$_2$} (l$_1$ : list T$_1$) (l$_2$ : list T$_2$) : list (T$_1$ * T$_2$) :=
  list_rect (fun _ : list T$_1$ => list T$_2$ $\rightarrow$ list (T$_1$ * T$_2$))
    (fun _ => nil)
    (fun t$_1$ tl$_1$ (zip_tl$_1$ : list T$_2$ $\rightarrow$ list (T$_1$ * T$_2$)) l$_2$ =>
      list_rect (fun _ : list T$_2$ => list (T$_1$ * T$_2$))
        nil
        (fun t$_2$ tl$_2$ (_ : list (T$_1$ * T$_2$)) =>
          cons (t$_1$, t$_2$) (zip_tl$_1$ tl$_2$))
        l$_2$)
    l$_1$
    l$_2$.
\end{lstlisting}
\caption{The list \lstinline{zip} function from Figure~\ref{fig:zip}, translated to use eliminators.}
\label{fig:zip-elim}
\end{figure}

For the sake of this thesis, however, I will assume \intro{primitive eliminators}: eliminators that are a part of the core syntax and theory itself,
rather than being defined by way of pattern matching and recursion.
Likewise, when I show Gallina code, from this point forward, I will favor functions that apply eliminators rather than pattern matching, like the \lstinline{length} function from Figure~\ref{fig:length}
on the right.
%I remove the \lstinline{Definition} and \lstinline{Fixpoint} keywords, since everything from here on out is a \lstinline{Definition}.

To handle practical code that uses pattern matching and recursion,
I preprocesss the code using a tool by \kl{Nate}.
This preprocessing tool transforms functions that use simple pattern matching and recursion
to instead use Coq's automatically generated eliminators (Section~\ref{sec:pi-details-trans}).
Figure~\ref{fig:zip-elim}, for example, shows the \lstinline{zip} function after running the preprocessing tool.
In the rest of this thesis, I skip the step of running this preprocessing tool in examples,
though the corresponding code invokes it explicitly.

\paragraph{Intensionality}

Gallina is based on what is called an \kl{intensional} type theory:
a type theory that distinguishes between equalities that hold by reducton (\intro{definitional equality}), and those that hold by proof (\intro{propositional} equality).
That is, two terms \lstinline{t} and \lstinline{t'} of type \lstinline{T} are definitionally equal if they
reduce\footnote{Reduction uses a sequence of predefined reduction rules, which are described by various Greek letters. $\beta$-reducton, for example,
has the standard meaning. $\delta$-reduction unfolds constants. Chapter 10 of Certified Programming with Dependent Types~\cite{chlipala:cpdt} includes a nice summary of reduction and its relationship to \kl{definitional equality}.} 
to the same normal form,
and propositionally equal if there is a proof that \lstinline{t = t'} using the inductive
equality type \lstinline{=} at type \lstinline{T}. Definitionally equal terms are necessarily propositionally equal, but 
the converse is not in general true. % TODO this should probably go into an earlier section

The inductive equality type also shows up inside of our proof development---that is what the \lstinline{=} sign in our theorem statement means.
The \lstinline{=} type has exactly one constructor, \lstinline{eq_refl} (reflexivity), which can be applied to two terms exactly when those terms are definitionally equal.
Propositional equality is more general, though, because it comes equipped with an eliminator \lstinline{eq_rect} (or the reverse, \lstinline{eq_rect_r})
that encodes rewriting: if \lstinline{t = t'}, and some motive \lstinline{P} holds on \lstinline{t}, then the motive must also hold on \lstinline{t'}.
It is possible to use this eliminator to prove equalities by sequences of rewrites, substituting equal terms for one another until the goal holds by reflexivity.

\paragraph{Back to our Proof}

With that in mind, we can now look back at the proof script and corresponding proof term in Figure~\ref{fig:zip-proof}.
There is a correspondence between the proof script and the proof term, highlighted in the same color:
The proof term is a function from the context to a body that proves the goal type.
Every call to \lstinline{induction} in the proof script shows up as an application of the \lstinline{list} eliminator 
\lstinline{list_rect}, with the cases corresponding to the appropriate arguments of the eliminator.
The first call to \lstinline{auto} compiles down to \lstinline{eq_refl}, the constructor for the inductive equality type.
The second call to \lstinline{auto} compiles down to \lstinline{eq_sym}, the proof of symmetry of equality in the Coq standard library. % TODO not using absurd then?
Rewrites compile down to applications of \lstinline{eq_rect_r}, an eliminator over the inductive equality type.

Still, the proof terms can be difficult for a proof engineer to write and understand.
In Section~\ref{sec:pi-implementation}, I will introduce a prototype decompiler by \kl{RanDair} from proof terms back up to proof scripts.
This decompiler will make it possible for \sysnamelong to work over highly structured Gallina terms,
but produce Ltac proof scripts that the proof engineer can use going forward.

\subsubsection{Calculus of Inductive Constructions}
\label{sec:cic}

\begin{figure*}
\small
\begin{grammar}
<i> $\in \mathbbm{N}$, <v> $\in$ Vars, <s> $\in$ \{ Prop, Set, Type<i> \}

<t> ::= <v>\\
| <s> \\
| $\Pi$ (<v> : <t>) . <t>\\
| $\lambda$ (<v> : <t>) . <t>\\
|  <t> <t>
\end{grammar}
\caption{Syntax for CoC$_\omega$ with (from top to bottom) variables, sorts, function types, functions, and application.}
\label{fig:coc-syntax}
\end{figure*}

The type theory that Gallina implements is \intro{CIC$_{\omega}$}, or the \kl{Calculus of Inductive Constructions}.
CIC$_{\omega}$ is based on the \intro{Calculus of Constructions} (CoC), a variant of the lambda calculus with two kinds of universally quantified function types: \intro{polymorphism} (types that dependent on types) and \intro{dependent types} (types that depend on terms)~\cite{coquand:inria-00076024}.
CoC with an infinite universe hieararchy---basically a trick for consistency---is 
called CoC$_{\omega}$.\footnote{The need for the infinite universe hiearchy is too distracting for me to explain in detail here, but it is also not important for this thesis.}
The syntax for CoC$_{\omega}$ is in Figure~\ref{fig:coc-syntax}.
Note that whereas in Gallina we represent universal quantification over terms or types with \ltacforall, here we represent it with $\Pi$, as is standard.

\begin{figure*}
\small
\begin{grammar}
<t> ::= \ldots\\
| Ind (<v> : <t>)\{<t>,\ldots,<t>\} \\
| Constr (<i>, <t>)\\
| Elim(<t>, <t>)\{<t>,\ldots,<t>\}
\end{grammar}
\caption{CIC$_\omega$ is CoC$_\omega$ with \kl{inductive types}, inductive \kl{constructors}, and \kl{primitive eliminators}.}
\label{fig:cic-syntax}
\end{figure*}

CIC$_{\omega}$ extends \kl{CoC$_{\omega}$} with \kl{inductive types}~\cite{inductive};
the syntax for CIC$_{\omega}$ (building on syntax from an existing paper~\cite{Timany2015FirstST}) is in Figure~\ref{fig:cic-syntax},
and the typing rules are standard and omitted.
As in Gallina, inductive types are defined by their constructors and eliminators.
Consider the inductive type \lstinline{nat} of unary natural numbers that we saw in Figure~\ref{fig:nat},
this time in CIC$_{\omega}$: % TODO explain sorts

\begin{lstlisting}
  Ind (nat : Set) {
    (@\codesimb{nat}@),
    (@\codesima{nat $\rightarrow$ nat}@)
  }
\end{lstlisting}
where the \lstinline{O} constructor type is the zeroth element in the list, and the \lstinline{S} constructor type is the first element.
Accordingly, the terms:

\begin{lstlisting}
  (@\codesimb{Constr (0, nat)}@)
\end{lstlisting}
and:

\begin{lstlisting}
  (@\codesima{Constr (1, nat)}@)
\end{lstlisting}
refer to the constructors \lstinline{O} and \lstinline{S}, respectively.

As in Gallina, inductive types like \lstinline{nat} come associated with \kl{eliminators}.
Unlike in Gallina, here we truly assume \kl{primitive eliminators}---that these eliminators
do not reduce at all.
Instead, we represent them explicitly with the \lstinline{Elim} construct.
Thus, to eliminate over a natural number \lstinline{n} with motive \lstinline{P},
we write: % TODO check

\begin{lstlisting}
  Elim (n, P) {
    (@\codesimb{f$_0$}@),
    (@\codesima{f$_1$}@)
  }
\end{lstlisting}
where functions:

\begin{lstlisting}
  (@\codesimb{f$_0$ : P (Constr (0, nat))}@)
\end{lstlisting}
and:

\begin{lstlisting}
  (@\codesima{f$_1$ : $\Pi$ (n : nat) (IHn : P n) . P ((Constr (1, nat)) n)}@)
\end{lstlisting}
prove the base and inductive cases, respectively.
When \lstinline{n}, \lstinline{P}, \lstinline{f}$_0$, and \lstinline{f}$_1$ are arbitrary,
this statement has the same type as \lstinline{nat_rect} in Gallina.

% TODO use knowledge below to index this stuff:
\paragraph{Conventions}
Throughout this thesis, I assume an inductive \kl{propositional} equality type \lstinline{=} with constructor \lstinline{eq_refl}
and eliminator \lstinline{eq_rect}.
I also assume an inductive type \intro{$\Sigma$} for existential or refinement types,
with constructor $\exists$ and projections $\pi_l$ and $\pi_r$.
For simplicity of presentation, when I write terms in CIC$_{\omega}$, I assume variables are names, and that all names are fresh.

I often present \kl{semantic differencing} algorithms and \kl{proof term transformations} over CIC$_{\omega}$ relationally, using a set of judgments; % TODO maybe move this up too
to turn these relations into algorithms, prioritize the rules by running the derivations in
order, falling back to the original term when no rules match.
I use $\vec{t}$ and $\{t_1, \ldots, t_n\}$ to denote lists of terms;
the default derivation for a list of terms is to run the derivation on each element of the list individually. 

\paragraph{From CIC$_{\omega}$ back to Gallina}
Gallina implements CIC$_{\omega}$, but with some important differences.
Three differences are especially relevant:
The first is that Gallina lacks \kl{primitive eliminators}, as I mentioned earlier.
The second is that Gallina has constants that define terms---later on, this will help with building optimizations for proof repair tools.
The third is that variables in Gallina are de Bruijn indices rather than names---the implementation handles this discrepancy.

Otherwise, a proof repair tool for Gallina can harness the power of CIC$_{\omega}$.
In particular, $\Pi$ makes it possible to quantify over both terms and types,
so that we can state powerful theorems and prove that they hold.
Inductive types make it possible to write proofs by induction.
Both of these constructs mean that terms in Gallina are extremely structured,
and as we will soon see, that structure makes a proof repair tool's job much easier.

\paragraph{From Gallina back to Ltac}
This very same structure that helps proof repair tools can be difficult for proof engineers to work with,
which is why proof engineers typically rely on the \kl{Ltac} \kl{tactics} we saw in Section~\ref{sec:mot-workflow}.
Chapter~\ref{chapt:pi} will introduce a prototype decompiler by \kl{RanDair} from \kl{Gallina} back up to \kl{Ltac},
so that proof repair tools can suggest tactics in the end.

Tactics more generally are a form of \intro{proof automation}, or tools that automatically search for proofs.
This proof automation makes it much simpler to develop proofs to begin with.
But it turns out this proof automation is a bit naive when it comes to \textit{maintaining} proofs
as programs and specifications change over time.
Proof repair is a new form of proof automation for maintaining proofs: it uses the rich type information carried by proof terms
to automatically fix broken proofs in response to change.





