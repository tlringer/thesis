\section*{The Next Era: Proof Engineering for All}

% 2030? could do decades for each of this lol. I guess only if energy
% Do just 2-3 sentences per project.

Proof engineering has come a long way, but it is still accessible mostly to experts, and perhaps the occasional practitioner.
Proof repair has made proof engineering easier for experts, and a bit easier for practitioners.
But there is a lot more that we as a community can do to bring proof engineering to all: not just experts and practitioners,
but also software engineers and potential users from in other domains.

I thus conclude with a discussion of 12 future project summaries building up to the next era.
If any of these ideas inspire you, please work with me to bring them to life.

\subsection*{Experts}

With proof repair, I have to some degree tamed the challenges of proof maintenance.
But there is much more to be done, even for experts.
The key to better serving experts with proof repair tools is lateral reach:
building proof repair tools that indespensible for proof engineers experts
across all proof assistants, that support an extremely broad and practical set of changes,
and that in the end produce proof scripts that match the styles of proof engineers.

\paragraph{Proof Assistants} Thoughts from PUMPKIN Pi on Isabelle/HOL, future work from PUMPKIN PATCH.

\paragraph{Changes} Version updates, isolating large changes (PUMPKIN PATCH), relations more general than equivalences (PUMPKIN Pi).

\paragraph{Styles} ML for decompiler (PUMPKIN Pi, REPLICA) : more for diverse proof styles (PUMPKIN PATCH). Note that this is a WIP, but sketch out project, challenges, future ideas, expectations, evaluation a bit.

\subsection*{Practitioners}

Proof engineering is just starting to reach practitioners, like proof engineers in industry---my proof repair tools have been a small part of that.
Usability has been central to that effort.
Still, usability remains a barrier to bringing not just proof repair tools, but also proof engineering 
more broadly to practitioners.
Some paths to better automation and workflow integration are already clear.
For the rest, I believe that empirical evaluation will hold the answers.

\paragraph{Automation} More search procedures for automatic configuration, e-graphs from PUMPKIN Pi, custom unification heuristics.

\paragraph{Integration} IDE \& CI integration, HCI for repair.

\paragraph{Evaluation} repair challenge, user studies ideas (PUMPKIN PATCH, REPLICA, panel w/ Benjamin Pierce, QED at large). (maybe look for more ideas, this can be merged with integration if need be).

\subsection*{Software Engineers}

Unifying theme: mixed methods verification, or the 2030 vision from Twitter thread. Some examples:

\paragraph{Gradual Verification} A continuum from testing to verification, tools to help with that.

\paragraph{Tool-Assisted Proof Development} Tool-assisted development to follow good design principles for verificattion (James Wilcox conversation, final REPLICA takeaway).

\paragraph{Specification Inference} Analysis to infer specs (TA1).

\subsection*{New Domains}

Unifying theme: collaboration, new abstractions for new domains). Some examples:

\paragraph{Machine Learning} Fairification \& other ML correctness properties. Some stuff here but more.

\paragraph{Cryptography} Lots of stuff here but not thinking broadly enough. What about cryptographic proof systems? ZK and beyond. Recall email thread.

\paragraph{Something Else} Look for more in survey paper, email, DARPA TAs, Twitter. Healthcare perhaps? % https://twitter.com/TaliaRinger/status/1373747841944883201
