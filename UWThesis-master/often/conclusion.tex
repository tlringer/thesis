\section{Conclusion}

The \toolnamec plugin addresses the main limitations of the \sysname prototype:
\sysname had limited support for patch application,
supported a narrow class of changes,
did not support typical proof engineering workflows like tactics,
and only retroactively \textit{could have} helped proof engineers in a few practical use cases.
\toolnamec, in contrast, supports patch application in a principled manner,
broadens the scope of \sysnamelong to include a large and flexible class of changes,
is built with workflow integration including tactics in mind,
and can save and in fact already has saved work for proof engineers in practical use cases.

At this point, it is fair to say that my thesis holds:

\begin{quote}
\textbf{Thesis}: Changes in programs, specifications, and proofs carry information that a tool can extract, generalize, and apply to fix other proofs broken by the same change (Sections~\ref{sec:pumpkin-approach},~\ref{sec:pumpkin-diff},~\ref{sec:pumpkin-trans},~\ref{sec:pi-approach},~\ref{sec:pi-diff},and~\ref{pi-trans}). A tool that automates this (Sections~\ref{sec:pumpkin-impl} and~\ref{sec:pi-implementation}) can save work for proof engineers relative to reference manual repairs in 
practical use cases (Sections~\ref{sec:pumpkin-results} and~\ref{sec:pi-results}).
\end{quote}
And so there really is \textit{reason to believe}. % TODO knowledge package

It is, importantly, not just me who believes this now.
Consider a recent article by a proof engineer saying just this (emphasis mine, again): % TODO link: https://galois.com/blog/2020/12/proofs-should-repair-themselves/

\begin{quote}
We have \textit{reason to think} such proof repair is tractable. Rather than trying to synthesize a complete proof from nothing---a problem known to be immensely difficult---we 
start from a correct proof of fairly similar software. We will be attempting proof reconstruction \textit{within a known neighborhood}.
\end{quote}
The proof engineer credited my work on Twitter, % TODO link to Twitter
but noted that there ought to be much, much more work in this space.
I agree, and I am excited to share some ideas for this in Chapter~\ref{chapt:conclusions}.

But first, I would like to back up a bit and talk about other work in this space,
in part because doing so is standard, and in part because it has been and continues to be a wonderful source of inspiration.
I would not skip this chapter---remember that I wrote a whole survey paper of proof engineering,
so I take related work quite seriously.
If you are still awake, please enjoy Chapter~\ref{sec:related}.
